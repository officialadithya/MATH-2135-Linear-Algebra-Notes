\documentclass[oneside]{book}
\usepackage[utf8]{inputenc}
\usepackage[a4paper]{geometry}
\usepackage[english]{babel}
\usepackage{cancel}
\usepackage{quotchap}
\usepackage{amsmath}
\usepackage{amssymb}
\usepackage{amsthm}
\usepackage{appendix}
\usepackage{array}
\usepackage{marvosym}
\usepackage[hidelinks]{hyperref}
\usepackage{color}
\usepackage[table]{xcolor}
\usepackage{amsthm}
\usepackage{graphicx}
\usepackage{polynom}
\usepackage{braket}
\usepackage{fontawesome}
\usepackage{mathtools}
\usepackage{tikz}
\usepackage{tkz-euclide}
\usepackage[most]{tcolorbox}
\usepackage{pgfplots}
\usepackage{multicol}
\usepgfplotslibrary{polar}
\usepgflibrary{shapes.geometric}
\usetikzlibrary{calc}
\pgfplotsset{def/.append style={axis x line=middle, axis y line=middle, xlabel={\(x\)}, ylabel={\(y\)}, axis equal}}
\pgfplotsset{compat=1.17}
\graphicspath{{./images/}}
\usetikzlibrary{fit,shapes}
\usetikzlibrary{graphs}

\DeclareMathOperator{\arcsec}{arcsec}
\DeclareMathOperator{\arccot}{arccot}
\DeclareMathOperator{\arccsc}{arccsc}
\DeclarePairedDelimiter{\ceil}{\lceil}{\rceil}
\DeclarePairedDelimiter{\floor}{\lfloor}{\rfloor}

\usepackage{fix-cm} 
\makeatletter
\newcommand\HUGE{\@setfontsize\Huge{50}{60}} 
\makeatother

\newcommand{\subsubsubsection}[1]{\paragraph{#1}\mbox{}\\}
\newcommand*{\?}{\stackrel{?}{=}}
\newcommand*{\dd}{\,\text{d}}
\newcommand*{\thus}{.^..}
%\renewcommand*{\and}{\wedge}
%\renewcommand*{\or}{\vee}
\newcommand*{\curl}{\text{curl\,}}
\newcommand*{\range}{\text{range\,}}
\renewcommand*{\div}{\text{div\,}}
\newcommand*{\Difficulty}{\Radioactivity}
%\newcommand*{\Difficulty}{\Frowny}
\newcommand*{\Stop}{\Stopsign}
\newcommand*{\DOTHISLATER}{\begin{center}\Stop\Stop\Stop\Stop\Stop\Stop\Stop\Stop\Stop\Stop\Stop\Stop\Stop\Stop\Stop\end{center}}
\newcommand*{\laplace}[2][]{\mathcal{L}^{#1}\left\{#2\right\}}
\newcommand*{\powerset}[1]{\mathcal{P}\left(#1\right)}
\renewcommand*{\vec}[1]{\overset{_{\rightharpoonup}}{#1}}
\renewcommand*{\bar}[1]{\overline{#1}}
\renewcommand*{\gcd}[2]{\text{gcd}\left(#1,#2\right)}
%\renewcommand*{\vec}[1]{\mathbf{#1}}
\newcommand*{\nin}{\notin}
\newcommand*{\nequiv}{\not\equiv}
\newcommand*{\Span}{\text{span\,}}
\newcommand*{\proj}{\text{proj\,}}
\newcommand*{\rank}{\text{rank\,}}
\newcommand*{\nullity}{\text{nullity\,}}
\newcommand*{\trace}{\text{trace\,}}
\newcommand*{\iprod}[2]{\left\langle #1,#2\right\rangle}


\usepackage{sfmath}
\renewcommand{\familydefault}{\sfdefault}

% \renewcommand*{\qedsymbol}{\(\blacksquare\)}

\newtcbtheorem[number within=section]{definition}{Definition}{
                lower separated=false,
                breakable,
                before skip=0.5cm,
                colback=white,
                colframe=black,fonttitle=\bfseries,
                colbacktitle=black,
                coltitle=white,
                enhanced,
                attach boxed title to top left={yshift=-0.1in,xshift=0.15in},
                }{def}
                
\newtcbtheorem[number within=section]{example}{Example}{
                lower separated=false,
                breakable,
                before skip=0.5cm,
                colback=gray!15,
                colframe=white!10!black,fonttitle=\bfseries,
                colbacktitle=black,
                coltitle=white,
                enhanced,
                attach boxed title to top left={yshift=-0.1in,xshift=0.15in},
                }{exa}
                
\newtcbtheorem[number within=section]{exercise}{Exercise}{
                lower separated=false,
                breakable,
                before skip=0.5cm,
                colback=gray!15,
                colframe=white!10!black,fonttitle=\bfseries,
                colbacktitle=black,
                coltitle=white,
                enhanced,
                attach boxed title to top left={yshift=-0.1in,xshift=0.15in},
                }{exe}
                
\newtcbtheorem[number within=section]{theorem}{Theorem}{
                lower separated=false,
                breakable,
                before skip=0.5cm,
                colback=white,
                colframe=black,fonttitle=\bfseries,
                colbacktitle=black,
                coltitle=white,
                enhanced,
                attach boxed title to top left={yshift=-0.1in,xshift=0.15in},
                }{thm}
                
\newtcbtheorem[number within=section]{solution}{Solution}{
                lower separated=false,
                before skip=0.5cm,
                breakable,
                colback=white,
                colframe=black,fonttitle=\bfseries,
                colbacktitle=black,
                coltitle=white,
                enhanced,
                attach boxed title to top left={yshift=-0.1in,xshift=0.15in},
                }{sol}
            
\usepackage{listings}
\usepackage{xcolor}

% Configuration for Syntax Highlighting
\definecolor{codegreen}{rgb}{0,0.6,0}
\definecolor{codegray}{rgb}{0.5,0.5,0.5}
\definecolor{codepurple}{rgb}{0.58,0,0.82}
\definecolor{backcolour}{rgb}{0.95,0.95,0.92}
\definecolor{bggray}{gray}{0.9}

\renewcommand{\ttdefault}{pcr}
\lstdefinestyle{code}{
    backgroundcolor=\color{bggray},   
    commentstyle=\color{gray},
    keywordstyle=\bfseries,
    morekeywords={def},
    numberstyle=\tiny\color{black},
    %stringstyle=\color{codepurple},
    basicstyle=\ttfamily\footnotesize,
    breakatwhitespace=false,         
    breaklines=true,                 
    captionpos=b,                    
    keepspaces=true,                 
    numbers=left,                    
    numbersep=5pt,                  
    showspaces=false,                
    showstringspaces=false,
    showtabs=false,                  
    tabsize=2
}

\lstset{style=code}

\newenvironment{code}{\fontfamily{lmtt}\selectfont}{\par}
\title{

    \rule{15cm}{1.6pt}\vspace*{-\baselineskip}\vspace*{2pt}
    \rule{15cm}{0.4pt}
	
	\vspace{0.75\baselineskip}
		
	\Huge{MATH2135: LINEAR ALGEBRA\\\vspace{3mm}}

	\rule{15cm}{0.4pt}\vspace*{-\baselineskip}\vspace{3.2pt}
	\rule{15cm}{1.6pt}

}

\author{ADITHYA BHASKARA\\\vspace{0.4cm}\small{PROFESSOR: ROBIN DEELEY}\\\vspace{1em}\small{TEXTBOOK: STEPHEN ANDRILLI \& DAVID HECKER}}

\date{}
\geometry{left=2.54cm}
\geometry{right=2.54cm}
\colorlet{chaptergrey}{black}

\begin{comment}
\usepackage{draftwatermark}

\SetWatermarkText{Draft: \today}
\SetWatermarkColor[gray]{0.5}
\SetWatermarkFontSize{1cm}
\SetWatermarkAngle{90}
\SetWatermarkHorCenter{20cm}
\end{comment}

\begin{document}

\begin{center} 
    \begin{minipage}{\textwidth}
        \maketitle
        \begin{center}
        \begin{tabular}{c}
            UNIVERSITY OF COLORADO BOULDER \\
            \hline \\
        \end{tabular}
        \\
        \includegraphics[scale=0.2]{Graphics/CU.png}
        \end{center}
    \end{minipage}
\end{center}

\vfill
\rightline{\textbf{EDITION 3}}
\frontmatter

\begin{center}
    \includegraphics[scale=1.5, width=\textwidth]{Graphics/latex.png}
\end{center}

\pagebreak

\chapter*{Preface}
\addcontentsline{toc}{chapter}{Preface}

To the interested reader,
\\
\\
This document is a compilation of lecture notes taken during the Fall 2022 semester for MATH2135: Linear Algebra for Mathematics Majors at the University of Colorado Boulder. The course used \cite{andrilli2016linear} as its primary text. Supplemental texts included \cite{treil2017linear,axler2024linear,olver2006applied}. As such, many theorems, definitions, and content may be quoted or derived from the aforementioned books. This course was taught by Robin Deeley, Ph. D.
\\
\\
Appendix~\ref{appendix:a} is provided as a nonexhaustive refresher on proof techniques commonly covered in a discrete mathematics course. Appendix~\ref{appendix:a} is a result of lecture notes compiled during the Summer 2022 semester for MATH2520: Discrete Mathematics at Front Range Community College. This course was taught by Kenneth M. Monks, Ph. D. The primary text was \cite{levin2021discrete}, but \cite{hammack2018book} was also used as supplement.
\\
\\
Appendix~\ref{appendix:b} is provided as a nonexhaustive refresher on functions, as preparation for Chapter~\ref{chapter:lintrans}, the chapter on linear transformations. Appendix~\ref{appendix:b} is a result of both content covered in \cite{andrilli2016linear} and MATH2520 lecture notes.
\\
\\
Appendix~\ref{appendix:c} is provided as an introduction to numerical linear algebra, as covered by APPM4600: Numerical Methods and Scientific Computation. This course was taught by Eduardo Corona, Ph. D. There was no official course text, but in writing the section, \cite{olver2006applied} was primarily referenced. It is worth noting that this appendix is less-polished than the rest of the document.
\\
\\
Each chapter in this text has an accompanying quote, or accompanying image, from \href{https://xkcd.com/license.html}{XKCD}. These additions are tied to the content of the chapter and are meant to add lighthearted humor to the material. 
\\
\\
For quick reference, the backmatter contains a list of theorems and definitions and the page numbers on which they are on.
\\
\\
The current edition of this text consists primarily of concepts covered in MATH2135. Appendix~\ref{appendix:c} contains some exposition on numerical linear algebra. In the future, however, the author would like to add additional topics, not covered in the courses, to provide for a more complete text. Possible future topics include, but are not limited to, optimization in \(\mathbb{R}^n\) with the Hessian matrix, applications of linear algebra to differential equations, and linear programming. As MATH2135 is a proof-based course, the author will prioritize rigor in the presentation of any additional topics and will refrain from implementing sections where the necessary level of rigor has not been reached.
\\
\\
Future editions are more likely to resolve errata or reframe the discussion of existing material instead of implementing new sections entirely. By the end of MATH2135, a student will have a degree of mathematical maturity enough to make inroads into additional topics in linear algebra without direct instruction.
\pagebreak
\vphantom
\\
\\
The author would like to provide the following additional resources for students currently taking a linear algebra course:
\begin{enumerate}
    \item \href{https://www.math.brown.edu/streil/papers/LADW/LADW_2017-09-04.pdf}{Sergei Treil's \textit{Linear Algebra Done Wrong}.}
    \item \href{https://linear.axler.net/LADR4e.pdf}{Sheldon Axler's \textit{Linear Algebra Done Right}.}
    \item \href{https://youtube.com/playlist?list=PL221E2BBF13BECF6C}{Gilbert Strang's Linear Algebra Lectures From Fall 2011.}
    \item \href{https://www.youtube.com/playlist?app=desktop&list=PLZHQObOWTQDPD3MizzM2xVFitgF8hE_ab}{3Blue1Brown's \textit{Essence of Linear Algebra}.}
\end{enumerate}
While much effort has been put in to remove typos and mathematical errors, it is very likely that some errors, both small and large, are present. Please keep in mind that the author wrote this resource during his undergraduate studies. If an error needs to be resolved, please contact Adithya Bhaskara at \href{mailto:adithya@colorado.edu}{adithya@colorado.edu}.
\\
\\
Finally, the author would like to dedicate this set of lecture notes to \textit{Aidan Janney}, \textit{Erika Sj\"{o}blom}, and \textit{Tate McDonald}, three of the author's closest friends who plan to take Linear Algebra in the upcoming semester; Spring 2023, at the time of writing.
\\
\\
\rightline{Best Regards,}
\rightline{Adithya Bhaskara}
\vfill
\rightline{\textbf{REVISED: \today}}

\pagebreak

\setcounter{tocdepth}{4}
\setcounter{secnumdepth}{4}
\tableofcontents

\mainmatter

\begin{savequote}
I go by the name of Vector. It's a mathematical term, represented by an arrow with both direction and magnitude. Vector! That's me, because I commit crimes with both direction and magnitude. Oh yeah!
\\
\\
Victor (Vector) Perkins
\end{savequote}
\chapter{Vectors and Matrices} \label{chapter:vecmat}

    \section{Lecture 1: August 22, 2022}

    \subsection{Notations, Definitions, and Conventions}

    Consider the following table for a very basic review of fundamental sets.
    \begin{center}
        \begin{tabular}{|c|c|}
            \hline
            \hline
            Symbol & Explanation \\
            \hline
            \hline
            \(\mathbb{N}\) & Natural Numbers (\(\mathbb{N}=\{1,2,\ldots\}\)) \\
            \hline
            \(\mathbb{Z}\) & Integers \\
            \hline
            \(\mathbb{Q}\) & Rationals \\
            \hline
            \(\mathbb{R}\) & Reals \\
            \hline
            \(\mathbb{R}^n\) & \(\{[v_1,v_2,\ldots,v_n]:v_1,\ldots,v_n\in\mathbb{R}\}\) \\
            \hline 
        \end{tabular}.
    \end{center}
    We will also note that there is an important distinction between vectors and points. Vectors describe ``movement,'' whereas points describe location. For example, the vector \([1,2]\) starting at the point \((1,1)\) ends at the point \((2,3)\). Consider the following diagram.
    \begin{center}
        \begin{tikzpicture}[scale=0.5]
            \draw[-to] (0, -5) -- (0, 5) node[above] {\(y\)};
            \draw[-to] (-5, 0) -- (5, 0) node[right] {\(x\)};
            \draw[-to] (1, 1) -- (2, 3) node[right] {\(\vec{v}\)};
            \tkzDefPoint(1,1){INIT};
            \tkzLabelPoint[right, xshift=0mm, yshift=0mm](INIT){\((1,1)\)};
            \node at (INIT)[circle, fill, inner sep=1.5pt]{};
            \tkzDefPoint(2,3){TERM};
            \tkzLabelPoint[left, xshift=0mm, yshift=0mm](TERM){\((2,3)\)};
            % \node at (TERM)[circle, fill, inner sep=1.5pt]{};
        \end{tikzpicture}
    \end{center}
    \vphantom
    \pagebreak
    \\
    \\
    Consider the following definitions and statements.
    \begin{definition}{\Stop\,\,The Zero Vector}{zerovec}

        We define 
        \begin{equation*}
            \vec{0}=[0,0,\ldots,0].
        \end{equation*}
        
    \end{definition}
    \begin{definition}{\Stop\,\,Vector Equality}{vecequal}
        Two vectors \(\vec{v}=[v_1,v_2,\ldots,v_n]\) and \(\vec{w}=[w_1,w_2,\ldots,w_n]\) are equal if and only if
        \begin{equation*}
            v_1=w_1,v_2=w_2,\ldots,v_n=w_n.
        \end{equation*}
    \end{definition}
    \begin{definition}{\Stop\,\,Vector Magnitude}{vecmagn}

        Given a \(\vec{v}=[v_1,\ldots, v_n]\in\mathbb{R}^n\), we define \(||\vec{v}||\), the magnitude, or norm, of \(\vec{v}\), as
        \begin{equation*}
            ||\vec{v}||=\sqrt{v_1^2+\cdots+v_n^2}.
        \end{equation*}
        
    \end{definition}
    \begin{definition}{\Stop\,\,Scalar Multiplication}{scalmult}

        Given a \(\vec{v}=[v_1,\ldots v_n]\in\mathbb{R}^n\), and a scalar \(c\in\mathbb{R}\), we define scalar multiplication as
        \begin{equation*}
            c\vec{v}=[cv_1,\ldots, cv_n].
        \end{equation*}
        
    \end{definition}
    \vphantom
    \\
    \\
    Consider the following theorem.
    \begin{theorem}{\Stop\,\,Scalar Multiplication and Magnitude}{scalmultandmagn}

        Given a scalar \(c\in\mathbb{R}\) and \(\vec{v}\in\mathbb{R}^n\), 
        \begin{equation*}
            ||c\vec{v}||=|c|||\vec{v}||.
        \end{equation*}
        \begin{proof}
        \begin{align*}
            ||c\vec{v}||&=\sqrt{(cv_1)^2+\cdots+(cv_n)^2} \\
            &=\sqrt{c^2v_1^2+\cdots+c^2v_n^2} \\
            &=\sqrt{c^2}\sqrt{v_1^2+\cdots+v_n^2} \\
            &=|c|||\vec{v}||.
        \end{align*}    
        The theorem is hence proved.
        \end{proof}
            
    \end{theorem}
    \pagebreak
    \vphantom
    \\
    \\
    Consider the following definitions.
    \begin{definition}{\Stop\,\,Vector Direction}{vectdir}

        Two nonzero vectors \(\vec{v},\vec{w}\in\mathbb{R}^n\) are
        \begin{enumerate}
            \item in the same direction if there exists \(c>0\) such that \(\vec{v}=c\vec{w}\).
            \item in opposite directions if there exists \(c<0\) such that \(\vec{v}=c\vec{w}\).
        \end{enumerate}
        
    \end{definition}
    \begin{definition}{\Stop\,\,Unit Vectors}{unitvec}
        The vector \(\vec{v}\in\mathbb{R}^n\) is a unit vector if and only if \(||\vec{v}||=1\).
    \end{definition}
    \vphantom
    \\
    \\
    Consider the following theorem.
    \begin{theorem}{\Stop\,\,Unit Vectors Represent Direction}{unitvecdir}
        Given a nonzero \(\vec{v}\in\mathbb{R}^n\), there exists a unique unit vector in the same direction as \(\vec{v}\).
        \begin{proof}
            Since \(\vec{v}\neq\vec{0}\), \(||\vec{v}||\neq0\). % CHECK THIS! 
            Let \(\vec{u}=\frac{1}{||\vec{v}||}\vec{v}\), meaning that
            \begin{align*}
                ||\vec{u}||&=\left|\left|\frac{1}{||\vec{v}||}\vec{v}\right|\right| \\
                &=\left|\frac{1}{||\vec{v}||}\right|||\vec{v}|| \\
                &=1.
            \end{align*}
            This means that \(\vec{u}\) is a unit vector and, because \(\frac{1}{||\vec{v}||}\) is a scalar, is in the same direction as \(\vec{v}\). We have now shown existence. For uniqueness, assume \(\vec{w}\) is a unit vector in the same direction as \(\vec{v}\). Then, for a positive \(c\in\mathbb{R}\), \(\vec{w}=c\vec{v}\) and \(||\vec{w}||=1\). That is,
            \begin{align*}
                ||\vec{w}||=|c|||\vec{v}||=1,
            \end{align*}
            meaning that \(|c|=\frac{1}{||\vec{v}||}\). Because \(c>0\), \(|c|=c\), so \(\vec{w}=\vec{u}\).
        \end{proof}
    \end{theorem}
    \pagebreak
    \vphantom
    \\
    \\
    In Theorem \ref{thm:unitvecdir}, we used the fact that for some \(\vec{v}\in\mathbb{R}^n\), \(\vec{v}\neq\vec{0}\implies||\vec{v}||\neq0\). We will now provide a proof.
    \begin{theorem}{\Stop\,\,Nonzero Vector Implies Nonzero Magnitudes}{nonzerovecnonzeromagn}

        For some \(\vec{v}=[v_1,\ldots,v_n]\in\mathbb{R}^n\), 
        \begin{equation*}
            \vec{v}\neq\vec{0}\implies||\vec{v}||\neq0.
        \end{equation*}
        \begin{proof}
            We will proceed by proving the contrapositive. Suppose that \(||\vec{v}||=0\), meaning that
            \begin{equation*}
                ||\vec{v}||=\sqrt{v_1^2+\cdots+v_n^2}=0.
            \end{equation*}
            The only way this is true is if all of \(v_1,\ldots,v_n\) are zero, which, by definition, means that \(\vec{v}=\vec{0}\).
        \end{proof}
        
    \end{theorem}
    

    \begin{comment}
    \vphantom
    \\
    \\
    The famous Cauchy-Schwarz Inequality is stated below.
    \begin{theorem}{\Stop\,\,The Cauchy-Schwarz Inequality}{cauchyschwarz}

        Let \(\vec{v},\vec{w}\in\mathbb{R}^n\). Then,
        \begin{equation*}
            \vec{v}\cdot\vec{w}\leq||\vec{v}||||\vec{w}||.
        \end{equation*}
        \begin{proof}
            We may rewrite the equation as
            \begin{align*}
                \sum_{k=1}^n v_kw_k&\leq\sqrt{\left(\sum_{k=1}^nv_k^2\right)\left(\sum_{k=1}^nw_k^2\right)} \\
                &\leq\sqrt{(v_1^2+v_2^2\cdots+v_n^2)(w_1^2+w_2^2+\cdots+w_n^2)} \\
                &\leq\sqrt{v_1^2w_1^2+v_1^2w_2^2+v_2^2w_1^2+v_2^2w_2^2+\cdots+v_1^2w_n^2+v_2^2w_n^2+v_n^2w_1^2+v_n^2w_2^2+v_n^2w_n^2}
            \end{align*}
            The terms \(v_1^2w_1^2, v_2^2w_2^2,\ldots,v_n^2w_n^2\) are all present on the left hand side, but do not make up all the addends on the right hand side.
        \end{proof}
        
    \end{theorem}
    \end{comment}

\pagebreak

\section{Lecture 2: August 24, 2022}

    \subsection{Vector Operations}

    As a review from the previous lecture, consider the following exercises.
    \begin{example}{\Difficulty\,\Difficulty\,\,Vector Representing Movement}{movevec}

        Find the vector representing the movement from \((3,1)\) to \((-1,2)\).
        \\
        \\
        We simply find the change in the \(y\) coordinates and the change in the \(x\) coordinates to find the vector \([-4, 1]\), \textit{starting from} \((3,1)\). This is illustrated below.
        \begin{center}
            \begin{tikzpicture}[scale=0.5]
                \draw[-to] (0, -5) -- (0, 5) node[above] {\(y\)};
                \draw[-to] (-5, 0) -- (5, 0) node[right] {\(x\)};
                \draw[-to] (3, 1) -- (-1, 2) node[above] {\(\vec{v}\)};
                \tkzDefPoint(3,1){INIT};
                \tkzLabelPoint[right, xshift=0mm, yshift=0mm](INIT){\((3,1)\)};
                \node at (INIT)[circle, fill, inner sep=1.5pt]{};
                \tkzDefPoint(-1, 2){TERM};
                \tkzLabelPoint[left, xshift=0mm, yshift=0mm](TERM){\((-1,2)\)};
                %\node at (TERM)[circle, fill, inner sep=1.5pt]{};
            \end{tikzpicture}
        \end{center}
        
    \end{example}
    \begin{example}{\Difficulty\,\Difficulty\,\,Finding a Unit Vector}{unitvec}

        Find the unit vector in the direction \([3,-1,-\pi]\).
        \\
        \\
        We construct the unit vector by normalization. This produces \(\frac{[3,-1,-\pi]}{\sqrt{3^2+(-1)^2+(-\pi)^2}}\).
    
    \end{example}
    \pagebreak
    \vphantom
    \\
    \\
    We will now define addition and subtraction with vectors and provide a few properties.
    \begin{definition}{\Stop\,\,Addition and Subtraction With Vectors}{addsubvec}

        Let \(\vec{v}=[v_1,\ldots,v_n]\) and \(\vec{w}=[w_1,\ldots,w_n]\) be vectors in \(\mathbb{R}^n\). Then,
        \begin{equation*}
            \vec{v}\pm\vec{w}=[v_1\pm w_1,\ldots,v_n\pm w_n].
        \end{equation*}
        
    \end{definition}
    \vphantom
    \\
    \\
    Geometrically, we may visualize vector addition as
    \begin{center}
    \begin{tikzpicture}[scale=0.5]
        \draw[-to] (0, -5) -- (0, 5) node[above] {\(y\)};
        \draw[-to] (-5, 0) -- (5, 0) node[right] {\(x\)};
        \draw[-to] (0, 0) -- (2, 1) node[below] {\(\vec{v}\)};
        \draw[-to] (2, 1) -- (1.5, 3) node[above] {\(\vec{w}\)};
        \draw[dashed, -to] (0, 0) -- (1.5, 3) node[right] {\(\vec{v}+\vec{w}\)};
        \tkzDefPoint(0,0){ORIGIN};
        \node at (ORIGIN)[circle, fill, inner sep=1.5pt]{};
        \tkzDefPoint(2,1){INIT};
        \tkzLabelPoint[right, xshift=0mm, yshift=0mm](INIT){\((v_1,v_2)\)};
        \node at (INIT)[circle, fill, inner sep=1.5pt]{};
        \tkzDefPoint(1.5, 3){TERM};
        \tkzLabelPoint[left, xshift=0mm, yshift=0mm](TERM){\((w_1,w_2)\)};
        % \node at (TERM)[circle, fill, inner sep=1.5pt]{};
        \end{tikzpicture}.
    \end{center}
    \begin{theorem}{\Stop\,\,Properties of Vector Addition and Scalar Multiplication}{addscalmult}

        Let \(\vec{u}=[u_1,\ldots,u_n]\), \(\vec{v}=[v_1,\ldots,v_n]\), and \(\vec{w}=[w_1,\ldots,w_n]\) be vectors in \(\mathbb{R}^n\). Let \(c\) and \(d\) be scalars. Then,
        \begin{enumerate}
            \item \(\vec{u}+\vec{v}=\vec{v}+\vec{u}\)
            \item \(\vec{u}+(\vec{v}+\vec{w})=(\vec{u}+\vec{v})+\vec{w}\)
            \item \(\vec{0}+\vec{u}=\vec{u}+\vec{0}=\vec{u}\)
            \item \(\vec{u}+(-\vec{u})=(-\vec{u})+\vec{u}=\vec{0}\)
            \item \(c(\vec{u}+\vec{v})=c\vec{u}+c\vec{v}\)
            \item \((c+d)\vec{u}=c\vec{u}+d\vec{u}\)
            \item \((cd)\vec{u}=c(d\vec{u})\)
            \item \(1\vec{u}=\vec{u}\)
        \end{enumerate}
        
    \end{theorem}
    \vphantom
    \\
    \\
    Note that \(\vec{0}\) is called the \textit{identity element} for addition and \(-\vec{u}\) is the \textit{additive inverse element of \(\vec{u}\)}.
    \pagebreak
    \\
    \\
    We present the proof of one of the components of Theorem \ref{thm:addscalmult}.
    \begin{example}{\Difficulty\,\Difficulty\,\,Commutativity of Addition}{addcommut}

        Let \(\vec{v}=[v_1,\ldots,v_n]\) and \(\vec{w}=[w_1,\ldots,w_n]\) be vectors in \(\mathbb{R}^n\). Prove that
        \begin{equation*}
            \vec{v}+\vec{w}=\vec{w}+\vec{v}.
        \end{equation*}
        \begin{proof}
            Consider the following.
            \begin{align*}
                \vec{v}+\vec{w}&=[v_1,\ldots,v_n]+[w_1,\ldots,w_n] \\
                &=[v_1+w_1,\ldots,v_n+w_n] \\
                &=[w_1+v_1,\ldots w_n+v_n] \\
                &=\vec{w}+\vec{v}.
            \end{align*}
            The proposition is hence proved.
        \end{proof}

    
    \end{example}
    \begin{theorem}{\Stop\,\,Scalar Multiplication Producing the Zero Vector}{scalmultzero}

        Let \(\vec{v}\in\mathbb{R}^n\) and let \(c\) be a scalar. Then,
        \begin{equation*}
            (c=0\vee\vec{v}=\vec{0})\iff c\vec{v}=\vec{0}.
        \end{equation*}
        \begin{proof}
            First, we wish to show that 
            \begin{equation*}
                (c=0\vee\vec{v}=\vec{0})\implies c\vec{v}=\vec{0}.
            \end{equation*}
            Suppose \(c=0\). We have
            \begin{equation*}
                \vec{0}=c\vec{v}=[cv_1,\ldots,cv_n].
            \end{equation*}
            We wish to show that all of \(cv_1,\ldots,cv_n\) must be zero, no matter the components of \(\vec{v}\). By basic arithmetic, zero multiplied by any other number is also zero, so if \(c=0\), \(c\vec{v}\) must be \(\vec{0}\), as all the components of \(c\vec{v}\) are zero. Suppose \(\vec{v}=\vec{0}\). We again have
            \begin{equation*}
                \vec{0}=c\vec{v}=[cv_1,\ldots,cv_n]
            \end{equation*}
            and wish to show that all of \(cv_1,\ldots,cv_n\) must be zero, no matter the value of \(c\). If \(\vec{v}=0\), all components of \(\vec{v}\) are zero, and again, zero multiplied by any other number is zero, so all components of \(c\vec{v}\) are zero. We have now shown that, indeed, \((c=0\vee\vec{v}=\vec{0})\implies c\vec{v}=\vec{0}\). Now, we wish to show that
            \begin{equation*}
                c\vec{v}=\vec{0}\implies(c=0\vee\vec{v}=\vec{0}).
            \end{equation*}
            Suppose that \(\vec{v}\neq\vec{0}\). Again, we have 
            \begin{equation*}
                \vec{0}=c\vec{v}=[cv_1,\ldots,cv_n].
            \end{equation*}
            There must be some nonzero \(v_1,\ldots,v_n\). Let this nonzero number be \(n\). The only way \(cn=0\) is if \(c=0\).
        \end{proof} 
        
    \end{theorem}

\pagebreak

\section{Lecture 3: August 26, 2022}

    \subsection{Matrices}

        Consider the following warm-up exercise.
        \begin{example}{\Difficulty\,\Difficulty\,\,General Vector Movement}{genvecmvt}

            What is the formula for the vector \(\vec{v}\) representing the movement from \(A=(a_1,\ldots,a_n)\) to \(B=(b_1,\ldots,b_n)\). Then, find the magnitude of \(\vec{v}\).
            \\
            \\
            We see that
            \begin{equation*}
                \vec{v}=[b_1-a_1,\ldots,b_n-a_n],
            \end{equation*}
            starting from \(A\). Then,
            \begin{equation*}
                ||\vec{v}||=\sqrt{(b_1-a_1)^2+\cdots+(b_n-a_n)^2}.
            \end{equation*}
        
        \end{example}
        \vphantom
        \\
        \\
        Consider the following definitions.
        \begin{definition}{\Stop\,\,Matrices}{matrices}
        
            Let \(m,n\in\mathbb{N}\). An \(m\times n\) matrix is a rectangular array of real numbers with \(m\) rows and \(n\) columns. Matrices are often denoted with capital letters. The elements are called entries, and are usually written as
            \begin{equation*}
                A=\begin{bmatrix} a_{11} & \cdots & a_{1n} \\ \vdots & \ddots & \vdots \\ a_{m1} & \cdots & a_{mn} \end{bmatrix}.
            \end{equation*}
            The main diagonal of \(A\) consists of \(a_{11},a_{22},a_{33},\ldots\).
        
        \end{definition}
        \begin{definition}{\Stop\,\,Square Matrices}{sqmatrices}
        
            A matrix is square if and only if \(m=n\).
        
        \end{definition}
        \begin{definition}{\Stop\,\,Diagonal Matrices}{diagmatrices}
        
            A matrix is diagonal if and only if it is square and \(a_{ij}=0\) whenever \(i\neq j\). That is, all elements not on the main diagonal are zero.
        
        \end{definition}
        \begin{definition}{\Stop\,\,The Identity Matrix}{identitymatrices}
        
            The identity matrix \(A\) is an \(n\times n\) matrix where
            \begin{equation*}
                a_{ij}=\begin{cases} 1 & i=j \\ 0 & i\neq j \end{cases}.
            \end{equation*}
        
        \end{definition}
        \begin{definition}{\Stop\,\,The Zero Matrix}{zeromatrices}
        
            The zero matrix \(A\) is an \(m\times n\) matrix where \(a_{ij}=0\) for all \(i\) and \(j\).
            
        \end{definition}
        \begin{definition}{\Stop\,\,Upper Triangular Matrices}{uppertriangularmatrices}
        
            An upper triangular matrix is a matrix such that \(a_{ij}=0\) for all \(i>j\).
            
        \end{definition}
        \begin{definition}{\Stop\,\,Lower Triangular Matrices}{lowertriangularmatrices}
        
            An lower triangular matrix is a matrix such that \(a_{ij}=0\) for all \(i<j\).
            
        \end{definition}
        \begin{definition}{\Stop\,\,The Set of All \(m\times n\) Matrices}{matset}
        
            The set \(\mathcal{M}_{mn}\) is the set of all \(m\times n\) matrices.
            
        \end{definition}
        \begin{definition}{\Stop\,\,Matrix Addition}{matrixaddition}
        
            Given \(A,B\in\mathcal{M}_{mn}\), we define \(A\pm B\) to be the matrix in \(\mathcal{M}_{mn}\) with entries \(a_{ij}\pm b_{ij}\).
        
        \end{definition}
        \vphantom
        \\
        \\
        Consider the following example.
        \begin{example}{\Difficulty\,\Difficulty\,\,Matrix Addition}{matadd}
        
            Consider the following addition.
            \begin{equation*}
                \begin{bmatrix} 3 & 1 & e \\ 0 & \pi & 5 \end{bmatrix}+\begin{bmatrix} -1 & 3 & e \\ 5 & 0 & -2 \end{bmatrix}=\begin{bmatrix} 2 & 4 & 2e \\ 5 & \pi & 3 \end{bmatrix}.
            \end{equation*}
        
        \end{example}
        \vphantom
        \\
        \\
        Consider the following definition.
        \begin{definition}{\Stop\,\,Scalar Multiplication With Matrices}{scalmultmat}
        
            Given \(A\in\mathcal{M}_{mn}\) and \(c\in\mathbb{R}\), we define \(cA\) as the matrix with elements \(ca_{ij}\).
        
        \end{definition}
        \vphantom
        \\
        \\
        Consider the following example.
        \begin{example}{\Difficulty\,\Difficulty\,\,Scalar Multiplication With Matrices}{scalmultmat}
        
            Consider the following scalar multiplication.
            \begin{equation*}
                5\begin{bmatrix} 5 & 1 \\ -1 & 0 \end{bmatrix}=\begin{bmatrix} 25 & 5 \\ -5 & 0 \end{bmatrix}.
            \end{equation*}
        
        \end{example}
        \pagebreak
        \vphantom
        \\
        \\
        Consider the following definition.
        \begin{definition}{\Stop\,\,Matrix Transpose}{transpose}
        
            Given \(A\in\mathcal{M}_{mn}\), we define \(A^T\in\mathcal{M}_{nm}\) to be the matrix with the \((i,j)\) entry equal to the \((j,i)\) entry of \(A\).
        
        \end{definition}
        \vphantom
        \\
        \\
        Consider the following example.
        \begin{example}{\Difficulty\,\Difficulty\,\,Matrix Transpose}{transpose}
        
            Consider the following transpose.
            \begin{equation*}
                \begin{bmatrix} 2 & -5 \\ 3 & -1 \\ \pi & -\pi \end{bmatrix}^T=\begin{bmatrix} 2 & 3 & \pi \\ -5 & -1 & -\pi \end{bmatrix}.
            \end{equation*}
        
        \end{example}
        \vphantom
        \\
        \\
        Consider the following definition.
        \begin{definition}{\Stop\,\,Symmetric and Skew-Symmetric Matrices}{symmetricmatrices}
        
            Suppose \(A\in\mathcal{M}_{nn}\). Then,
            \begin{enumerate}
                \item \(A\) is symmetric if and only if \(A=A^T\)
                \item \(A\) is skew-symmetric if and only if \(A=-A^T\).
            \end{enumerate}
            \vphantom
            \\
            \\
            Note that \(A\) being square is necessary for the above conditions, but this condition is not sufficient.
        
        \end{definition}
        \begin{example}{\Difficulty\,\,A Symmetric Matrix}{symmat}
        
            Let \(A=\begin{bmatrix} 2 & 1 \\ 1 & 1\end{bmatrix}\). Is \(A\) symmetric?
            \\
            \\
            We see that
            \begin{equation*}
                A^T=\begin{bmatrix} 2 & 1 \\ 1 & 1\end{bmatrix},
            \end{equation*}
            so \(A\) is symmetric.
        
        \end{example}
        \pagebreak
        \begin{example}{\Difficulty\,\,A Skew-Symmetric Matrix}{skewsymmat}
        
            Let \(A=\begin{bmatrix} 0 & -1 \\ 1 & 0\end{bmatrix}\). Is \(A\) symmetric?
            \\
            \\
            We see that
            \begin{equation*}
                A^T=\begin{bmatrix} 0 & 1 \\ -1 & 0\end{bmatrix}=-A,
            \end{equation*}
            so \(A\) is skew-symmetric.
        
        \end{example}
        \vphantom
        \\
        \\
        Consider the following theorems.
        \begin{theorem}{\Stop\,\,Transpose Properties}{transprop}
        
            Suppose \(A,B\in\mathcal{M}_{mn}\) and \(c\in\mathbb{R}\). Then,
            \begin{enumerate}
                \item \((A^T)^T=A\)
                \item \((A+B)^T=A^T+B^T\)
                \item \((cA)^T=cA^T\).
            \end{enumerate}
            \begin{proof}
                Consider the matrix 
                \begin{equation*}
                    A=\begin{bmatrix} a_{11} & \cdots & a_{1n} \\ \vdots & \ddots & \vdots \\
                    a_{m1} & \cdots & a_{mn} \end{bmatrix}.
                \end{equation*}
                We see that
                \begin{equation*}
                    A^T=\begin{bmatrix} a_{11} & \cdots & a_{m1} \\ \vdots & \ddots & \vdots \\
                    a_{1n} & \cdots & a_{mn} \end{bmatrix}.
                \end{equation*}
                We will take the transpose of \(A^T\), that is \((A^T)^T\), to yield
                \begin{align*}
                    (A^T)^T&=\begin{bmatrix} a_{11} & \cdots & a_{1n} \\ \vdots & \ddots & \vdots \\
                    a_{m1} & \cdots & a_{mn} \end{bmatrix} \\
                    &=A.
                \end{align*}
                We have proved the first property. Let the matrix
                \begin{equation*}
                    B=\begin{bmatrix} b_{11} & \cdots & b_{1n} \\ \vdots & \ddots & \vdots \\
                    b_{m1} & \cdots & b_{mn} \end{bmatrix},
                \end{equation*}
                and therefore,
                \begin{equation*}
                    A+B=\begin{bmatrix} a_{11}+b_{11} & \cdots & a_{1n}+b_{1n} \\ \vdots & \ddots & \vdots \\
                    a_{m1}+b_{m1} & \cdots & a_{mn}+b_{mn} \end{bmatrix}.
                \end{equation*}
                Then,
                \begin{align*}
                    (A+B)^T&=\begin{bmatrix} a_{11}+b_{11} & \cdots & a_{m1}+b_{m1} \\ \vdots & \ddots & \vdots \\ a_{1n}+b_{1n} & \cdots & a_{mn}+b_{mn} \\
                    \end{bmatrix} \\
                    &=\begin{bmatrix} a_{11} & \cdots & a_{m1} \\
                    \vdots & \ddots & \vdots \\
                    a_{1n} & \cdots & a_{mn} \end{bmatrix}+\begin{bmatrix} b_{11} & \cdots & b_{m1} \\
                    \vdots & \ddots & \vdots \\
                    b_{1n} & \cdots & b_{mn} \end{bmatrix} \\
                    &=A^T+B^T.
                \end{align*}
                Thus, we have proved the second property. Finally, consider the matrix
                \begin{equation*}
                    cA=\begin{bmatrix} ca_{11} & \cdots & ca_{1n} \\ \vdots & \ddots & \vdots \\
                    ca_{m1} & \cdots & ca_{mn} \end{bmatrix}.
                \end{equation*}
                Then,
                \begin{align*}
                    (cA)^T&=\begin{bmatrix} ca_{11} & \cdots & ca_{m1} \\ \vdots & \ddots & \vdots \\
                    ca_{1n} & \cdots & ca_{mn} \end{bmatrix} \\
                    &=c\begin{bmatrix} a_{11} & \cdots & a_{m1} \\ \vdots & \ddots & \vdots \\
                    a_{1n} & \cdots & a_{mn} \end{bmatrix} \\
                    &=cA^T.
                \end{align*}
                The final property is hence proved.
            \end{proof}
        
        \end{theorem}
        \vphantom
        \\
        \\
        Consider the following definition.
        \begin{definition}{\Stop\,\,Trace}{trace}
            
            Let \(A\in\mathcal{M}_{nn}\). Then,
            \begin{equation*}
                \trace A=\sum_{i=1}^na_{ii}.
            \end{equation*}
            That is, the sum of the elements on the main diagonal.
            
        \end{definition}
        \pagebreak
        \begin{theorem}{\Stop\,\,Sum and Difference of Matrices and Their Transpose}{sumdiftrans}
        
            Suppose \(A\) is an \(n\times n\) matrix. Then,
            \begin{enumerate}
                \item \(A+A^T\) is symmetric.
                \item \(A-A^T\) is skew-symmetric.
            \end{enumerate}
            \begin{proof}
                Consider the transpose of \(A+A^T\). That is,
                \begin{align*}
                    (A+A^T)^T&=A^T+(A^T)^T \\
                    &=A^T+A \\
                    &=A+A^T.
                \end{align*}
                This means that \(A+A^T\) is symmetric. Similarly, Consider the transpose of \(A-A^T\). That is,
                \begin{align*}
                    (A-A^T)^T&=A^T-(A^T)^T \\
                    &=A^T-A \\
                    &=-(A-A^T).
                \end{align*}
                This means that \(A-A^T\) is skew-symmetric.
            \end{proof}
        \end{theorem}
        \begin{theorem}{\Stop\,\,The Relation Between Square, Symmetric, and Skew-Symmetric}{sqsymskew}
        
            Suppose \(A\) is a square matrix. There exists a symmetric matrix \(S\) and a skew-symmetric matrix \(R\) such that
            \begin{equation*}
                A=S+R.
            \end{equation*}
            \begin{proof}
                Note that
                \begin{equation*}
                    2A=(A+A^T)+(A-A^T).
                \end{equation*}
                Dividing both sides by \(2\) produces
                \begin{equation*}
                    A=\frac{1}{2}(A+A^T)+\frac{1}{2}(A-A^T),
                \end{equation*}
                meaning \(S=\frac{1}{2}(A+A^T)\) and \(R=\frac{1}{2}(A-A^T)\). By Theorem \ref{thm:sumdiftrans}, and basic properties of the transpose operation, \(S\) is symmetric and \(R\) is skew-symmetric, as desired. We have now shown existence, and will now show uniqueness. Suppose \(P\) is a symmetric matrix and \(Q\) is a skew-symmetric matrix, and \(A=P+Q\). Then, \(A^T=P^T+Q^T=P-Q\). This must mean that \(P=\frac{1}{2}(A+A^T)\) and \(Q=\frac{1}{2}(A-A^T)\).
            \end{proof}
        \end{theorem}
        
\pagebreak
        
\section{Lecture 4: August 29, 2022}

    \subsection{The Dot Product}
    
    We will now define another vector operation: the dot product. We will also provide some important properties.
        \begin{definition}{\Stop\,\,The Dot Product}{dotprod}
    
            Let \(\vec{v}=[v_1,\ldots,v_n]\) and \(\vec{w}=[w_1,\ldots,w_n]\) be vectors in \(\mathbb{R}^n\). The dot product, or inner product, of \(\vec{v}\) and \(\vec{w}\) is given by
            \begin{equation*}
                \vec{v}\cdot\vec{w}=\sum_{k=1}^nv_kw_k.
            \end{equation*}
        
        \end{definition}
        \vphantom
        \\
        \\
        Note that \(\vec{v}\) and \(\vec{w}\) are orthogonal if and only if \(\vec{v}\cdot\vec{w}=0\). Consider the following examples.
        \begin{example}{\Difficulty\,\,Dot Product 1}{dp1}
        
            Find \([1,\pi]\cdot[-1,\pi]\).
            \\
            \\
            We see that \([1,\pi]\cdot[-1,\pi]=-1+\pi^2\).
        
        \end{example}
         \begin{example}{\Difficulty\,\,Dot Product 2}{dp2}
        
            Find \([1,0]\cdot[0,1]\).
            \\
            \\
            We see that \([1,0]\cdot[0,1]=0\).
        
        \end{example}
        \begin{theorem}{\Stop\,\,Properties of the Dot Product}{dotprodprop}
    
            Let \(\vec{u}=[u_1,\ldots,u_n]\), \(\vec{v}=[v_1,\ldots,v_n]\), and \(\vec{w}=[w_1,\ldots,w_n]\) be vectors in \(\mathbb{R}^n\). Let \(c\) be a scalar. Then,
            \begin{enumerate}
                \item \(\vec{u}\cdot\vec{v}=\vec{v}\cdot\vec{u}\)
                \item \(\vec{u}\cdot\vec{u}=||\vec{u}||^2\geq0\) 
                \item \(\vec{u}\cdot\vec{u}=0\iff\vec{u}=\vec{0}\)
                \item \(c(\vec{u}\cdot\vec{v})=(c\vec{u})\cdot\vec{v}=\vec{u}\cdot(c\vec{v})\)
                \item \(\vec{u}\cdot(\vec{v}+\vec{w})=(\vec{u}\cdot\vec{v})+(\vec{u}\cdot\vec{w})\)
                \item \((\vec{u}+\vec{v})\cdot\vec{w}=(\vec{u}\cdot\vec{w})+(\vec{v}\cdot\vec{w})\)
            \end{enumerate}
        
        \end{theorem}
        \pagebreak
        \vphantom
        \\
        \\
        Consider the following example.
        \begin{example}{\Difficulty\,\Difficulty\,\,Proving a Dot Product Property}{pfdpprop}
                
            Let \(\vec{v}=[v_1,\ldots,v_n]\) and \(\vec{w}=[w_1,\ldots,w_n]\) be vectors in \(\mathbb{R}^n\). Let \(c\) be a scalar. Prove that
            \begin{equation*}
                c(\vec{v}\cdot\vec{w})=(c\vec{v})\cdot\vec{w}.
            \end{equation*}
            \begin{proof}
                Consider the following transitive chain of equality:
                \begin{align*}
                     c(\vec{v}\cdot\vec{w})&=c(v_1w_1+\cdots+v_nw_n) \\
                     &=(cv_1w_1+\cdots+cv_nw_n) \\
                     &=((cv_1)w_1+\cdots+(cv_n)w_n) \\
                     &=(c\vec{v})\cdot\vec{w}.
                \end{align*}
                The proposition is hence proved.
            \end{proof}
        
        \end{example}
        \vphantom
        \\
        \\
        Consider the following theorem regarding the angle between two vectors.
        \begin{theorem}{\Stop\,\,The Angle Between Two Vectors}{angletwovec}
        
            Given nonzero \(\vec{v},\vec{w}\in\mathbb{R}^n\),
            \begin{equation*}
                \vec{v}\cdot\vec{w}=||\vec{v}||||\vec{w}||\cos\theta,
            \end{equation*}
            where \(\theta\) is the angle between the two vectors.
            \begin{proof}
                Consider two unit vectors \(\hat{v}\) and \(\hat{w}\) where
                \begin{equation*}
                    \hat{v}=[\cos\alpha,\sin\alpha],\quad\hat{w}=[\cos\beta,\sin\beta].
                \end{equation*}
                The angle between the two vectors is \(\theta=\beta-\alpha\). The dot product between the two vectors is
                \begin{align*}
                    \hat{v}\cdot\hat{w}&=\cos\alpha\cos\beta+\sin\beta\sin\alpha \\
                    &=\cos(\beta-\alpha) \\
                    &=\cos\theta.
                \end{align*}
                Now, if we consider \(\vec{v}=c_1\hat{v}\) and \(\vec{w}=c_2\hat{w}\), meaning that \(||\vec{v}||=c_1\) and \(||\vec{w}||=c_2\), we simply scale the above result with \(\hat{v}\) and \(\hat{v}\) by the magnitudes of \(\vec{v}\) and \(\vec{w}\) to produce
                \begin{equation*}
                \vec{v}\cdot\vec{w}=||\vec{v}||||\vec{w}||\cos\theta,
            \end{equation*}
                The proposition is hence proved, but it is also of note to realize that the same proposition can be proved with the law of cosines.
            \end{proof}
        
        \end{theorem}
        \pagebreak
        \vphantom
        \\
        \\
        We will now state the famous Cauchy-Schwarz Inequality.
        \begin{theorem}{\Stop\,\,The Cauchy-Schwarz Inequality}{cauchyschwarzineq}

            If \(\vec{v},\vec{w}\in\mathbb{R}^n\), 
            \begin{equation*}
                |\vec{v}\cdot\vec{w}|\leq||\vec{v}||||\vec{w}||.
            \end{equation*}
            \begin{proof}
            
                Consider the following lemma. If \(\vec{v},\vec{w}\in\mathbb{R}^n\),
                with \(||\vec{v}||=||\vec{w}||=1\), 
                \begin{equation*}
                    -1\leq\vec{v}\cdot\vec{w}\leq 1.
                \end{equation*}
                \begin{proof}
                    We will start by showing that \(-1\leq\vec{v}\cdot\vec{w}\). Consider \(||\vec{v}+\vec{w}||^2\geq0\). We have that 
                    \begin{align*}
                        ||\vec{v}+\vec{w}||^2&=(\vec{v}+\vec{w})\cdot(\vec{v}+\vec{w}) \\
                        &=\vec{v}\cdot\vec{v}+\vec{v}\cdot\vec{w}+\vec{w}\cdot\vec{v}+\vec{w}\cdot\vec{w} \\
                        &=||\vec{v}||^2+2\vec{v}\cdot\vec{w}+||\vec{w}||^2 \\
                        &=1+2\vec{v}\cdot\vec{w}+1 \\
                        &=2+2\vec{v}\cdot\vec{w} \\
                        &\geq 0.
                    \end{align*}
                    We then solve the resulting inequality which yields
                    \begin{equation*}
                        2\vec{v}\cdot\vec{w}\geq-2 \implies \vec{v}\cdot\vec{w}\geq -1.
                    \end{equation*}
                    Similarly, consider that \(||\vec{v}-\vec{w}||^2\geq0\). We then have
                    \begin{align*}
                        ||\vec{v}-\vec{w}||^2&=(\vec{v}-\vec{w})\cdot(\vec{v}-\vec{w}) \\
                        &=||\vec{v}||^2-2\vec{v}\cdot\vec{w}+||\vec{w}||^2 \\
                        &=2-2\vec{v}\cdot\vec{w} \\
                        &\geq 0.
                    \end{align*}
                    Then, we have
                    \begin{equation*}
                        2-2\vec{v}\cdot\vec{w}\geq 0 \implies \vec{v}\cdot\vec{w}\leq 1
                    \end{equation*}
                    as desired.
                \end{proof}
                Consider the following cases. If \(\vec{v}=\vec{0}\) or \(\vec{w}=\vec{0}\), then, \(\vec{v}\cdot\vec{w}=0\) and \(||\vec{v}||\vec{w}||=0\). Thus, in this case, \(|\vec{v}\cdot\vec{w}|=||\vec{v}||||\vec{w}||=0\). If \(\vec{v}\neq\vec{0}\) and \(\vec{w}\neq\vec{0}\), Consider the unit vectors \(\frac{1}{||\vec{v}||}\vec{v}\) and \(\frac{1}{||\vec{w}||}\vec{w}\). Hence, by the lemma, 
                \begin{equation*}
                    -1\leq \frac{1}{||\vec{v}||}\vec{v}\cdot \frac{1}{||\vec{w}||}\vec{w} \leq 1.
                \end{equation*}
                This implies that 
                \begin{equation*}
                    -||\vec{v}||||\vec{w}||\leq\vec{v}\cdot\vec{w}\leq||\vec{v}||||\vec{w}||.
                \end{equation*}
                That is,
                \begin{equation*}
                    |\vec{v}\cdot\vec{w}|\leq||\vec{v}||||\vec{w}||,
                \end{equation*}
                which is precisely the statement of the theorem.
            \end{proof}
            
        \end{theorem}
        \pagebreak
        \vphantom
        \\
        \\
        We will now state the famous Triangle Inequality, or Minkowski's Inequality.
        \begin{theorem}{\Stop\,\,The Triangle Inequality}{triineq}
        
            If \(\vec{v},\vec{w}\in\mathbb{R}^n\),
            \begin{equation*}
                ||\vec{v}+\vec{w}||\leq||\vec{v}||+||\vec{w}||.
            \end{equation*}
            \begin{proof}
                Since \(f(x)=\sqrt{x}\) is always increasing, we need only prove
                \begin{equation*}
                    ||\vec{v}+\vec{w}||^2\leq(||\vec{v}||+||\vec{w}||)^2.
                \end{equation*}
                We have
                \begin{align*}
                    ||\vec{v}+\vec{w}||^2&=||\vec{v}||^2+2\vec{v}\cdot\vec{w}+||\vec{w}||^2 \\
                    &\leq||\vec{v}||^2+2|\vec{v}\cdot\vec{w}|+||\vec{w}||^2 \\
                    &\leq||\vec{v}||^2+2||\vec{v}||||\vec{w}||+||\vec{w}||^2 \\
                    &=(||\vec{v}||+||\vec{w}||)^2.
                \end{align*}
                The theorem is hence proved.
            \end{proof}
        
        \end{theorem}
        \vphantom
        \\
        \\
        We will now define the projection onto a vector.
        \begin{definition}{\Stop\,\,The Projection Onto a Vector}{proj}
        
            Given \(\vec{v},\vec{w}\in\mathbb{R}^n\), where \(\vec{w}\neq\vec{0}\), we conclude that the projection of \(\vec{v}\) onto \(\vec{w}\) is
            \begin{align*}
                \proj_{\vec{w}}{\vec{v}}&=||\vec{v}||\cos\theta\frac{\vec{w}}{||\vec{w}||} \\
                &=||\vec{v}||\frac{\vec{v}\cdot\vec{w}}{||\vec{v}||||\vec{w}||}\frac{\vec{w}}{||\vec{w}||} \\
                &=\frac{\vec{v}\cdot\vec{w}}{||\vec{w}||^2}\vec{w} \\
                &=\frac{\vec{v}\cdot\vec{w}}{\vec{w}\cdot\vec{w}}\vec{w}.
            \end{align*}
            
        \end{definition}
        \vphantom
        \\
        \\
        Consider the following example.
        \begin{example}{\Difficulty \,\,Projections}{projes}
        
            Given \(\vec{v}=[x,y]\), find \(\proj_{[1,0]}\vec{v}\) and \(\proj_{[0,1]}\vec{v}\).
            \\
            \\
            We see that
            \begin{equation*}
                \proj_{[1,0]}\vec{v}=[x,0],\quad\proj_{[0,1]}\vec{v}=[0,y].
            \end{equation*}
        
        \end{example}
        \vphantom
        \\
        \\
        Consider the following theorems.
        \begin{theorem}{\Stop\,\,Sum of Parallel and Perpendicular Projections}{sumparperpproj}
        
            Given \(\vec{w}\neq0\) where \(\vec{v},\vec{w}\in\mathbb{R}^n\), 
            \begin{equation*}
                \vec{v}=\proj_{\vec{w}}\vec{v}+(\vec{v}-\proj_{\vec{w}}\vec{v}).
            \end{equation*}
            The first addend returns a vector parallel to \(\vec{w}\), and the second added returns a vector perpendicular to \(\vec{w}\).
            \begin{proof}
                Consider the expansion of the right hand side, that is,
                \begin{align*}
                    \proj_{\vec{w}}\vec{v}+(\vec{v}-\proj_{\vec{w}}\vec{v})&=\frac{\vec{v}\cdot\vec{w}}{\vec{w}\cdot\vec{w}}\vec{w}+\left(\vec{v}-\frac{\vec{v}\cdot\vec{w}}{\vec{w}\cdot\vec{w}}\vec{w}\right) \\
                    &=\vec{v}.
                \end{align*}
                We have arrived at the desired result.
            \end{proof}
            
        
        \end{theorem}
        \begin{theorem}{\Stop\,\,Projections Depend on Lines}{projline}
        
            Given \(\vec{w}\neq0\) where \(\vec{v},\vec{w}\in\mathbb{R}^n\), and \(c\in\mathbb{R}\) where \(c\neq0\),
            \begin{equation*}
                \proj_{\vec{w}}\vec{v}=\proj_{c\vec{w}}\vec{v}.
            \end{equation*}
            \begin{proof}
                Consider the expansion of the left hand side, that is,
                \begin{align*}
                    \proj_{c\vec{w}}\vec{v}&=\frac{\vec{v}\cdot c\vec{w}}{c\vec{w}\cdot c\vec{w}}c\vec{w} \\
                    &=\frac{\vec{v}\cdot \vec{w}}{\vec{w}\cdot \vec{w}}\vec{w} \\
                    &=\proj_{\vec{w}}\vec{v}.
                \end{align*}
                The proposition is hence proved.
            \end{proof}
        
        \end{theorem}
\pagebreak

\section{Lecture 5: August 31, 2022}
        
    \subsection{Matrix Multiplication}
    
        Consider the following definition of matrix multiplication.
        \begin{definition}{\Stop\,\,Matrix Multiplication}{matmul}
        
            For \(A\in\mathcal{M}_{mn}\) and \(B\in\mathcal{M}_{np}\), such that
            \begin{equation*}
                A=\begin{bmatrix} 
                     a_{11} &  a_{12} & \cdots &  a_{1n} \\
                     a_{21} &  a_{22} & \cdots &  a_{2n} \\
                    \vdots & \vdots & \ddots & \vdots \\
                     a_{m1} &  a_{m2} & \cdots &  a_{mn} \\
                \end{bmatrix},\quad
                B=\begin{bmatrix} 
                     b_{11} &  b_{12} & \cdots &  b_{1p} \\
                     b_{21} &  b_{22} & \cdots &  b_{2p} \\
                    \vdots & \vdots & \ddots & \vdots \\
                     b_{n1} &  b_{n2} & \cdots &  b_{np} \\
                \end{bmatrix},
            \end{equation*}
            the matrix product \(C=AB\) is defined such that \(C\in\mathcal{M}_{mp}\) and is given by
            \begin{equation*}
                C=\begin{bmatrix} 
                     c_{11} &  c_{12} & \cdots &  c_{1p} \\
                     c_{21} &  c_{22} & \cdots &  c_{2p} \\
                    \vdots & \vdots & \ddots & \vdots \\
                     c_{m1} &  c_{m2} & \cdots &  c_{mp} \\
                \end{bmatrix}
            \end{equation*}
            such that \(c_{ij}=a_{i1}b_{1j}+a_{i2}b_{2j}+\cdots+a_{in}b_{nj}=\sum_{k=1}^na_{ik}b_{kj}\). The matrix product is only defined if the number of columns of the matrix on the left is equal to the number of the rows of the matrix on the right. Note that matrix multiplication is not commutative. We may also say that the entry in \(c_{ij}\) is the dot product of the \(i\)th row of \(A\) and the \(j\)th column of \(B\).
            
        \end{definition}
        \vphantom
        \\
        \\
        Consider the following examples.
        \begin{example}{\Difficulty\,\Difficulty\,\,Matrix Multiplication 1}{matmul1}
        
            Find \(\begin{bmatrix} 2 & 1 \\ 1 & 1 \end{bmatrix}\begin{bmatrix} 3 \\ -1 \end{bmatrix}\).
            \\
            \\
            We see that
            \begin{equation*}
                \begin{bmatrix} 2 & 1 \\ 1 & 1 \end{bmatrix}\begin{bmatrix} 3 \\ -1 \end{bmatrix}=\begin{bmatrix} 5 \\ 2 \end{bmatrix}.
            \end{equation*}
                
        \end{example}
        \begin{example}{\Difficulty\,\Difficulty\,\,Matrix Multiplication 2}{matmul2}
        
            Find \(\begin{bmatrix} 3 & 1 & -1 \end{bmatrix}\begin{bmatrix} \pi \\ -3 \\ 7 \end{bmatrix}\).
            \\
            \\
            We see that
            \begin{equation*}
               \begin{bmatrix} 3 & 1 & -1 \end{bmatrix}\begin{bmatrix} \pi \\ -3 \\ 7 \end{bmatrix}=\begin{bmatrix} 3\pi-10 \end{bmatrix}.
            \end{equation*}
                
        \end{example}
        \begin{example}{\Difficulty\,\Difficulty\,\,Matrix Multiplication 3}{matmul3}
        
            Find \(\begin{bmatrix} 3 & 1 & -1 \end{bmatrix}\begin{bmatrix} 2 & 1 \\ 1 & 7 \\ 1 & 3 \end{bmatrix}\).
            \\
            \\
            We see that
            \begin{equation*}
               \begin{bmatrix} 3 & 1 & -1 \end{bmatrix}\begin{bmatrix} 2 & 1 \\ 1 & 7 \\ 1 & -3 \end{bmatrix}=\begin{bmatrix} 6 & 13 \end{bmatrix}.
            \end{equation*}
                
        \end{example}
        \begin{example}{\Difficulty\,\Difficulty\,\,Matrix Multiplication 4}{matmul4}
        
            Find \(\begin{bmatrix} 3 & 1 \\ -1 & 5 \end{bmatrix}\begin{bmatrix} 1 & 0 \\ 0 & 1 \end{bmatrix}\).
            \\
            \\
            We see that
            \begin{equation*}
               \begin{bmatrix} 3 & 1 \\ -1 & 5 \end{bmatrix}\begin{bmatrix} 1 & 0 \\ 0 & 1 \end{bmatrix}=\begin{bmatrix} 3 & 1 \\ -1 & 5 \end{bmatrix}.
            \end{equation*}
                
        \end{example}
        \begin{example}{\Difficulty\,\Difficulty\,\,Matrix Multiplication 5}{matmul5}
        
            Find \(\begin{bmatrix} 3 & 1 \\ -1 & 5 \end{bmatrix}\begin{bmatrix} 0 & 0 \\ 0 & 0 \end{bmatrix}\).
            \\
            \\
            We see that
            \begin{equation*}
               \begin{bmatrix} 3 & 1 \\ -1 & 5 \end{bmatrix}\begin{bmatrix} 0 & 0 \\ 0 & 0 \end{bmatrix}=\begin{bmatrix} 0 & 0 \\ 0 & 0 \end{bmatrix}.
            \end{equation*}
                
        \end{example}
        \vphantom
        \\
        \\
        Consider the following properties of matrix multiplication.
        \begin{theorem}{\Stop\,\,Properties of Matrix Multiplication}{propmatmul}
        
            Suppose \(A\), \(B\), and \(C\) are matrices such that matrix multipication is well-defined. Then,
            \begin{enumerate}
                \item \(A(BC)=(AB)C\)
                \item \(A(B+C)=AB+AC\)
                \item \((A+B)C=AC+BC\) 
                \item \(c(AB)=(cA)B=A(cB)\).
            \end{enumerate}
        
        \end{theorem}
        \pagebreak
        \vphantom
        \\
        \\
        We now ponder two questions.
        \begin{enumerate}
            \item Does \(AB=BA\)? No.
            \item If \(AB=0\), does \(A=0\) or \(B=0\)? No.
        \end{enumerate}
        \vphantom
        \\
        \\
        Consider the following definition.
        \begin{definition}{\Stop\,\,Raising a Matrix to a Power}{matpow}
            Consider an \(n\times n\) matrix \(A\). Then,
            \begin{equation*}
                A^k=\underbrace{A\cdot A\cdot A\cdot\cdots\cdot A}_{k \text{ times}}
            \end{equation*}
            and \(A^k=I\).
        \end{definition}
        
        

\begin{savequote}
\includegraphics[scale=0.45]{Graphics/ml.png}
\end{savequote}
\chapter{Systems of Linear Equations} \label{chapter:syslineq}

    \section{Lecture 6: September 2, 2022}

    \subsection{Systems of Linear Equations}
    
        Consider the following definitions.
        \begin{definition}{\Stop\,\,Linear Equations}{lineq}
        
            A linear equation is an equation of the form 
            \begin{equation*}
                a_1x_1+\cdots+a_nx_n=b.
            \end{equation*}
            
        \end{definition}
        \begin{definition}{\Stop\,\,Systems of Linear Equations}{syslineq}
        
            A system of linear equations is a system of the form 
            \begin{align*}
                a_{11}x_1+\cdots+a_{1n}x_n&=b_1 \\
                &\vdots \\
                a_{m1}x_1+\cdots+a_{mn}x_n&=b_m.
            \end{align*}
            
        \end{definition}
        \pagebreak
        
\pagebreak
        
\section{Lecture 7: September 7, 2022}

    \subsection{Systems of Linear Equations as Matrices}

        We may write systems of linear equations in terms of matrices as
        \begin{equation*}
            A\begin{bmatrix} x_1 \\ \vdots \\ x_n \end{bmatrix}=\begin{bmatrix} b_1 \\ \vdots \\ b_m \end{bmatrix},
        \end{equation*}
        where \(A\) is the matrix with entries \(a_{ij}\). We will use the convention that \(X=[x_1,\ldots,x_n]^T\) and \(B=[b_1,
        \ldots,b_m]^T\). Consider the following theorem.
        \begin{theorem}{\Stop\,\,Characterizing Solutions of Linear Systems}{charsollinsys}
            
            A system of linear equations can either have
            \begin{enumerate}
                \item No solution.
                \item One unique solution.
                \item Infinitely many solutions.
            \end{enumerate}
            
        \end{theorem}

    \subsection{Matrix Row Operations}
    
        Consider the following operations.
        \begin{enumerate}
            \item Multiplication of a row by a nonzero scalar. Notated as \(c\langle r_1\rangle\to\langle r_1\rangle\).
            \item Addition of a scalar multiple of one row to another. Notated as \(\langle r_1\rangle +(c)\langle r_2\rangle\to\langle r_1\rangle\).
            \item Switching the elements of two rows. Notated as \(\langle r_1 \rangle \leftrightarrow \langle r_2 \rangle\).
        \end{enumerate}
        \vphantom
        \\
        \\
        Consider the following examples.
        \begin{example}{\Difficulty\,\Difficulty\,\,Row Operation 1}{rowop1}
        
            Consider the matrix \(\begin{bmatrix} 3 & 1 & -1 \\ 1 & 0 & 1 \\ -1 & 1 & 5 \end{bmatrix}\). Find \(4\langle 3\rangle\to\langle 3\rangle\).
            \\
            \\
            We obtain
            \begin{equation*}
                \begin{bmatrix}
                    3 & 1 & -1 \\
                    1 & 0 & 1 \\
                    -4 & 4 & 20
                \end{bmatrix}.
            \end{equation*}
    
        \end{example}
        \pagebreak
        \begin{example}{\Difficulty\,\Difficulty\,\,Row Operation 2}{rowop2}
        
            Consider the matrix \(\begin{bmatrix} 3 & 1 & -1 \\ 1 & 0 & 1 \\ -1 & 1 & 5 \end{bmatrix}\). Find \(\langle 1\rangle +(-3)\langle 2\rangle\to\langle 1\rangle\).
            \\
            \\
            We obtain
            \begin{equation*}
                \begin{bmatrix}
                    0 & 1 & -4 \\
                    1 & 0 & 1 \\
                    -1 & 1 & 5
                \end{bmatrix}.
            \end{equation*}
        \end{example}
        \begin{example}{\Difficulty\,\Difficulty\,\,Row Operation 3}{rowop3}
        
            Consider the matrix \(\begin{bmatrix} 3 & 1 & -1 \\ 1 & 0 & 1 \\ -1 & 1 & 5 \end{bmatrix}\). Find \(\langle 2\rangle\leftrightarrow\langle 3\rangle\).
            \\
            \\
            We obtain
            \begin{equation*}
                \begin{bmatrix}
                    3 & 1 & -1 \\
                    -1 & 1 & 5 \\
                    1 & 0 & 1
                \end{bmatrix}.
            \end{equation*}
    
        \end{example}
        
\pagebreak
        
\section{Lecture 8: September 9, 2022}

    \subsection{Solving Linear Systems}
    
        Given a linear system of equations, we solve by the following steps.
        \begin{enumerate}
            \item Convert the linear system into the matrix equation \(AX=B\), written \([A|B]\).
            \item Use the three row operations to reduce \([A|B]\) to one with ``lots of zeroes and ones.''
            \item Perform back substitution and analyze the solution set.
        \end{enumerate}
        \vphantom
        \\
        \\
        Consider the following examples.
        \begin{example}{\Difficulty\,\Difficulty\,\,No Solution}{nosols}
            
            Consider the matrix
            \begin{equation*}
                \begin{bmatrix} 
                3 & -6 & 0 & 3 & | & 9 \\
                -2 & 4 & 2 & -1 & | & -11 \\
                4 & -8 & 6 & 7 & | & -5
                \end{bmatrix}.
            \end{equation*}
            By row operations, we obtain
            \begin{equation*}
                \begin{bmatrix} 
                1 & -2 & 0 & 1 & | & 3 \\
                0 & 0 & 1 & \frac{1}{2} & | & -\frac{5}{2} \\
                0 & 0 & 0 & 0 & | & -2
                \end{bmatrix}.
            \end{equation*}
            Looking at the last row, we see the equation \(0=-2\), which is not true. Hence, the system has no solution.
            
        \end{example}
        \begin{example}{\Difficulty\,\Difficulty\,\,Infinitely Many Solutions}{infmanysols}
            
            Consider the matrix
            \begin{equation*}
                \begin{bmatrix} 
                3 & 1 & 7 & 2 & | & 13 \\
                2 & -4 & 14 & -1 & | & -10 \\
                5 & 11 & -7 & 8 & | & 59 \\
                2 & 5 & -4 & -3 & | & 39
                \end{bmatrix}.
            \end{equation*}
            By row operations, we obtain
            \begin{equation*}
                \begin{bmatrix} 
                1 & \frac{1}{3} & \frac{7}{3} & \frac{2}{3} & | & \frac{13}{3} \\
                0 & 1 & -2 & \frac{1}{2} & | & 4 \\
                0 & 0 & 0 & 1 & | & -2 \\
                0 & 0 & 0 & 0 & | & 0
                \end{bmatrix}.
            \end{equation*}
            We see that \(x_1\), \(x_2\), and \(x_4\) are determined because their respective column has a \(1\) in the correct position. In contrast, \(x_3\) is a free variable. To find the solution set, let \(x_3=c\in\mathbb{R}\) and solve for \(x_1\), \(x_2\), and \(x_4\) in terms of \(c\). We have \(x_4=-2\). Then, to find \(x_2\) we have
            \begin{equation*}
                x_2-2x_3+\frac{1}{2}x_4=4\implies x_2-2+\frac{1}{2}(-2)=4\implies x_2=2c+5.
            \end{equation*}
            For \(x_1\), we have
            \begin{equation*}
                x_1+\frac{1}{3}x_2+\frac{7}{3}x_3+\frac{2}{3}x_4=\frac{13}{3}\implies x_1=-3c+4.
            \end{equation*}
            The solution set is then \(\{(-3c+4,2c+5,c,-2):c\in\mathbb{R}\}\).
        \end{example}
        \vphantom
        \\
        \\
        We generally agree that back substitution is not much fun. Consider the following example.
        \begin{example}{\Difficulty\,\Difficulty\,\,No More Back Substitution}{nomorebacksub}
            Note that the matrix 
            \begin{equation*}
                \begin{bmatrix} 3 & -3 & -2 &| &23 \\ -6 & 4 & 3 &| &-40 \\ -2 & 1 & 1 &| &-12 \end{bmatrix}
            \end{equation*}
            reduces into
            \begin{equation*}
                \begin{bmatrix} 1 & -1 & -\frac{2}{3} &| &\frac{23}{3} \\ 0 & 1 & \frac{1}{3} &| &-\frac{10}{3} \\ 0 & 0 & 1 &| &2 \end{bmatrix}.
            \end{equation*}
            We may now ``get rid of'' \(-1\), \(-\frac{2}{3}\), and \(\frac{1}{3}\). We perform \(-\frac{1}{3}\langle 3\rangle+\langle 2\rangle\to\langle2\rangle\) which produces
            \begin{equation*}
                \begin{bmatrix} 1 & -1 & -\frac{2}{3} &| &\frac{23}{3} \\ 0 & 1 & 0 &| & -4 \\ 0 & 0 & 1 & | & 2 \end{bmatrix}.
            \end{equation*}
            Then, we will perform \(\langle 1\rangle + \langle 2\rangle\to\langle1\rangle\), yielding
            \begin{equation*}
                \begin{bmatrix} 1 & 0 & -\frac{2}{3} &| & \frac{23}{3}-4 \\ 0 & 1 & 0 &| & -4 \\ 0 & 0 & 1 & | & 2 \end{bmatrix}.
            \end{equation*}
            Finally, we will perform \(\frac{2}{3}\langle3\rangle+\langle1\rangle\to 1\), obtaining
            \begin{equation*}
                \begin{bmatrix}
                1 & 0 & 0 & | & 5 \\
                0 & 1 & 0 & | & -4 \\
                0 & 0 & 1 & | & 2
            \end{bmatrix}
            \end{equation*}
            which means \(x_1=5\), \(x_2=-4\), and \(x_3=2\).
        \end{example}
        \vphantom
        \\
        \\
        Consider the following theorem.
        \begin{theorem}{\Stop\,\,Row Operations}{rowops}
        
            Suppose \(A\in\mathcal{M}_{mn}\) and \(B\in\mathcal{M}_{np}\). Then,
            \begin{enumerate}
                \item If \(R\) is a row operation, \(R(AB)=(R(A))B\).
                \item If \(R_1,\ldots,R_n\) are row operations, \(R_n(
                \ldots(R_2(R_1(AB)))\ldots)=(R_n(\ldots(R_2(R_1(A)))\ldots))B\).
            \end{enumerate}
            \vphantom
            \\
            \\
            Note that this result follows from the associativity of matrix multiplication, as any row operation can be represented by a multiplication of two matrices.
        \end{theorem}
        
    \pagebreak
    \vphantom
    \\
    \\
    Consider the following examples of solving linear systems.
    \begin{example}{\Difficulty\,\Difficulty\,\,Linear System 1}{linsys1}
        Solve the following system:
        \begin{equation*}
            \begin{bmatrix}
                2 & -1 & 1 & | & 0 \\
                1 & 3 & 4 & | & 0
            \end{bmatrix}
        \end{equation*}
        We first perform the row operation \(\frac{1}{2}\langle 1\rangle\to\langle 1\rangle\), which produces
        \begin{equation*}
            \begin{bmatrix}
                1 & -\frac{1}{2} & \frac{1}{2} & | & 0 \\
                1 & 3 & 4 & | & 0
            \end{bmatrix}.
        \end{equation*}
        Then, we perform \(\langle 1\rangle -\langle 2\rangle\to\langle 2\rangle\). We obtain
        \begin{equation*}
            \begin{bmatrix}
                1 & -\frac{1}{2} & \frac{1}{2} & | & 0 \\
                0 & -\frac{7}{2} & -\frac{7}{2} & | & 0
            \end{bmatrix}.
        \end{equation*}
        Next, we have \(-\frac{2}{7}\langle 2\rangle\to\langle 2\rangle\). This yields
        \begin{equation*}
            \begin{bmatrix}
                1 & -\frac{1}{2} & \frac{1}{2} & | & 0 \\
                0 & 1 & 1 & | & 0
            \end{bmatrix}. 
        \end{equation*}
        From here, let \(x_3=c\). Then, \(x_2=-c\). To find \(x_1\), we use the equation 
        \begin{equation*}
            x_1-\frac{1}{2}(-c)+\frac{1}{2}c=0,
        \end{equation*}
        which implies that \(x_1=-c\). Thus, the solution set is \(\{(-c,-c,c):c\in\mathbb{R}\}\).
    \end{example}
    \pagebreak
    \begin{example}{\Difficulty\,\Difficulty\,\,Linear System 2}{linsys2}
        Solve the following system:
        \begin{equation*}
            \begin{bmatrix}
                1 & -2 & 1 & 2 & | & 1 \\
                1 & 1 & -1 & 1 & | & 2 \\
                1 & 7 & -5 & -1 & | & 3
            \end{bmatrix}
        \end{equation*}
        First, we perform \(\langle 1\rangle -\langle 2\rangle\to\langle 2\rangle\), yielding
        \begin{equation*}
            \begin{bmatrix}
                1 & -2 & 1 & 2 & | & 1 \\
                0 & -3 & 2 & 1 & | & -1 \\
                1 & 7 & -5 & -1 & | & 3
            \end{bmatrix}.
        \end{equation*}
        Then, we perform \(\langle 1\rangle-\langle 3\rangle\to\langle 3\rangle\). We obtain
        \begin{equation*}
            \begin{bmatrix}
                1 & -2 & 1 & 2 & | & 1 \\
                0 & -3 & 2 & 1 & | & -1 \\
                0 & -9 & 6 & 3 & | & -2 
            \end{bmatrix}.
        \end{equation*}
        Then, we have \(-\frac{1}{3}\langle 2\rangle\to\langle 2\rangle\). This produces
        \begin{equation*}
            \begin{bmatrix}
                1 & -2 & 1 & 2 & | & 1 \\
                0 & 1 & -\frac{2}{3} & -\frac{1}{3} & | & \frac{1}{3} \\
                0 & -9 & 6 & 3 & | & -2
            \end{bmatrix}.
        \end{equation*}
        Our final row operation is \(\langle3\rangle+9\langle2\rangle\to\langle 3\rangle\). This provides us with
        \begin{equation*}
            \begin{bmatrix}
                 1 & -2 & 1 & 2 & | & 1 \\
                 0 & 1 & -\frac{2}{3} & -\frac{1}{3} & | & \frac{1}{3} \\
                 0 & 0 & 0 & 0 & | & 1
            \end{bmatrix},
        \end{equation*}
        meaning that there is no solution to the system.
    \end{example}
    \pagebreak
    \begin{example}{\Difficulty\,\Difficulty\,\,Linear System 3}{linsys3}
        Solve the following system:
        \begin{equation*}
            \begin{bmatrix}
                 1 & -1 & 2 & | & 1 \\
                 2 & 0 & 2 & | & 1 \\
                 1 & -3 & 4 & | & 2
            \end{bmatrix}
        \end{equation*}
        Our first row operation is \(2\langle1\rangle-\langle2\rangle\to\langle2\rangle\). This produces
        \begin{equation*}
            \begin{bmatrix}
                 1 & -1 & 2 & | & 1 \\
                 0 & -2 & 2 & | & 1 \\
                 1 & -3 & 4 & | & 2
            \end{bmatrix}.
        \end{equation*}
        Then, we have \(\langle1\rangle-\langle3\rangle\to\langle3\rangle\), providing
        \begin{equation*}
            \begin{bmatrix}
                 1 & -1 & 2 & | & 1 \\
                 0 & -2 & 2 & | & 1 \\
                 0 & 2 & -2 & | & -1
            \end{bmatrix}.
        \end{equation*}
        Next, we perform \(\langle2\rangle+\langle3\rangle\to\langle3\rangle\). We obtain
        \begin{equation*}
            \begin{bmatrix}
                 1 & -1 & 2 & | & 1 \\
                 0 & -2 & 2 & | & 1 \\
                 0 & 0 & 0 & | & 0
            \end{bmatrix}.
        \end{equation*}
        We perform another row operation, \(-\frac{1}{2}\langle 2\rangle\to\langle 2\rangle\). This yields
        \begin{equation*}
            \begin{bmatrix}
                 1 & -1 & 2 & | & 1 \\
                 0 & 1 & -1 & | & -\frac{1}{2} \\
                 0 & 0 & 0 & | & 0
            \end{bmatrix}.
        \end{equation*}
        Next, we have \(\langle2\rangle+\langle1\rangle\to\langle1\rangle\), which gives
        \begin{equation*}
            \begin{bmatrix}
                 1 & 0 & 1 & | & \frac{1}{2} \\
                 0 & 1 & -1 & | & -\frac{1}{2} \\
                 0 & 0 & 0 & | & 0
            \end{bmatrix}.
        \end{equation*}
        Let \(x_3=c\). Then, \(x_2=-\frac{1}{2}+c\) and \(x_1=\frac{1}{2}-c\). This means the solution set is \(\{\left(\frac{1}{2}-c,-\frac{1}{2}+c,c\right):c\in\mathbb{R}\}\).
    \end{example}

\pagebreak

\section{Lecture 9: September 12, 2022}

    \subsection{Formalizing Previous Notions: Part I}

    Consider the following definitions.
    \begin{definition}{\Stop\,\,Row Echelon Form}{rowechelon}
        
        A matrix \(A\) is in row echelon form if and only if
        \begin{enumerate}
            \item All rows consisting of only zeroes are at the bottom.
            \item The leading coefficient, or the pivot, of a nonzero row is always strictly to the right of the leading coefficient of the row above it.
        \end{enumerate}
    
    \end{definition}
    \begin{definition}{\Stop\,\,Reduced Row Echelon Form}{redrowechelon}
    
        A matrix \(A\) is in reduced row echelon form if and only if
        \begin{enumerate}
            \item The first nonzero entry in each row is one.
            \item Each successive row has its first nonzero entry in a later column.
            \item All entries above and below the first nonzero entry are zero.
            \item All rows consisting of only zeroes are at the bottom.
        \end{enumerate}
        
    \end{definition}
    \vphantom
    \\
    \\
    Note that every matrix has a unique reduced row echelon form.
    \\
    \\
    Consider the following theorems and definitions.
    \begin{theorem}{\Stop\,\,Number of Solutions to a Linear System}{numsollinsys}
    
        Let \(AX=B\) be a system of linear equations. Let \(C\) be the reduced row echelon form augmented matrix obtained by row reducing \([A|B]\). Then,
        \begin{enumerate}
            \item If there is a row of \(C\) having all zeroes to the left of the augmentation bar but with its last entry nonzero, \(AX=B\) has no solution.
            \item If not, and if one of the columns of \(C\) to the left of the augmentation bar has no nonzero pivot entry, \(AX=B\) has an infinite number of solutions. The nonpivot columns correspond to (independent) variables that can take on any value, and the values of the remaining (dependent) variables are determined from those.
            \item Otherwise \(AX=B\) has a unique solution.
        \end{enumerate}
    
    \end{theorem}
    \pagebreak
    \vphantom
    \\
    \\
    Consider the following definitions.
    \begin{definition}{\Stop\,\,Homogeneous Systems}{homosys}
    
        Given \(A\in\mathcal{M}_{mn}\), the homogeneous system associated with \(A\) is
        \begin{equation*}
            AX=\begin{bmatrix}
            0 \\
            \vdots \\
            0
            \end{bmatrix}.
        \end{equation*}
    
    \end{definition}
    \begin{theorem}{\Stop\,\,Solutions to Homogeneous Systems}{solstohomosys}
    
        Given \(A\in\mathcal{M}_{mn}\), the homogeneous system always has at least one solution, called the \textit{trivial solution}. Namely,
        \begin{equation*}
            x_1=0,\quad\ldots,\quad x_n=0.
        \end{equation*}
        Also, consider the following.
        \begin{enumerate}
            \item If \(m<n\), the solution set is infinite. 
            \item If \(X=\begin{bmatrix} x_1 \\ \vdots \\ x_n \end{bmatrix}\) and \(\tilde{X}=\begin{bmatrix} \tilde{x}_1 \\ \vdots \\ \tilde{x}_n \end{bmatrix}\) are solutions,
            \begin{equation*}
                cX+\tilde{X}
            \end{equation*}
            is a solution for any \(c\in\mathbb{R}\).
            \item If \(AX=\begin{bmatrix} 0 \\ \vdots \\ 0 \end{bmatrix}\) and \(A\hat{X}=B=\begin{bmatrix} b_1 \\ \vdots \\ b_m \end{bmatrix}\),
            \begin{equation*}
                cX+\hat{X}
            \end{equation*}
            is a solution to \([A|B]\). Notice that (2) is a special case of (3).
            \begin{proof}
                Consider \(A(cX+\hat{X})\). We wish to show that \(A(cX+\hat{X})=B\). We see that
                \begin{align*}
                    A(cX+\hat{X})&=A(cX)+A\hat{X} \\
                    &=cAX+A\hat{X} \\
                    &=\begin{bmatrix} 0 \\ \vdots \\ 0 \end{bmatrix}+\begin{bmatrix} b_1 \\ \vdots \\ b_m \end{bmatrix} \\
                    &=B,
                \end{align*}
                as desired.
            \end{proof}
            This process is analogous to solving homogeneous differential equations to solve nonhomogeneous differential equations.
        \end{enumerate}
    
    \end{theorem}
    \begin{definition}{\Stop\,\,Equivalence of Linear Systems}{sysequiv}
        Two systems \([A|B]\) and \([\tilde{A}|\tilde{B}]\) are equivalent if and only if
        \begin{equation*}
            AX=B\wedge \tilde{A}X=\tilde{B}.
        \end{equation*}
        That is, if they have the same solution sets.
    \end{definition}
    \begin{definition}{\Stop\,\,Row Equivalence}{rowequiv}
        A matrix \(A\) is row equivalent to a matrix \(B\) if \(B\) can be obtained by a finite number of row operations conducted on \(A\).
    \end{definition}
    \vphantom
    \\
    \\
    For example, Gaussian Elimination and Gauss-Jordan Elimination produce matrices that are row equivalent to the original matrix.
    \\
    \\
    One may ask: What is the relationship between these relations? We see that row equivalence implies system equivalence. But, two systems can have the same solution set, but have different sizes, making row equivalence impossible. For the latter case, consider two matrices of different sizes, but with an empty solution set. Recall the following definition from discrete mathematics.
    \begin{definition}{\Stop\,\,Equivalence Relations}{equivrel}
        
        A relation \(\sim\) on a set \(S\) is an equivalence relation on \(S\) if and only if \(\sim\) is reflexive, symmetric, and transitive. That is, if
        \begin{enumerate}
            \item \(\forall a\in S, a\sim a\).
            \item \(\forall a, b\in S, a\sim b\implies b\sim a\)
            \item \(\forall a, b, c\in S, a\sim b\wedge b\sim c\implies a\sim c\).
        \end{enumerate}
        
    \end{definition}
    \pagebreak
    \vphantom
    \\
    \\
    Consider the following theorem.
    \begin{theorem}{\Stop\,\,System Equivalence and Row Equivalence are Equivalence Relations}{equivrelrowsysequiv}
    
        First, consider the following table.
        \begin{center}
            \begin{tabular}{|c|c|}
                \hline
                \hline
                Row Operation & Reverse Operation \\
                \hline
                \hline
                \(c\langle i\rangle \to \langle i\rangle\) & \(\frac{1}{c}\langle i\rangle \to \langle i \rangle\) \\
                \hline
                \(c\langle i \rangle+\langle j \rangle \to\langle j\rangle\) & \(-c\langle i \rangle+\langle j \rangle\to\langle j \rangle\) \\
                \hline
                \(\langle i \rangle \leftrightarrow \langle j \rangle\) & \(\langle i \rangle \leftrightarrow \langle j \rangle\) \\
                \hline
            \end{tabular}
        \end{center}
        \vphantom
        \\
        \\
        \begin{proof}
            We will consider row equivalence first, and wish to show that row equivalence is reflexive, symmetric, and transitive. Reflexivity is trivial. We can simply not perform any row operations on a matrix \(A\), and we are left with \(A\). The above table can be used to show that row equivalence is symmetric. If a sequence of row operations is carried out on \(A\) and produces a matrix \(B\), we can simply carry out the reverse operations on \(B\) to lead us back to \(A\). For transitivity, if a sequence of row operations is carried out on \(A\) and leads to \(B\), and a second sequence of row operations is performed on \(B\) and leads to \(C\), we simply carry out the operations, in sequence, on \(A\) to get us to \(C\).
            \\
            \\
            Now, we consider system equivalence. The system \([A|B]\), of course, has the same solution set as itself. If the system \([A|B]\) has the same solution set as \([C|D]\), \([C|D]\) has the same solution set as \([A|B]\). If \([A|B]\) has the same solution set as \([C|D]\) and \([C|D]\) has the same solution set as the system \([E|F]\), \([A|B]\) has the same solution set as \([E|F]\).
        \end{proof}
        
    \end{theorem}

\pagebreak

\section{Lecture 10: September 14, 2022}

    \subsection{Formalizing Previous Notions: Part II}
    
        Consider the following formalization of our last discoveries.
        \begin{theorem}{\Stop\,\,Row Equivalence Implies System Equivalence}{rowsyseq}
        
            If \([A|B]\) is row equivalent to \([C|D]\), \([A|B]\) is equivalent to \([C|D]\).

        \end{theorem}
        \begin{theorem}{\Stop\,\,Uniqueness of Reduced Row Echelon Form}{uniquenessredrow}
        
            Every matrix is row equivalent to a unique matrix in reduced row echelon form. Two matrices are row equivalent if and only if they have the same reduced row echelon form.
            
        \end{theorem}
        \begin{definition}{\Stop\,\,Rank}{rank}
        
            Given \(A\in\mathcal{M}_{mn}\), \(\rank{A}\) is the number of nonzero rows in the unique matrix that is row equivalent to \(A\) and is in reduced row echelon form.
            
        \end{definition}
        \vphantom
        \\
        \\
        Consider the following example.
        \begin{example}{\Difficulty\,\Difficulty\,\,Rank}{rank}
        
            Consider
            \begin{equation*}
                A=\begin{bmatrix}
                3 & 1 & 0 & 1 \\
                0 & -2 & 12 & -5 \\
                2 & -3 & 22 & -14
                \end{bmatrix}.
            \end{equation*}
            By row reduction, we have the matrix
            \begin{equation*}
                \begin{bmatrix}
                1 & 0 & 2 & -1 \\
                0 & 1 & -6 & 4 \\
                0 & 0 & 0 & 0
                \end{bmatrix},
            \end{equation*}
            and see that \(\rank A=2\).
        \end{example}
        \vphantom
        \\
        \\
        Consider the following theorem. 
        \begin{theorem}{\Stop\,\,Number of Solutions to Homogeneous Systems}{numsolshomosys}
        
            If \(A\in\mathcal{M}_{mn}\),
            \begin{enumerate}
                \item If \(\rank A< n\), \(AX=0\) has an infinite solution set.
                \item If \(\rank A= n\), \(AX=0\) has only the trivial solution.
            \end{enumerate}
        
        \end{theorem}
        \pagebreak
        \vphantom
        \\
        \\
        We will now define linear combinations of vectors.
        \begin{definition}{\Stop\,\,Linear Combinations}{lincomb}
    
            Let \(\vec{v}_1,\vec{v}_2,\ldots,\vec{v}_k\in\mathbb{R}^n\). The vector \(\vec{v}\) is a linear combination of \(\vec{v}_1,\vec{v}_2,\ldots,\vec{v}_k\) if and only if there are scalars \(c_1,c_2,\ldots,c_k\) such that 
            \begin{equation*}
                \vec{v}=c_1\vec{v}_1+\cdots+c_k\vec{v}_k.
            \end{equation*}
            
        \end{definition}
        \vphantom
        \\
        \\
        In general, \(\{c\vec{v}:c\in\mathbb{R}\}\) is a line unless \(\vec{v}=\vec{0}\). Also, \(\{c_1\vec{v}_1+c_2\vec{v}_2:c_1,c_2\in\mathbb{R}\}\) is usually a plane, but could be either a point or a line. This pattern is an introduction to the concept of linear independence, which will be elaborated on later in the text. Consider the following example.
        \begin{example}{\Difficulty\,\Difficulty\,\,Is a Vector a Linear Combination of Others? 1}{veclincombothers1}
        
            Let \(\vec{v}=[1,0]\), \(\vec{v}_1=[\pi,1]\), and \(\vec{v}_2=[2,1]\). Is \(\vec{v}\) a linear combination of \(\vec{v}_1\) and \(\vec{v}_2\)?
            \\
            \\
            Notice that \(\vec{v}_1-\vec{v}_2=[\pi-2,0]\). Then,
            \begin{equation*}
                \frac{1}{\pi-2}[\pi-2,0]=[1,0]=\vec{v}.
            \end{equation*}
            We then have that
            \begin{equation*}
                \vec{v}=\frac{1}{\pi-2}[\pi,1]-\frac{1}{\pi-2}[2,1].
            \end{equation*}
            Therefore, \(\vec{v}\) is a linear combination of \(\vec{v}_1\) and \(\vec{v}_2\).
        \end{example}
        \vphantom
        \\
        \\
        The above solution used a bit of trickery. Instead, given \(\vec{v}_1,\ldots,\vec{v}_k\), we form the equation
        \begin{equation*}
            \begin{bmatrix}
                \vec{v}_1,\ldots,\vec{v}_k
            \end{bmatrix}
            \begin{bmatrix}
                c_1 \\ \vdots \\ c_k
            \end{bmatrix}
            =\begin{bmatrix} \vec{v} \end{bmatrix}
        \end{equation*}
        and solve for the necessary constants.
        \pagebreak
        \\
        \\
        Consider the following examples.
        \begin{example}{\Difficulty\,\Difficulty\,\,Is a Vector a Linear Combination of Others? 2}{veclincombothers2}
        
            Let \(\vec{v}=[1,0,0]\), \(\vec{v}_1=[-4,2,0]\), and \(\vec{v}_2=[2,1,1]\). Is \(\vec{v}\) a linear combination of \(\vec{v}_1\) and \(\vec{v}_2\)?
            \\
            \\
            Consider the system
            \begin{equation*}
                \begin{bmatrix}
                    -4 & 2 & | & 1 \\
                    2 & 1 & | & 0 \\
                    0 & 1 & | & 0
                \end{bmatrix}.
            \end{equation*}
            We first perform the row operation \(-\frac{1}{4}\langle1\rangle\to\langle1\rangle\) to obtain
            \begin{equation*}
                \begin{bmatrix}
                    1 & -\frac{1}{2} & | & -\frac{1}{4} \\
                    2 & 1 & | & 0 \\
                    0 & 1 & | & 0
                \end{bmatrix}.
            \end{equation*}
            Then, we have \(2\langle1\rangle-\langle2\rangle\to\langle2\rangle\), producing
            \begin{equation*}
                \begin{bmatrix}
                    1 & -\frac{1}{2} & | & -\frac{1}{4} \\
                    0 & -2 & | & -\frac{1}{2} \\
                    0 & 1 & | & 0
                \end{bmatrix}.
            \end{equation*}
            Next, we will carry out \(-\frac{1}{2}\langle2\rangle\to\langle2\rangle\) to yield
            \begin{equation*}
                \begin{bmatrix}
                    1 & -\frac{1}{2} & | & -\frac{1}{4} \\
                    0 & 1 & | & \frac{1}{4} \\
                    0 & 1 & | & 0
                \end{bmatrix}.
            \end{equation*}
            We will then compute \(\langle2\rangle-\langle3\rangle\to\langle3\rangle\); we have
            \begin{equation*}
                \begin{bmatrix}
                    1 & -\frac{1}{2} & | & -\frac{1}{4} \\
                    0 & 1 & | & \frac{1}{4} \\
                    0 & 0 & | & \frac{1}{4}
                \end{bmatrix}.
            \end{equation*}
            There is no solution, so \(\vec{v}\) is not a linear combination of \(\vec{v}_1\) and \(\vec{v}_2\).
            
        \end{example}
        \pagebreak
        \begin{example}{\Difficulty\,\Difficulty\,\,Is a Vector a Linear Combination of Others? 3}{veclincombothers3}
        
            Let \(\vec{v}=[14,-21,7]\), \(\vec{v}_1=[2,-3,1]\), and \(\vec{v}_2=[-4,6,2]\). Is \(\vec{v}\) a linear combination of \(\vec{v}_1\) and \(\vec{v}_2\)?
            \\
            \\
            Consider the system
            \begin{equation*}
                \begin{bmatrix}
                    2 & -4 & | & 14 \\
                    -3 & 6 & | & -21 \\
                    1 & 2 & | & 7
                \end{bmatrix}.
            \end{equation*}
            We first perform the row operation \(\frac{1}{2}\langle1\rangle\to\langle1\rangle\) to obtain
            \begin{equation*}
                \begin{bmatrix}
                    1 & -2 & | & 7 \\
                    -3 & 6 & | & -21 \\
                    1 & 2 & | & 7
                \end{bmatrix}.
            \end{equation*}
            Then, we have \(3\langle1\rangle+\langle2\rangle\to\langle2\rangle\), producing
            \begin{equation*}
                \begin{bmatrix}
                    1 & -2 & | & 7 \\
                    0 & 0 & | & 0 \\
                    1 & 2 & | & 7
                \end{bmatrix}.
            \end{equation*}
            Next, we will carry out \(\langle2\rangle\leftrightarrow\langle3\rangle\) to yield
            \begin{equation*}
                \begin{bmatrix}
                    1 & -2 & | & 7 \\
                    1 & 2 & | & 7 \\
                    0 & 0 & | & 0
                \end{bmatrix}.
            \end{equation*}
            We will then compute \(\langle1\rangle+\langle2\rangle\to\langle2\rangle\); we have
            \begin{equation*}
                \begin{bmatrix}
                    1 & -2 & | & 7 \\
                    2 & 0 & | & 14 \\
                    0 & 0 & | & 0
                \end{bmatrix}.
            \end{equation*}
            Then, we will execute \(2\langle1\rangle-\langle2\rangle\to\langle2\rangle\), and we obtain
            \begin{equation*}
                \begin{bmatrix}
                    1 & -2 & | & 7 \\
                    0 & -4 & | & 0 \\
                    0 & 0 & | & 0
                \end{bmatrix}.
            \end{equation*}
            We then have, by \(-\frac{1}{4}\langle2\rangle\to\langle2\rangle\),
            \begin{equation*}
                \begin{bmatrix}
                    1 & -2 & | & 7 \\
                    0 & 1 & | & 0 \\
                    0 & 0 & | & 0
                \end{bmatrix}.
            \end{equation*}
            Finally, we have the operation \(\langle1\rangle+2\langle2\rangle\to\langle1\rangle\), which produces
            \begin{equation*}
                \begin{bmatrix}
                    1 & 0 & | & 7 \\
                    0 & 1 & | & 0 \\
                    0 & 0 & | & 0
                \end{bmatrix}.
            \end{equation*}
            Here, we see that \(\vec{v}=7\vec{v}_1\). We note that it would have been simple to conclude this based on the problem statement, but the method shown is the systematic algorithm for answering such questions.
            
        \end{example}
        \pagebreak
        \begin{example}{\Difficulty\,\Difficulty\,\,Is a Vector a Linear Combination of Others? 4}{veclincombothers4}
        
            Let \(\vec{v}=[14,-21,7]\), \(\vec{v}_1=[2,-3,1]\), and \(\vec{v}_2=[-4,6,-2]\). Is \(\vec{v}\) a linear combination of \(\vec{v}_1\) and \(\vec{v}_2\)?
            \\
            \\
            Consider the system
            \begin{equation*}
                \begin{bmatrix}
                    2 & -4 & | & 14 \\
                    -3 & 6 & | & -21 \\
                    1 & -2 & | & 7
                \end{bmatrix}.
            \end{equation*}
            By row reduction, we finally obtain
            \begin{equation*}
                \begin{bmatrix}
                    1 & -2 & | & 7 \\
                    0 & 0 & | & 0 \\
                    0 & 0 & | & 0
                \end{bmatrix}.
            \end{equation*}
            Here, the solution set is \(\{(2c+7,c):c\in\mathbb{R}\}\). There are thus infinitely many ways to express \(\vec{v}\) as a linear combination of \(\vec{v}_1\) and \(\vec{v}_2\).
            
        \end{example}
        \vphantom
        \\
        \\
        Consider the following definition.
        \begin{definition}{\Stop\,\,Row Space}{rowspace}
        
            Suppose \(A\in\mathcal{M}_{mn}\). The row space of \(A\) is the subset of \(\mathbb{R}^n\) consisting of the linear combinations of the rows of \(A\).
            
        \end{definition}
        \vphantom
        \\
        \\
        Consider the following examples.
        \begin{example}{\Difficulty\,\,Row Space 1}{rsp1}
        
            Consider
            \begin{equation*}
                A=\begin{bmatrix}
                    1 & 2 \\
                    5 & 10
                \end{bmatrix}.
            \end{equation*} 
            The row space of \(A\) is 
            \begin{equation*}
                 \{c_1[1,2]+c_2[5,10]:c_1,c_2\in\mathbb{R}\}.
            \end{equation*}
            In this case, the row space of \(A\) is a line. Generally, though, with two vectors, the row space will be a plane.
        \end{example}
        \begin{example}{\Difficulty\,\,Row Space 2}{rsp2}
        
            Consider
            \begin{equation*}
                A=\begin{bmatrix}
                    1 & 3 \\
                    5 & 10
                \end{bmatrix}.
            \end{equation*} 
            The row space of \(A\) is 
            \begin{equation*}
                 \{c_1[1,3]+c_2[5,10]:c_1,c_2\in\mathbb{R}\}.
            \end{equation*}
            In this case, the row space of \(A\) is a plane.
        \end{example}
        \pagebreak
        \vphantom
        \\
        \\
        To determine if a vector is in the row space of a matrix \(A\), we consider the system \([A^T|X]\). One may ask: why? Well, considering \(A\) instead of \(A^T\) would provide the wrong system of equations to solve. All we are doing when determining if a vector is in the row space of \(A\) is asking if the vector can be written as a linear combination of the rows of \(A\). Consider the following example.
        \begin{example}{\Difficulty\,\Difficulty\,\,Are Vectors in the Row Space?}{vecrowspace}
            Consider
            \begin{equation*}
                A=\begin{bmatrix}
                    1 & 2 \\
                    5 & 10
                \end{bmatrix}
            \end{equation*} 
            and recall that the row space of \(A\) is 
            \begin{equation*}
                 \{c_1[1,2]+c_2[5,10]:c_1,c_2\in\mathbb{R}\}.
            \end{equation*}
            Is \([3,6]\) in the row space of \(A\)? Is \([1,0]\) in the row space of \(A\)? We consider
            \begin{equation*}
            \begin{bmatrix}
                1 & 2 \\
                5 & 10
            \end{bmatrix}^T\begin{bmatrix} x_1 \\ x_2 \end{bmatrix} = \begin{bmatrix} b_1 \\ b_2 \end{bmatrix}.
        \end{equation*}
        For \([3,6]\), we have
        \begin{equation*}
            \begin{bmatrix}
                1 & 5 & | & 3 \\
                2 & 10 & | & 6
            \end{bmatrix},
        \end{equation*}
        which reduces to
        \begin{equation*}
            \begin{bmatrix}
                1 & 5 & | & 3 \\
                0 & 0 & | & 0
            \end{bmatrix}.
        \end{equation*}
        The system has (infinitely many) solutions, so \([3,6]\) is in the row space of \(A\). For \([1,0]\), we have
        \begin{equation*}
            \begin{bmatrix}
                1 & 5 & | & 1 \\
                2 & 10 & | & 0
            \end{bmatrix},
        \end{equation*}
        which reduces to
        \begin{equation*}
            \begin{bmatrix}
                1 & 5 & | & 1 \\
                0 & 0 & | & -2
            \end{bmatrix}.
        \end{equation*}
        The system has no solution, so \([1,0]\) is not in the row space of \(A\).
        \end{example}
        \pagebreak
        \vphantom
        \\
        \\
        Consider the following theorems.
        \begin{theorem}{\Stop\,\,Transitivity of Linear Combinations}{translincomb}
        
            Suppose that \(\vec{x}\) is a linear combination of \(\vec{q}_1,\ldots,\vec{q}_k\), and suppose also that each of \(\vec{q}_1,\ldots,\vec{q}_k\) is itself a linear combination of \(\vec{r}_1,\ldots,\vec{r}_\ell\). Then, \(\vec{x}\) is a linear combination of \(\vec{r}_1,\ldots,\vec{r}_\ell\).
            \begin{proof}
                Because \(\vec{x}\) is a linear combination of \(\vec{q}_1,\ldots,\vec{q}_k\), 
                \begin{equation*}
                    \vec{x}=c_1\vec{q}_1+\cdots+c_k\vec{q}_k
                \end{equation*}
                for \(c_1,\ldots,c_k\in\mathbb{R}\). Then, since each of \(\vec{q}_1,\ldots,\vec{q}_k\) can be written as a linear combination of \(\vec{r}_1,\ldots,\vec{r}_\ell\), there exist scalars \(d_{11},\ldots,d_{k\ell}\) such that
                \begin{align*}
                    \vec{q}_1&=d_{11}\vec{r}_1+d_{12}\vec{r}_2+\cdots+d_{1\ell}\vec{r}_\ell \\
                    \vec{q}_2&=d_{21}\vec{r}_1+d_{22}\vec{r}_2+\cdots+d_{2\ell}\vec{r}_\ell \\
                    &\vdots \\
                    \vec{q}_k&=d_{k1}\vec{r}_1+d_{k2}\vec{r}_2+\cdots+d_{k\ell}\vec{r}_\ell
                \end{align*}
                Then,
                \begin{align*}
                    \vec{x}&=c_1(d_{11}\vec{r}_1+d_{12}\vec{r}_2+\cdots+d_{1\ell}\vec{r}_\ell) \\
                    &\quad+c_2(d_{21}\vec{r}_1+d_{22}\vec{r}_2+\cdots+d_{2\ell}\vec{r}_\ell) \\
                    &\quad\,\,\,\vdots \\
                    &\quad+c_k(d_{k1}\vec{r}_1+d_{k2}\vec{r}_2+\cdots+d_{k\ell}\vec{r}_\ell) \\
                    &=(c_1d_{11}+c_2d_{21}+\cdots+c_kd_{k1})\vec{r}_1 \\
                    &\quad+(c_1d_{12}+c_2d_{22}+\cdots+c_kd_{k2})\vec{r}_2 \\
                    &\quad\,\,\,\vdots \\\
                    &\quad+(c_1d_{1\ell}+c_2d_{1\ell}+\cdots+c_kd_{k\ell})\vec{r}_\ell.
                \end{align*}
                We have just written \(\vec{x}\) as a linear combination of \(\vec{r}_1,\ldots,\vec{r}_\ell\).
            \end{proof}
            Note that this theorem may be rephrased as follows: If \(\vec{x}\) is in the row space of a matrix \(Q\) and each row of \(Q\) is in the row space of a matrix \(R\), \(\vec{x}\) is in the row space of \(R\).
        \end{theorem}
        \begin{theorem}{\Stop\,\,Row Equivalence Implies Equal Row Space}{rowequivequalrowspc}
            
            Suppose \(A\) and \(B\) are row equivalent. Then, the row space of \(A\) is equal to the row space of \(B\).
        
        \end{theorem}

\pagebreak

\section{Lecture 11: September 16, 2022}

    \subsection{Linear Maps}
    
        Consider the following definition.
        \begin{definition}{\Stop\,\,Linear Maps}{linmaps}
        
            Given \(A\in\mathcal{M}_{mn}\), we define
            \begin{align*}
                T_A&:\mathbb{R}^n\to\mathbb{R}^m \\
                &:\begin{bmatrix} x_1 \\ \vdots \\ x_n \end{bmatrix}\mapsto A\begin{bmatrix} x_1 \\ \vdots \\ x_n \end{bmatrix}.
            \end{align*}
            
        \end{definition}
        \vphantom
        \\
        \\
        Consider the following example.
        \begin{example}{\Difficulty\,\Difficulty\,\,Some Special Maps in \(\mathbb{R}^2\)}{specialmapsr2}
            
            Consider the following maps, and name them.
            \begin{enumerate}
                \item If \(A=\begin{bmatrix} 0 & 0 \\ 0 & 0 \end{bmatrix}\), \(T_A:\mathbb{R}^2\to\mathbb{R}^2\) is the zero map.
                \item If \(A=\begin{bmatrix} 1 & 0 \\ 0 & 1 \end{bmatrix}\), \(T_A:\mathbb{R}^2\to\mathbb{R}^2\) is the identity map.
                \item If \(A=\begin{bmatrix} 1 & 0 \\ 0 & 0 \end{bmatrix}\), \(T_A:\mathbb{R}^2\to\mathbb{R}^2\) is the projection onto the \(x\) axis.
                \item If \(A=\begin{bmatrix} 0 & 0 \\ 0 & 1 \end{bmatrix}\), \(T_A:\mathbb{R}^2\to\mathbb{R}^2\) is the projection onto the \(y\) axis.
            \end{enumerate}
            
        \end{example}
        \vphantom
        \\
        \\
        We will revisit linear maps in much greater detail in Chapter \ref{chapter:lintrans}.

\pagebreak

\section{Lecture 12: September 19, 2022}

    \subsection{Inverses of Matrices}

        Consider the following definitions and theorems.
        \begin{definition}{\Stop\,\,Multiplicative Inverse of a Matrix}{inverse}

            Let \(A\in\mathcal{M}_{nn}\). Then, \(B\in\mathcal{M}_{nn}\) is a multiplicative inverse of \(A\) if and only if 
            \begin{equation*}
                AB=BA=I_n.
            \end{equation*}
            
        \end{definition}
        \vphantom
        \\
        \\
        Consider the following examples.
        \begin{example}{\Difficulty\,\Difficulty\,\,The Inverse of the Identity}{inviden}

            Let \(A=I_n\). Find the inverse of \(A\).
            \begin{proof}
                Since \(I_nI_n=I_nI_n=I_n\), \(I_n\) is the inverse of \(A\).
            \end{proof}
            
        \end{example}
        \begin{example}{\Difficulty\,\Difficulty\,\,The Inverse of the Zero}{invzero}

            Let \(A=0_n\). Show that \(A\) does not have an inverse.
            \begin{proof}
                For all \(B\in\mathcal{M}_{nn}\), \(AB=0_nB=0_n\neq I_n\).
            \end{proof}
            
        \end{example}
        \begin{theorem}{\Stop\,\,Inverse Commutativity}{invcommute}
            
            Let \(A,B\in\mathcal{M}_{nn}\). If either \(AB\) or \(BA\) equals \(I_n\), the other product also equals \(I_n\), and \(A\) and \(B\) are inverses of each other.
            
        \end{theorem}
        \begin{definition}{\Stop\,\,Singularity}{singularity}

            A matrix is \textit{singular} if and only if it is square and does not have an inverse. A matrix is \textit{nonsingular} if and only if it is square and has an inverse.
            
        \end{definition}
        \begin{theorem}{\Stop\,\,Uniqueness of the Inverse}{uniquenessinv}

            If \(B\) and \(C\) are both inverses of \(A\in\mathcal{M}_{nn}\), \(B=C\).
            \begin{proof}
                    \(B=BI_n=B(AC)=(BA)C=I_nC=C\).
             \end{proof}
            
        \end{theorem}
        \pagebreak
        \vphantom
        \\
        \\
        We denote the unique inverse of \(A\) as \(A^{-1}\). We can use the inverse to define negative integral powers of a nonsingular matrix \(A\). Consider the following definition.
        \begin{definition}{\Stop\,\,Negative Integral Powers of a Nonsingular Matrices}{nonsingmat}

            Let \(A\) be a nonsingular matrix. Then, the negative integral powers of \(A\) are given as follows: \(A^{-1}\) is the unique inverse of \(A\). For \(k\geq2\), \(A^{-k}=(A^{-1})^k\).
            
        \end{definition}
        \begin{theorem}{\Stop\,\,Properties of Nonsingular Matrices}{propnonsingmat}

            Let \(A\) and \(B\) be nonsingular \(n\times n\) matrices. Then,
            \begin{enumerate}
                \item \(A^{-1}\) is nonsingular, and \((A^{-1})^{-1}=A\).
                \begin{proof}
                    We have that \(A^{-1}A=AA^{-1}=I_n\) since \(A^{-1}\) is the inverse of \(A\), hence \(A^{-1}\) is nonsingular and \((A^{-1})^{-1}=A\).
                \end{proof}
                \item \(A^k\) is nonsingular, and \((A^k)^{-1}=(A^{-1})^k=A^{-k}\), for \(k\in\mathbb{Z}\).
                \begin{proof}
                    We have that \(A^kA^{-k}=A^{-k+k}=A^0=I_n\). Hence, \(A^k\) is nonsingular and \((A^k)^{-1}=A^{-k}\).
                \end{proof}
                \item \(AB\) is nonsingular, and \((AB)^{-1}=B^{-1}A^{-1}\).
                \begin{proof}
                    We have that \((AB)(B^{-1}A^{-1})=A(BB^{-1})A^{-1}=AI_nA^{-1}=AA^{-1}=I_n\). Hence, \(AB\) is nonsingular and \((AB)^{-1}=B^{-1}A^{-1}\).
                \end{proof}
                \item \(A^T\) is nonsingular, and \((A^T)^{-1}=(A^{-1})^T\).
                \begin{proof}
                    We have that \((A^T)(A^{-1})^T=(A^{-1}A)^T=I_n^T=I_n\).
                \end{proof}
            \end{enumerate}
            
        \end{theorem}
        \vphantom
        \\
        \\
        Now, we will fully provide statements of matrix exponent laws.
        \begin{theorem}{\Stop\,\,Matrix Exponent Laws}{matexplaw}

            If \(A\) is nonsingular and \(p,q\in\mathbb{Z}\),
            \begin{enumerate}
                \item \(A^{p+q}=(A^p)(A^q)\).
                \item \((A^p)^q=A^{pq}=(A^q)^p\).
            \end{enumerate}
            
        \end{theorem}
        \pagebreak
        \vphantom
        \\
        \\
        Consider the following theorem.
        \begin{theorem}{\Stop\,\,\(2\times 2\) Inverse}{2by2inv}

            Suppose \(A=\begin{bmatrix} a & b \\ c & d \end{bmatrix}\). Then, \(A\) is nonsingular if and only if \(ad-bc\neq0\). In this case,
            \begin{equation*}
                A=\frac{1}{ad-bc}\begin{bmatrix} d & -b \\ -c & a \end{bmatrix}
            \end{equation*}
            \begin{proof}
                Suppose \(ad-bc\neq0\). We consider
                \begin{equation*}
                    \frac{1}{ad-bc}\begin{bmatrix} d & -b \\ -c & a \end{bmatrix}\begin{bmatrix} a & b \\ c & d \end{bmatrix}=\begin{bmatrix}
                        1 & 0 \\ 
                        0 & 1
                    \end{bmatrix}.
                \end{equation*}
                Thus, \(A^{-1}\) exists and the formula holds. Now, suppose that \(ad-bc=0\). We wish to show that \(A\) is singular. Consider
                \begin{equation*}
                    \begin{bmatrix} d & -b \\ -c & a \end{bmatrix}\begin{bmatrix} a & b \\ c & d \end{bmatrix}=\begin{bmatrix} ad-bc & 0 \\ 0 & ad-bc \end{bmatrix}=\begin{bmatrix} 0 & 0 \\ 0 & 0 \end{bmatrix}.
                \end{equation*}
                Suppose \(A\) has an inverse. Then,
                \begin{equation*}
                    \begin{bmatrix} d & -b \\ -c & a \end{bmatrix}\left(AA^{-1}\right)=\begin{bmatrix} d & -b \\ -c & a \end{bmatrix}.
                \end{equation*}
                But,
                \begin{equation*}
                    \left(\begin{bmatrix} d & -b \\ -c & a \end{bmatrix}A\right)A^{-1}=\begin{bmatrix} 0 & 0 \\ 0 & 0 \end{bmatrix}A^{-1}=\begin{bmatrix} 0 & 0 \\ 0 & 0 \end{bmatrix}.
                \end{equation*}
                This implies \(a=b=c=d=0\), meaning \(A=0_2\). From Example \ref{exa:invzero}, we know that the matrix \(0_n\) is singular.
            \end{proof}
        \end{theorem}
        
\pagebreak

\section{Lecture 13: September 21, 2022}

    \subsection{Finding Inverses of Matrices}

        We have provided a lot of theory about the inverse of a matrix, but given \(A\in\mathcal{M}_{nn}\), how do we determine if \(A\) is invertible. That is, how do we determine if \(A\) is nonsingular? If it is, how do we find \(A^{-1}\)? Consider the matrix equation
        \begin{equation*}
            A\begin{bmatrix}
                x_{11} & \cdots & x_{1n} \\
                \vdots & \ddots & \vdots \\
                x_{n1} & \cdots & x_{nn}
            \end{bmatrix}=
            \begin{bmatrix}
                1 & \cdots & 0 \\
                \vdots & \ddots & \vdots \\
                0 & \cdots & 1
            \end{bmatrix}=I_n.
        \end{equation*}
        Our goal is to solve for the elements \(x_{ij}\). We thus have \(n\) linear systems to solve. That is,
        \begin{equation*}
            A\begin{bmatrix} x_{11} \\ \vdots \\ x_{n1} \end{bmatrix} = \begin{bmatrix} 1 \\ \vdots \\ 0 \end{bmatrix}\quad,\ldots,\quad A\begin{bmatrix} x_{1n} \\ \vdots \\ x_{nn} \end{bmatrix} = \begin{bmatrix} 0 \\ \vdots \\ 1 \end{bmatrix}.
        \end{equation*}
        To solve equations of the sort, consider the following steps.
        \begin{enumerate}
            \item Form \([A|I_n]\).
            \item Row reduce \([A|I_n]\) to \([C|D]\) until we have reduced row echelon form.
            \item If \(C=I_n\), \(A^{-1}\) exists and \(A^{-1}=D\). Otherwise, \(A^{-1}\) does not exist.
        \end{enumerate}
        \pagebreak
        \vphantom
        \\
        \\
        Consider the following example.
        \begin{example}{\Difficulty\,\Difficulty\,\,Finding a \(2\times 2\) Inverse}{findinv22}
            
            Consider the matrix 
            \begin{equation*}
                \begin{bmatrix}
                    2 & 1 \\
                    1 & 1
                \end{bmatrix}.
            \end{equation*}
            First we form
            \begin{equation*}
                \begin{bmatrix}
                    2 & 1 & | & 1 & 0 \\
                    1 & 1 & | & 0 & 1
                \end{bmatrix}.
            \end{equation*}
            We perform the row operation \(\langle 2\rangle\leftrightarrow\langle1\rangle\), yielding
            \begin{equation*}
                \begin{bmatrix}
                    1 & 1 & | & 0 & 1 \\
                    2 & 1 & | & 1 & 0
                \end{bmatrix}.
            \end{equation*}
            Then, we execute \(2\langle1\rangle-\langle2\rangle\to\langle2\rangle\), which yields
            \begin{equation*}
                \begin{bmatrix}
                    1 & 1 & | & 0 & 1 \\
                    0 & 1 & | & -1 & 2
                \end{bmatrix}.
            \end{equation*}
            We then compute \(\langle2\rangle-\langle1\rangle\to\langle 1\rangle\), producing
            \begin{equation*}
                \begin{bmatrix}
                    -1 & 0 & | & -1 & 1 \\
                    0 & 1 & | & -1 & 2
                \end{bmatrix}.
            \end{equation*}
            Finally, we simply perform \(-\langle1\rangle\to\langle1\rangle\), which obtains
            \begin{equation*}
                \begin{bmatrix}
                    1 & 0 & | & 1 & -1 \\
                    0 & 1 & | & -1 & 2
                \end{bmatrix}.
            \end{equation*}
            Therefore, 
            \begin{equation*}
                \begin{bmatrix}
                    1 & -1 \\
                    -1 & 2
                \end{bmatrix}
            \end{equation*}
            is the inverse of 
            \begin{equation*}
                \begin{bmatrix}
                    2 & 1 \\
                    1 & 1
                \end{bmatrix}.
            \end{equation*}
            Note that here, we did not apply the formula in Theorem \ref{thm:2by2inv}, but instead, the general algorithm.

        \end{example}
        \pagebreak
        \begin{example}{\Difficulty\,\Difficulty\,\,Finding a \(3\times 3\) Inverse}{findinv33}
            
            Consider the matrix 
            \begin{equation*}
                \begin{bmatrix}
                    2 & -6 & 5 \\
                    -4 & 12 & -9 \\
                    2 & -9 & 8
                \end{bmatrix}.
            \end{equation*}
            First, we form
            \begin{equation*}
                \begin{bmatrix}
                    2 & -6 & 5 & | & 1 & 0 & 0 \\
                    -4 & 12 & -9 & | & 0 & 1 & 0 \\
                    2 & -9 & 8 & | & 0 & 0 & 1
                \end{bmatrix}.
            \end{equation*}
            After row reduction, we have
            \begin{equation*}
                \begin{bmatrix}
                    1 & 0 & 0 & | & \frac{5}{2} & \frac{1}{2} & -1 \\
                    0 & 1 & 0 & | & \frac{7}{3} & 1 & -\frac{1}{3} \\
                    0 & 0 & 1 & | & 2 & 1 & 0
                \end{bmatrix}.
            \end{equation*}
            Thus, 
            \begin{equation*}
                \begin{bmatrix}
                    2 & -6 & 5 \\
                    -4 & 12 & -9 \\
                    2 & -9 & 8
                \end{bmatrix}
                \begin{bmatrix}
                    \frac{5}{2} & \frac{1}{2} & -1 \\
                    \frac{7}{3} & 1 & -\frac{1}{3} \\
                    2 & 1 & 0
                \end{bmatrix}=I_3.
            \end{equation*}

            
        \end{example}
        \vphantom
        \\
        \\
        Finally, we present an important theorem about the existence and uniqueness of solutions to \(AX=B\).
        \begin{theorem}{\Stop\,\,Uniqueness of Solutions to Linear Systems}{uniquenessofsol}

            Let \(A\in\mathcal{M}_{nn}\) and \(AX=B\) be a linear system. If \(A\) is nonsingular, the system has a unique solution. If \(A\) is singular, the system either has no solution or infinitely many solutions. That is, \(AX=B\) has a unique solution if and only if \(A\) is nonsingular.
            \begin{proof}
                Consider the case where \(A\) is nonsingular; by definition, \(A^{-1}\) exists. Consider \(X=A^{-1}B\) as a prospective solution. To verify it, we have
                \begin{equation*}
                    A(A^{-1}B)=(AA^{-1})B=I_nB=B.
                \end{equation*}
                Thus, \(X=A^{-1}B\) is a valid solution. Now, for uniqueness, suppose that \(X=Y\) is a solution. That is, suppose \(AY=B\). We multiply both sides by \(A^{-1}\), on the left, to obtain
                \begin{equation*}
                    A^{-1}(AY)=A^{-1}B \implies (A^{-1}A)Y=A^{-1}B \implies I_nY=A^{-1}B \implies Y=A^{-1}B,
                \end{equation*}
                showing that \(X=A^{-1}B\) is a unique solution. Now, if \(A\) is singular, \(\rank A<n\), meaning we will have no solution or infinitely many solutions. This is because we will either have a row of all zeroes with a nonzero entry after the augmentation bar yielding an empty solution set, or if the system has at least one solution, there will be at least one free variable, guaranteeing infinitely many solutions.
            \end{proof}

        \end{theorem}


\addtocontents{toc}{\protect\newpage}
\begin{savequote}
    \includegraphics[scale=0.4]{Graphics/eigenvectorxkcd.png}
\end{savequote}
\chapter{Determinants and Eigenvalues} \label{chapter:deteigen}

    \section{Lecture 13: September 21, 2022}

    \subsection{Defining the Determinant}

        Consider the following theorems and definitions.
        \begin{theorem}{\Stop\,\,The Determinant Determines the Area in \(\mathbb{R}^2\)}{areadet}

            Consider \(\vec{x}=[x_1,x_2]\) and \(\vec{y}=[y_1,y_2]\). If we form
            \begin{equation*}
                A=\begin{bmatrix}
                    x_1 & x_2 \\
                    y_1 & y_2 
                \end{bmatrix},
            \end{equation*}
            \begin{equation*}
                |\det A| = ||\vec{x}\times\vec{y}||.
            \end{equation*}
            That is, \(|\det A|\) provides the area of the parallelogram determined by \(\vec{x}\) and \(\vec{y}\).
            
        \end{theorem}
        \begin{theorem}{\Stop\,\,The Determinant Determines the Volume in \(\mathbb{R}^3\)}{voldet}

            Consider \(\vec{x}=[x_1,x_2,x_3]\), \(\vec{y}=[y_1,y_2,y_3]\), and \(\vec{z}=[z_1,z_2,z_3]\).
            If we form
            \begin{equation*}
                A=\begin{bmatrix}
                    x_1 & x_2 & x_3 \\
                    y_1 & y_2 & y_3 \\
                    z_1 & z_2 & z_3
                \end{bmatrix},
            \end{equation*}
            \begin{equation*}
                |\det A| = \vec{x}\cdot(\vec{y}\times\vec{z}).
            \end{equation*}
            That is, \(|\det A|\) provides the volume of the parallelepiped determined by \(\vec{x}\), \(\vec{y}\), and \(\vec{z}\).
            
        \end{theorem}
        \begin{definition}{\Stop\,\,The \((i,j)\) Submatrix}{submatrix}

            Suppose \(A\in\mathcal{M}_{nn}\). The \((i,j)\) submatrix of \(A\) is the \((n-1)\times(n-1)\) matrix obtained by removing the \(i\)th row and the \(j\)th column. We denote this by \(A_{(i,j)}\) 
            
        \end{definition}
        \begin{definition}{\Stop\,\,The \((i,j)\) Minor}{minor}

            Suppose \(A\in\mathcal{M}_{nn}\). The \((i,j)\) minor of \(A\) is the determinant of the \((i,j)\) submatrix of \(A\).
            
        \end{definition}
        \begin{definition}{\Stop\,\,The \((i,j)\) Cofactor}{cofactor}

            Suppose \(A\in\mathcal{M}_{nn}\). The \((i,j)\) cofactor of \(A\) is
            \begin{equation*}
                A_{ij}=(-1)^{i+j}\det (A_{(i,j)}).
            \end{equation*}
            
        \end{definition}
        \begin{definition}{\Stop\,\,The Determinant}{det}
            Suppose \(A\in\mathcal{M}_{nn}\). Then,
            \begin{enumerate}
                \item If \(n=1\), and \(A=\begin{bmatrix} a_{11} \end{bmatrix}\), \(\det A = a_{11}\).
                \item If \(n=2\), and \(A=\begin{bmatrix} a_{11} & a_{12} \\ a_{21} & a_{22} \end{bmatrix}\), \(\det A = a_{11}a_{22}-a_{12}a_{21}\).
                \item If \(n>2\), and \(A=\begin{bmatrix} a_{11} & \cdots & a_{1n} \\ \vdots & \ddots & \vdots \\ a_{n1} & \cdots & a_{nn} \end{bmatrix}\), \(\det A = (a_{11}A_{11}+\cdots+a_{1n}A_{1n})+\cdots+(a_{n1}A_{n1}+\cdots+a_{nn}A_{nn})\).
            \end{enumerate}
        \end{definition}
        \pagebreak
        \vphantom
        \\
        \\
        For fun, consider the following Python 3 implementation of computing the determinant of any \(n\times n\) matrix.
        \lstinputlisting[language=Python]{Graphics/matrixdet.py}
        
        \pagebreak

\section{Lecture 14: September 23, 2022}

    \subsection{Determinants of Upper Triangular Matrices}

        Consider the following theorems.
        \begin{theorem}{\Stop\,\,The Determinant of an Upper Triangular Matrix}{uppertriangulardet}

            If \(A\in\mathcal{M}_{nn}\) is upper triangular, 
            \begin{equation*}
                \det A = a_{11} a_{22} \cdots a_{nn}.
            \end{equation*}
            Recall that \(A\) is upper triangular if and only if all elements below the main diagonal are zero.
            \begin{proof}
                We proceed by induction on \(n\). For \(n=1\), \(A=\begin{bmatrix} a_{11} \end{bmatrix}\), so \(\det A = a_{11}\). Suppose for all \(k\in\mathbb{N}\), and some upper triangular \(A\in\mathcal{M}_{kk}\),
                \begin{equation*}
                    a_{11} a_{22} \cdots a_{kk}.
                \end{equation*}
                Consider 
                \begin{equation*}
                    B=\begin{bmatrix}
                        b_{11} & \cdots & b_{1(k+1)} \\
                        \vdots & \ddots & \vdots \\
                        0 & \cdots & b_{(k+1)(k+1)}
                    \end{bmatrix}.
                \end{equation*}
                Then, we compute \(\det B\) using the last row as our ``first row.''
                \begin{align*}
                    \det B &= (-1)^{1+1}b_{11}\det\begin{bmatrix} b_{22} & b_{23} & \cdots & a_{2(k+1)} \\ 0 & b_{33} & \cdots & b_{3(k+1)} \\ \vdots & \vdots & \vdots & \vdots \\ 0 & 0 & 0 & b_{(k+1)(k+1)} \end{bmatrix} \\
                    &= b_{11}\underbrace{b_{22}\cdots b_{(k+1)(k+1)}}_{\text{By the inductive hypothesis.}}.
                    %\begin{comment}\det B&= 0+\cdots+0+a_{(k+1)(k+1)}(-1)^{(k+1)+(k+1)}\det B_{(k+1),(k+!)} \\ &=a_{(k+1)(k+1)}(-1)^{2(k+1)}\det A.\end{comment}
                \end{align*}
                This is precisely the stipulation of the theorem when \(n=k+1\).
            \end{proof}

        \end{theorem}
        \pagebreak
        \begin{theorem}{\Stop\,\,Determinants and Row Operations}{detrowops}

            Suppose \(A\in\mathcal{M}_{nn}\) and let \(R\) be a row operation. Then,
            \begin{enumerate}
                \item If \(R\) is \(c\langle i\rangle\to\langle i \rangle\) for some \(c\in\mathbb{R}\), 
                \begin{equation*}
                    \det R(A) = c\det A.
                \end{equation*}
                Note that if \(c=0\), \(R\) is not a valid row operation.
                \item If \(R\) is \(c\langle i\rangle+\langle j \rangle\to\langle j \rangle\) for some \(c\in\mathbb{R}\),
                \begin{equation*}
                    \det R(A) = \det A.
                \end{equation*}
                Note that if \(c=0\), \(R\) is not a valid row operation.
                \item If \(R\) is \(\langle i \rangle\leftrightarrow\langle j \rangle\),
                \begin{equation*}
                    \det R(A) = -\det A.
                \end{equation*}
            \end{enumerate}
            \vphantom
            \\
            \\
            Note that if \(\det A\neq 0\) and \(R\) is a row operation, \(\det R(A)\neq 0\).
        \end{theorem}
        \begin{theorem}{\Stop\,\,Properties of Determinants}{detprop}

            Suppose \(A,B\in\mathcal{M}_{nn}\). Then,
            \begin{enumerate}
                \item \(\det (AB) = \det A\det B\).
                \item \(\det (A^T) = \det A\).
                \item \(\underbrace{\det (A^{-1}) = \frac{1}{\det A}}_{\text{Suppose that \(A\) is nonsingular.}}\).
            \end{enumerate}
            
        \end{theorem}
        \begin{theorem}{\Stop\,\,Inverses and Determinants}{invdet}
            
            Suppose that \(A\in\mathcal{M}_{nn}\). Then, \(A\) is nonsingular if and only if \(\det A \neq 0\).
            \begin{proof}
                If \(A\) is nonsingular, we can row reduce \(A\) to produce \(I_n\). Since \(\det I_n\neq 0\), by Theorem \ref{thm:detrowops}, \(\det A\neq 0\). If \(\det A\neq 0\), we form the system \([A|B]\) and reduce it to \([C|D]\). We know that \(C\) either is the identity matrix or has at least one zero row. Because \(\det A\neq 0\), we realize that \(C=I_n\). Because we were able to row reduce \(A\) to \(I_n\), \(A\) is nonsingular.
            \end{proof}
            
        \end{theorem}
        \pagebreak
        \vphantom
        \\
        \\
        We can use Theorem \ref{thm:detrowops} in conjunction with row operations to compute the determinant of a matrix easily. We simply use row operations to create an upper triangular matrix, while keeping track of how the determinant changes. We then apply Theorem \ref{thm:uppertriangulardet}. Consider the following examples.
        \begin{example}{\Difficulty\,\Difficulty\,\,Computing a Determinant by Row Reduction}{compdetrowred}
            
            Compute \(\det\begin{bmatrix} 1 & 1 & 1 \\ 2 & 3 & -2 \\ 4 & 9 & 4 \end{bmatrix}\).
            \\
            \\
            Let 
            \begin{equation*}
                A_0=\begin{bmatrix} 1 & 1 & 1 \\ 2 & 3 & -2 \\ 4 & 9 & 4 \end{bmatrix}.
            \end{equation*}
            Consider the following table.
            \begin{center}
                \begin{tabular}{||c|c|c||}
                    \hline
                    \hline
                    Row Operation & Resultant Matrix & Effect on the Determinant \\
                    \hline
                    \(2\langle1\rangle-\langle2\rangle\to\langle2\rangle\) & \(A_1=\begin{bmatrix} 1 & 1 & 1 \\ 0 & -1 & 4 \\ 4 & 9 & 4 \end{bmatrix}\) & \(\det A_1=\det A_0\) \\
                    \hline
                    \(4\langle1\rangle-\langle3\rangle\to\langle3\rangle\) & \(A_2=\begin{bmatrix} 1 & 1 & 1 \\ 0 & -1 & 4 \\ 0 & -5 & 0 \end{bmatrix}\) & \(\det A_2=\det A_1\) \\
                    \hline
                    \(5\langle2\rangle-\langle3\rangle\to\langle3\rangle\) & \(A_3=\begin{bmatrix} 1 & 1 & 1 \\ 0 & -1 & -4 \\ 0 & 0 & 20 \end{bmatrix}\) & \(\det A_3=\det A_2\) \\
                    \hline
                \end{tabular}
                \DOTHISLATER
            \end{center}
            \vphantom
            \\
            \\
            Thus, \(\det A_0=-20\).

        \end{example}

\begin{savequote}
    \includegraphics[scale=0.4]{Graphics/puremathxkcd.png}
\end{savequote}
\chapter{Finite Dimensional Vector Spaces} \label{chapter:vcspcs}

    \section{Lecture 20: October 7, 2022}

    \subsection{The Process of Abstraction}

        Consider the following definition.
        \begin{definition}{\Stop\,\,Vector Spaces}{vecspc}

            Let \(\mathbb{F}\) be a field of scalars. For now, \(\mathbb{F}=\mathbb{R}\vee\mathbb{C}\). A vector space \(V\) over \(\mathbb{F}\) is a set with two operations: 
            \begin{enumerate}
                \item Vector Addition: \(V\times V\to V, (\vec{v},\vec{w})\mapsto \vec{v}+\vec{w}\).
                \item Scalar Multiplication: \(\mathbb{F}\times V\to V, (c,\vec{v})\mapsto c\vec{v}\).
            \end{enumerate}
            \vphantom
            \\
            \\
            The following must hold for each \(\vec{u},\vec{v},\vec{w}\in V\) and \(c_1,c_2\in\mathbb{F}\).
            \begin{enumerate}
                \item \(\vec{u}+\vec{v}=\vec{v}+\vec{u}\)
                \item \(\vec{u}+(\vec{v}+\vec{w})=(\vec{u}+\vec{v})+\vec{w}\).
                \item \(\exists \vec{0}\in V, \forall \vec{v}\in\mathbb{V}, \vec{0}+\vec{v}=\vec{v}\)
                \item \(\forall \vec{v}\in V, \exists! (-\vec{v})\in V, \vec{v}+(-\vec{v})=\vec{0}\).
                \item \(c_1(\vec{u}+\vec{v})=c_1\vec{u}+c_1\vec{v}\).
                \item \((c_1+c_2)\vec{u}=c_1\vec{u}+c_2\vec{u}\).
                \item \((c_1c_2)\vec{u}=c_1(c_2\vec{u})\).
                \item \(1\vec{u}=\vec{u}\).
            \end{enumerate}
            
        \end{definition}
        \vphantom
        \\
        \\
        We remark that \(0\vec{v}=\vec{0}\) is \textit{not} an axiom of a vector space; we must \textit{prove} that it holds. Now, we will justify our use of abstraction.
        \begin{enumerate}
            \item The set \(\mathbb{R}^n\) is a vector space.
            \item The set \(\mathbb{C}^n\) is a vector space.
            \item The set \(\{\vec{0}\}\), with \(\vec{0}+\vec{0}+\vec{0}\) and \(c\vec{0}=\vec{0}\), is a vector space.
            \item The set \(\mathcal{M}_{mn}\) is a vector space.
            \item Let \(S\) be a non-empty set and \(F(S)=\{f:S\to\mathbb{R}\}\) with \((f+g)(s)=f(s)+g(s)\) and \((cf)(s)=cf(s)\). The set \(F(S)\) is a vector space.
            \item The set \(\{a_nx^n+\cdots+a_0x^0:a_0,\ldots a_n\in\mathbb{R}\}\) is a vector space.
            \item The set \(\{a_nx^n+\cdots+a_0x^0:a_0,\ldots a_n\in\mathbb{R}, \text{degree is less than or equal to \(n\)}\}\) is a vector space.
        \end{enumerate}
        \vphantom
        \\
        \\
        We note that all the above examples need \textit{proof}.

\addtocontents{toc}{\protect\newpage}
\begin{savequote}
    \includegraphics[scale=0.45]{Graphics/rotationmatrix.png}
\end{savequote}
\chapter{Linear Transformations} \label{chapter:lintrans}

    \section{Lecture 30: November 4, 2022}

    \subsection{An Introduction to Linear Transformations}

        Before proceeding into linear transformations, for a review of functions and associated terminology, consult Appendix \ref{appendix:b}. Consider the following definition.
        \begin{definition}{\Stop\,\,Linear Transformations}{lineartransformation}

            Let \(V\) and \(W\) be vector spaces. Let \(F:V\to W\) be a function. Then, \(F\) is a linear transformation if and only if both the following conditions hold:
            \begin{enumerate}
                \item \(\forall \vec{v}_1,\vec{v}_2\in V, F(\vec{v}_1+\vec{v}_2)=F(\vec{v}_1)+F(\vec{v}_2)\).
                \item \(\forall c\in\mathbb{F},\forall\vec{v}\in V, F(c\vec{v})=cF(\vec{v})\).
            \end{enumerate}
            
        \end{definition}
        \vphantom
        \\
        \\
        We remark that a linear transformation ``preserves'' the operations that give structure to the vector spaces involved: vector addition and scalar multiplication.
        \pagebreak
        \\
        \\
        Consider the following examples.
        \begin{example}{\Difficulty\,\Difficulty\,\,Is it a Linear Transformation? 1}{lintrans1}

            Let \(F:\mathcal{M}_{mn}\to \mathcal{M}_{nm}\) where \(F(A)=A^T\). Is \(F\) a linear transformation?
            \\
            \\
            For matrices \(A_1,A_2\in\mathcal{M}_{mn}\) and scalar \(c\in\mathbb{R}\), we have
            \begin{align*}
                F(A_1+A_2)&=(A_1+A_2)^T \\
                &=A_1^T+A_2^T \\
                &=F(A_1)+F(A_2)
            \end{align*}
            and
            \begin{align*}
                F(cA_1)&=(cA_1)^T \\
                &=cA_1^T \\
                &=cF(A_1).
            \end{align*}
            Thus, \(F\) is a linear transformation.
            
        \end{example}
        \begin{example}{\Difficulty\,\Difficulty\,\,Is it a Linear Transformation? 2}{lintrans2}

            Let \(F:\mathcal{P}_n\to\mathcal{P}_{n-1}\) where \(F(\vec{p})=\vec{p}'\), the derivative of \(\vec{p}\). Is \(F\) a linear transformation?
            \\
            \\
            For \(\vec{p}_1,\vec{p}_2\in \mathcal{P}_n\), we know, from Calculus, that the derivative of a sum is the sum of the derivatives, so
            \begin{align*}
                F(\vec{p}_1+\vec{p_2})&=(\vec{p}_1+\vec{p}_2)' \\
                &=\vec{p}_1'+\vec{p}_2' \\
                &=F(\vec{p}_1)+F(\vec{p}_2).
            \end{align*}
            For \(c\in\mathbb{R}\), the constant multiple rule, from Calculus, tells us that
            \begin{align*}
                F(c\vec{p}_1)&=(c\vec{p}_1)' \\
                &=c\vec{p_1}' \\
                &=cF(\vec{p_1}).
            \end{align*}
            Thus, \(F\) is a linear transformation.
        \end{example}
        \pagebreak
        \begin{example}{\Difficulty\,\Difficulty\,\,Is it a Linear Transformation? 3}{lintrans3}

            Let \(F:\mathcal{P}_{n}\to W\) where \(W=\Span\left\{\frac{1}{s},\ldots,\frac{1}{s^{n+1}}\right\}\) and
            \begin{align*}
                F(\vec{p})&=\laplace{\vec{p}(t)}(s) \\
                &=\int_0^\infty e^{-st}\vec{p}(t)\dd t.
            \end{align*}
            Is \(F\) a linear transformation?
            \\
            \\
            For \(\vec{p}_1(t),\vec{p}_2(t)\in\mathcal{P}_n\), we have
            \begin{align*}
                F(\vec{p}_1(t)+\vec{p}_2(t))&=\int_0^\infty e^{-st}(\vec{p}_1(t)+\vec{p}_2(t))\dd t \\
                &=\int_0^\infty e^{-st}\vec{p}_1(t)+e^{-st}\vec{p}_2(t)\dd t \\
                &=\int_0^\infty e^{-st}\vec{p}_1(t)\dd t+\int_0^\infty e^{-st}\vec{p}_2(t)\dd t \\
                &=F(\vec{p}_1(t))+F(\vec{p}_2(t)).
            \end{align*}
            For \(c\in\mathbb{R}\), we have 
            \begin{align*}
                F(c\vec{p}_1(t))&=\int_0^\infty ce^{-st}\vec{p}_1(t)\dd t \\
                &=c\int_0^\infty e^{-st}\vec{p}_1(t)\dd t \\
                &=cF(\vec{p}_1(t)).
            \end{align*}
            Thus, \(F\) is a linear transformation.
        \end{example}
        \pagebreak
        \begin{example}{\Difficulty\,\Difficulty\,\,Is it a Linear Transformation? 4}{lintrans4}

            Let \(V\) be a vector space with \(\dim V=n\). Let \(B\) be an ordered basis for \(V\). Then, every \(\vec{v}\in V\) has coordinatization \([\vec{v}]_B\) with respect to \(B\). Consider the function \(F:V\to\mathbb{R}^n\) given by 
            \begin{equation*}
                F(\vec{v})=[\vec{v}]_B.
            \end{equation*}
            Is \(F\) a linear transformation?
            \\
            \\
            For \(\vec{v}_1,\vec{v}_2\in V\), we have
            \begin{align*}
                F(\vec{v}_1+\vec{v}_2)&=[\vec{v}_1+\vec{v}_2]_B \\
                &=[\vec{v}_1]_B+[\vec{v}_2]_B \\
                &=F(\vec{v}_1)+F(\vec{v}_2),
            \end{align*}
            by Theorem \ref{thm:propcoords}. Then, for \(c\in\mathbb{R}\), we have
            \begin{align*}
                F(c\vec{v}_1)&=[c\vec{v}_1]_B \\
                &=c[\vec{v}_1]_B \\
                &=cF(\vec{v}),
            \end{align*}
            also by by Theorem \ref{thm:propcoords}. Thus, \(F\) is a linear transformation.

        \end{example}
        \pagebreak
        \vphantom
        \\
        \\
        We now state some properties of linear transformations.
        \begin{theorem}{\Stop\,\,Properties of Linear Transformations}{proplintrans}

            Let \(V\) and \(W\) be vector spaces, and let \(L:V\to W\) be a linear transformation. Let \(\vec{0}_V\) be the zero vector in \(V\) and \(\vec{0}_W\) be the zero vector in \(W\). Then,
            \begin{enumerate}
                \item \(L(\vec{0}_V)=L(\vec{0}_W)\).
                \begin{proof}
                    Consider \(L(\vec{0}_V)=L(0\vec{0}_V)=0L(\vec{0}_V)=\vec{0}_W\), as desired.
                \end{proof}
                \item \(L(-\vec{v})=-L(\vec{v})\).
                \begin{proof}
                    Consider \(L(-\vec{v})=L(-1\vec{v})=-L(\vec{v})\), as desired.
                \end{proof}
                \item \(L(c_1\vec{v}_1+\cdots+c_n\vec{v}_n)=c_1L(\vec{v}_1)+\cdots+c_nL(\vec{v}_n)\) for \(c_1,\ldots,c_n\in\mathbb{F}\) and \(\vec{v}_1,\ldots,\vec{v}_n\in V\) with \(n\geq2\).
                \begin{proof}
                    We proceed by induction. For the base case when \(n=2\), we have
                    \begin{align*}
                        L(c_1\vec{v}_1+c_2\vec{v}_2)&=L(c_1\vec{v}_1)+L(c_2\vec{v}_2) \\
                        &=c_1L(\vec{v}_1)+c_2L(\vec{v}_2).
                    \end{align*}
                    Then, suppose that for all \(n=k\), 
                    \begin{equation*}
                        L(c_1\vec{v}_1+\cdots+c_k\vec{v}_k)=c_1L(\vec{v}_1)+\cdots+c_kL(\vec{v}_k).
                    \end{equation*}
                    Then, we have
                    \begin{align*}
                        L(c_1\vec{v}_1+\cdots+c_k\vec{v}_k+c_{k+1}v_{k+1})&=c_1L(\vec{v}_1)+\cdots+c_kL(\vec{v}_k)+L(c_{k+1}\vec{v}_{k+1}) \\
                        &=c_1L(\vec{v}_1)+\cdots+c_kL(\vec{v}_k)+c_{k+1}L(\vec{v}_{k+1}),
                    \end{align*}
                    as desired.
                \end{proof}
            \end{enumerate}
            
        \end{theorem}
        \vphantom
        \\
        \\
        We remark that not every function between vector spaces is a linear transformation. To show that some function between vector spaces is not a linear transformation, we must show a counterexample of the conditions in Definition \ref{def:lineartransformation}.
        \pagebreak
        \\
        \\
        We now turn to compositions of linear transformations.
        \begin{theorem}{\Stop\,\,Compositions of Linear Transformations}{compositionslintrans}

            Let \(V_1\), \(V_2\), and \(V_3\) be vector spaces and \(L_1:V_1\to V_2\) and \(L_2:V_2\to V_3\) be linear transformations. Then, \((L_2\circ L_1):V_1\to V_3\) with \((L_2\circ L_1)(\vec{v})=L_2(L_1(\vec{v}))\) is a linear transformation for \(\vec{v}\in V_1\).
            \begin{proof}
                For \(\vec{v}_1,\vec{v}_2\in V_1\), we have
                \begin{align*}
                    (L_2\circ L_1)(\vec{v}_1+\vec{v}_2)&=L_2(L_1(\vec{v}_1+\vec{v}_2)) \\
                    &=L_2(L_1(\vec{v}_1)+L_1(\vec{v}_2)) \\
                    &=L_2(L_1(\vec{v}_1))+L_2(L_1(\vec{v}_2)) \\
                    &=(L_2\circ L_1)(\vec{v}_1)+(L_2\circ L_1)(\vec{v}_2).
                \end{align*}
                Then, for \(c\in\mathbb{F}\), we have
                \begin{align*}
                    (L_2\circ L_1)(c\vec{v}_1)&=L_2(L_1(c\vec{v}_1)) \\
                    &=L_2(cL_1(\vec{v}_1)) \\
                    &=cL_2(L_1(\vec{v}_1)) \\
                    &=c(L_2\circ L_1)(\vec{v}_1),
                \end{align*}
                as desired.
            \end{proof}            
        \end{theorem}
        \vphantom
        \\
        \\
        We now define a special case of linear transformations: linear operators.
        \begin{definition}{\Stop\,\,Linear Operators}{linearoperator}

            Let \(V\) be a vector space. A linear operator on \(V\) is a linear transformation whose domain and codomain are both \(V\).
            
        \end{definition}
        \vphantom
        \\
        \\
        Two special linear operators are the identity linear operator and the zero linear operator. Consider the following definitions.
        \begin{definition}{\Stop\,\,The Identity Linear Operator}{idlinop}

            Let \(V\) be a vector space. Then, the function \(i:V\to V,\vec{v}\mapsto\vec{v}\) is the identity linear operator.
            
        \end{definition}
        \begin{definition}{\Stop\,\,The Zero Linear Operator}{zerolinop}

            Let \(V\) be a vector space. Then, the function \(z:V\to V,\vec{v}\mapsto\vec{0}_V\) is the zero linear operator.
            
        \end{definition}
        \pagebreak
        \vphantom
        \\
        \\
        We end this section with a result about subspaces and linear transformations.
        \begin{theorem}{\Stop\,\,Linear Transformations and Subspaces}{lintranssubspc}

            Let \(L:V\to W\) be a linear transformation. Then,
            \begin{enumerate}
                \item If \(V'\) is a subspace of \(V\), \(L(V')=\{L(\vec{v}):\vec{v}\in V'\}\), the image of \(V'\) in \(W\), is a subspace of \(W\). That is, the range of \(L\) is a subspace of \(W\).
                \begin{proof}
                    We know \(\vec{0}_V\in V'\) since \(V'\) is a subspace. Then, \(\vec{0}_W\in L(V')\) since \(L(\vec{0}_V)=\vec{0}_W\). Next, we take \(\vec{w}_1,\vec{w}_2\in L(V')\). By definition, \(\vec{w}_1=L(\vec{v}_1)\) and \(\vec{w}_2=L(\vec{v}_2)\) for some \(\vec{v}_1,\vec{v}_2\in V'\). Then,
                    \begin{align*}
                        \vec{w}_1+\vec{w}_2&=L(\vec{v}_1)+L(\vec{v}_2) \\
                        &=L(\vec{v}_1+\vec{v}_2).
                    \end{align*}
                    Since \(V'\) is a subspace, \(\vec{v}_1+\vec{v}_2\in V'\). Then, \(\vec{w}_1+\vec{w}_2\) is the image of \(\vec{v}_1+\vec{v}_2\), so \((\vec{w}_1+\vec{w}_2)\in L(V')\). Now, for \(c\in\mathbb{F}\), we have \(c\vec{w}_1=cL(\vec{v}_1)=L(c\vec{v}_1)\). Since \(V'\) is a subspace, \(c\vec{v}_1\in V'\), so \(c\vec{w}_1\) is the image of \(c\vec{v}_1\), so \(c\vec{w}_1\in L(V')\).
                \end{proof}
                \item If \(W'\) is subspace of \(W\), then \(L^{-1}(W')=\{\vec{v}\in V:L(\vec{v})\in W'\}\), the pre-image of \(W'\) in \(V\), is a subspace of \(V\).
                \begin{proof}
                    We know \(\vec{0}_W\in W'\) since \(W'\) is a subspace. Then, \(\vec{0}_V\in L^{-1}(W')\) since \(L(\vec{0}_V)=\vec{0}_W\in W'\). Next, we take \(\vec{v}_1,\vec{v}_2\in L^{-1}(W')\), hence \(L(\vec{v}_1),L(\vec{v}_2)\in W'\). Since \(W'\) is a subspace,
                    \begin{equation*}
                        L(\vec{v}_1)+L(\vec{v}_2)\in W.
                    \end{equation*}
                    Because \(L\) is linear, \(L(\vec{v}_1)+L(\vec{v}_2)=L(\vec{v}_1+\vec{v}_2)\). Thus, \(L(\vec{v}_1+\vec{v}_2)\in W'\). That is, \((\vec{v}_1+\vec{v}_2)\in L^{-1}(W')\). Finally, for \(c\in\mathbb{F}\), we have \(L(c\vec{v}_1)=cL(\vec{v}_1)\). Since \(W'\) is a subspace and \(L(\vec{v}_1)\in W'\), we have that \(cL(\vec{v}_1)\in W'\). Thus, \(L(c\vec{v}_1)\in W'\) , so \(c\vec{v}_1\in L^{-1}(W')\).
                \end{proof}
            \end{enumerate}
            
        \end{theorem}
        \pagebreak
\section{Lecture 31: November 7, 2022}

    \subsection{Linear Transformations and Bases}

        We begin with an important theorem.
        \begin{theorem}{\Stop\,\,Linear Transformations and Bases}{lineartransformationsbases}

            Let \(B=\{\vec{v}_1,\ldots,\vec{v}_n\}\) be a basis for a vector space \(V\). Let \(W\) be a vector space with arbitrary \(\vec{w}_1,\ldots,\vec{w}_n\in W\). Then, there exists a unique linear transformation \(L:V\to W\) such that
            \begin{equation*}
                L(\vec{v}_1)=\vec{w}_1,\ldots,L(\vec{v}_n)=\vec{w}_n.
            \end{equation*}
            \begin{proof}
                Let \(L:V\to W\) be a linear transformation with
                \begin{equation*}
                    (c_1\vec{v}_1+\cdots+c_n\vec{v}_n)\mapsto(c_1\vec{w}_1+\cdots+c_n\vec{w}_n)
                \end{equation*}
                for scalars \(c_1,\ldots,c_n\). We note that \(L\) is well-defined because \(c_1,\ldots,c_n\) are unique. We will show \(L\) is linear by considering
                \begin{align*}
                    L(\vec{v}+\vec{v}')&=L(c_1\vec{v}_1+\cdots+c_n\vec{v}_n+c_1'\vec{v}_1+\cdots+c_n'\vec{v}_n) \\
                    &=L((c_1+c_1')\vec{v}_1+\cdots+(c_n+c_n')\vec{v}_n) \\
                    &=(c_1+c_1')\vec{w}_1+\cdots+(c_n+c_n')\vec{w}_n \\
                    &=c_1\vec{w}_1+c_1'\vec{w}_1+\cdots+c_n\vec{w}_n+c_2'\vec{w}_n \\
                    &=c_1\vec{w}_1+\cdots+c_n\vec{w}_n+c_1'\vec{w}_1+\cdots+c_n'\vec{w}_n \\
                    &=L(\vec{v})+L(\vec{v}').
                \end{align*}
                We now consider
                \begin{align*}
                    L(c\vec{v})&=L(cc_1\vec{v}_1+\cdots+cc_n\vec{v}_n) \\
                    &=cc_1\vec{w}_1+\cdots+cc_n\vec{w}_n \\
                    &=c(c_1\vec{w}_1+\cdots+c_n\vec{w}_n) \\
                    &=cL(\vec{v}).
                \end{align*}
                Now, we will show that \(L(\vec{v}_i)=\vec{w}_i\). We have that
                \begin{align*}
                    L(\vec{v}_i)&=L(0\vec{v}_1+\cdots+1\vec{v}_i+\cdots+0\vec{v}_n) \\
                    &=\vec{w}_i.
                \end{align*}
                We have shown existence, and now, will show uniqueness. Suppose \(R:V\to W\) is a linear transformation and \(R(\vec{v}_i)=\vec{w}_i\) for \(1\leq i\leq n, i\in \mathbb{N}\). We will show that \(R\) and \(L\) are equal. Let \(\vec{v}\in V\). Then,
                \begin{align*}
                    R(\vec{v})&=R(c_1\vec{v}_1+\cdots+c_n\vec{v}_n) \\
                    &=c_1R(\vec{v}_1)+\cdots+c_nR(\vec{v}_n) \\
                    &=c_1\vec{w}_1+\cdots+c_n\vec{w}_n \\
                    &=L(\vec{v}).
                \end{align*}
                Thus, \(L\) and \(R\) are the same transformation.
            \end{proof}
            
            
        \end{theorem}

\pagebreak

\section{Lecture 32: November 9, 2022}

    \subsection{The Matrix of a Linear Transformation}

        Consider the following theorem. 
        \begin{theorem}{\Stop\,\,Matrices and Linear Transformations}{matlintrans}
            
            Let \(V\) and \(W\) be nontrivial vector spaces. Let \(B=(\vec{v}_1,\ldots,\vec{v}_n)\) and \(C=(\vec{w}_1,\ldots,\vec{w}_m)\) be ordered bases for \(V\) and \(W\), respectively. Let \(L:V\to W\) be a linear transformation. Then, there exists a unique \(A_{BC}\in\mathcal{M}_{mn}\) such that
            \begin{equation*}
                A_{BC}[\vec{v}]_{B}=[L(\vec{v})]_{C}.
            \end{equation*}
            For \(1\leq i\leq n\), the \(i\)th column of \(A_{BC}\) is \([L(\vec{v}_i)]_C\).
            \begin{proof}
                Consider \(A_{BC}\in\mathcal{M}_{mn}\) with \(i\)th column \([L(\vec{v}_i)]_C\), for \(1\leq i\leq n\). We will first show that \(A_{BC}[\vec{v}]_B=[L(\vec{v})]_C\). Suppose that \([\vec{v}]_B=[c_1,\ldots,c_n]\). Then,
                \begin{equation*}
                    \vec{v}=c_1\vec{v}_1+\cdots+c_n\vec{v}_n.
                \end{equation*}
                Then, we have
                \begin{equation*}
                    L(\vec{v})=c_1L(\vec{v}_1)+\cdots+c_nL(\vec{v}_n).
                \end{equation*}
                Next,
                \begin{align*}
                    [L(\vec{v})]_C&=[c_1L(\vec{v}_1)+\cdots+c_nL(\vec{v}_n)]_C \\
                    &=c_1[L(\vec{v}_1)]_C+\cdots+c_n[L(\vec{v}_n)]_C \\
                    &=A_{BC}\begin{bmatrix} c_1 \\ \vdots \\ c_n \end{bmatrix} \\
                    &=A_{BC}[\vec{v}]_B.
                \end{align*}
                Note that the third step in the above transitive chain comes from the fact that the \(i\)th column of \(A_{BC}\) is \([L(\vec{v}_i)]_C\). For uniqueness, suppose \(H\in\mathcal{M}_{nn}\) such that \(H[\vec{v}]_B=[L(\vec{v})]_C\). We can show that \(H=A_{BC}\) if we can show that the \(i\)th column of \(H\) is the \(i\)th column of \(A_{BC}\), or equivalently, \([L(\vec{v}_i)]_C\). Consider \(\vec{v}_i\in B\). We know \([\vec{v}_i]_B=\vec{e}_i\). Then, the \(i\)th column of \(H\) is \(H\vec{e}_i=H[\vec{v}_i]_B=[L(\vec{v}_i)]_C\), which is also the \(i\)th column of \(A_{BC}\).
            \end{proof}
            
        \end{theorem}
        \vphantom
        \\
        \\
        As a remark, Theorem \ref{thm:matlintrans} shows that once we have picked ordered bases for \(V\) and \(W\), each linear transformation \(L:V\to W\) is equivalent to multiplication by a unique corresponding matrix. This matrix, \(A_{BC}\) is called the matrix of the linear transformation \(L\) with respect to the ordered bases \(B\) and \(C\). To compute \(A_{BC}\), we simply apply the linear transformation on each basis element \(\vec{v}_i\), and then express the result with respect to \(C\) to get the respective columns of \(A_{BC}\).
        \pagebreak
        \vphantom
        \\
        \\
        Consider the following example.
        \begin{example}{\Difficulty\,\Difficulty\,\,Finding a Matrix for a Linear Transformation 1}{findmat1}
            
            Consider
            \begin{equation*}
                L:\mathbb{R}^2\to\mathbb{R}^2, \begin{bmatrix} x \\ y \end{bmatrix} \mapsto \begin{bmatrix} x+y \\ x \end{bmatrix}.
            \end{equation*}
            Find the matrix for the linear transformation \(L\) with respect to the ordered bases \(B=([1,0],[0,1])\) and \(C=([1,0],[0,1])\).
            \\
            \\
            Consider
            \begin{align*}
                A_{BC}&=\begin{bmatrix}
                    L\left(\begin{bmatrix} 1 \\ 0 \end{bmatrix}\right) & L\left(\begin{bmatrix} 0 \\ 1 \end{bmatrix}\right)
                \end{bmatrix} \\
                &=\begin{bmatrix}
                    1 & 1 \\
                    1 & 0 \\
                \end{bmatrix}.
            \end{align*}
            Note that we did not have to explicitly coordinatize after finding the image of each element in \(B\) under \(L\) since we are using the standard basis for the codomain as well.
        \end{example}
        \begin{example}{\Difficulty\,\Difficulty\,\,Finding a Matrix for a Linear Transformation 2}{findmat2}
            
            Consider
            \begin{equation*}
                L:\mathbb{R}^2\to\mathbb{R}^2, \begin{bmatrix} x \\ y \end{bmatrix} \mapsto \begin{bmatrix} x+y \\ x \end{bmatrix}.
            \end{equation*}
            Find the matrix for the linear transformation \(L\) with respect to the ordered bases \(B=([1,0],[0,1])\) and \(C=([0,1],[1,0])\).
            \\
            \\
            Consider
            \begin{align*}
                A_{BC}&=\begin{bmatrix}
                    \left[L\left(\begin{bmatrix} 1 \\ 0 \end{bmatrix}\right)\right]_C & \left[L\left(\begin{bmatrix} 0 \\ 1 \end{bmatrix}\right)\right]_C
                \end{bmatrix} \\
                &=\begin{bmatrix}
                    \left[\begin{bmatrix} 1 \\ 1 \end{bmatrix}\right]_C & \left[\begin{bmatrix} 1 \\ 0 \end{bmatrix}\right]_C
                \end{bmatrix} \\
                &=\begin{bmatrix}
                    1 & 0 \\
                    1 & 1
                \end{bmatrix}.
            \end{align*}
            Note that \(L\) is the same linear transformation as the one given in Example \ref{exa:findmat1}; however, we did need to explicitly coordinatize, since we had a nonstandard basis.
        \end{example}
        \pagebreak
        \begin{example}{\Difficulty\,\Difficulty\,\,Finding a Matrix for a Linear Transformation 3}{findmat3}
            
            Consider
            \begin{equation*}
                L:\mathcal{P}_3\to\mathbb{R}^3, c_0+c_1x+c_2x^2+c_3x^3\mapsto[c_0+c_1,2c_2,c_3-c_0].
            \end{equation*}
            Find the matrix for the linear transformation \(L\) with respect to the ordered bases \(B=(x^3,x^2,x,1)\) for \(\mathcal{P}_{3}\) and \(C=([1,0,0],[0,1,0],[0,0,1])\) for \(\mathbb{R}^3\).
            \\
            \\
            By the definition of \(L\), we see that \(L(x^3)=[0,0,1]\), \(L(x^2)=[0,2,0]\), \(L(x)=[1,0,0]\), and \(L(1)=[1,0,-1]\). We need not perform any explicit coordinatization since we are using the standard basis for \(\mathbb{R}^3\), so,
            \begin{equation*}
                A_{BC}=\begin{bmatrix}
                    0 & 0 & 1 & 1 \\
                    0 & 2 & 0 & 0 \\
                    1 & 0 & 0 & -1
                \end{bmatrix}.
            \end{equation*}
            
        \end{example}
        \begin{example}{\Difficulty\,\Difficulty\,\,Finding a Matrix for a Linear Transformation 4}{findmat4}
            
            Consider
            \begin{equation*}
                L:\mathcal{P}_3\to\mathbb{R}^3, c_0+c_1x+c_2x^2+c_3x^3\mapsto[c_0+c_1,2c_2,c_3-c_0].
            \end{equation*}
            Find the matrix for the linear transformation \(L\) with respect to the ordered bases \(B=(x^3+x^2,x^2+x,x+1,1)\) for \(\mathcal{P}_{3}\) and \(C=([-2,1,-3],[1,-3,0],[3,-6,2])\) for \(\mathbb{R}^3\).
            \\
            \\
            By the definition of \(L\), we see that \(L(x^3+x^2)=[0,2,0]\), \(L(x^2+x)=[1,2,0]\), \(L(x+1)=[2,0,-1]\), and \(L(1)=[1,0,-1]\). Now, we must find \([0,2,1]_C\), \([1,2,0]_C\), \([2,0,-1]_C\), and \([1,0,-1]_C\). We consider
            \begin{equation*}
                \begin{bmatrix}
                    -2 & 1 & 3 & | & 0 & 1 & 2 & 1 \\
                    1 & -3 & -6 & | & 2 & 2 & 0 & 0 \\
                    -3 & 0 & 2 & | & 1 & 0 & -1 & -1
                \end{bmatrix}\underbrace{\to}_{\text{RREF}}\begin{bmatrix}
                    1 & 0 & 0 & | & -1 & -10 & -15 & -9 \\
                    0 & 1 & 0 & | & 1 & 26 & 41 & 25 \\
                    0 & 0 & 1 & | & -1 & -15 & -23 & -14
                \end{bmatrix}
            \end{equation*}
            Thus,
            \begin{equation*}
                A_{BC}=\begin{bmatrix}
                    -1 & -10 & -15 & -9 \\
                    1 & 26 & 41 & 25 \\
                    -1 & -15 & -23 & -14
                \end{bmatrix}.
            \end{equation*}
            
        \end{example}
        \pagebreak
        \vphantom
        \\
        \\
        Consider the following theorem.
        \begin{theorem}{\Stop\,\,Matrices for Linear Transformation, Considering Different Bases}{matricesconsdiffbases}
            Let \(V\) and \(W\) be nontrivial vector spaces with \(B\) and \(D\) be distinct ordered bases for \(V\) and \(C\) and \(E\) be distinct ordered bases for \(W\). Suppose \(L:V\to W\) is a linear transformation with matrix \(A_{BC}\). Then,
            \begin{equation*}
                A_{DE}=QA_{BC}P^{-1}
            \end{equation*}
            where \(P\) is the transition matrix from \(B\) to \(D\) and \(Q\) is the transition matrix from \(C\) to \(E\).
            \begin{proof}
                For \(\vec{v}\in V\), consider \(A_{BC}[\vec{v}]_B=[L(\vec{v})]_C\). First, we see that \(P^{-1}[\vec{v}]_D=[\vec{v}]_B\), so we may substitute to obtain \(A_{BC}P^{-1}[\vec{v}]_D=[L(\vec{v})]_C\). If we multiply by \(Q\) on both sides, on the left, we have
                \begin{align*}
                    QA_{BC}P^{-1}[\vec{v}]_D&=Q[L(\vec{v})]_C \\
                    &=[L(\vec{v})]_E.
                \end{align*}
                Since \(A_{DE}\) is the unique matrix such that \(A_{DE}[\vec{v}]_D=[L(\vec{v})]_E\), \(A_{DE}=QA_{BC}P^{-1}\).
            \end{proof}
        \end{theorem}
        \pagebreak
        \vphantom
        \\
        \\
        Consider the following example.
        \begin{example}{\Difficulty\,\Difficulty\,\Difficulty\,\,Finding a Matrix for a Linear Transformation 5}{findmat5}
            
            Consider
            \begin{equation*}
                L:\mathcal{P}_3\to\mathbb{R}^3, c_0+c_1x+c_2x^2+c_3x^3\mapsto[c_0+c_1,2c_2,c_3-c_0].
            \end{equation*}
            As seen in Example \ref{exa:findmat3}, the matrix for \(L\) using the standard bases for \(\mathcal{P_3}\) and \(\mathbb{R}^3\) was
            \begin{equation*}
                A_{BC}=\begin{bmatrix}
                    0 & 0 & 1 & 1 \\
                    0 & 2 & 0 & 0 \\
                    1 & 0 & 0 & -1
                \end{bmatrix},
            \end{equation*}
            where, again \(B=(x^3,x^2,x,1)\) and \(C=([1,0,0],[0,1,0],[0,0,1])\). We will now check our work in Example \ref{exa:findmat4}, where we saw the matrix for \(L\), with respect to the bases \(D=(x^3+x^2,x^2+x,x+1,1)\) and \(E=([-2,1,-3],[1,3,0],[3,-6,2])\) was
            \begin{equation*}
                A_{DE}=\begin{bmatrix}
                    -1 & -10 & -15 & -9 \\
                    1 & 26 & 41 & 25 \\
                    -1 & -15 & -23 & -14
                \end{bmatrix}.
            \end{equation*}
            To calculate the transition matrix \(P^{-1}\) from \(D\) to \(B\), we have
            \begin{equation*}
                \begin{bmatrix}
                    1 & 0 & 0 & 0 & | & 1 & 0 & 0 & 0 \\
                    0 & 1 & 0 & 0 & | & 1 & 1 & 0 & 0 \\
                    0 & 0 & 1 & 0 & | & 0 & 1 & 1 & 0 \\
                    0 & 0 & 0 & 1 & | & 0 & 0 & 1 & 1
                \end{bmatrix},
            \end{equation*}
            so
            \begin{equation*}
                P^{-1}=\begin{bmatrix}
                    1 & 0 & 0 & 0 \\
                    1 & 1 & 0 & 0 \\
                    0 & 1 & 1 & 0 \\
                    0 & 0 & 1 & 1
                \end{bmatrix}.
            \end{equation*}
            To calculate the transition matrix from \(C\) to \(E\), we have
            \begin{equation*}
                \begin{bmatrix}
                    -2 & 1 & 3 & | & 1 & 0 & 0 \\
                    1 & 3 & -6 & | & 0 & 1 & 0 \\
                    -3 & 0 & 2 & | & 0 & 0 & 1
                \end{bmatrix}\underbrace{\to}_\text{RREF}\begin{bmatrix}
                    1 & 0 & 0 & | & -6 & -2 & 3 \\
                    0 & 1 & 0 & | & 16 & 5 & -9 \\
                    0 & 0 & 1 & | & -9 & -3 & 5
                \end{bmatrix}.
            \end{equation*}
            so
            \begin{equation*}
                Q=\begin{bmatrix}
                    -6 & -2 & 3 \\
                    16 & 5 & -9 \\
                    -9 & -3 & 5
                \end{bmatrix}.
            \end{equation*}
            Then,
            \begin{align*}
                A_{DE}=QA_{BC}P^{-1}&=\begin{bmatrix}
                    -6 & -2 & 3 \\
                    16 & 5 & -9 \\
                    -9 & -3 & 5
                \end{bmatrix}\begin{bmatrix}
                    0 & 0 & 1 & 1 \\
                    0 & 2 & 0 & 0 \\
                    1 & 0 & 0 & -1
                \end{bmatrix}\begin{bmatrix}
                1 & 0 & 0 & 0 \\
                1 & 1 & 0 & 0 \\
                0 & 1 & 1 & 0 \\
                0 & 0 & 1 & 1
            \end{bmatrix} \\
            &=\begin{bmatrix}
                -1 & -10 & -15 & -9 \\
                1 & 26 & 41 & 25 \\
                -1 & -15 & -23 & -14
            \end{bmatrix},
            \end{align*}
            as desired.
            
        \end{example}
        \vphantom
        \\
        \\
        We now revisit and consider similar matrices.
        \begin{theorem}{\Stop\,\,Similar Matrices and Linear Operators}{simmatlinops}

            Let \(V\) be a vector space with bases \(C\) and \(D\). Let \(L:V\to V\) be a linear operator, so there exists some \(A_{CC}\) and \(A_{DD}\). Let \(P\) be the transition matrix from \(D\) to \(C\). Then, by Theorem \ref{thm:matricesconsdiffbases}, 
            \begin{equation*}
                A_{CC}=PA_{DD}P^{-1}
            \end{equation*}
            and
            \begin{equation*}
                A_{DD}=P^{-1}A_{CC}P.
            \end{equation*}
            Thus, \(A_{CC}\) and \(A_{DD}\) are similar. Generally, any two matrices for the same linear operator, with respect to different bases, are similar, by Definition \ref{def:similarity}.
             
        \end{theorem}
        \vphantom
        \\
        \\
        Finally, we present an important result about compositions of linear transformations and matrix multiplication.
        \begin{theorem}{\Stop\,\,The Matrix of a Composition of Linear Transformations}{matcomplintrans}

            Let \(V_1\), \(V_2\) and \(V_3\) be nontrivial finite dimensional vector spaces with ordered bases \(B\), \(C\), and \(D\), respectively. Let \(L_1:V_1\to V_2\) be a linear transformation with matrix \(A_{BC}\), and let \(L_2:V_2\to V_3\) be a linear transformation with matrix \(A_{CD}\). Then, the matrix, \(A_{BD}\), for the composite linear transformation \(L_2\circ L_1:V_1\to V_3\), with respect to bases \(B\) and \(D\), is \(A_{CD}A_{BC}\).
            
        \end{theorem}

\pagebreak

\section{Lecture 33: November 11, 2022}

    \subsection{Kernel and Range}

        We now define some important concepts.
        \begin{definition}{\Stop\,\,Kernel}{kernel}

            The kernel of a linear transformation \(L:V\to W\), is given by
            \begin{equation*}
                \ker(L)=\{\vec{v}\in V:L(\vec{v})=\vec{0}_W\}.
            \end{equation*}
            
        \end{definition}
        \begin{definition}{\Stop\,\,Range}{range}

            The range of a linear transformation \(L:V\to W\), is given by
            \begin{equation*}
                \range(L)=\{\vec{w}\in W:\exists\vec{v}\in V, L(\vec{v})=\vec{w}\}.
            \end{equation*}
            
        \end{definition}
        \pagebreak
        \vphantom
        \\
        \\
        Consider the following theorem.
        \begin{theorem}{\Stop\,\,Kernel and Range are Subspaces}{kerransubspc}

            Let \(L:V\to W\) be a linear transformation. Then, \(\ker(L)\) is a subspace of \(V\) and \(\range(L)\) is a subspace of \(W\).
            \begin{proof}
                For \(\ker(L)\), we have \(\vec{0}_V\in\ker(L)\) since \(L(\vec{0}_V)=\vec{0}_W\). Then, for \(\vec{v}_1,\vec{v}_2\in \ker(L)\), we have
                \begin{align*}
                    L(\vec{v}_1+\vec{v}_2)&=L(\vec{v}_1)+L(\vec{v}_2) \\
                    &=\vec{0}_W+\vec{0}_W \\
                    &=\vec{0}_W,
                \end{align*}
                so \(\vec{v}_1+\vec{v}_2\in\ker(L)\). Then, we also have
                \begin{align*}
                    L(c\vec{v}_1)&=cL(\vec{v}_1) \\
                    &=c\vec{0}_W \\
                    &=\vec{0}_W,
                \end{align*}
                for some \(c\in\mathbb{F}\). Thus, \(c\vec{v}_1\in\ker(L)\), as desired. For \(\range(L)\), we have \(\vec{0}_W\in\range(L)\) since \(L(\vec{0}_V)=\vec{0}_W\). Then, if \(\vec{w}_1,\vec{w}_2\in\range(L)\), \(L(\vec{v}_1)=\vec{w}_1\) and \(L(\vec{v}_2)=\vec{w}_2\) for some \(\vec{v}_1,\vec{v}_2\in V\). Then,
                \begin{align*}
                    L(\vec{v}_1+\vec{v}_2)&=L(\vec{v}_1)+L(\vec{v}_2) \\
                    &=\vec{w}_1+\vec{w}_2,
                \end{align*}
                meaning \(\vec{w}_1+\vec{w}_2\in \range(L)\). Then, we also have
                \begin{align*}
                    L(c\vec{v}_1)&=cL(\vec{v}_1) \\
                    &=c\vec{w}
                \end{align*}
                for some \(c\in\mathbb{F}\). Thus, \(c\vec{w}\in\range(L)\), as desired.
            \end{proof}
        
        \end{theorem}
        \pagebreak
        \vphantom
        \\
        \\
        Consider the linear transformation
        \begin{equation*}
            L_A:\mathbb{R}^n\to\mathbb{R}^m, \vec{v}\mapsto A\vec{v}
        \end{equation*}
        for \(A\in\mathcal{A}_{mn}\). Now, we want to find \(\ker(L_A)\) and \(\range(L_A)\). Consider the following theorems.
        \begin{theorem}{\Stop\,\,Finding the Kernel of a Linear Transformation}{findker}

            Let \(A\in\mathcal{M}_{mn}\). Let \(L_A\) be a linear transformation with
            \begin{equation*}
                L_A:\mathbb{R}^n\to\mathbb{R}^m, \vec{v}\mapsto A\vec{v}.
            \end{equation*}
            We find a basis of \(\ker(L)\) by finding particular solutions to \([A|\vec{0}]\). We find each particular solution \(\vec{v}_i\) by setting the \(i\)th free variable in the system to \(1\) and the other free variables to \(0\). We end up with the set \(\{\vec{v}_1,\ldots,\vec{v}_k\}\) as a basis for \(\ker(L)\). Then, \(\ker(L)=\Span(\{\vec{v}_1,\ldots,\vec{v}_k\})\). As a remark, \(\dim(\ker(L))\) is the number of free variables in the homogeneous solution set.
        \end{theorem}
        \begin{theorem}{\Stop\,\,Finding the Range of a Linear Transformation}{findrange}

            Let \(A\in\mathcal{M}_{mn}\). Let \(L_A\) be a linear transformation with
            \begin{equation*}
                L_A:\mathbb{R}^n\to\mathbb{R}^m, \vec{v}\mapsto A\vec{v}.
            \end{equation*}
            To find \(\range(L)\), we need to row reduce \(A\). The columns in \(A\) corresponding to the pivot columns after row reduction is complete form a basis of \(\range(L)\). This is because the span of the columns of \(A\) is \(\range(L)\), and we must then form a basis by keeping only the pivot columns. As a remark, \(\dim(\range(L))=\rank A\), the number of pivot columns.

        \end{theorem}
        \pagebreak
        \vphantom
        \\
        \\
        Consider the following examples.
        \begin{example}{\Difficulty\,\Difficulty\,\,Find Kernel}{findker}

            Let \(L:\mathbb{R}^5\to\mathbb{R}^4,\vec{v}\mapsto A\vec{v}\), where
            \begin{equation*}
                A=\begin{bmatrix}
                    8 & 4 & 16 & 32 & 0 \\
                    4 & 2 & 10 & 22 & -4 \\
                    -2 & -1 & -5 & -11 & 7 \\
                    6 & 3 & 15 & 33 & -7
                \end{bmatrix}.
            \end{equation*}
            Find \(\ker(L)\).
            \\
            \\
            To find \(\ker(L)\), we solve \([A|\vec{0}]\) by row reduction to obtain
            \begin{equation*}
                \begin{bmatrix}
                    1 & \frac{1}{2} & 0 & -2 & 0 & | & 0 \\
                    0 & 0 & 1 & 3 & 0 & | & 0 \\
                    0 & 0 & 0 & 0 & 1 & | & 0 \\
                    0 & 0 & 0 & 0 & 0 & | & 0 
                \end{bmatrix}.
            \end{equation*}
            We see that there are free variables in the second and fourth columns. The solution set to the system is
            \begin{equation*}
                \left\{\left[-\frac{1}{2}c_1+2c_2,c_1,-3c_2,c_2,0\right]:c_1,c_2\in\mathbb{R}\right\}.
            \end{equation*}
            If \(c_1=1\) and \(c_2=0\), we have the particular solution \(\vec{v}_1=[-\frac{1}{2},1,0,0,0]\). If \(c_1=0\) and \(c_2=1\), we have \(\vec{v}_2=[2,0,-3,1,0]\). Thus,
            \begin{equation*}
                \ker(L)=\left\{c_1\left[-\frac{1}{2},1,0,0,0\right]+c_2[2,0,-3,1,0]:c_1,c_2\in\mathbb{R}\right\}.
            \end{equation*}
            It is worth noting that the initial solution set was indeed also \(\ker(L)\), but it is nice to see \(\ker(L)\) as the span of a basis of \(\ker(L)\). We will further simplify to obtain
            \begin{equation*}
                \ker(L)=\left\{c_1\left[-1,2,0,0,0\right]+c_2[2,0,-3,1,0]:c_1,c_2\in\mathbb{R}\right\}.
            \end{equation*}

        \end{example}
        \pagebreak
        \begin{example}{\Difficulty\,\Difficulty\,\,Find Range}{findran}

            Let \(L:\mathbb{R}^5\to\mathbb{R}^4,\vec{v}\mapsto A\vec{v}\), where
            \begin{equation*}
                A=\begin{bmatrix}
                    8 & 4 & 16 & 32 & 0 \\
                    4 & 2 & 10 & 22 & -4 \\
                    -2 & -1 & -5 & -11 & 7 \\
                    6 & 3 & 15 & 33 & -7
                \end{bmatrix}.
            \end{equation*}
            Find \(\range(L)\).
            \\
            \\
            To find \(\range(L)\), row reduce \(A\) to obtain
            \begin{equation*}
                \begin{bmatrix}
                    1 & \frac{1}{2} & 0 & -2 & 0 \\
                    0 & 0 & 1 & 3 & 0 \\
                    0 & 0 & 0 & 0 & 1 \\
                    0 & 0 & 0 & 0 & 0 
                \end{bmatrix}.
            \end{equation*}
            We see that there are pivots in the first, third, and fifth columns. Thus, 
            \begin{equation*}
                \range(L)=\{c_1[8,4,-2,6]+c_2[16,10,-5,15]+c_3[0,-4,7,-7]:c_1,c_2,c_3\in\mathbb{R}\}.
            \end{equation*}
            
        \end{example}
        \pagebreak
        \vphantom
        \\
        \\
        The next theorems combine the results of Theorem \ref{thm:findker} and Theorem \ref{thm:findrange} to find 
        \begin{equation*}
            \dim(\ker(L))+\dim(\range(L)).
        \end{equation*}
        But first, consider the following definitions.
        \begin{definition}{\Stop\,\,Nullity of a Linear Transformation}{nullity}

            Suppose \(V\) and \(W\) are finite dimensional vector spaces and \(L:V\to W\) is a linear transformation. Then,
            \begin{equation*}
                \nullity(L)=\dim(\ker(L)).
            \end{equation*}
            
        \end{definition}
        \begin{definition}{\Stop\,\,Rank of a Linear Transformation}{ranklintrans}

            Suppose \(V\) and \(W\) are finite dimensional vector spaces and \(L:V\to W\) is a linear transformation. Then,
            \begin{equation*}
                \rank(L)=\dim(\range(L)).
            \end{equation*}
            
        \end{definition}
        \vphantom
        \\
        \\
        Now, consider the following theorems.
        \begin{theorem}{\Stop\,\,The Dimension Theorem (The Rank-Nullity Theorem), in \(\mathbb{R}^n\)}{dimthmrn}

            Let \(A\in\mathcal{M}_{mn}\). Let \(L_A\) be a linear transformation with
            \begin{equation*}
                L_A:\mathbb{R}^n\to\mathbb{R}^m, \vec{v}\mapsto A\vec{v}.
            \end{equation*}
            Then,
            \begin{enumerate}
                \item \(\dim(\range(L))=\rank A\).
                \item \(\dim(\ker(L))=n-\rank A\).
                \item \(\dim(\ker(L))+\dim(\range(L))=n\).
            \end{enumerate}
            These results are verified by Theorems \ref{thm:findker} and \ref{thm:findrange}.
            
        \end{theorem}
        \begin{theorem*}{\Stop\,\,The Dimension Theorem (The Rank-Nullity Theorem)}

            Suppose \(V\) and \(W\) are finite dimensional vector spaces and \(L:V\to W\) is a linear transformation. Then,
            \begin{equation*}
                \dim(\ker(L))+\dim(\range(L))=\dim V.
            \end{equation*}
            
        \end{theorem*}
        \vphantom
        \\
        \\
        We will postpone the proof of the above theorem to a later section; hence, we have omitted the reference number.

    \pagebreak

    \subsection{Injections, Surjections, Bijections, and Isomorphisms}

        We state two theorems about if linear transformations are injective or surjective.
        \begin{theorem}{\Stop\,\,Determining Injectivity and Surjectivity}{detinjsurj}

            Suppose \(V\) and \(W\) are finite dimensional vector spaces and \(L:V\to W\) is a linear transformation. Then,
            \begin{enumerate}
                \item The linear transformation \(L\) is injective if and only if \(\ker(L)=\{\vec{0}_V\}\).
                \begin{proof}
                    Suppose \(L\) is injective and let \(\vec{v}\in\ker(L)\). Now, \(L(\vec{v})=\vec{0}_W\). Similarly, \(L(\vec{0}_V)=\vec{0}_W\), and since \(L\) is injective, \(\vec{v}=\vec{0}_V\). Now, we suppose \(\ker(L)=\{\vec{0}_V\}\). We must show \(L\) is injective. Let \(\vec{v}_1,\vec{v}_2\in V\) with \(L(\vec{v}_1)=L(\vec{v}_2)\). We wish to show \(\vec{v}_1=\vec{v}_2\). Now, we have \(L(\vec{v}_1)-L(\vec{v}_2)=\vec{0}_W\), implying that \(L(\vec{v}_1-\vec{v}_2)=\vec{0}_W\). Thus, \(\vec{v}_1-\vec{v}_2\in\ker(L)\). Since \(\ker(L)=\{\vec{0}_V\}\), \(\vec{v}_1-\vec{v}_2=\vec{0}_V\), and so, \(\vec{v}_1=\vec{v}_2\), as desired.
                \end{proof}
                \item The linear transformation \(L\) is surjective if and only if \(\dim(\range(L))=\dim W\).
                \begin{proof}
                    By definition, \(L\) is surjective if and only if \(\range(L)=W\). Then, since \(\range(L)\) is a subspace of \(W\), \(\range(L)=W\) if and only if \(\dim(\range(L))=\dim W\) by Theorem \ref{thm:dimsubspc}.
                \end{proof}
            \end{enumerate}
            
        \end{theorem}
        \begin{theorem}{\Stop\,\,Determining Injectivity and Surjectivity With Equivalent Dimensions}{detinjsurjequivdim}
            
            Suppose \(V\) and \(W\) are finite dimensional vector spaces with \(\dim V=\dim W\). Let \(L:V\to W\) be a linear transformation. Then, \(L\) is injective if and only if \(L\) is surjective.
            \begin{proof}
                We know \(L\) is injective if and only if \(\ker(L)=\{\vec{0}_V\}\), meaning \(\dim (\ker(L))=0\). By the dimension theorem, \(\dim V=\dim(\range(L))+\dim(\ker(L))=\dim(\range(L))\). Since \(\dim V=\dim W\), \(\dim W=\dim(\range(L))\), meaning \(L\) is surjective, by definition. Conversely, if \(L\) is surjective, \(\dim(\ker(L))=0\), meaning that \(\ker(L)=\{\vec{0}_V\}\), which is equivalent to \(L\) being injective.
            \end{proof}

        \end{theorem}
        \pagebreak
        \vphantom
        \\
        \\
        Consider the following theorem about linear independence and spanning, with regards to linear transformations.
        \begin{theorem}{\Stop\,\,Injectivity Implies Linear Independence, Surjectivity Implies Spanning}{injlinindepsurjspan}

            Suppose \(V\) and \(W\) are vector spaces and \(L:V\to W\) is a linear transformation. Then,
            \begin{enumerate}
                \item If \(L\) is injective, and \(T\) is a linearly independent subset of \(V\), \(L(T)\) is linearly independent in \(W\).
                \begin{proof}
                    Suppose that \(L\) is injective, and \(T\) is a linearly independent subset of \(V\). We wish to show that all finite subsets of \(L(T)\) are linearly independent. Suppose \(\{L(\vec{v}_1),\ldots,L(\vec{v}_n)\}\subseteq L(T)\) for \(\vec{v}_1,\ldots,\vec{v}_n\in T\). Suppose
                    \begin{equation*}
                        c_1L(\vec{v}_1)+\cdots+c_nL(\vec{v}_n)=\vec{0}_W,
                    \end{equation*}
                    which implies 
                    \begin{equation*}
                        L(c_1\vec{v}_1+\cdots+c_n\vec{v}_n)=\vec{0}_W
                    \end{equation*}
                    for scalars \(c_1,\ldots,c_n\). Thus, \((c_1\vec{v}_1+\cdots+c_n\vec{v}_n)\in\ker(L)\). Since \(\ker(L)=\{\vec{0}_V\}\) because \(L\) is injective,
                    \begin{equation*}
                        c_1\vec{v}_1+\cdots+c_n\vec{v}_n=\vec{0}_V.
                    \end{equation*}
                    Since \(\{\vec{v}_1,\ldots,\vec{v}_n\}\subseteq T\) are linearly independent, \(c_1=\cdots=c_n=0\). Thus, \(\{L(\vec{v}_1),\ldots,L(\vec{v}_n)\}\) is linearly independent as well, meaning that \(L(T)\) is linearly independent, as desired.
                \end{proof}
                \item If \(L\) is surjective and \(S\) spans \(V\), \(L(S)\) spans \(W\).
                \begin{proof}
                    Suppose that \(L\) is surjective and \(S\) spans \(V\). We wish to show that all \(\vec{w}\in W\) can be written as a linear combination of vectors in \(L(T)\). Since \(L\) is surjective, there exists \(\vec{v}\in V\) with \(L(\vec{v})=\vec{w}\). Since \(S\) spans \(V\), we have
                    \begin{equation*}
                        \vec{v}=c_1\vec{v}_1+\cdots+c_n\vec{v}_n
                    \end{equation*}
                    for \(\vec{v}_1,\ldots,\vec{v}_n\in S\) and scalars \(c_1,\ldots,c_n\). Then, 
                    \begin{align*}
                        \vec{w}=L(\vec{v})&=L(c_1\vec{v}_1+\cdots+c_n\vec{v}_n) \\
                        &=c_1L(\vec{v}_1)+\cdots+c_nL(\vec{v}_n).
                    \end{align*}
                    We have written arbitrary \(\vec{w}\in W\) as a linear combination of elements in \(L(S)\), so \(L(S)\) spans \(W\), as desired.
                \end{proof}
            \end{enumerate}
            
        \end{theorem}
        \vphantom
        \\
        \\
        Consider the following definition.
        \begin{definition}{\Stop\,\,Isomorphisms}{isomorphisms}

            Suppose \(V\) and \(W\) are finite dimensional vector spaces and \(L:V\to W\) is a linear transformation; \(L\) an isomorphism from \(V\) to \(W\) if and only if \(L\) is both injective and surjective, or bijective.
            
        \end{definition}
        \begin{definition}{\Stop\,\,Invertible Linear Transformations}{invtrans}

            Suppose \(V\) and \(W\) are finite dimensional vector spaces and \(L:V\to W\) is a linear transformation. Then, \(L\) is an invertible linear transformation if and only if there exists some function \(M:W\to V\) such that
            \begin{equation*}
                (M\circ L)(\vec{v})=\vec{v}
            \end{equation*}
            for all \(\vec{v}\in V\) and
            \begin{equation*}
                (L\circ M)(\vec{w})=\vec{w}
            \end{equation*}
            for all \(\vec{w}\in W\).
            
        \end{definition}
        \begin{theorem}{\Stop\,\,Isomorphism If And Only If Invertible}{isoinv}
            
            Let \(L:V\to W\) be a linear transformation. Then, \(L\) is an isomorphism if and only if \(L\) is an invertible linear transformation. If \(L\) is invertible, \(L^{-1}\) is also a linear transformation.
            \begin{proof}
                The first part of this theorem follows from Theorem \ref{thm:existinv}, as by definition, \(L\) is an isomorphism if and only if \(L\) is bijective. Now, we just seek to show that \(L^{-1}\) is a linear transformation. First, consider \(\vec{w}_1,\vec{w}_2\in W\). Since \(L\) is surjective, we have \(\vec{w}_1=L(\vec{v}_1)\) and \(\vec{w}_2=L(\vec{v}_2)\) for \(\vec{v}_1,\vec{v}_2\in V\). We have
                \begin{align*}
                    L^{-1}(\vec{w}_1+\vec{w}_2)&=L^{-1}(L(\vec{v}_1)+L(\vec{v}_2)) \\
                    &=L^{-1}(L(\vec{v}_1+\vec{v}_2)) \\
                    &=\vec{v}_1+\vec{v}_2 \\
                    &=L^{-1}(\vec{w}_1)+L^{-1}(\vec{w}_2).
                \end{align*}
                Now, for some \(c\in\mathbb{F}\), we have
                \begin{align*}
                    L^{-1}(c\vec{w}_1)&=L^{-1}(cL(\vec{v}_1)) \\
                    &=cL^{-1}(L(\vec{v}_1)) \\
                    &=c\vec{v}_1 \\
                    &=cL^{-1}(\vec{w}_1),
                \end{align*}
                as desired. Note that for the last step of both transitive chains, we used the fact that \(L\) is injective.
            \end{proof}

        \end{theorem}
        \vphantom
        \\
        \\
        The following theorem allows us to determine whether a linear transformation between finite dimensional vector spaces is an isomorphism, and if so, how to find the inverse.
        \begin{theorem}{\Stop\,\,Finding an Inverse, if it Exists}{findinv}

            Suppose \(V\) and \(W\) are nontrivial finite dimensional vector spaces with ordered bases \(B\) and \(C\), respectively. Let \(L:V\to W\) be a linear transformation. Then, \(L\) is an isomorphism if and only if the matrix representation \(A_{BC}\) associated to \(L\) is nonsingular. If \(L\) is indeed an isomorphism, the matrix \(A_{CB}\) for \(L^{-1}\) is \(A_{BC}^{-1}\).
            
        \end{theorem}
        \pagebreak
        \vphantom
        \\
        \\
        Consider the following examples.
        \begin{example}{\Difficulty\,\Difficulty\,\,Find Inverse 1}{findinv1}

            Consider
            \begin{equation*}
                L:\mathbb{R}^2\to\mathbb{R}^2,\begin{bmatrix} x \\ y \end{bmatrix}\mapsto\begin{bmatrix} 3x+y \\ x+y \end{bmatrix}.
            \end{equation*}
            Is \(L\) invertible? If so, find its inverse.
            \\
            \\
            Let \(B\) be an ordered basis of \(\mathbb{R}^2\) with
            \begin{equation*}
                B=\left(\begin{bmatrix} 1 \\ 0 \end{bmatrix}, \begin{bmatrix} 0 \\ 1 \end{bmatrix} \right).
            \end{equation*}
            Then,
            \begin{equation*}
                A_{BB}=\begin{bmatrix} 3 & 1 \\ 1 & 1 \end{bmatrix}.
            \end{equation*}
            Since this matrix is invertible, as \(\det A_{BB}\neq 0\), the inverse is given by
            \begin{align*}
                L^{-1}:\mathbb{R}^2\to\mathbb{R}^2&,\begin{bmatrix} x \\ y \end{bmatrix}\mapsto A_{BB}^{-1}\begin{bmatrix} x \\ y \end{bmatrix} \\
                &,\begin{bmatrix} x \\ y \end{bmatrix}\mapsto \begin{bmatrix} \frac{1}{2} & -\frac{1}{2} \\ -\frac{1}{2} & \frac{3}{2} \end{bmatrix}\begin{bmatrix} x \\ y \end{bmatrix}.
            \end{align*}
        \end{example}
        \begin{example}{\Difficulty\,\Difficulty\,\,Find Inverse 2}{findinv2}

            Consider
            \begin{equation*}
                L:\mathbb{R}^3\to\mathbb{R}^3,\begin{bmatrix} x \\ y \\ z \end{bmatrix}\mapsto A\begin{bmatrix} x \\ y \\ z \end{bmatrix}.
            \end{equation*}
            where
            \begin{equation*}
                A=\begin{bmatrix}
                    1 & 0 & 3 \\
                    0 & 1 & 3 \\
                    0 & 0 & 1
                \end{bmatrix}.
            \end{equation*}
            Is \(L\) invertible? If so, find its inverse.
            \\
            \\
            Since this matrix is invertible, as \(\det A_{BB}\neq 0\), the inverse is given by
            \begin{align*}
                L^{-1}:\mathbb{R}^2\to\mathbb{R}^2&,\begin{bmatrix} x \\ y \\ z \end{bmatrix}\mapsto A^{-1}\begin{bmatrix} x \\ y  \\ z \end{bmatrix} \\
                &,\begin{bmatrix} x \\ y \\ z \end{bmatrix}\mapsto \begin{bmatrix} 1 & 0 & -3 \\ 0 & 1 & -3 \\ 0 & 0 & 1 \end{bmatrix}\begin{bmatrix} x \\ y \\ z \end{bmatrix}.
            \end{align*}
        \end{example}
        \pagebreak
        \vphantom
        \\
        \\
        We end with an important, yet unsurprising, theorem.
        \begin{theorem}{\Stop\,\,Isomorphisms Preserve Linear Independence and Span}{isopreslinindepspan}
            
            Let \(V\) and \(W\) be vector spaces, and let \(L:V\to W\) be an isomorphism.
            \begin{enumerate}
                \item If \(T\) is a linearly independent subset of \(V\), \(L(T)\) is linearly independent in \(W\).
                \begin{proof}
                    Because \(L\) is an isomorphism, \(L\) is injective. Therefore, \(L(T)\) is linearly independent in \(W\) by the first part of Theorem \ref{thm:injlinindepsurjspan}.
                \end{proof}
                \item If \(S\) spans \(V\), \(L(S)\) spans \(W\).
                \begin{proof}
                    Because \(L\) is an isomorphism, \(L\) is surjective. Therefore, \(L(S)\) spans \(W\) by the second part of Theorem \ref{thm:injlinindepsurjspan}.
                \end{proof}
                \item If \(B\) is a basis of \(V\), \(L(B)\) is a basis for \(W\).
                \begin{proof}
                    If \(B\) is a basis for \(V\), \(B\) is linearly independent, so \(L(B)\) is linearly independent. We also have that \(\Span(B)=V\), so \(\Span(L(B))=W\). Since \(L(B)\) is linearly independent and spans \(W\), \(L(B)\) is a basis for \(W\).
                \end{proof}
            \end{enumerate}

        \end{theorem}

        \pagebreak

\section{Lecture 34, November 18, 2022}

    \subsection{Isomorphic Vector Spaces}

        We will now define the notion of equivalence of vector spaces.
        \begin{definition}{\Stop\,\,Isomorphic Vector Spaces}{isovec}

            Suppose \(V\) and \(W\) are vector spaces. Then, \(V\) is isomorphic to \(W\), that is, \(V\cong W\), if and only if there exists some linear transformation \(L:V\to W\) that is an isomorphism.
            
        \end{definition}
        \begin{theorem}{\Stop\,\,\(\cong\) is an Equivalence Relation}{isoequivrel}

            Suppose \(V\) and \(W\) are vector spaces. Then, \(\cong\) is an equivalence relation.
            \begin{proof}
                We must show that \(\cong\) is reflexive, symmetric and transitive. Consider the following.
                \begin{enumerate}
                    \item We wish to show \(V\cong V\). Consider the linear transformation \(i:V\to V,\vec{v}\mapsto\vec{v}\), the identity linear operator, defined by Definition \ref{def:idlinop}. We wish to show that \(i:V\to V\) is an isomorphism. We need only show that \(i\) is injective. Since \(i(\vec{0}_V)=\vec{0}_V\), \(\{\vec{0}_V\}\subseteq\ker(i)\).
                    \item Suppose \(V\cong W\) via \(L:V\to W\), and we must show that \(L^{-1}:W\to V\) is an isomorphism from \(W\) to \(V\).
                    \item Suppose \(V_1 \cong V_2\) via \(L_1:V_1\to V_2\) and \(V_2\cong V_3\) via \(L_2:V_2\to V_3\). Show that \(L_2\circ L_1:V_1\to V_3\) is an isomorphism.
                \end{enumerate}
                \DOTHISLATER
            \end{proof}

        \end{theorem}
        \pagebreak
        \vphantom
        \\
        \\
        We will now restate the Dimension Theorem, or the Rank-Nullity Theorem, with an accompanying proof.
        \begin{theorem}{\Stop\,\,The Dimension Theorem (The Rank-Nullity Theorem)}{dimensionthm}

            Suppose \(V\) and \(W\) are finite dimensional vector spaces and \(L:V\to W\) is a linear transformation. Then,
            \begin{equation*}
                \dim(\ker(L))+\dim(\range(L))=\dim V.
            \end{equation*}
            \begin{proof}
                Let \(B_{\ker(L)}=\{\vec{u}_1,\ldots,\vec{u}_m\}\) be a basis for \(\ker(L)\). We can extend \(B_{\ker(L)}\) to form a basis for \(V\), \(B_V\), with \(B_V=\{\vec{u}_1,\ldots,\vec{u}_m,\vec{v}_1,\ldots,\vec{v}_n\}\). Note that \(\dim(\ker(L))=m\) and \(\dim V=m+n\). Then, for some \(\vec{v}\in V\), we have
                \begin{equation*}
                    \vec{v}=c_1\vec{u}_1+\cdots+c_m\vec{u}_m+c_{m+1}\vec{v}_1+\cdots+c_n\vec{v}_n
                \end{equation*}
                for \(c_1,\ldots,c_m,c_{m+1},\ldots,c_n\in\mathbb{F}\). If we apply \(L\) to both sides, we have
                \begin{align*}
                    L(\vec{v})&=L(c_1\vec{u}_1+\cdots+c_m\vec{u}_m+c_{m+1}\vec{v}_1+\cdots+c_n\vec{v}_n) \\
                    &=c_1L(\vec{u}_1)+\cdots+c_mL(\vec{u}_m)+c_{m+1}L(\vec{v}_1)+\cdots+c_nL(\vec{v}_n).
                \end{align*}
                Since \(B_{\ker(L)}\) is a basis of \(\ker(L)\), \(L(\vec{u}_1)=\cdots=L(\vec{u_m})=\vec{0}_W\) and we have
                \begin{equation*}
                    L(\vec{v})=c_{m+1}L(\vec{v}_1)+\cdots+c_nL(\vec{v}_n).
                \end{equation*}
                Therefore, since we have written arbitrary \(L(\vec{v})\) as a linear combination of \(L(\vec{v}_1),\ldots,L(\vec{v}_n)\), we have 
                \begin{equation*}
                    \range(L)=\Span(\{L(\vec{v}_1),\ldots,L(\vec{v}_n)\}).
                \end{equation*}
                Consider
                \begin{align*}
                    \vec{0}_W&=b_1L(\vec{v}_1)+\cdots+b_nL(\vec{v}_n) \\
                    &=L(b_1\vec{v}_1+\cdots+b_n\vec{v}_n)
                \end{align*}
                for \(b_1,\ldots,b_n\in\mathbb{F}\). We see that \(b_1\vec{v}_1+\cdots+b_n\vec{v}_n\in\ker(L)\), so
                \begin{equation*}
                    b_1\vec{v}_1+\cdots+b_n\vec{v}_n=d_1\vec{u}_1+\cdots+d_m\vec{u}_m.
                \end{equation*}
                If we subtract the right hand side from both sides, we have
                \begin{equation*}
                    b_1\vec{v}_1+\cdots+b_n\vec{v}_n+(-d_1\vec{u}_1)+\cdots+(-d_m)\vec{u}_m=\vec{0}_V
                \end{equation*}
                Since \(B_V\) is a basis for \(V\), \(B_V\) is linearly independent, so \(b_1=\cdots=b_n=d_1=\cdots=d_m=0\). Since \(b_1=\cdots=b_n=0\), \(B_{\range(L)}:=\{L(\vec{v}_1),\ldots,L(\vec{v}_n)\}\) is linearly independent. Since \(B_{\range(L)}\) both spans \(\range(L)\) and is linearly independent, \(B_{\range(L)}\) is a basis for \(\range(L)\). We see that \(\dim(\range(L))=n\), so
                \begin{align*}
                    \dim V&=m+n \\
                    &=\dim(\ker(L))+\dim(\range(L)),
                \end{align*}
                as desired.

            \end{proof}

        \end{theorem}
        \vphantom
        \\
        \\
        Consider the following important theorems.
        \begin{theorem}{\Stop\,\,Isomorphism Implies Equivalent Dimension}{equivdim}

            Suppose \(V\) and \(W\) are finite dimensional vector spaces. Then, \(V\cong W\) if and only if \(\dim V=\dim W\).
            \begin{proof}
                Suppose \(V\cong W\). Then, there exists some linear transformation \(L:V\to W\) where \(L\) is an isomorphism. Therefore, we have that \(\dim(\range(L))=\dim W\). We also have \(\ker(L)=\{\vec{0}_V\}\), so \(\dim(\ker(L))=0\). Therefore, by Theorem \ref{thm:dimensionthm},
                \begin{align*}
                    \dim V&=\dim(\ker(L))+\dim(\range(L)) \\
                    &=0+\dim W \\
                    &=\dim W.
                \end{align*}
                Now, suppose \(\dim V=\dim W\). Let \(B_V=\{\vec{v}_1,\ldots,\vec{v}_n\}\) be a basis for \(V\) and \(B_W=\{\vec{w}_1,\ldots,\vec{w}_n\}\) be a basis for \(W\). Let 
                \begin{equation*}
                    L:V\to W, c_1\vec{v}_1+\cdots+c_n\vec{v}_n\mapsto c_1\vec{w}_1+\cdots+c_n\vec{w}_n
                \end{equation*}
                for \(c_1,\ldots,c_n\in\mathbb{F}\). Consider arbitrary \(\vec{w}\in W\). We wish to find some \(\vec{v}\in V\) such that \(L(\vec{v})=\vec{w}\) to show that \(L\) is surjective. Under the supposition \(\dim V=\dim W\), \(L\) is an isomorphism if and only if \(L\) is surjective by Theorem \ref{thm:detinjsurjequivdim}. We have
                \begin{equation*}
                    \vec{w}=c_1\vec{w}_1+\cdots+c_n\vec{w}_n.
                \end{equation*}
                By the definition of \(L\), we know
                \begin{equation*}
                    L(c_1\vec{v}_1+\cdots+c_n\vec{v}_n)=c_1\vec{w}_1+\cdots+c_n\vec{w}_n,
                \end{equation*}
                so \(L\) is surjective and is an isomorphism, so \(V\cong W\), as desired.
            \end{proof}
            
        \end{theorem}
        \begin{theorem}{\Stop\,\,All \(n\)-Dimensional Vector Spaces are Isomorphic to \(\mathbb{R}^n\)}{allisorn}
            
            Suppose \(V\) is a finite dimensional vector space with \(\dim V=n\). Then, \(V\cong \mathbb{R}^n\).
            \begin{proof}
                We have that \(\dim V=n\), and we know that \(\dim\mathbb{R}^n=n\), so by Theorem \ref{thm:equivdim}, \(V\cong\mathbb{R}^n\).
            \end{proof}

        \end{theorem}

      %  \begin{example}{\Stop\,\,Find Inverse 1}{findinv1}
%
      %      Consider
      %      \begin{equation*}
       %         L:\mathbb{R}^2\to\mathbb{R}^2,\begin{bmatrix} x \\ y \end{bmatrix}\mapsto\begin{bmatrix} 3x+y \\ x+y \end{bmatrix}.
       %     \end{equation*}
       %     Is \(L\) invertible, if so, find its inverse.
       %     \\
       %     \\
        %    Let \(B\) be an ordered basis of \(\mathbb{R}^2\) with
       %     \begin{equation*}
              % B=\left(\begin{bmatrix} 1 \\ 0 \end{bmatrix}, \begin{bmatrix} 0 \\ 1 \end{bmatrix} \right).
         %   \end{equation*}
          %  Then,
           % \begin{equation*}
             %   [L(\vec{v})]_{BB}=\begin{bmatrix} 3 & 1 \\ 1 & 1 \end{bmatrix}.
            %\end{equation*}
            %Since this matrix is invertible, as \(\det [L(\vec{v})]_{BB} \neq 0\), the inverse is given by
           % \begin{equation*}
            %    L^{-1}:\mathbb{R}^2\to\mathbb{R}^2,\begin{bmatrix} x \\ y \end{bmatrix}\mapsto([L(\vec{v})]_{BB})^{-1}\begin{bmatrix} x \\ y \end{bmatrix}.
           % \end{equation*}
        %\end{example}

    \pagebreak
    \subsection{Diagonalization of Linear Operators}

        Consider the following definitions, similar to the notions discussed in Chapter \ref{chapter:deteigen}, but in the context of linear transformations.
        \begin{definition}{\Stop\,\,Eigenvalues and Eigenvectors}{eigenvaluesandvectorslintrans}

            Suppose \(V\) is a vector space. Let \(L:V\to V\) be a linear operator. A scalar \(\lambda\) is an eigenvalue of \(L\) if and only if there exists \(\vec{v}\in V\), where \(\vec{v}\neq\vec{0}_V\), such that \(L(\vec{v})=\lambda \vec{v}\). If \(\lambda\) is an eigenvalue of \(L\), \(\vec{v}\) is an eigenvector of \(L\) with eigenvalue \(\lambda\). 

        \end{definition}
        \begin{definition}{\Stop\,\,Eigenspace}{eigenspacelintrans}

            Suppose \(V\) is a vector space. Let \(L:V\to V\) be a linear operator. The eigenspace of a given eigenvalue \(\lambda\) is
            \begin{equation*}
                E_\lambda=\{\vec{v}\in V:L(\vec{v})=\lambda \vec{v}\}\cup\{\vec{0}_V\}.
            \end{equation*}
            Note that ``\(\cup\{\vec{0}_V\}\)'' is somewhat redundant, as it will always satisfy the equation. However, the zero vector is never an eigenvector.
        
        \end{definition}
        \vphantom
        \\
        \\
        Our goal is to find all the eigenvalues and eigenspaces of \(L\). Consider the following theorem.
        \begin{theorem}{\Stop\,\,Finding Eigenvectors and Eigenvalues}{findeigenvs}
    
            Let \(L\) be a linear operator on a nontrivial finite dimensional vector space \(V\). Suppose \(A\in\mathcal{M}_{nn}\) is the matrix representation of \(L\) with respect to some ordered basis of \(V\). The scalar \(\lambda\) is an eigenvalue of \(L\) if and only if \(\lambda\) satisfies
            \begin{equation*}
                \det(A-\lambda I_n)=0.
            \end{equation*}
            
        \end{theorem}
        \vphantom
        \\
        \\
        We will now define what it means for a linear operator to be diagonizable, while also providing a theorem to provide an equivalent condition.
        \begin{definition}{\Stop\,\,Diagonalizability of a Linear Operator}{diaglintrans}

            A linear operator \(L\) on a finite dimensional vector space is diagonizable if and only if the matrix representation of \(L\) with respect to some ordered basis for \(V\) is a diagonal matrix.
            
        \end{definition}
        \pagebreak
        \begin{theorem}{\Stop\,\,Diagonalizability of a Linear Operator}{diaglintrans}

            Suppose \(L\) is a linear operator on an \(n\)-dimensional vector space \(V\). Then, \(L\) is diagonalizable if and only if there exists a set of \(n\) linearly independent eigenvectors for \(L\).
            \begin{proof}
                Suppose \(L\) is diagonalizable. Then, there exists some ordered basis \(B=(\vec{v}_1,\ldots,\vec{v}_n)\) for \(V\) such that the matrix representation for \(L\) is a diagonal matrix \(D\). Because \(B\) is a basis, \(B\) is linearly independent. We wish to show that each \(\vec{v}_i\in B\), with \(1\leq i\leq n\), is an eigenvector corresponding to some eigenvalue for \(L\). Let \(d_{ii}\) be the \((i,i)\) element of \(D\). For each \(\vec{v}_i\), we have
                \begin{equation*}
                    [L(\vec{v_i})]_B=D[\vec{v_i}]_B=D\vec{e}_i=d_{ii}\vec{e}_i=d_{ii}[\vec{v_i}]_B=[d_{ii}\vec{v}_i]_B.
                \end{equation*}
                We have shown that \(L(\vec{v}_i)=d_{ii}\vec{v}_i\). That is, we have shown that each \(\vec{v}_i\in B\) is an eigenvector of \(L\) with eigenvalue \(d_{ii}\). Thus, \(B\) is a set of \(n\) linearly independent eigenvectors for \(L\). Conversely, suppose \(B=\{\vec{w}_1,\ldots,\vec{w}_n\}\) is a set of \(n\) linearly independent eigenvectors for \(L\), corresponding to eigenvalues \(\lambda_1,\ldots,\lambda_n\). These eigenvalues need not be distinct. We also note that \(B\) is a basis for \(V\), by Theorem \ref{thm:lindim2}. We wish to show that the matrix representation for \(L\), with respect to \(B\) is diagonal. The \(i\)th column for \(A\) is given by
                \begin{equation*}
                    [L(\vec{w}_i)]_B=[\lambda_i\vec{w}_i]_B=\lambda_i[\vec{w}_i]_B=\lambda_i\vec{e}_i.
                \end{equation*}
                Thus, \(A\) is diagonal, and \(L\) is diagonalizable, as desired.
            \end{proof}

        \end{theorem}
        \pagebreak
        \vphantom
        \\
        \\
        Note that Theorem \ref{thm:diaglintrans} requires that we find ``enough'' linearly independent eigenvectors. We now provide a theorem guaranteeing the linear independence of eigenvectors in certain conditions.
        \begin{theorem}{\Stop\,\,Eigenvectors With Distinct Eigenvalues are Linearly Independent}{eigenvecsdisteigenvalslinindep}

            Suppose \(L\) is a linear operator on \(V\). Let \(\lambda_1,\ldots,\lambda_n\) be distinct eigenvalues for \(L\). If \(\vec{v}_1,\ldots,\vec{v}_n\) are eigenvectors for \(L\) corresponding to \(\lambda_1,\ldots,\lambda_n\), respectively, the set \(\{\vec{v}_1,\ldots,\vec{v}_n\}\) is linearly independent.
            \begin{proof}
                
                We proceed by induction on \(n\). For \(n=1\), any eigenvector \(\vec{v}_1\), for any eigenvalue, by definition is nonzero, so \(\{\vec{v}_1\}\) is linearly independent. Suppose that for all \(k\in\mathbb{N}\), the proposition holds. Now, for distinct eigenvalues \(\lambda_1,\ldots,\lambda_{k+1}\). We wish to show that \(\{\vec{v}_1,\ldots,\vec{v}_{k+1}\}\) is linearly independent. Consider
                \begin{equation*}
                    c_1\vec{v}_1+\cdots+c_{k+1}\vec{v}_{k+1}=\vec{0}_V
                \end{equation*}
                for scalars \(c_1,\ldots,c_n\). If we apply \(L\) to both sides, we have
                \begin{align*}
                    L(\vec{0}_V)&=L(c_1\vec{v}_1+\cdots+c_{k+1}\vec{v}_{k+1}) \\
                    &=c_1L(\vec{v}_1)+\cdots+c_{k+1}L(\vec{v}_{k+1})
                \end{align*}
                which implies
                \begin{equation*}
                    c_1\lambda_1\vec{v}_1+\cdots+c_{k+1}\lambda_{k+1}\vec{v}_{k+1}=\vec{0}_V.
                \end{equation*}
                If we multiply \(c_1\vec{v}_1+\cdots+c_{k+1}\vec{v}_{k+1}=\vec{0}_V\) by \(\lambda_{k+1}\), we have
                \begin{equation*}
                    c_1\lambda_{k+1}\vec{v}_1+\cdots+c_{k+1}\lambda_{k+1}\vec{v}_{k+1}=\vec{0}_V=c_1\lambda_1\vec{v}_1+\cdots+c_{k+1}\lambda_{k+1}\vec{v}_{k+1}=\vec{0}_V.
                \end{equation*}
                which we can rewrite as
                \begin{equation*}
                    c_1(\lambda_1-\lambda_{k+1})\vec{v}_1+\cdots+c_k(\lambda_k-\lambda_{k+1})\vec{v}_k=\vec{0}_V
                \end{equation*}
                By the inductive hypothesis,
                \begin{equation*}
                    c_1(\lambda_1-\lambda_{k+1})=\cdots=c_k(\lambda_k-\lambda_{k+1})=0.
                \end{equation*}
                Since \(\lambda_1,\ldots,\lambda_{k+1}\) are distinct, none of the differences in the above equation can be zero, so \(c_1=\cdots=c_k=0\). Thus, for
                \begin{equation*}
                    c_1\vec{v}_1+\cdots+c_{k+1}\vec{v}_{k+1}=\vec{0}_V,
                \end{equation*}
                we have \(c_{k+1}\vec{v}_{k+1}=\vec{0}_V\). Since \(\vec{v}_{k+1}\neq\vec{0}_V\), we have \(c_{k+1}=0\), as desired.

            \end{proof}
            
        \end{theorem}
        \vphantom
        \\
        \\
        Note that Theorem \ref{thm:eigenvecsdisteigenvalslinindep} provides that if \(L\) is linear operator on an \(n\)-dimensional vector space and \(L\) has \(n\) distinct eigenvalues, \(L\) is diagonalizable. The converse is false.
        \pagebreak
        \\
        \\
        Consider the following theorems.
        \begin{theorem}{\Stop\,\,Union and Intersection of Bases for Eigenspaces}{unioninterbaseseigenspc}

            Suppose \(V\) is a finite dimensional vector space. Let \(L:V\to V\) be a linear operator, and let \(B_1,\ldots,B_k\) be bases for eigenspaces \(E_{\lambda_1},\ldots,E_{\lambda_k}\) for \(L\), where \(\lambda_1,\ldots,\lambda_k\) are distinct eigenvalues for \(L\). Then, \(B_i\cap B_j=\emptyset\) for \(1\leq i<j\leq k\), and \(B_1\cup \cdots\cup B_k\) is a linearly independent subset of \(V\).
            
        \end{theorem}
        \begin{theorem}{\Stop\,\,The Process of Diagonalization for Linear Operators}{procdiaglinops}

            Let \(V\) be an \(n\)-dimensional vector space and let \(L:V\to V\) be a linear operator. Consider the following steps.
            \begin{enumerate}
                \item Find a basis, \(C\), for \(V\). Then, find the matrix representation \(A\) of \(L\) with respect to \(C\).
                \item Apply the steps of Theorem \ref{thm:diagonalization} on \(A\) to find the eigenvalues \(\lambda_1,\ldots,\lambda_k\) and a basis in \(\mathbb{R}^n\) for each eigenspace \(E_\lambda\). If \(\left|\bigcup_{i}E_{\lambda_i}\right|<n\), \(L\) is not diagonalizable. Otherwise, let \(Z=(\vec{w}_1,\ldots,\vec{w}_n)=\bigcup_i E_{\lambda_i}\) be an ordered basis for \(\mathbb{R}^n\).
                \item Find an ordered basis \(B=(\vec{v}_1,\ldots,\vec{v}_n)\) of \(V\) such that \([\vec{v}_i]_C=\vec{w}_i\).
                \item Form \(D\) by finding the matrix representation for \(L\) with respect to \(B\).
                \item If needed, form \(P=\begin{bmatrix} [\vec{v}_1]_C & \cdots & [\vec{v}_n]_C \end{bmatrix}=\begin{bmatrix} \vec{w}_1 & \cdots & \vec{w}_n \end{bmatrix}\).
            \end{enumerate}

        \end{theorem}

\begin{savequote}
    Richard Friedman: I think that issue is entirely orthogonal to the issue here because the Commonwealth is acknowledging--
    \\
    \\
    Chief Justice Roberts: I'm sorry. Entirely what?
    \\
    \\
    Richard Friedman: Orthogonal. Right angle. Unrelated. Irrelevant.
    \\
    \\
    Chief Justice Roberts: Oh.
    \\
    \\
    Justice Scalia: What was that adjective? I liked that.
    \\
    \\
    Richard Friedman: Orthogonal.
    \\
    \\
    Chief Justice Roberts: Orthogonal.
    \\
    \\
    Richard Friedman: Right, right.
    \\
    \\
    Justice Scalia: Orthogonal, ooh.
    \\
    \\
    Justice Kennedy: I knew this case presented us a problem.
\end{savequote}
\chapter{Orthogonality} \label{chapter:ortho}

    \section{Lecture 35: November 28, 2022}

    \subsection{Inner Product Spaces}
    \DOTHISLATER

        Consider the following definition.
        \begin{definition}{\Stop\,\,Inner Products and Inner Product Spaces}{innerprod}

            An \(\mathbb{F}\) valued inner product on a vector space \(V\) is a function 
            \begin{equation*}
                \langle\cdot,\cdot\rangle:V\times V\to \mathbb{F}
            \end{equation*}
            such that
            \begin{enumerate}
                \item \(\forall \vec{v}\in V,\langle \vec{v},\vec{v}\rangle\geq 0\).
                \item \(\vec{v}=\vec{0}_V\iff \langle \vec{v},\vec{v}\rangle\).
                \item \(\forall \vec{u},\vec{v}\in V, \langle \vec{u},\vec{v}\rangle=\bar{\langle \vec{v}, \vec{u}\rangle}\).
                \item \(\forall \vec{u},\vec{v},\vec{w}\in V, \langle \vec{u}+\vec{v}, \vec{z}\rangle=\langle \vec{u},\vec{w}\rangle+\langle \vec{v},\vec{w}\rangle\).
                \item \(\forall c\in\mathbb{F},\forall \vec{u},\vec{v}\in V, \langle c\vec{u},\vec{v}\rangle=c\langle \vec{u},\vec{v}\rangle\).
            \end{enumerate}
            \vphantom
            \\
            \\
            The pair \((V,\langle\cdot,\cdot\rangle)\) is called an inner product space.
            
        \end{definition}
        \vphantom
        \\
        \\
        Consider the following examples.
        \begin{example}{An Inner Product of \(\mathbb{R}^n\)}{rninnerprod}
            
            The pair \((\mathbb{R}^n, \langle \vec{u},\vec{v}\rangle=\vec{u}\cdot\vec{v})\) is a real inner product space.

        \end{example}
        \begin{example}{An Inner Product of \(\mathbb{C}^n\)}{cninnerprod}
            
            The pair \((\mathbb{C}^n, \langle \vec{u},\vec{v}\rangle=\vec{u}\cdot\vec{v})\) is a complex inner product space.

        \end{example}
        \begin{example}{An Inner Product of \(\{f:[0,1]\to\mathbb{R}\}\)}{funcinnerprod}

            Let \(V=\{f:[0,1]\to\mathbb{R}\}\). The pair
            \begin{equation*}
                \left(V,\langle f, g\rangle=\int_0^1 f(x)g(x)\dd x\right)
            \end{equation*}
            is a real inner product space.
            
        \end{example}
        \begin{theorem}{\Stop\,\,}{...}

            Let \(V\) be an inner product space. Suppose \(\vec{u},\vec{v},\vec{w}\in V\) and \(c\in\mathbb{F}\). Then,
            \begin{enumerate}
                \item \(\langle\vec{0},\vec{v}\rangle=\vec{0}=\langle\vec{v},\vec{0}\rangle\).
                \item \(\langle \vec{v},\vec{w}+\vec{z}\rangle=\langle \vec{v},\vec{u}\rangle+\langle\vec{v},\vec{z}\rangle\).
                \item \(\langle\vec{v},c\vec{w}\rangle=\bar{c}\langle\vec{v},\vec{w}\rangle\).
            \end{enumerate}
            
        \end{theorem}
        \vphantom
        \\
        \\
        Consider the following definitions and theorems.
        \begin{definition}{\Stop\,\,Norms}{norms}

            The norm associated to the inner product \(\langle\cdot,\cdot\rangle\) is
            \begin{equation*}
                ||\cdot||:V\to[0,\infty),\vec{v}\mapsto\sqrt{\langle\vec{v},\vec{v}\rangle}.
            \end{equation*}
            
        \end{definition}
        \begin{theorem}{\Stop\,\,}{...2}
            
            Let \(V\) be an inner product space. Suppose \(\vec{v},\vec{w}\in V\) and \(c\in\mathbb{F}\). Then,
            \begin{enumerate}
                \item \(||c\vec{v}||=|c|||\vec{v}||\).
                \item \(\langle \vec{v},\vec{w}\rangle\leq ||\vec{v}||||\vec{w}||\).
                \item \(||\vec{v}+\vec{w}||\leq||\vec{v}||+||\vec{w}||\).
            \end{enumerate}

        \end{theorem}
        \begin{definition}{\Stop\,\,Distance}{distance}

            Let \(V\) be an inner product space. The distance betwen \(\vec{v},\vec{w}\in V\) is \(||\vec{v}-\vec{w}||\).
            
        \end{definition}
        \begin{definition}{\Stop\,\,Angle}{angle}

            Let \(V\) be an inner product space. Let \(\mathbb{F}=\mathbb{R}\). The angle betwen \(\vec{v},\vec{w}\neq\vec{0}_V\) is given by
            \begin{equation*}
                \theta =\arccos\left(\frac{\langle\vec{v},\vec{w}\rangle}{||\vec{v}||||\vec{w}||}\right)
            \end{equation*}
            
        \end{definition}
        \begin{definition}{\Stop\,\,Orthogonality}{orthogonality}

            Let \(V\) be an inner product space. Then, 
            \begin{enumerate}
                \item \(\vec{v},\vec{w}\in V\) are orthogonal if and only if \(\langle\vec{v},\vec{w}\rangle=0\).
                \item A set \(S\subseteq V\) is orthogonal if and only if for each \(\vec{v},\vec{w}\in S\), \(\langle\vec{v},\vec{w}\rangle=0\).
                \item A set \(S\subseteq V\) is orthonormal if and only if it is orthogonal and each \(\vec{v}\in S\) has \(||\vec{v}||=1\).
            \end{enumerate}

        \end{definition}
        \vphantom
        \\
        \\
        Consider the following example.
        \begin{example}{\Difficulty\,\,The Standard Basis of \(\mathbb{R}^n\)}{stdbasisrn}

            The set \(\{\vec{e}_1,\ldots,\vec{e}_n\}\) is orthonormal when considered as a subset of \(\mathbb{R}^n\) or \(\mathbb{C}^n\).
            
        \end{example}
        \begin{theorem}{\Stop\,\,Orthonormal Implies Linearly Independent}{orthlinindep}
            
            If \(\{\vec{v}_1,\ldots,\vec{v}_n\}\) is orthonormal, \(\{\vec{v}_1,\ldots,\vec{v}_n\}\) is linearly independent.
            \begin{proof}
                Suppose
                \begin{equation*}
                    c_1\vec{v}_1+\cdots+c_n\vec{v}_n=\vec{0}_V
                \end{equation*}
                for some \(c\in\mathbb{F}\). We must show \(c_1=\cdots=c_n=0\). For each \(i\), \(1\leq i\leq n\), consider 
                \begin{align*}
                    0&=\langle c_1\vec{v}_1+\cdots+c_n\vec{v}_n,\vec{v}_i \rangle \\
                    &=\langle c_1\vec{v}_1,\vec{v}_i\rangle+\cdots+\langle c_n\vec{v}_n,\vec{v}_i\rangle \\
                    &=c_1\langle \vec{v}_1,\vec{v}_i\rangle+\cdots+c_n\langle \vec{v}_n,\vec{v}_i\rangle \\
                    &=c_i\langle \vec{v}_i,\vec{v}_i\rangle
                \end{align*}
                We see that \(\langle\vec{v}_i,\vec{v}_i\rangle>0\), since \(\vec{v}_i\neq\vec{0}_V\), we have \(0=c_i||\vec{v}_i||^2\), implying that \(c_i=0\) for each \(i\).
            \end{proof}
            The converse is false.

        \end{theorem}
        \pagebreak
        \vphantom
        \\
        \\
        Given a linearly indepedent set \(S_1=\{\vec{w}_1,\ldots,\vec{w}_n\}\), we seek to construct a set \(S_2=\{\vec{v}_1,\ldots,\vec{v}_k\}\) such that \(S\) is orthogonal and \(\Span(S_1)=\Span(S_2)\). We present the Gram-Schmidt process.
        \begin{theorem}{\Stop\,\,Gram-Schmidt Process}{gramschmidt}
            
            Let \(S_1=\{\vec{w}_1,\ldots,\vec{w}_n\}\) such that \(S_1\) is linearly independent.
            \begin{enumerate}
                \item Let \(\vec{v}_1=\vec{w}_1\)
                \item Let \(\vec{v}_2=\vec{w}_2-\left(\frac{\langle\vec{w}_2,\vec{v}_1\rangle}{\langle(\vec{v}_1,\vec{v}_1\rangle)}\right)\vec{v}_1\)
                \item \(\vdots\)
                \item Let \(\vec{v}_k=\)
            \end{enumerate}
            \DOTHISLATER

        \end{theorem}
        \vphantom
        \\
        \\
        We can also normalize an orthogonal set into an orthonormal one by just dividing by the magnitude of each \(\vec{v}_i\).
        \DOTHISLATER
        \begin{definition}{\Stop\,\,Orthonormal Bases}{orthonormalbasis}

            A set \(B\subseteq V\) is an orthonormal basis if and only if \(B\) is a basis of \(V\) and \(B\) is orthonormal.
            
        \end{definition}
        \begin{theorem}{\Stop\,\,Every Inner Product Space Has an Orthonormal Basis}{innerproductspaceorthobasis}

            Every inner product space has an orthonormal basis.
            \begin{proof}
                Suppose \(V\) is a finite dimensional inner product space. Let \(B=\{\vec{w}_1,\ldots,\vec{w}_n\}\) be a basis for \(V\). Apply the Gram-Schmidt process to find orthogonal \(\{\vec{v}_1,ldots,\vec{v}_n\}\) and normalize it. This set is orthonormal and, therefore, linearly indepedent, and the spans are the same.
                \DOTHISLATER
            \end{proof}
            
        \end{theorem}

        \pagebreak

\section{Lecture 36: November 30, 2022}

    \subsection{}

\pagebreak

\chapter*{}
\addcontentsline{toc}{chapter}{Linear Algebra as a Word Cloud}

\vspace*{\fill}
\begin{center}
    \includegraphics[scale=0.8]{Graphics/wordcloudlinearalgebra.png}
\end{center}
\vspace*{\fill}

\pagebreak

\appendix
\appendixpage
\noappendicestocpagenum
\addappheadtotoc

\begin{savequote}
    \includegraphics[scale=0.5]{Graphics/proofxkcd.png}
\end{savequote}
\chapter{Introduction to Proofs} \label{appendix:a}

    \section{Introduction to Proofs}
    
    Before we delve into techniques to write proofs, let us first define what a proof is. 
    \begin{definition}{\Stop\,\,Proofs}{proofs}
    
        Mathematical proofs are logical arguments to show that stated premises guarantee that a mathematical statement must be true.
    
    \end{definition}
    \vphantom
    \\
    \\
    There are multiple techniques to write proofs, but here, we will explore the Proof by Induction, the Direct Proof, the Proof by Contrapositive, the Proof by Contradiction, and the Proof by Cases.
    
\section{Proof by Induction}

    We will use quantifiers to state induction.
    \\
    \\
    Let \(P(n)\) be a statement with \(n\in\mathbb{N}\). Consider the following Rule of Inference.
    \begin{center}
        \begin{tabular}{c}
            \hline
            \(P(0)\) \\
            \(P(0) \implies P(1)\) \\
            \(P(1) \implies P(2)\) \\
            \(\vdots\) \\
            \(P(n) \implies P(n+1)\) \\
            \hline
            \(\thus \forall n\in\mathbb{N}, P(n)\). \\
            \hline
        \end{tabular}.
    \end{center}
    \pagebreak
    \vphantom
    \\
    \\
    This may be further collapsed into
    \begin{center}
        \begin{tabular}{c}
            \hline
            \(P(0)\) \\
            \(\forall k\in\mathbb{N}, P(k)\implies P(k+1)\) \\
            \hline
            \(\thus \forall n\in\mathbb{N}, P(n)\). \\
            \hline
        \end{tabular}.
    \end{center}
    \vphantom
    \\
    \\
    Generally, in Proofs by Induction, we follow the following steps.
    \begin{itemize}
        \item Start with an iterative propostition that depends on some \(n\in\mathbb{N}\), or \(P(n)\).
        \item Prove that the proposition is true for some base case \(n=n_0\). That is, show that the proposition is true for the smallest fixed number that the proposition makes sense for.
        \item Prove the inductive step. Suppose that the proposition holds true for \(n=k\), and then prove that the proposition holds for \(n=k+1\). Essentially, suppose \(P(k)\) is true, and prove that \(P(k+1)\) is true.
        \item Then, the proposition is proved \(\forall n\in\mathbb{N}\), where \(n \geq n_0\).
    \end{itemize}
    \vphantom
    \\
    \\
    Consider the following examples and exercises.
    \begin{example}{\Difficulty\,\Difficulty\,\,Gauss' Formula}{gaussformula}
    
        Prove that the sum of consecutive integers starting at \(1\) can be found by Gauss' formula. That is,
        \begin{equation*}
            1+2+3\cdots+n=\frac{n(n+1)}{2}.
        \end{equation*}
        \begin{proof}
            Consider the base case \(n=1\). Then the left hand side is \(1\), and the right hand side is
            \begin{equation*}
                \frac{1(1+1)}{2}=1.
            \end{equation*}
            Therefore, the left hand side is equal to the right hand side, proving the case base.
            \\
            \\
            We suppose that the relationship is true for \(n=k\) where \(k\in\mathbb{N}\). That is, we suppose that
            \begin{equation*}
                1+2+3+\cdots+k=\frac{k(k+1)}{2}.
            \end{equation*}
            If we add \(k+1\) to both sides, we obtain
            \begin{align*}
                1+2+3+\cdots+k+k+1&=\frac{k(k+1)}{2}+k+1 \\
                &=\frac{k(k+1)+2k+2}{2} \\
                &=\frac{k^2+3k+2}{2} \\
                &=\frac{(k+2)(k+1)}{2} \\
                &=\frac{(k+1)((k+1)+1)}{2}.
            \end{align*}
            This result is the proposition where \(n=k+1\). Therefore, the inductive step is true. Therefore, Gauss' formula is true for all \(n\in\mathbb{N}\) where \(n \geq 1\).
        \end{proof}
    
    \end{example}
    \pagebreak
    \begin{example}{\Difficulty\,\Difficulty\,\,Sum of Consequtive Squares}{sumconseqsqs}
    
        Prove that for all \(n\in\mathbb{N}, n \geq 1\),
        \begin{equation*}
            1^2+2^2+3^2\cdots+n^2=\frac{n(n+1)(2n+1)}{6}.
        \end{equation*}
        \begin{proof}
            Consider the base case \(n=1\). Then the left hand side is \(1\), and the right hand side is
            \begin{equation*}
                \frac{1(1+1)(2+1)}{6}=1.
            \end{equation*}
            Therefore, the left hand side is equal to the right hand side, proving the base case.
            \\
            \\
            We suppose that the relationship is true for \(n=k\) where \(k\in\mathbb{N}\). That is, we suppose that
            \begin{equation*}
                1^2+2^2+3^2\cdots+k^2=\frac{k(k+1)(2k+1)}{6}.
            \end{equation*}
            If we add \(k+1\) to both sides, we obtain
            \begin{align*}
                1^2+2^2+3^2\cdots+k^2+(k+1)^2&=\frac{k(k+1)(2k+1)}{6}+(k+1)^2 \\
                &=\frac{k(k+1)(2k+1)}{6}+k^2+2k+1 \\
                &=\frac{k(k+1)(2k+1)+6k^2+12k+6}{6} \\
                &=\frac{2k^3+9k^2+13k+6}{6} \\
                &=\frac{(k+1)(k+2)(2k+3)}{6} \\
                &=\frac{(k+1)((k+1)+1)(2(k+1)+1)}{6}.
            \end{align*}
            This result is the proposition where \(n=k+1\). Therefore, the inductive step is true. Therefore, the above formula is true for all \(n\in\mathbb{N}\) where \(n \geq 1\).
        \end{proof}
    
    \end{example}
    \pagebreak
    \begin{exercise}{\Difficulty\,\Difficulty\,\,The Power Rule for Derivatives}{powerrule}
    
        Prove that for all \(n\in\mathbb{N}, n \geq 0\),
        \begin{equation*}
            \frac{\dd}{\dd x}x^n=nx^{n-1}.
        \end{equation*}
        \begin{proof}
            Consider the base case \(n=0\). Then the left hand side is \(0\), as the derivative of any constant is zero, and the right hand side is
            \begin{equation*}
                0x^{0-1}=0.
            \end{equation*}
            Therefore, the left hand side is equal to the right hand side, proving the base case.
            \\
            \\
            We suppose that the relationship is true for \(n=k\) where \(k\in\mathbb{N}\). That is, we suppose that
            \begin{equation*}
                \frac{\dd}{\dd x}x^k=kx^{k-1}.
            \end{equation*}
            Consider \(\frac{\dd}{\dd x}[x^{k+1}]\), or \(\frac{\dd}{\dd x}[xx^{k}]\). Then we have
            \begin{align*}
                \frac{\dd}{\dd x}[x^{k+1}]&=\frac{\dd}{\dd x}[xx^k] \\
                &=x^k+x(kx^{k-1}) \\
                &=x^k+kx^k \\
                &=(k+1)x^k.
            \end{align*}
            This result is the proposition where \(n=k+1\). Therefore, the inductive step is true. Therefore, the power rule for derivatives is true for all \(n\in\mathbb{N}\) where \(n \geq 0\).
        \end{proof}
    
    \end{exercise}
    \pagebreak
    \begin{exercise}{\Difficulty\,\Difficulty\,\,\(n\)th Derivative}{nthderiv}
    
        Prove that the \(n\)th Derivative of \(f(x)=\frac{1}{x}\) is
        \begin{equation*}
            f^{(n)}(x)=\frac{(-1)^nn!}{x^{n+1}}.
        \end{equation*}
        \begin{proof}
            Consider the base case \(n=0\). The zeroth derivative of \(f(x)\) is \(f(x)\) itself. Using the formula, we have
            \begin{align*}
                f^{(0)}(x)&=\frac{(-1)^00!}{x^{0+1}} \\
                &=\frac{1}{x} \\
                &=f(x).
            \end{align*}
            Therefore, the base case is true. We suppose that the relationship is true for \(n=k\) where \(k\in\mathbb{N}\). That is, we suppose that
            \begin{equation*}
                f^{(k)}(x)=\frac{(-1)^kk!}{x^{k+1}}.
            \end{equation*}
            To find the \((k+1)\)th derivative, we differentiate \(f^{(k)}\), producing
            \begin{align*}
                f^{(k+1)}(x)&=\frac{\dd}{\dd x}\frac{(-1)^kk!}{x^{k+1}} \\
                &=\frac{(-1)^kk!}{x^{k+1+1}}(-(k+1)) \\
                &=\frac{(-1)^{(k+1)}(k+1)!}{x^{(k+1)+1}}.
            \end{align*}
            This result is the proposition where \(n=k+1\). Therefore, the inductive step is true. Therefore, the proposition is proved for all \(n\in\mathbb{N}\) where \(n\geq0\).
        \end{proof}
    
    \end{exercise}
    \pagebreak
    \begin{exercise}{\Difficulty\,\Difficulty\,\Difficulty\,\,Reduction}{reduction}
    
        Prove that for all \(n\in\mathbb{N}, n \geq 0\),
        \begin{equation*}
            \int x^ne^{-x}\dd x = -e^{-x}\left(x^n+nx^{n-1}+n(n-1)x^{n-2}+n(n-1)(n-2)x^{n-3}+\cdots + n! \right)+C.
        \end{equation*}
        \begin{proof}
            Consider the base case \(n=0\). Then, the left hand side is equal to
            \begin{equation*}
                \int e^{-x} \dd x=-e^{-x}+C.
            \end{equation*}
            The right hand side is equal to \(-e^{-x}+C\). Therefore, the left hand side is equal to the right hand side, proving the base case.
            \\
            \\
            We assume that the relationship is true for \(n=k\). That is, we assume that
            \begin{equation*}
                \int x^ke^{-x}\dd x = -e^{-x}(x^k+kx^{k-1}+k(k-1)x^{k-2}+k(k-1)(k-2)x^{k-3}+\cdots + k!)+C.
            \end{equation*}
            Let
            \begin{equation*}
                u=e^{-x}(x^k+kx^{k-1}+k(k-1)x^{k-2}+k(k-1)(k-2)x^{k-3}+\cdots + k!).
            \end{equation*}
            Then,
            \begin{align*}
                \int x^{k+1}e^{-x}\dd x&=-x^{k+1}e^{-x}-\int -e^{-x}x^k(k+1) \dd x \\
                &=-x^{k+1}e^{-x}-(k+1)\int -x^ke^{-x} \dd x \\
                &=-x^{k+1}e^{-x}+(k+1)\int x^ke^{-x} \dd x \\
                &=-x^{k+1}e^{-x}-e^{-x}(k+1)\frac{u}{e^{-x}}+C \\
                &=-e^{-x}\left(x^{k+1}+\frac{u(k+1)}{e^{-x}}\right)+C \\
                &=-e^{-x}\left(x^{k+1}+(k+1)x^k+k(k+1)x^{k-1}+\cdots+(k+1)!\right)+C.
            \end{align*}
                This result is the proposition where \(n=k+1\). Therefore, the inductive step is true. Therefore, the above formula is true for all \(n\in\mathbb{N}\) where \(n \geq 0\).
        \end{proof}
    \end{exercise}
    \pagebreak
    \begin{exercise}{\Difficulty\,\Difficulty\,\Difficulty\,\,The Shoelace Lemma}{shoelace}
    
        The following is a statement of the Shoelace Lemma.
        \begin{quote}
            Consider a simple polygon with vertices \((x_1, y_1), (x_2, y_2), \ldots, (x_n, y_n)\), oriented clockwise. Let \((x_{n+1}, y_{n+1})=(x_1, y_1)\). The area of the polygon is given by
            \begin{equation*}
                A_n=\frac{1}{2}\left[\sum_{i=1}^n x_iy_{i+1}-x_{i+1}y_i \right].
            \end{equation*}
        \end{quote}
        Prove the above proposition.
        \begin{proof}
            Consider a polygon with three vertices: \((x_1, y_1)\), \((x_2, y_2)\), and \((x_3, y_3)\). The area, given by the Shoelace Lemma, is
            \begin{equation*}
                A_3=\frac{1}{2}\left[\sum_{i=1}^3 x_iy_{i+1}-x_{i+1}y_i \right]=\frac{1}{2}\left[x_1y_2-x_2y_1+x_2y_3-x_3y_2+x_3y_1-y_3x_1\right].
            \end{equation*}
            Then, if we define two vectors \((\vec{v},\vec{w})\in\mathbb{R}^3\) such that
            \begin{equation*}
                \vec{v}=(x_2-x_1, y_2-y_1, 0),\quad\vec{w}=(x_3-x_1, y_3-y_1, 0),
            \end{equation*}
            we may see that the area of the parallelogram formed by the two vectors is given by
            \begin{equation*}
                A_{||GRAM}=||\vec{v} \times \vec{w}||
            \end{equation*}
            Either of the two triangles formed by the parallelogram's diagonals correspond to our polygon. The area is then given by
            \begin{align*}
                A_3&=\frac{1}{2}||\vec{v} \times \vec{w}|| \\
                &=\frac{1}{2}\left|\left|(0, 0, x_1y_2-x_2y_1+x_2y_3-x_3y_2+x_3y_1-y_3x_1)\right|\right| \\
                &=\frac{1}{2}\left[x_1y_2-x_2y_1+x_2y_3-x_3y_2+x_3y_1-y_3x_1\right].
            \end{align*}
            Therefore, we have proved the Shoelace Lemma in the case of a polygon with three vertices. By induction, we suppose that for a polygon with \(k\) vertices \((x_1, y_1), (x_2, y_2), \ldots, (x_k, y_k)\), the area is
            \begin{equation*}
                A_k=\frac{1}{2}\left[\sum_{i=1}^k x_iy_{i+1}-x_{i+1}y_i \right].
            \end{equation*}
            The area of a polygon with vertices \((x_1, y_1), (x_2, y_2), \ldots, (x_{k+1}, y_{k+1})\) is given by the sum of the area of the polygon with vertices \((x_1, y_1), (x_2, y_2), \ldots, (x_k, y_k)\) and the area of the polygon with vertices \((x_1, y_1)\), \((x_k, y_k)\), and \((x_{k+1}, y_{k+1})\). That is,
            \begin{align*}
                A_{k+1}&=\frac{1}{2}\left[\sum_{i=1}^k x_iy_{i+1}-x_{i+1}y_i \right]+\frac{1}{2}\left[x_1y_k-x_ky_1+x_ky_{k+1}-x_{k+1}y_k+x_{k+1}y_1-y_{k+1}x_1\right] \\
                &=\frac{1}{2}\left[\sum_{i=1}^{k+1} x_iy_{i+1}-x_{i+1}y_i \right].
            \end{align*}
            The above result is the consequence of the Shoelace Lemma in the case of a polygon with \(k+1\) vertices. Therefore, the Shoelace Lemma is proved.
        \end{proof}
    \end{exercise}
    \begin{exercise}{\Difficulty\,\Difficulty\,\Difficulty\,\,Fibonacci}{fibonacci}
    
        Let \(f_n\) represent the sequence of Fibonacci numbers, which is defined recursively as
        \begin{equation*}
            f_0=1,\,f_1=1,\,f_n=f_{n-1}+f_{n-2}.
        \end{equation*}
        Prove that
        \begin{equation*}
            \sum_{i=0}^n(f_i)^2=f_nf_{n+1}.
        \end{equation*}
        \begin{proof}
            Consider the base case \(n=0\). Then the left hand side is equal to \(1\), and the right hand side is
            \begin{equation*}
                (1)f_1=1.
            \end{equation*}
            Therefore, the left hand side is equal to the right hand side, proving the first base case. Then, consider the base case \(n=1\). The left hand side is equal to \(2\), and the right hand side is \(1\cdot f_2\) where \(f_2=f_1+f_0=2\). Therefore the right hand side is \(2\) and is equal to the left hand side proving the second base case.
            \\
            \\
            We suppose that the relationship is true for \(n=k\) where \(k\in\mathbb{N}\). That is, we suppose that
            \begin{equation*}
                \sum_{i=0}^k(f_i)^2=f_kf_{k+1}.
            \end{equation*}
            If we add \((f_{k+1})^2\) to both sides, we have
            \begin{align*}
                (f_{k+1})^2+\sum_{i=0}^k(f_i)^2&=f_kf_{k+1}+(f_{k+1})^2 \\
                &=f_{k+1}(f_k+f_{k+1}) \\
                &=f_{k+1}f_{k+2}.
            \end{align*}
            This result is the proposition where \(n=k+1\). Therefore, the inductive step is true. Therefore, the above formula is true for all \(n\in\mathbb{N}\) where \(n \geq 0\).
        \end{proof}
    \end{exercise}
    
\pagebreak    
\section{Direct Proofs}

    Direct Proofs are the simplest style of proofs, and are especially useful when proving implications. Consider the following examples.
    
    \begin{example}{\Difficulty\,\Difficulty\,\,Direct Proof 1}{dirproof1}
    
    Prove that for all integers \(n\), if \(n\) is even, then \(n^2\) is even.
    
    \begin{proof}
        Let \(n\in\mathbb{Z}\) and suppose that \(n\) is even. Let \(m\in\mathbb{Z}\). Thus, \(n=2m\). Then, \(n^2=(2m)^2=4m^2=2(2m^2)\). Because \(2m^2\in\mathbb{Z}\), \(n^2\) is even.
    \end{proof}
    
    \end{example}
    \begin{example}{\Difficulty\,\Difficulty\,\,Direct Proof 2}{dirproof2}
    
    Prove that for all integers \(a\), \(b\), and \(c\), if \(a|b\) and \(b|c\), then \(a|c\).
    
    \begin{proof}
        Let \((a,b,c,p,q,r)\in\mathbb{Z}\) and suppose that \(a|b\) and\(b|c\). Because \(a|b\), \(b=pa\). Because \(b|c\), \(c=qb=pqa\). Because \(c\) is an integer multiple of \(a\), \(a|c\).
    \end{proof}
    
    \end{example}
    \vphantom
    \\
    \\
    Consider the following exercises.
    \begin{exercise}{\Difficulty\,\Difficulty\,\,Direct Proof 1}{dirproof1}
    
    Prove that for any two odd integers, their sum is even.
    
    \begin{proof}
        Let \((m,n)\in\mathbb{Z}:m\bmod2\neq0:n\bmod2\neq0\) and let \((p,q)\in\mathbb{Z}\). Because \(m\) and \(n\) are odd, \(m=2p+1\) and \(n=2q+1\). Therefore,
        \begin{align*}
            m+n&=(2p+1)+(2q+1) \\
            &=2p+2q+2 \\
            &=2(p+q+1).
        \end{align*}
        Because \((p+q+1)\in\mathbb{Z}\), \(m+n\) is even.
    \end{proof}
    
    \end{exercise}
    \begin{exercise}{\Difficulty\,\Difficulty\,\,Direct Proof 2}{dirproof2}
    
    Prove that for all integers \(n\), if \(n\) is odd, then \(n^2\) is odd.
    
    \begin{proof}
        Let \(n\in\mathbb{Z}:n\bmod2\neq0\) and let \(p\in\mathbb{Z}\). Because \(n\) is odd, \(n=2p+1\). Therefore,
        \begin{align*}
            n^2&=(2p+1)^2 \\
            &=4p^2+4p+1 \\
            &=2(2p^2+2p)+1.
        \end{align*}
        Because \((2p^2+2p)\in\mathbb{Z}\), \(n^2\) is odd.
    \end{proof}
    
    \end{exercise}

\section{Proof by Contrapositive}

    Recall that for two statements \(P\) and \(Q\), \((P\implies Q)\iff(\neg Q\implies \neg P)\). In a Proof by Contrapositive, we produce a direct proof of the contrapositive of the implication. This is equivalent to proving the implication, because the implication is logically equivalent to the contrapositive. Consider the following examples.
    \begin{example}{\Difficulty\,\Difficulty\,\,Proof by Contrapositive 1}{proofcontrap1}
    
    Prove that for all integers \(n\), if \(n^2\) is even, then \(n\) is even.
    
    \begin{proof}
        Let \(n\in\mathbb{Z}:n\bmod2\neq0\). By Exercise \ref{exe:dirproof2}, \(n^2\) is odd.
    \end{proof}
    
    \end{example}
    \begin{example}{\Difficulty\,\Difficulty\,\,Proof by Contrapositive 2}{proofcontrap2}
    
    Prove that for all integers \(a\) and \(b\), if \(a+b\) is odd, then \(a\) is odd or \(b\) is odd.
    
    \begin{proof}
        Let \((a, b,p,q)\in\mathbb{Z}\). Suppose that \(a\) is even and \(b\) is even. Then, \(a=2p\) and \(b=2q\). We see that
        \begin{align*}
            a+b&=2p+2q \\
            &=2(p+q).
        \end{align*}
        Because \((p+q)\in\mathbb{Z}\), \(a+b\) is even.
    \end{proof}
    
    \end{example}
    \vphantom
    \\
    \\
    Consider the following exercises.
    \begin{exercise}{\Difficulty\,\Difficulty\,\,Proof by Contrapositive 1}{proofcontrap1}
    
    Prove that for real numbers \(a\) and \(b\), if \(ab\) is irrational, then \(a\) or \(b\) must be an irrational number.
    
    \begin{proof}
        Let \((p,q,r,s)\in\mathbb{Z}\). Suppose that \((a,b)\in\mathbb{Q}\). Therefore, \(a=\frac{p}{q}\) and \(b=\frac{r}{s}\). We see that
        \begin{equation*}
            ab=\frac{pr}{qs}\in\mathbb{Q}
        \end{equation*}
        Therefore \(ab\) is rational.
    \end{proof}
    
    \end{exercise}
    \pagebreak
    \begin{exercise}{\Difficulty\,\Difficulty\,\,Proof by Contrapositive 2}{proofcontrap2}
    
    Prove that for integers \(a\) and \(b\), if \(ab\) is even, then \(a\) or \(b\) must be even.
    
    \begin{proof}
        Let \((a,b,p,q)\in\mathbb{Z}\). Suppose that \(a=2p+1\) and \(b=2q+1\). We see that
        \begin{align*}
            ab&=(2p+1)(2q+1) \\
            &=4pq+2p+2q+1 \\
            &=2(2pq+p+q)+1
        \end{align*}
        Because \((2pq+p+q)\in\mathbb{Z}\), \(ab\) is odd.
    \end{proof}
    
    \end{exercise}
    \begin{exercise}{\Difficulty\,\Difficulty\,\,Proof by Contrapositive 3}{proofcontrap3}
    
    Prove that for any integer \(a\), if \(a^2\) is not divisible by \(4\), then \(a\) is odd.
    
    \begin{proof}
        Let \(a,p\in\mathbb{Z}\). Suppose that \(a\) is even, and \(a=2p\). Then, \(a^2=4p^2\), and \(4|4p^2\), so \(4|a^2\).
    \end{proof}
    
    \end{exercise}
    
\section{Proof by Contradiction}

    Sometimes, a statement, \(P\) cannot be rephrased as an implication. In these cases, it may be useful to prove that \(P\implies Q\), and also prove that \(P\implies \neg Q\). Then, we conclude \(\neg P\). Consider the following example.
    \begin{example}{\Difficulty\,\Difficulty\,\,Proof by Contradiction 1}{proofcontrad1}
    
        Prove that \(\sqrt{2}\) is irrational.
        
        \begin{proof}
            Suppose that \(\sqrt{2}\) is rational. Then,
            \begin{equation*}
                \sqrt{2}=\frac{p}{q}
            \end{equation*}
            where \((p,q)\in\mathbb{Z}\) and \(\frac{p}{q}\) is in lowest terms. By squaring both sides of the equation, we have
            \begin{equation*}
                2=\frac{p^2}{q^2}.
            \end{equation*}
            This means that
            \begin{equation*}
                2q^2=p^2,
            \end{equation*}
            and as \(q^2\in\mathbb{Z}\), \(p^2\) is even, which means that by Example \ref{exa:proofcontrap1}, \(p\) is even. We see that \(p=2k\) for some \(k\in\mathbb{Z}\). Then, we have
            \begin{equation*}
                2q^2=(2k)^2=4k^2
            \end{equation*}
            meaning that
            \begin{equation*}
                q^2=2k^2.
            \end{equation*}
            Therefore, \(q\) is even. If \(p\) and \(q\) are both even, \(\frac{p}{q}\) is not in lowest terms. Therefore, \(\sqrt{2}\) is irrational.
        \end{proof}
    
    \end{example}

\begin{savequote}
    \includegraphics[scale=0.7]{Graphics/functions.png}
\end{savequote}
\chapter{Functions} \label{appendix:b}

    \section{An Introduction to the Terminology of Functions}

    Consider the following definitions.
    \begin{definition}{\Stop\,\,Functions, Domains, and Codomains}{functions}

        A function \(F\), from a domain \(A\) to a codomain \(B\), that is, \(F:A\to B\) is a map from the elements of a set \(A\) to a set \(B\) such that for all \(a\in A\), there exists a unique \(b\in B\) such that \(a\) is mapped to \(b\) by \(F\).

    \end{definition}
    \begin{definition}{\Stop\,\,Images and Pre-Images}{imagespreimage}

        Let \(F:A\to B\) be a function. For \(a\in A\), the image of \(a\) is written as \(F(a)\) and is the unique element of \(B\) to which \(a\) is mapped to by \(F\). For \(b\in B\), the pre-images of \(b\) are the elements of \(A\) that map to \(b\) by \(F\).

    \end{definition}
    \begin{definition}{\Stop\,\,Range}{range}

        Let \(F:A\to B\) be a function. The image of the domain, \(X\), is the range of \(F\).
        
    \end{definition}
    \pagebreak
    \vphantom
    \\
    \\
    Consider the following example.
    \begin{example}{\Difficulty\,\Difficulty\,\,Is it a Function?}{functanal}
        
        Define \(R:\mathbb{N}\to\mathbb{N}\) where for \(a,b\in\mathbb{N}\), \(a\sim_Rb\iff b=a^2\). Determine if \(R\) is a function. State the domain, codomain, and range of \(R\).
        \\
        \\
        Consider the following diagram.
        \begin{center}
            \begin{tikzpicture}[
                >=stealth,
                bullet/.style={
                  fill=black,
                  circle,
                  minimum width=1pt,
                  inner sep=1pt
                },
                projection/.style={
                  ->,
                  thick,
                  shorten <=2pt,
                  shorten >=2pt
                },
                every fit/.style={
                  ellipse,
                  draw,
                  inner sep=0pt
                },
                scale=0.8
                ]
                \node[bullet,label=above:\(\mathbb{N}\)] (A) at (0,-1) {};
                \node[bullet,label=above:\(\mathbb{N}\)] (B) at (4,-1) {};
                \node[bullet,label=below:\(\vdots\)] (aEND) at (0,-5) {};
                \node[bullet,label=below:\(\vdots\)] (bEND) at (4,-5) {};
                \foreach \y/\l in {1/0,2/1,3/2,4/3,5/4}
                  \node[bullet,label=left:$\l$] (a\y) at (0,-1*\y) {};
            
                \foreach \y/\l in {1/0,2/1,3/4,4/9,5/16}
                  \node[bullet,label=right:$\l$] (b\y) at (4,-1*\y) {};
            
                \node[draw,fit=(a1) (a2) (a3) (a4) (a5) (aEND), minimum width=2cm] {} ;
                \node[draw,fit=(b1) (b2) (b3) (b4) (b5) (bEND), minimum width=2cm] {} ;
            
                \draw[projection] (a1) -- (b1);
                \draw[projection] (a2) -- (b2);
                \draw[projection] (a3) -- (b3);
                \draw[projection] (a4) -- (b4);
                \draw[projection] (a5) -- (b5);
            \end{tikzpicture}
        \end{center}
        \vphantom
        \\
        \\
        We recognize that \(R\) is a function, as each output only has one input. Notice that if \(R\) were a relation on \(\mathbb{Z}\), this would not be true. Furthermore, the domain and codomain of \(R\) is \(\mathbb{N}\). The range of \(R\) is the set of all perfect squares.
    \end{example}

\pagebreak

\section{Injections, Surjections, and Bijections}

    Consider the following definitions.
    \begin{definition}{\Stop\,\,Injective Functions}{injectivefunctions}
        
        Given a function \(F:A\to B\), \(F\) is injective, or one-to-one, if and only if
        \begin{equation*}
            \forall a_1,a_2\in A,a_1\neq a_2\implies F(a_1)\neq F(a_2).
        \end{equation*}
        That is, \(F\) is injective if and only if
        \begin{equation*}
            \forall a_1,a_2\in A, F(a_1)=F(a_2)\implies a_1=a_2.
        \end{equation*}
        
    \end{definition}
    \begin{definition}{\Stop\,\,Surjective Functions}{surjectivefunctions}
    
        Given a function \(F:A\to B\), \(F\) is surjective, or onto, if and only if
        \begin{equation*}
            \range F = B.
        \end{equation*}
        That is, \(F\) is surjective if and only if
        \begin{equation*}
            \forall b\in B, \exists a\in A,F(a)=b.
        \end{equation*}
        
    \end{definition}
    \begin{definition}{\Stop\,\,Bijective Functions}{bijectivefunctions}
    
        Given a function \(F:A\to B\), \(F\) is bijective, if and only if \(F\) is both injective and surjective. That is, \(F\) is bijective if and only if
        \begin{equation*}
            (\forall a_1,a_2\in A, F(a_1)=F(a_2)\implies a_1=a_2) \wedge (\forall b\in B, \exists a\in A,F(a)=b).
        \end{equation*}
        
    \end{definition}
    \vphantom
    \\
    \\
    Consider the following examples.
    \begin{example}{\Difficulty\,\Difficulty\,\,The Arctangent: Part I}{arctan1}
        
        Consider \(F:\mathbb{R}^2\to\mathbb{R}^2\) given by \(F(x)=\arctan x\). Determine if \(F\) is injective, surjective, or bijective.
        \begin{itemize}
            \item Injective: \(F\) is injective.
            \begin{itemize}
                \item Suppose \(\arctan(a_1)=\arctan(a_2)\). If we take the tangent of both sides, we see that \(a_1=a_2\).
                \item Alternatively, if \(a_1\neq a_2\), we observe that \(\arctan(a_1)\neq\arctan(a_2)\) because \(\arctan x\) is monotonically increasing.
            \end{itemize}
            \item Surjective: \(F\) is not surjective.
            \begin{itemize}
                \item We see that \(\range F=\left\{x:-\frac{\pi}{2}<x<\frac{\pi}{2}\right\}\).
            \end{itemize}
            \item Bijective: \(F\) is not bijective.
        \end{itemize}
    \end{example}
    \begin{example}{\Difficulty\,\Difficulty\,\,The Arctangent: Part II}{arctan2}
        
        Consider \(F:\mathbb{R}^2\to\left(-\frac{\pi}{2},\frac{\pi}{2}\right)\) given by \(F(x)=\arctan x\). Determine if \(F\) is injective, surjective, or bijective.
        \begin{itemize}
            \item Injective: \(F\) is injective.
            \item Surjective: \(F\) is surjective.
            \item Bijective: \(F\) is bijective.
        \end{itemize}
    
    \end{example}
    \vphantom
    \\
    \\
    We note that to make a function injective, we can often restrict the domain. Similarly, to make a function surjective, we can modify the codomain.
    
\pagebreak

\section{Composition and Inverses}

    Consider the following definitions and theorems.
    \begin{definition}{\Stop\,\,Compositions}{comp}

        For functions \(F:A\to B\) and \(G:B\to C\), the composition of \(F\) and \(G\) is \(G\circ F:A\to C\), which is given by
        \begin{equation*}
            (G\circ F)(a)=G(F(a)).
        \end{equation*}
        
    \end{definition}
    \begin{theorem}{\Stop\,\,Compositions, Injectivity, and Surjectivity}{compinjsur}

        Let \(F:A\to B\) and \(G:B\to C\) be functions. Then,
        \begin{enumerate}
            \item If both \(F\) and \(G\) are injective, \(G\circ F:A\to C\) is injective.
            \begin{proof}
                Suppose both \(F\) and \(G\) are injective. Now, suppose 
                \begin{equation*}
                    (G\circ F)(a_1)=G(F(a_1))=(G\circ F)(a_2)=G(F(a_2))
                \end{equation*}
                 for \(a_1,a_2\in A\). We wish to show that \(a_1=a_2\). We see that \(F(a_1)=F(a_2)\) since \(G\) is injective. Then, since \(F\) is injective, \(a_1=a_2\).
            \end{proof}
            \item If both \(F\) and \(G\) are surjective, \(G\circ F:A\to C\) is surjective.
            \begin{proof}
                Suppose both \(F\) and \(G\) are surjective. Consider some arbitrary \(c\in C\). We wish to find some \(a\in A\) such that \((G\circ F)(a)=G(F(a))=c\). Since \(G\) is surjective, there exists some \(b\in B\) such that \(G(b)=c\). Since \(F\) is surjective, there exists some \(a\in A\) such that \(F(a)=b\). Thus, \(G(F(a))=G(b)=c\).
            \end{proof}
        \end{enumerate}
        
    \end{theorem}
    \begin{definition}{\Stop\,\,Inverse Functions}{invfunc}

        The functions \(F:A\to B\) and \(G:B\to A\) are inverses of each other if and only if, for all \(a\in A\) and \(b\in B\),
        \begin{equation*}
            (G\circ F)(a)=a
        \end{equation*}
        and
        \begin{equation*}
            (F\circ G)(b)=b.
        \end{equation*}
        
    \end{definition}
    \pagebreak
    \begin{theorem}{\Stop\,\,Existence of Inverse Functions}{existinv}
        
        The function \(F:A\to B\) has an inverse \(G:B\to A\) if and only if \(F\) is bijective.
        \begin{proof}
            Suppose \(F:A\to B\) has an inverse \(G:B\to A\). Suppose \(F(a_1)=F(a_2)\) for \(a_1,a_2\in A\). Since \(F(a_1)=F(a_2)\), \(G(F(a_1))=G(F(a_2))\), but since \(G\) is an inverse of \(F\),
            \begin{equation*}
                G(F(a_1))=a_1=G(F(a_2))=a_2.
            \end{equation*}
            Thus, \(F\) is injective. Consider some arbitrary \(b\in B\). Then, we have \(G(b)=a\) since \(G\) will map all \(b\in B\) to some \(a\in A\). Since \(F\) and \(G\) are inverses, we have \(F(a)=F(G(b))=b\), so \(F\) is surjective. We have, at this point, shown that \(F\) is bijective. Now, suppose \(F\) is bijective. Consider some arbitrary \(b\in B\). Then, since \(F\) is surjective, there exists some \(a\in A\) such that \(F(a)=b\). Since \(F\) is surjective, \(a\) is unique. Now, consider the map \(G:Y\to X\) which maps each \(b\in B\) to its unique pre-image \(a\in A\) under \(F\). Then, \((F\circ G)(b)=F(G(b))=F(a)=b\). We also have \((G\circ F)(a)\) to be the unique pre-image of \(F(a)\) under \(F\). We have that \(a\) is the unique pre-image, so \((G\circ F)(a)=a\), so \(F\) and \(G\) are inverses. 
        \end{proof}

    \end{theorem}
    \begin{theorem}{\Stop\,\,Uniqueness of Inverse Functions}{uniqueinv}
        
        If \(F:A\to B\) has an inverse \(G:B\to A\), \(G\) is the only inverse of \(X\).
        \begin{proof}
            Suppose \(G_1:B\to A\) and \(G_2:B\to A\) are both inverses of \(F\). We wish to show that for all \(b\in B\), \(G_1(b)=G_2(b)\). We have \((G_2\circ F)(a)=a\) for all \(a\in A\), since \(F\) and \(G_2\) are inverses. Similarly, we have \((F\circ G_1)(b)=b\). We know \(G_1(b)\in A\), so
            \begin{equation*}
                G_1(b)=(G_2\circ F)(G_1(b))=G_2(F(G_1(b)))=G_2((F\circ G_1)(b))=G_2(b),
            \end{equation*}
            as desired.
        \end{proof}

    \end{theorem}

\pagebreak

\section{The Pigeonhole Principle}

    Consider the following theorem.
    \begin{theorem}{\Stop\,\,The Pigeonhole Principle}{pigeonhole}

        Let \(F:A\to B\) be a function with finite sets \(A\) and \(B\). If \(|A|>|B|\), \(F\) is not injective.
        
    \end{theorem}
    \vphantom
    \\
    \\
    The above result gets its name from the conceptual problem of a function that maps pigeons to holes. If there are more pigeons than holes, there exists a hole with more than one pigeon. To visualize this, consider the following diagrams.
    \begin{center}
        \begin{tikzpicture}[
            >=stealth,
            bullet/.style={
                fill=black,
                circle,
                minimum width=1pt,
                inner sep=1pt
            },
            projection/.style={
                ->,
                thick,
                shorten <=2pt,
                shorten >=2pt
            },
            every fit/.style={
                ellipse,
                draw,
                inner sep=0pt
            },
            scale=0.7
            ]
            \node[bullet,label=above:\(A\)] (A) at (0,-1) {};
            \node[bullet,label=above:\(B\)] (B) at (4,-1) {};
            \foreach \y/\l in {1/,2/,3/,4/,5/}
                \node[bullet,label=left:$\l$] (a\y) at (0,-1*\y) {};
        
            \foreach \y/\l in {1/,2/,3/,4/}
                \node[bullet,label=right:$\l$] (b\y) at (4,-1*\y) {};
                \node[] (b5) at (4,-1*5) {};
        
            \node[draw,fit=(a1) (a2) (a3) (a4) (a5), minimum width=1.5cm] {} ;
            \node[draw,fit=(b1) (b2) (b3) (b4) (b5), minimum width=1.5cm] {} ;
        
            \draw[projection] (a1) -- (b1);
            \draw[projection] (a2) -- (b2);
            \draw[projection] (a3) -- (b3);
            \draw[projection] (a4) -- (b4);
            \draw[projection] (a5) -- (b4);
            
            \node[bullet,label=above:\(A\)] (A) at (8,-1) {};
            \node[bullet,label=above:\(B\)] (B) at (12,-1) {};
            \foreach \y/\l in {1/,2/,3/,4/}
                \node[bullet,label=left:$\l$] (c\y) at (8,-1*\y) {};
                \node[] (c5) at (8,-1*5) {};
        
            \foreach \y/\l in {1/,2/,3/,4/,5/}
                \node[bullet,label=right:$\l$] (d\y) at (12,-1*\y) {};
        
            \node[draw,fit=(c1) (c2) (c3) (c4) (c5), minimum width=1.5cm] {} ;
            \node[draw,fit=(d1) (d2) (d3) (d4) (d5), minimum width=1.5cm] {} ;
        
            \draw[projection] (c1) -- (d1);
            \draw[projection] (c2) -- (d2);
            \draw[projection] (c3) -- (d3);
            \draw[projection] (c4) -- (d4);
            
        \end{tikzpicture}
    \end{center}
    \vphantom
    \\
    \\
    While the Pigeonhole Principle may seem trivial, it may be used to construct various proofs. There are three parts to every Pigeonhole argument.
    \begin{enumerate}
        \item Define \(A\), the set of pigeons.
        \item Define \(B\), the set of pigeonholes, such that \(|A|>|B|\).
        \item Define \(F:A\to B\), the method of assigning pigeons to pigeonholes.
    \end{enumerate}
    \vphantom
    \\
    \\
    We may then conclude that
    \begin{equation*}
        \exists a_1,a_2\in A,a_1\neq a_2\wedge F(a_1)=F(a_2).
    \end{equation*}
    \pagebreak
    Consider the following examples.
    \begin{example}{\Difficulty\,\Difficulty\,\,Hairs}{hairs}
        
        Prove that two people from the state of Colorado have the same number of hairs on their head.
        \begin{proof}
            The state of Colorado, at the time of writing, has roughly \(5.8\times10^6\) people, and it is safe to assume that the number of human hairs is less than \(5\times10^5\). Let \(A\) be the set of people in Colorado, with \(|A|=5.8\times10^6\) and let \(B\) be the set of all integers from \(0\) to \(5\times10^5\), inclusive, noninclusive. We note that \(|B|=5\times10^5\). Let the function \(F:A\to B\) be the function that maps a given person to the number of hairs on their head. We see that \(F\) is non-injective, therefore, at least two people from the state of Colorado must have the same number of hairs on their head.
        \end{proof}
    
    \end{example}
    \begin{example}{\Difficulty\,\Difficulty\,\,Sphere}{sphere}
    
        Prove that given \(5\) points on the surface of a sphere, there exists a hemisphere containing at least four of them. Any point on the boundary between the hemispheres is simultaneously in both hemispheres.
        \begin{proof}
            Pick two points, and cut the sphere in half such that the two points lie on the cut. Let \(A\) be the set of the three remaining points, and let \(B\) be the set of the two pieces of the sphere--the hemispheres. Let \(F:A\to B\) map the points to their corresponding hemisphere. As \(|A|=3\) and \(|B|=2\), we see that \(F\) is non-injective, meaning that at least two remaining points will fall on the same hemisphere. These two points add to the two points that lie on the cut, giving four points in the hemisphere.
        \end{proof}
    
    \end{example}
    \pagebreak
    \begin{example}{\Difficulty\,\Difficulty\,\Difficulty\,\,1978 Putnam}{1978putnam}
        
        Prove that any \(20\) distinct integers chosen from the set \(S=\{1,4,7,10,\ldots,100\}\) will contain a pair that sums to \(104\).
        \\
        \\
        Before delving into the proof itself, we will proceed with some informal experimentation. Consider the following pairs \(S\) that sum to \(104\).
        \begin{align*}
            104&=\underbrace{4}_{1+1(3)}+100 \\
            &=\underbrace{7}_{1+2(3)}+97 \\
            &=\underbrace{10}_{1+3(3)}+94 \\
            &=\underbrace{13}_{1+4(3)}+91 \\
            &\qquad\qquad\vdots \\
            &=\underbrace{49}_{1+16(3)}+55.
        \end{align*}
        Note that there are \(16\) pairs of numbers that add to \(104\). Also, \(1\) and \(52\) are not able to be used in a pair that sums to \(104\). Therefore, any choice of \(20\) distinct integers from \(S\) will contain at least \(18\) distinct integers selected from \(S-\{1,52\}\). We are now ready to begin our proof.
        \begin{proof}
            Let \(A\) be \(18\) distinct integers chosen from \(S-\{1,52\}\). Let \(B\) be the set of pairs of integers that sum to \(104\). That is,
            \begin{align*}
                B&=\{\{4,100\},\{7,97\},\{10,94\},\ldots,\{49,55\}\} \\
                &=\{\{1+3n, 103-3n\}:n\in\{1,2,3,\ldots,16\}\}.
            \end{align*}
            Note that \(|A|=18\) and \(|B|=16\). Let \(F:A\to B\) be given by 
            \begin{equation*}
                F(a)=\{a,104-a\}.
            \end{equation*}
            for \(a\in A\). The function \(F\) is non-injective by the Pigeonhole Principle, so there are two distinct elements of \(A\) that are mapped to the same element of \(B\). These two elements are the pair that will sum to \(104\).
        \end{proof}
    
    \end{example}
    \vphantom
    \\
    \\
    The Extended Pigeonhole Principle is, well, an extended form of the Pigeonhole Principle. Consider the following statement.
    \begin{theorem}{\Stop\,\,The Extended Pigeonhole Principle}{extpigeonhole}
    
        If \(n\) ``pigeons'' land into \(k\) ``pigeonholes,'' there exists at least one pigeonhole with at least \(\floor{\frac{n-1}{k}}=\ceil{\frac{n}{k}}\) pigeons.
    
    \end{theorem}
    \vphantom
    \\
    \\
    We may use Theorem \ref{thm:extpigeonhole} to better quantify the ``population'' of the holes. Consider the following exercise.
    \begin{exercise}{\Difficulty\,\Difficulty\,\Difficulty\,\,Equilateral Triangle}{equtri}
    
        Prove that given \(9\) points in an equilateral triangle with unit sides, there exist \(3\) that define a triangle of area less than or equal to \(\frac{\sqrt{3}}{8}\).
        \begin{proof}
            Let \(A\) be the set of the nine points in the triangle. Let \(B\) be the set of four equilateral triangles given by the first iteration of Sierpinski's Triangle. Let \(F:A\to B\) map each point to the triangle that contains the point. There must exist a triangle containing \(\ceil{\frac{9}{4}}=3\) points. The four equilateral triangles have area \(\frac{\sqrt{3}}{4\cdot2}\), and the triangle formed by the three points is therefore less than or equal to \(\frac{\sqrt{3}}{4\cdot2}\).
        \end{proof}
    
    \end{exercise}

\begin{savequote}
    Numerical analysis is often considered neither beautiful nor, indeed, profound. 
    \\
    \\
    Pure mathematics is beautiful if your heart goes after the joy of abstraction, applied mathematics is beautiful if you are excited by mathematics as a means to explain the mystery of the world around us.
    \\
    \\
    But numerical analysis? Surely, we compute only when everything else fails, when mathematical theory cannot deliver an answer in a comprehensive, pristine form and thus we are compelled to throw a problem onto a number-crunching computer and produce boring numbers by boring calculations. This, I believe, is nonsense.
    \\
    \\
    Arieh Iserles
\end{savequote}
\addtocontents{toc}{\protect\newpage}
\chapter{Numerical Linear Algebra} \label{appendix:c}

    In this section, we provide lecture notes for numerical linear algebra, as covered by APPM4600: Numerical Methods and Scientific Computation. We assume basic familiarity with the fundamentals of numerical computation. Lectures were presented by Eduardo Corona, Ph. D, and we additionally referred to \cite{olver2006applied}, especially for pseudocode. This set of notes is significantly less polished than the rest of this document.
\section{Lecture 1: Nov. 13, 2024}

    \subsection{Gaussian Elimination and the \(LU\) Decomposition}

        Recall that in Chapter~\ref{chapter:syslineq}, we discussed systems of linear equations. There, we described the classic Gaussian elimination method. Here, we formalize this procedure as an algorithm and discuss its efficiency and stability. Recall that given an upper triangular system, we may easily recover our unknowns. That is, given \(A\vec{x}=\vec{b}\) with \(A\) being an upper triangular matrix, we have that
        \begin{equation*}
            x_n=\frac{b_n}{a_{nn}},\quad x_j=\frac{1}{a_{jj}}\left(b_j-\sum_{k>j}a_{jk}x_k\right).
        \end{equation*}
        We refer to this procedure as ``back substitution.'' An analogous procedure, called ``forward substitution,'' follows if \(A\) is lower triangular. If \(A\in\mathcal{M}_{nn}\), back substitution takes roughly \(2(1+\cdots+(n-1))+n\in O(n^2)\) operations. There are \(n\) divisions total, and for row \(j\), there are \(j\) multiplication and \(j\) addition operations.
        \\
        \\
        The idea behind Gaussian elimination is to perform row operations to transform the matrix \(A\) into an upper triangular matrix, and then apply back substitution. We refer the reader to examples in Chapter~\ref{chapter:syslineq}. To formalize this, consider the below pseudocode, which reveals the \(O(n^3)\) complexity.
        \begin{algorithm}[H] 
            \begin{algorithmic}[1]
                \Require \(A\in\mathcal{M}_{nn}\) 
                \Procedure{Gaussian\_Elimination}{$A,\vec{b}$} 
                    \State \(M\gets [A|\vec{b}]\)
                    \For{\(j\in\{1,\ldots,n\}\)}
                        \If{\(m_{kj}=0\) for all \(k\geq j\)} \Return{Error: \(A\) is Singular}
                        \EndIf
                        \If{\(m_{jj}=0\) but \(m_{kj}\neq0\) for some \(k>j\)} \(\langle k\rangle \leftrightarrow \langle j\rangle\)
                        \EndIf
                        \For{\(i \in\{j+1,\ldots, n\}\)}
                            \State \(\ell_{ij}=\frac{m_{ij}}{m_{jj}}\)
                            \State \(\langle i\rangle\gets -\ell_{ij}\langle j\rangle+\langle i\rangle\)
                        \EndFor
                    \EndFor
                \EndProcedure 
            \end{algorithmic}
            \caption{Gaussian Elimination}
            \label{alg:gaussianelim}
        \end{algorithm}
        \vphantom
        \\
        \\
        Note that \(\langle i\rangle\), as used in Algorithm~\ref{alg:gaussianelim}, refers to the \(i\)th row of the system, just like the convention used in Chapter~\ref{chapter:syslineq}.
        \\
        \\
        Furthermore, note that if we have several linear systems to solve with the same matrix \(A\), we can perform one sequence of row operations on \(A\), and apply these to all systems in parallel. 
        \\
        \\
        Recall that by performing row operations, what we are really doing is multiplying \(A\) by elementary matrices.
        \begin{definition}{\Stop\,\,Elementary Matrices}{elemmat}
            The elementary matrix associated with a row operation on a matrix with \(m\) rows is the \(m\times m\) matrix obtained by applying the row operation to \(I_m\).
        \end{definition}
        \vphantom
        \\
        \\
        To transform \(A\) into an upper triangular matrix, the elementary matrices are all lower triangular. Formally, we can write
        \begin{equation*}
            U=(L_{n-1}\cdots L_1)A, \quad \vec{c}=(L_{n-1}\cdots L_1)\vec{b}
        \end{equation*} 
        where \(L_i\) are the lower triangular elementary matrices and \(U\) is the upper triangular matrix we desire. So,
        \begin{equation*}
            A=L_1^{-1}\cdots L_{n-1}^{-1}U.
        \end{equation*}
        It is not too difficult to show that the product of lower triangular matrices is lower triangular. Let \(L=L_1^{-1}\cdots L_{n-1}^{-1}\). So, with \(A=LU\), we have factorized \(A\) as a product of a lower triangular and an upper triangular matrix. This procedure is called the \(LU\) factorization. Going back to our system of equations, if \(A\vec{x}=\vec{b}\), we indeed have \(LU\vec{x}=\vec{b}\). Let \(\vec{y}=U\vec{x}\). To find \(\vec{y}\), we may use forward substitution on \(L\) and \(\vec{b}\) in \(O(n^2)\) time, and similarly to find \(\vec{x}\), we can use back substitution on \(U\) and \(\vec{y}\) in \(O(n^2)\) time. The \(O(n^3)\) time to compute the \(LU\) factorization is amortized over several solves.

\pagebreak

\section{Lecture 2: Nov. 15, 2024}

    \subsection{Combatting Instability: Pivoting Strategies}

        Here, we show that the standard Gaussian elimination is unstable. Consider
        \begin{equation*}
            A_\epsilon=\begin{bmatrix}
                \epsilon & 1 \\
                1 & -\epsilon
            \end{bmatrix},\quad\vec{x}_\epsilon=\frac{1}{1+\epsilon^2}\begin{bmatrix}
                2+\epsilon \\ 1-2\epsilon
            \end{bmatrix},\quad
            \vec{b}=\begin{bmatrix}
                1 \\ 2
            \end{bmatrix}.
        \end{equation*}
        Note
        \begin{equation*}
            A_\epsilon^{-1}=\frac{1}{1+\epsilon^2}A_\epsilon.
        \end{equation*}
        If we take \(\epsilon=10^{-6}\). Then, the Gaussian elimination procedure on \([A|\vec{b}]\) gives us the triangular system
        \begin{equation*}
            \begin{bmatrix}
                10^{-6} & 1 & | & 1 \\
                0 & -10^{-6}-10^6 & | & 2-10^6
            \end{bmatrix}
        \end{equation*}
        But, say our machine only has \(5\) digits of precision. So, really, we instead have
        \begin{equation*}
            \begin{bmatrix}
                10^{-6} & 1 & | & 1 \\
                0 & -10^6 & | & -10^6
            \end{bmatrix},
        \end{equation*}
        yielding \(x_2=1\) and \(x_1=10^6(1-1)=0\), which is disastrous. With more precision, we'd get \(x_2\approx 1\) and \(x_1\approx 10^6(1-x_2)\). Nothing about \(A_\epsilon\) should bring about this bad behavior; the condition number of \(A_\epsilon\) is \(1\).
        \\
        \\
        The issue arises from pivoting on the small entry \(10^{-6}\). It turns out that small pivots are bad for stability. This issue gives rise to two stategies for selecting a pivot:
        \begin{enumerate}
            \item Partial Pivoting: For each column, find the maximal element in absolute value, and permute rows until the maximal element is in the pivot position. Then, pivot on the maximal element.
            \item Total (Complete) Pivoting: Find the maximal element in absolute value in the matrix, and permute rows and columns until the maximal element is in the pivot position. Then, pivot on the maximal element.
        \end{enumerate}
        For both methods, we must be sure to store the permutations in some way; permutation matrices take a lot of space, so we can store the data in vectors. Let \(\vec{\sigma}_i\) be a pointer to the \(i\)th row in memory\footnote{Here, \(\vec{\sigma}\) contains all pointers, and \(\vec{\sigma}_i\) is the \(i\)th entry of \(\vec{\sigma}\).}, and similarly let \(\vec{\tau}_j\) be the pointer to the \(j\)th column. We initialize \(\vec{\sigma}_i=i\) and \(\vec{\tau}_j=j\). For partial pivoting, we only need to keep track of \(\vec{\sigma}\), but for total pivoting, we need both \(\vec{\sigma}\) and \(\vec{\tau}\). We provide pseudocode for partial pivoting in~\ref{alg:gaussianelimpp}.
        \begin{algorithm}[H] 
            \begin{algorithmic}[1]
                \Require \(A\in\mathcal{M}_{nn}\) 
                \Procedure{Gaussian\_Elimination\_Partial\_Pivoting}{$A,\vec{b}$} 
                    \State \(M\gets [A|\vec{b}]\)
                    \State \((\vec{\sigma}_i)_{i=1}^n\gets (i)_{i=1}^n\)
                    \For{\(j\in\{1,\ldots,n\}\)}
                        \If{\(m_{\vec{\sigma}_kj}=0\) for all \(k\geq j\)} \Return{Error: \(A\) is Singular}
                        \State \(i\gets \argmax_{i>j} |m_{\vec{\sigma}_ij}|\)
                        \State \(\vec{\sigma}_i\leftrightarrow \vec{\sigma}_j\)  
                        \EndIf
                        \For{\(i \in\{j+1,\ldots, n\}\)}
                            \State \(\ell_{\vec{\sigma}_ij}=\frac{m_{\vec{\sigma}_ij}}{m_{\vec{\sigma}_jj}}\)
                            \For{\(k\in\{j+1,\ldots,n+1\}\)}
                                \State \(m_{\vec{\sigma}_ik}\gets-\ell_{\vec{\sigma}_ij}m_{\vec{\sigma}_jk}+m_{\vec{\sigma}_ik}\)
                            \EndFor
                        \EndFor
                    \EndFor
                \EndProcedure 
            \end{algorithmic}
            \caption{Gaussian Elimination with Partial Pivoting}
            \label{alg:gaussianelimpp}
        \end{algorithm}
        \vphantom
        \\
        \\
        Note that while Algorithm~\ref{alg:gaussianelimpp} may look complicated, it is really just a modification of Algorithm~\ref{alg:gaussianelim} where we have selected the largest column element to be the pivot, and interchanged rows, storing the changes in \(\vec{\sigma}\), as necessary. Both pivoting strategies give rise to different equivalent systems, which we can solve as previously shown. If we use \(LU\) factorization with partial pivoting, it is important to note \(A\neq LU\), but \(A_{\vec{\sigma}}=LU\) where \(A_{\vec{\sigma}}\) has the elements of \(A\), but with the permutation of the rows we computed. Really, we're solving the system \(A_{\vec{\sigma}}\vec{x}=LU\vec{x}=\vec{b}_\sigma\) by forward solving on \(L\) and \(\vec{b}_\sigma\), and backward solving on \(U\) and \(\vec{y}\).
        \\
        \\
        Partial pivoting is generally stable, as it avoids the small pivots that led to catastrophe in the first example. It indeed almost solves all of our problems; however, there exist pathological examples where partial pivoting is unstable, but these examples are isolated, and peturbations fix the issues. Total pivoting truly fixes all our problems, but is not used as much in practice due to the added cost of searching for the maximal element.

\pagebreak

\section{Lecture 3: Nov. 18, 2024}

    \subsection{Projection Matrices and the \(QR\) Decomposition}

        In this section, we revisit the Gram-Schmidt procedure stated in Theorem~\ref{thm:gramschmidt}. Recall that given an arbitrary basis \(B=\{\vec{v}_1,\ldots,\vec{v}_n\}\) of a vector space \(V\), we can use the Gram-Schmidt procedure to find an orthonormal basis. Suppose that \(B_\perp=\{\vec{q}_1,\ldots,\vec{q}_n\}\) is an orthonormal basis. Then, 
        \begin{equation*}
            Q=\begin{bmatrix}
                \vec{q}_1 & \cdots & \vec{q}_n
            \end{bmatrix}
        \end{equation*}
        is unitary. Note that with respect to \(B_\perp\), we can write
        \begin{equation*}
            \vec{v}_k=\sum_{i=1}^k\iprod{\vec{q}_i}{\vec{v}_i}\vec{q}_i.
        \end{equation*}
        Now, we can form
        \begin{equation*}
            A=\begin{bmatrix}
                \vec{v}_1 & \cdots & \vec{v}_n
            \end{bmatrix}.
        \end{equation*}
        Then, with 
        \begin{equation*}
            R=\begin{bmatrix}
                \iprod{\vec{q}_1}{\vec{v}_1} & \iprod{\vec{q}_1}{\vec{v}_2} & \iprod{\vec{q}_1}{\vec{v}_3}  & \cdots & \iprod{\vec{q}_1}{\vec{v}_n} \\
                0 &  \iprod{\vec{q}_2}{\vec{v}_2} & \iprod{\vec{q}_2}{\vec{v}_3} & \cdots & \iprod{\vec{q}_2}{\vec{v}_n} \\
                0 & 0 & \iprod{\vec{q}_3}{\vec{v}_3} & \cdots & \iprod{\vec{q}_3}{\vec{v}_n} \\
                \vdots & \vdots & \vdots & \ddots & \vdots \\
                0 & 0 & 0 & \cdots & \iprod{\vec{q}_n}{\vec{v}_n}
            \end{bmatrix},
        \end{equation*}
        we have that
        \begin{equation*}
            A=QR.
        \end{equation*}
        This procedure is called the \(QR\) decomposition of \(A\); however, the Gram-Schmidt process is very unstable.
        \\
        \\
        We will, instead, consider the problem of finding \(Q\) such that \(R=Q^*A\). We study the process using Householder reflector matrices, but we remark that this may be done via Givens rotations.
        Consider the following definition.
        \begin{definition}{\Stop\,\,Projector Matrices}{projectormatrices}
            We say that a matrix \(P\in\mathcal{M}_{nn}\) is a projector matrix if \(P^2=P\). Furthermore, \(P\) is an orthogonal projector if \(P\) is a projector and \(P^*=P\).
        \end{definition}
        \vphantom
        \\
        \\
        Note that
        \begin{equation*}
            \ker P=\{\vec{y}\in\mathbb{C}^n:\vec{y}=\vec{x}-P\vec{x}\}.
        \end{equation*}
        since
        \begin{equation*}
            P(I-P)=P-P^2=P-P=0.
        \end{equation*}
        \pagebreak
        We have that \(P\) and \(I-P\) are both projector matrices with \(\range(P)=\ker(I-P)\) and \(\range(I-P)=\ker(P)\). If \(P\) is an orthogonal projector, we have the following useful properties.
        \begin{enumerate}
            \item \(\range(P)\perp \ker(P)\), and
            \item \(\range(P)+\ker(P)=\mathbb{C}^n\).
        \end{enumerate}
        We use the notion of a Householder reflector.
        \begin{definition}{\Stop\,\,Householder Reflectors}{householderreflector}
            Given \(\vec{u}\in \mathbb{R}^n\) with \(||\vec{u}||=1\). Then, define the Householder reflector
            \begin{equation*}
                H_{\vec{u}}=I-2\vec{u}\vec{u}^*.
            \end{equation*}
        \end{definition}
        \vphantom
        \\
        \\
        Geometrically, the Householder matrix \(H_{\vec{u}}\) is a reflection about \(\vec{u}^\perp=\{\vec{v}\in\mathbb{R}^n:\iprod{\vec{v}}{\vec{u}}=0\}\). It is straightforward to check that \(H_{\vec{u}}^2=H_{\vec{u}}\) and \(H_{\vec{u}}^*=H_{\vec{u}}\). Consider the following lemma.
        \begin{lemma}{\Stop\,\,Householder Interchange}{householderinterchange}
            Let \(\vec{v},\vec{w}\in\mathbb{R}^n\) with \(||\vec{v}||=||\vec{w}||\). Let \(\vec{u}=\frac{\vec{v}-\vec{w}}{||\vec{v}-\vec{w}||}\). Then,
            \begin{equation*}
                H_{\vec{u}}\vec{v}=\vec{w},\quad H_{\vec{u}}\vec{w}=\vec{v}.
            \end{equation*}
            \begin{proof}
                We have
                \begin{align*}
                    H_{\vec{u}}\vec{v}&=(I-2\vec{u}\vec{u}^*)\vec{v} \\
                    &=\vec{v}-\frac{2(\vec{v}-\vec{w})(\vec{v}-\vec{w})^*\vec{v}}{||\vec{v}-\vec{w}||^2} \\
                    &=\vec{v}-\frac{2(||\vec{v}||^2-\iprod{\vec{w}}{\vec{v}})(\vec{v}-\vec{w})}{2||\vec{v}||^2-2\iprod{\vec{v}}{\vec{w}}} \\
                    &=\vec{v}-(\vec{v}-\vec{w})=\vec{w}.
                \end{align*}
                The proof of the other equation is analogous.
            \end{proof}
        \end{lemma}
        \vphantom
        \\
        \\
        Let
        \begin{equation*}
            A=\begin{bmatrix}
                \vec{v}_1 & \cdots & \vec{v}_n
            \end{bmatrix}.
        \end{equation*}
        We will now form \(R\) by repeatedly applying Householder matrices. Define, if \(\vec{v}\neq c\vec{e}_1\), for some constant \(c\),
        \begin{equation*}
            \vec{w}_1=||\vec{v}_1||\vec{e}_1,\quad \vec{u}_1=\frac{\vec{v}_1-||\vec{v}_1||\vec{e}_1}{||\vec{v}_1-||\vec{v}_1||\vec{e}_1||},\quad H_1=I-2\vec{u}_1\vec{u}_1^*.
        \end{equation*}
        If \(\vec{v}=c\vec{e}_1\), let \(\vec{u}_1=\vec{0}\) and \(H_1=I\). By Lemma \ref{lem:householderinterchange}, \(H_1\vec{v}_1=\vec{w}_1\). So, the first column of \(A_2=H_1A\) is in the desired upper triangular form. We have
        \begin{equation*}
           A_2=H_1A=\begin{bmatrix}
                r_{11} & \star & \cdots & \star \\
                       0 &       &        &       \\
                  \vdots &       &     A' &       \\
                       0 &       &        &
              \end{bmatrix}.
        \end{equation*}
        We want to iterate this process to eventually get an upper triangular matrix. But, we don't want to mess up our previous work. So, at the \(k\)th step of this process, let \(\tilde{\vec{v}}_k\) be the \(k\)th column of \(A\) with the first \(k-1\) entries set to \(0\). Then, let
        \begin{equation*}
            \vec{w}_1=\tilde{\vec{v}}_k-||\tilde{\vec{v}}_k||\vec{e}_k,\quad \vec{u}_k=\begin{cases} \frac{\vec{w}_k}{||\vec{w}_k||} & \vec{w}_k\neq\vec{0} \\ \vec{0} & \vec{w}_k=\vec{0} \end{cases}, \quad H_k=I-2\vec{u}_k\vec{u}_k^*,\quad A_{k+1}=H_kA_k.
        \end{equation*}
        In \(n-1\) steps, we have 
        \begin{equation*}
            R=H_{n-1}\cdots H_1A=Q^*A,\quad Q=H_1\cdots H_{n-1},
        \end{equation*}
        so
        \begin{equation*}
            A=QR,
        \end{equation*}
        as desired. This algorithm for \(QR\) decomposition is much more stable than the naive form given by Gram-Schmidt. The complexity is \(O(n^3)\).
        % Given \(\vec{v}\in\mathbb{R}^n\), we can write \(\vec{v}=\alpha\vec{w}+\vec{y}\) with \(\iprod{\vec{y}}{\vec{w}}=0\). Then,
        % \begin{equation*}
        %     H_{\vec{w}}\vec{w}=(I-2\vec{w}\vec{w}^*)\vec{w}=\vec{w}-2\vec{w}(\vec{w}^*\vec{w})=-\vec{w}
        % \end{equation*}
        % and
        % \begin{equation*}
        %     H_{\vec{w}}\vec{y}=(I-2\vec{w}\vec{w}^*)\vec{y}=\vec{y}-2\vec{w}(\vec{w}^*\vec{y})=\vec{y}.
        % \end{equation*}
        % So, 
        % \begin{equation*}
        %     H_{\vec{w}}\vec{y}=\vec{y}-\alpha\vec{w}.
        % \end{equation*}
        % Now, we wish to find \(\vec{w}_1=H_{\vec{w}_1}\vec{a}_1=r_{11}\vec{e}_1\), so \(|r_{11}|=||\vec{a}_1||\). We have
        % \begin{equation*}
        %     \vec{\tilde{w}}_1=\vec{a}_1+\sgn(a_{11})+\vec{e}_1.
        % \end{equation*}
        % Then, \(\vec{w}_1=\frac{\vec{\tilde{w}}_1}{||\vec{\tilde{w}}_1||}\). Now, we consider
        % \begin{equation*}
        %     H_{\vec{w}_1}A=\begin{bmatrix}
        %         r_{11} & \star & \cdots & \star \\
        %                0 &       &        &       \\
        %           \vdots &       &     A' &       \\
        %                0 &       &        &
        %       \end{bmatrix}.
        % \end{equation*}
        % We can continue this process to get
        % \begin{equation*}
        %     H_k=\begin{bmatrix}
        %         I_{k-1} & 0 \\
        %         0 & H_{\vec{w}_k}
        %     \end{bmatrix}.
        % \end{equation*}
        % Finally, 
        % \begin{equation*}
        %     H_n\cdots H_1A=R.
        % \end{equation*}
        % So, 
        % \begin{equation*}
        %     A=\underbrace{(H_1\cdots H_n)}_{Q}R.
        % \end{equation*}

\pagebreak

\section{Lecture 4: Nov. 20, 2024}

    \subsection{Eigendecomposition} \label{sec:eigendecomposition}

        Recall Definitions,~\ref{def:eigenvaluesandvectors},~\ref{def:eigenspace}, and Theorem~\ref{thm:findeigenvs}. These definitions define eigenvalues, eigenvectors, and the eigenspace of a matrix. The theorem shows how to find eigenspaces. In this section, we want to write a matrix \(A\in\mathcal{M}_{nn}^\mathbb{C}\) as
        \begin{equation*}
            A=UDU^*
        \end{equation*}
        where \(U\) is unitary and \(D\) is diagonal. We introduce a few basic definitions.
        \begin{definition}{\Stop\,\,Spectrums}{spectrums}
            Let \(A\in\mathcal{M}_{nn}^\mathbb{C}\). The set of eigenvalues of \(A\) is called the spectrum \(\sigma(A)\) of \(A\).
        \end{definition}
        \begin{definition}{\Stop\,\,Eigenpair}{eigenpair}
            Let \(A\in\mathcal{M}_{nn}^\mathbb{C}\). If \(\lambda\) is an eigenvalue of \(A\) with eigenvector \(\vec{v}\), then \((\lambda,\vec{v})\) is called an eigenpair of \(A\).
        \end{definition}
        \vphantom
        \\
        \\
        Recall that we can characterize the eigenspace of a matrix \(A\) as \(E_\lambda=\ker(A-\lambda I)\). We also recall the notions of algebraic and geometric multiplicity in Definitions~\ref{def:algmultdiag} and~\ref{def:geomultdiag}, and the notion of similarity from Definition~\ref{def:similarity}; intuitively, if \(A\) is similar to \(B\), then \(A\) and \(B\) represent the same linear transformation, just with respect to different bases. When these bases are orthonormal, we have the change of basis matrix \(P\) as unitary. It is not difficult to show that the spectrums of similar matrices are equivalent.
        \\
        \\
        Not all matrices are diagonalizable, see Theorems~\ref{thm:diagonalizability},~\ref{thm:diagandrank},~\ref{thm:diaglintrans}, and~\ref{thm:alggeomultdiag}. Proceeding, we introduce a similar notion in Theorem~\ref{thm:schurdecomp}, which will help us state and prove the spectral theorems. We turn to the notation of linear transformations, for clarity. We reference \cite{treil2017linear}.
        \\
        \\
        First, consider the following definitions, analogous to Definitions~\ref{def:adjoint} and~\ref{def:hermitskewhermit}.
        \begin{definition}{\Stop\,\,The Adjoint of a Linear Transformation}{adjointlintrans}
            Let \(L:V\to W\) be a linear transformation from the inner product spaces \(V\) and \(W\). Then, the adjoint \(L^*:W\to V\) is the transformation satisfying
            \begin{equation*}
                \iprod{L(\vec{v})}{\vec{w}}=\iprod{\vec{v}}{L^*(\vec{w})}
            \end{equation*}
            for all \(\vec{v}\in V\) and \(\vec{w}\in W\). Note that \(L^*\) is well-defined. Let \(B_V\) and \(B_W\) be orthonormal bases for \(V\) and \(W\) respectively. Let \(A_{B_VB_W}\) be the matrix of \(L\) with respect to \(B_V\) and \(B_W\). Then, we can define \(L^*\) by defining its matrix \(A^*_{B_WB_V}\) as
            \begin{equation*}
                A^*_{B_WB_V}=A_{B_VB_W}^*.
            \end{equation*}
        \end{definition}
        \begin{definition}{\Stop\,\,Self-Adjoint Operators}{selfadjointops}
            Let \(L:V\to V\) be a linear operator on the inner product space \(V\). We say \(L\) is self-adjoint if \(L=L^*\).
        \end{definition}
        \begin{definition}{\Stop\,\,Normal Operators}{normalops}
            Let \(L:V\to V\) be a linear operator on the inner product space \(V\). We say \(L\) is normal if \(L\circ L^*=L^*\circ L\).
        \end{definition}
        \begin{theorem}{\Stop\,\,Schur Decompositions}{schurdecomp}
            Let \(L:V\to V\) be a linear operator on the complex inner product space \(V\). There exists an orthonormal basis \(\{\vec{v}_1,\ldots,\vec{v}_n\}\) of \(V\) such that the matrix representation of \(L\) in this basis is upper triangular. That is, any matrix \(A\in\mathcal{M}_{nn}^{\mathbb{C}}\) can be written as
            \begin{equation*}
                A=UTU^*
            \end{equation*}
            where \(U\) is unitary and \(T\) is upper triangular.
            \begin{proof}
                We proceed by induction on \(n=\dim V\). With \(n=1\), the theorem is trivial, since any \(1\times 1\) matrix is upper triangular.
                \\
                \\
                Suppose that for \(\dim E=n-1\), the theorem is true. Let \(\lambda_1\) be an eigenvalue of \(L\) with corresponding unit eigenvector \(\vec{u}_1\). Let \(E=\vec{u}_1^\perp\), and let \(\{\vec{v}_2,\ldots,\vec{v}_n\}\) be an arbitrary orthonormal basis of \(E\); note that \(\dim E=n-1\). We have that
                \begin{equation*}
                    \{\vec{u}_1,\vec{v}_2,\ldots,\vec{v}_n\}
                \end{equation*}
                is an orthonormal basis of \(V\), and in this basis, the matrix of \(L\) is
                \begin{equation*}
                    A_0=\begin{bmatrix}
                            \lambda_1 & \star & \cdots & \star \\
                                0 &       &        &       \\
                            \vdots &       &     A_1 &       \\
                                0 &       &        &
                        \end{bmatrix}.
                \end{equation*}
                The matrix \(A_1\) corresponds to a linear transformation in \(E\), say \(L_1:E\to E\). Since \(\dim E=n-1\), there exists an basis \(\{\vec{u}_2,\ldots,\vec{u}_n\}\) where the matrix of \(L_1\) with respect to this basis is upper triangular. Then, in the basis of \(V\) given by \(\{\vec{u}_1,\ldots,\vec{u}_n\}\), \(L\) has the form of \(A_0\) with \(A_1\) being upper triangular. So, the matrix of \(L\) with respect to the basis \(\{\vec{u}_1,\ldots,\vec{u}_n\}\) is upper triangular, and we are done.
            \end{proof}
        \end{theorem}
        \vphantom
        \\
        \\
        Note that the Schur decomposition exists for all \(A\in\mathcal{M}_{nn}^\mathbb{C}\). We now introduce the spectral theorems, and prove them with the Schur decomposition.
        \begin{theorem}{\Stop\,\,Spectral Theorem (Self-Adjoint)}{spectralthmselfadj}
            Let \(L:V\to V\) be a self-adjoint operator on the inner product space \(V\). Then, all eigenvalues of \(L\) are real and there exists an orthonormal basis of eigenvectors of \(L\) in \(V\). In matrix form, if \(A\in\mathcal{M}_{nn}^\mathbb{C}\) is Hermitian, then we may write \(A=UDU^*\) where \(U\) is unitary and \(D\) is diagonal with real entries.
            \begin{proof}
                First, apply Theorem~\ref{thm:schurdecomp} on \(L\) to find an orthonormal basis in \(V\) where the matrix \(T\) of \(L\) with respect to the basis is upper triangular. Then, \(T\) must also be self-adjoint (i.e. Hermitian). Upper triangular matrices are Hermitian if and only if they are diagonal with real entries. So, \(D=T\) is diagonal with real entries, as desired.
            \end{proof}
        \end{theorem}
        \begin{theorem}{\Stop\,\,Spectral Theorem (Normal)}{spectralthmnormal}
            Let \(L:V\to V\) be a normal operator on the complex inner product space \(V\). Then, there exists an orthonormal basis of eigenvectors of \(L\) in \(V\). In matrix form, if \(A\in\mathcal{M}_{nn}^\mathbb{C}\) is normal, then we may write \(A=UDU^*\) where \(U\) is unitary and \(D\) is diagonal.
            \begin{proof}
                First, apply Theorem~\ref{thm:schurdecomp} on \(L\) to find an orthonormal basis in \(V\) where the matrix \(T\) of \(L\) with respect to the basis is upper triangular. Normal upper triangular matrices are diagonal, and we are done, taking \(D=T\).
            \end{proof}
        \end{theorem}
        \vphantom
        \\
        \\
        The spectral theorems, Theorems \ref{thm:spectralthmselfadj} and \ref{thm:spectralthmnormal} provide sufficient conditions on linear operators, and equivalently \(n\times n\) matrices, to be diagonalizable. As a note, if the matrix in Theorem \ref{thm:spectralthmselfadj} is real, \(U\) can also be chosen to be real. If \(D\) is real in Theorem \ref{thm:spectralthmnormal}, then the matrix must be Hermitian.
        \\
        \\
        We conclude this section by providing an impossibility result on analytically finding eigendecompositions in Theorem~\ref{thm:impossibilityeigendecomp}.
        \begin{theorem}{\Stop\,\,Impossibility: Direct Methods for Eigendecomposition}{impossibilityeigendecomp}
            There are no analytic methods to find the eigendecomposition, or all eigenvalues, of \(A\in\mathcal{M}_{nn}^\mathbb{C}\) when \(n\geq 5\).
            \begin{proof}
                Suppose such a method exists. It is easy to use this method to algebraically solve any polynomial of degree greater than or equal to \(5\). This is not possible.
            \end{proof}
        \end{theorem}

    \pagebreak

\section{Lecture 5: Nov. 22, 2024}

    \subsection{The \(QR\) Iteration to Compute Schur Decompositions}

        Consider the following algorithm to compute the Schur decomposition of \(A\).
        \begin{algorithm}[H]
            \begin{algorithmic}[1]
                \Require \(A\in\mathcal{M}_{nn}\) 
                \Procedure{QR\_Iteration}{$A,\epsilon$} 
                    \State \(A_0\gets A\)
                    \While{\(\max\{\text{lower triangular part of } A_k\}<\epsilon\)}
                        \State \((Q_k,R_k)\gets \Call{QR}{A_k}\)
                        \State \(A_{k+1}\gets R_kQ_k\) \Comment{\(R_kQ_k=Q_k^*AQ_k\)}
                    \EndWhile
                \EndProcedure 
            \end{algorithmic}
            \caption{\(QR\) Iteration to Compute Schur Decompositions}
            \label{alg:qriteration}
        \end{algorithm}
        \vphantom
        \\
        \\
        Then, if \(A\) is Hermitian, \(\lim_{k\to\infty}A_k=T\) and \(\lim_{k\to\infty}(Q_1\cdots Q_k)=U\), where \(T\) and \(U\) are as in Theorem~\ref{thm:schurdecomp}. The convergence of this algorithm is linear; we don't have that kind of time. Importantly, the algorithm may not even terminate, as sometimes, for non-Hermitian matrices, we see \(2\times 2\) block matrices that correspond to complex conjugate eigenvalue pairs; these will not go to \(0\). On the bright side, Algorithm \ref{alg:qriteration} is stable if one uses the Householder method of finding \(QR\) decompositions.
        \\
        \\
        It turns out that the convergence of the \(QR\) iteration is dependent on the ratios of the eigenvalues \(\frac{\lambda_i}{\lambda_{i+1}}\) when the \(\lambda_i\) are ordered by absolute value. The \(QR\) iteration also seems to order the eigenvalues! Furthermore, we need to address thed fact that the cost per iteration is \(O(n^3)\), and there are a large number of iterations.
        \\
        \\
        Given these problems, how do we accelerate the \(QR\) iteration? Maybe \(A_0=A\) isn't a good idea. What about selecting \(A_0\) similar to \(A\) such that \(\textsc{QR}(A_0)\) is cheap? What we may do is use Householder reflections, but as similarity transformations; that is, note that \(A\) and \(HAH\) are similar. However, using \(H_1,\ldots,H_k\) to make \(A\) upper triangular, as in the Householder \(QR\) method does not work, as it would imply a direct method for eigendecomposition, contradicting Theorem~\ref{thm:impossibilityeigendecomp}. The next best thing is to make \(A\) upper Hessenberg.
        \begin{definition}{\Stop\,\,Upper Hessenberg Matrices}{upperhessenberg}
            Let \(A\in\mathcal{M}_{nn}\). Then, \(A\) is upper Hessenberg if \(a_{ij}=0\) for all \(i>j+1\); that is, all entries above the first superdiagonal are \(0\).
        \end{definition}
        \pagebreak
        \vphantom
        \\
        \\
        Starting with the matrix \(A\), define
        \begin{equation*}
            \vec{x}_1=\begin{bmatrix}
                0 & a_{21} & \cdots & a_{n1}
            \end{bmatrix}^\top,\quad \vec{y}_1=\begin{bmatrix}
                0 & \pm ||\vec{x}_1|| & \cdots & 0
            \end{bmatrix}^\top.
        \end{equation*}
        As a note, we usually choose the \(\pm ||\vec{x}_1||\) term to be opposite sign to \(a_{21}\). Now, let
        \begin{equation*}
            H_1=I-2\vec{u}_1\vec{u}_1^*,\quad \vec{u}_1=\frac{\vec{x}_1-\vec{y}_1}{||\vec{x}_1-\vec{y}_1||}.
        \end{equation*}
        Then, setting \(A_1=H_1AH_1\) produces another matrix. Let \(a_{ij,1}\) denote the \(ij\)th entry of \(A_1\). We continue the iteration as follows. Let
        \begin{equation*}
            \vec{x}_{k}=\begin{bmatrix}
                0 & 0 & a_{k+1,k,k-1} & \cdots & a_{n,k,k-1}
            \end{bmatrix}^\top,\quad \vec{y}_{k}=\begin{bmatrix}
                0 & 0 & \pm ||\vec{x}_{k}|| & \cdots & 0
            \end{bmatrix}^\top.
        \end{equation*}
        with \(\pm ||\vec{x}_k||\) chosen to be opposite in sign to \(a_{k+1,k,k}\). Then, let
        \begin{equation*}
            H_k=I-2\vec{u}_k\vec{u}_k^*,\quad \vec{u}_k=\frac{\vec{x}_k-\vec{y}_k}{||\vec{x}_k-\vec{y}_k||}.
        \end{equation*}
        Proceeding, we get the upper Hessenberg matrix
        \begin{equation*}
            \mathcal{H}=H_{n}\cdots H_1 A H_1\cdots H_{n}.
        \end{equation*}
        Then, we may apply Algorithm \ref{alg:qriteration} with \(A_0=\mathcal{H}\). If \(A\) is Hermitian, then \(\mathcal{H}\) turns out to be tridiagonal.

\pagebreak

\section{Lecture 5: Dec. 2, 2024}

    \subsection{The Power Method}

        In this section, suppose \(A\in\mathcal{M}_{nn}^{\mathbb{R}}\) is diagonalizable. It may be useful to recall the results of Theorems \ref{thm:diagonalizability}, \ref{thm:diaglintrans}, and \ref{thm:raisematpow}. The first two theorems state that matrices and linear operators, respectively, are diagonalizable if and only if there exists a set of \(n\) linearly independent eigenvectors. The third provides a method for us to easily raise diagonalizable matrices to powers.
        \\
        \\
        Let \(\sigma(A)=\{\lambda_1,\ldots,\lambda_n\}\) with corresponding eigenvectors \(\vec{v}_1,\ldots,\vec{v}_n\). Now, inductively define, for \(\vec{v}\), 
        \begin{equation*}
            \vec{v}^{(0)}=\vec{v},\quad \vec{v}^{(k)}=A\vec{v}^{(k-1)}.
        \end{equation*}
        It is easy to verify that \(\vec{v}^{(k)}=A^k\vec{v}\). Now, we expand this equation in terms of the eigenbasis \(\{\vec{v}_1,\ldots,\vec{v}_n\}\). We may write
        \begin{align*}
            \vec{v}^{(k)}=A^k\vec{v}&=A^k(c_1\vec{v}_1+\cdots+c_n\vec{v}_n) \\
            &=c_1\lambda_1^k\vec{v}_1+\cdots+c_n\lambda_n^k\vec{v}_n.
        \end{align*}
        Now, suppose that \(\lambda_1\in\mathbb{R}\) and \(|\lambda_1|>|\lambda_2|\geq\cdots\geq|\lambda_n|\). Then, as we increase \(k\), we see that
        \begin{equation*}
            \vec{v}^{(k)}=A^k\vec{v}\approx c_1\lambda_1^k\vec{v}_1.
        \end{equation*}
        since
        \begin{equation*}
            \vec{v}^{(k)}=\lambda_1^k\left(c_1\vec{v}_1+c_2\left(\frac{\lambda_2}{\lambda_1}\right)^k+\cdots+c_{n-1}\left(\frac{\lambda_{n-1}}{\lambda_1}\right)^k+c_n\left(\frac{\lambda_n}{\lambda_1}\right)^k\vec{v}_n\right).
            % \frac{\lambda_1^k\left(c_1\vec{v}_1+c_2\left(\frac{\lambda_2}{\lambda_1}\right)^k+\cdots+c_{n-1}\left(\frac{\lambda_{n-1}}{\lambda_1}\right)^k+c_n\left(\frac{\lambda_n}{\lambda_1}\right)^k\vec{v}_n\right)}{|\lambda_1|^k\left|\left|c_1\vec{v}_1+c_2\left(\frac{\lambda_2}{\lambda_1}\right)^k+\cdots+c_{n-1}\left(\frac{\lambda_{n-1}}{\lambda_1}\right)^k+c_n\left(\frac{\lambda_n}{\lambda_1}\right)^k\vec{v}_n\right|\right|}.
        \end{equation*}
        The order of convergence is linear and depends on the spectral gap \(\left|\frac{\lambda_1}{\lambda_2}\right|\).
        \\
        \\
        This leads us to the power method. But first, consider the following theorem.
        \begin{theorem}{\Stop\,\,Rayleigh Quotient}{rayleighquotient}
            Let \(A\in\mathcal{M}_{nn}\). Then, if \(\vec{v}\) is an eigenvector of \(A\), then the corresponding eigenvalue \(\lambda\) is given by
            \begin{equation*}
                \lambda=\frac{\iprod{A\vec{x}}{\vec{x}}}{\iprod{\vec{x}}{\vec{x}}}.
            \end{equation*}
            \begin{proof}
                We see that
                \begin{equation*}
                    \lambda=\frac{\lambda\iprod{\vec{x}}{\vec{x}}}{\iprod{\vec{x}}{\vec{x}}}=\frac{\iprod{\lambda\vec{x}}{\vec{x}}}{\iprod{\vec{x}}{\vec{x}}}=\frac{\iprod{A\vec{x}}{\vec{x}}}{\iprod{\vec{x}}{\vec{x}}},
                \end{equation*}
                as desired.
            \end{proof}
        \end{theorem}
        \pagebreak
        \vphantom
        \\
        \\
        We provide pseudocode for the power method with initial guess \(\vec{v}\) in Algorithm \ref{alg:powermethod}.
        \begin{algorithm}[H] 
            \begin{algorithmic}[1]
                \Require \(A\in\mathcal{M}_{nn}\), \(A\) Diagonalizable 
                \Procedure{Power Method}{$A,\vec{v},k$} 
                    \For{\(i\in\{1,\ldots,k\}\)}
                        \State \(\vec{v}\gets A\vec{v}\)
                        \State \(\vec{v}\gets \frac{\vec{v}}{||\vec{v}||}\)
                    \EndFor
                    \State \(\lambda_1\gets \frac{\iprod{A\vec{v}}{\vec{v}}}{\iprod{\vec{v}}{\vec{v}}}\)
                    \State \Return \(\left(\lambda_1,\vec{v}\right)\)
                \EndProcedure 
            \end{algorithmic}
            \caption{Power Method}
            \label{alg:powermethod}
        \end{algorithm}
        \vphantom
        \\
        \\
        Importantly, to converge, note that \(A\) must have a dominant eigenvalue with \(\lambda_1\in\mathbb{R}\) and \(|\lambda_1|>|\lambda_2|\geq\cdots\geq|\lambda_n|\). Furthermore, we also require \(\iprod{\vec{v}}{\vec{v}_1}\neq0\) where \(\vec{v}_1\) is an eigenvector associated with \(\lambda_1\). The second requirement is easy to satisfy, as we can, and often do, pick \(\vec{v}\) at random.
        \\
        \\
        Say, now we want to find \(\lambda_n\in\mathbb{R}\) with \(\lambda_n<\lambda_{n-1}\leq\cdots\leq \lambda_1\). It turns out that we can apply the power method on \(A^{-1}\). Consider the following lemma.
        \begin{lemma}{\Stop\,\,Eigenvalues of an Inverse Matrix}{eigenvaluesinversemat}
            Let \(A\in\mathcal{M}_{nn}\) be nonsingular. Then, \(\lambda\) is an eigenvalue of \(A\) with eigenvector \(\vec{v}\) if and only if \(\frac{1}{\lambda}\) is an eigenvalue of \(A^{-1}\) with eigenvector \(\vec{v}\).
            \begin{proof}
                Suppose \(\lambda\) is an eigenvalue of \(A\) with eigenvector \(\vec{v}\). Let \(\gamma\) be an eigenvalue of \(A^{-1}\) with eigenvector \(\vec{v}\). We show that \(\gamma=\frac{1}{\lambda}\). We have that
                \begin{align*}
                    \vec{v}=A^{-1}A\vec{v}=A^{-1}\lambda\vec{v}=\lambda A^{-1}\vec{v}=\lambda\gamma\vec{v}.
                \end{align*}
                So, \(1=\gamma\lambda\), and \(\gamma=\frac{1}{\lambda}\), as desired. The other direction is identical.
            \end{proof}
        \end{lemma}
        \vphantom
        \\
        \\
        Then, invoking \(\Call{Power\_Method}{A^{-1},\cdot,\cdot}\) will provide us the largest eigenvalue, say \(\gamma\), of \(A^{-1}\). By an immediate corollary of Lemma \ref{lem:eigenvaluesinversemat}, \(\lambda=\frac{1}{\gamma}\) is an eigenvalue of \(A\) with the same eigenvector. Note \(\lambda\) is maximal since \(\gamma\) is minimal.
        \\
        \\
        We briefly discuss the shifted power method. For \(\mu\in\mathbb{R}\), we can invoke \(\Call{Power\_Method}{A-\mu I,\cdot,\cdot}\). The eigenvalues of \(A-\mu I\) are \(\lambda_1-\mu,\ldots,\lambda_n-\mu\). Choosing \(\mu\) carefully may improve the rate of convergence.
        \\
        \\
        What if we do both?
        \\
        \\
        The inverse shifted power method invokes \(\Call{Power\_Method}{(A-\mu I)^{-1},\cdot,\cdot}\). The eigenvalues are \(\frac{1}{\lambda_1-\mu},\ldots,\frac{1}{\lambda_n-\mu}\). The iteration converges to \((\lambda_k,\vec{v}_k)\) for which \(\mu\) is closest to \(\lambda_k\); this follows since we will find the largest eigenvalue of the form \(\frac{1}{\lambda_k-\mu_k}\). But then \(|\lambda_k-\mu_k|\) must be minimized. The rate of convergence is \(\frac{\frac{1}{|\lambda_j-\mu|}}{\frac{1}{\lambda_k-\mu}}=\left|\frac{\lambda_k-\mu}{\lambda_j-\mu}\right|\) where \(|\lambda_k-\mu|\) is closest to \(0\) and \(|\lambda_j-\mu|\) is the next closest to \(0\). 
        \\
        \\
        To further optimize, we may use the variable inverse shifted power method. Let \(\mu_0\) be an input parameter, and for step \(k\), let \(\mu_{k-1}=\lambda_{k-1}\). Then, apply the inverse shifted power method. This is superlinear convergence. If \(A\in\mathcal{M}_{nn}\), we have quadratic convergence, and if \(A\) is symmetric, and we use Rayleigh shifts, we have cubic convergence.

% \pagebreak

% \section{Lecture \(\spadesuit\): Bonus}

%     \subsection{The Polar and Singular Value Decompositions (SVD)}

%         We believe that this set of notes is incomplete without the coverage of the Singular Value Decomposition (SVD). Hence, we provide it here, despite it not being discussed in either MATH2135 or APPM4600. Consider the following definition. We follow both \cite{treil2017linear} and \cite{olver2006applied}.
%         \begin{definition}{\Stop\,\,Positive Definite Operators}{positivedefinite}
%             A self-adjoint operator \(L:V\to V\) is positive semidefinite if
%             \begin{equation*}
%                 \iprod{L(\vec{v})}{\vec{v}}\geq0
%             \end{equation*}
%             for all \(\vec{v}\in V\). Additionally, we may say that \(L\) is positive definite if
%             \begin{equation*}
%                 \iprod{L(\vec{v})}{\vec{v}}>0
%             \end{equation*}
%             for \(\vec{v}\neq\vec{0}\). We write \(L\succeq \vec{0}\) if \(L\) is positive semidefinite, and \(L\succ \vec{0}\) if \(L\) is positive definite.
%         \end{definition}
%         \begin{theorem}{\Stop\,\,Positive Definiteness and Eigenvalues}{positivedefiniteeigs}
%             Let \(L:V\to V\) be a self-adjoint operator. Then, \(L\) is positive definite (semidefinite) if and only if all eigenvalues of \(L\) are positive (nonnegative).
%             \begin{proof}
%                 By Theorem \ref{thm:spectralthmselfadj}, we may find an orthonormal basis of eigenvectors such that the matrix of \(L\) in this basis is diagonal. Then, we note that a diagonal matrix is positive definite (semidefinite) if and only if its entries, the eigenvalues of \(L\), are positive (nonnegative).
%             \end{proof}
%         \end{theorem}

\pagebreak

\backmatter

\chapter{List of Theorems and Definitions}

    \begin{multicols}{3}
   \begin{center}
      \textbf{Chapter} \ref{chapter:vecmat}, Page \pageref{chapter:vecmat} \\
      \textit{Definition} \ref{def:zerovec}, Page \pageref{def:zerovec} \\
      \textit{Definition} \ref{def:vecequal}, Page \pageref{def:vecequal} \\
      \textit{Definition} \ref{def:vecmagn}, Page \pageref{def:vecmagn} \\
      \textit{Definition} \ref{def:scalmult}, Page \pageref{def:scalmult} \\
      \textit{Theorem} \ref{thm:scalmultandmagn}, Page \pageref{thm:scalmultandmagn} \\
      \textit{Definition} \ref{def:vectdir}, Page \pageref{def:vectdir} \\
      \textit{Definition} \ref{def:unitvec}, Page \pageref{def:unitvec} \\
      \textit{Theorem} \ref{thm:nonzerovecnonzeromagn}, Page \pageref{thm:nonzerovecnonzeromagn} \\
      \textit{Theorem} \ref{thm:unitvecdir}, Page \pageref{thm:unitvecdir} \\
      \textit{Definition} \ref{def:addsubvec}, Page \pageref{def:addsubvec} \\
      \textit{Theorem} \ref{thm:addscalmult}, Page \pageref{thm:addscalmult} \\
      \textit{Theorem} \ref{thm:scalmultzero}, Page \pageref{thm:scalmultzero} \\
      \textit{Definition} \ref{def:matrices}, Page \pageref{def:matrices} \\
      \textit{Definition} \ref{def:sqmatrices}, Page \pageref{def:sqmatrices} \\
      \textit{Definition} \ref{def:diagmatrices}, Page \pageref{def:diagmatrices} \\
      \textit{Definition} \ref{def:identitymatrices}, Page \pageref{def:identitymatrices} \\
      \textit{Definition} \ref{def:zeromatrices}, Page \pageref{def:zeromatrices} \\
      \textit{Definition} \ref{def:uppertriangularmatrices}, Page \pageref{def:uppertriangularmatrices} \\
      \textit{Definition} \ref{def:lowertriangularmatrices}, Page \pageref{def:lowertriangularmatrices} \\
      \textit{Definition} \ref{def:matset}, Page \pageref{def:matset} \\
      \textit{Definition} \ref{def:matrixaddition}, Page \pageref{def:matrixaddition} \\
      \textit{Definition} \ref{def:scalmultmat}, Page \pageref{def:scalmultmat} \\
      \textit{Definition} \ref{def:transpose}, Page \pageref{def:transpose} \\
      \textit{Definition} \ref{def:symmetricmatrices}, Page \pageref{def:symmetricmatrices} \\
      \textit{Theorem} \ref{thm:transprop}, Page \pageref{thm:transprop} \\
      \textit{Definition} \ref{def:trace}, Page \pageref{def:trace} \\
      \textit{Theorem} \ref{thm:sumdiftrans}, Page \pageref{thm:sumdiftrans} \\
      \textit{Theorem} \ref{thm:sqsymskew}, Page \pageref{thm:sqsymskew} \\
      \textit{Definition} \ref{def:dotprod}, Page \pageref{def:dotprod} \\
      \textit{Theorem} \ref{thm:dotprodprop}, Page \pageref{thm:dotprodprop} \\
      \textit{Theorem} \ref{thm:angletwovec}, Page \pageref{thm:angletwovec} \\
      \textit{Theorem} \ref{thm:lemmacauchyschwarzineq}, Page \pageref{thm:lemmacauchyschwarzineq} \\
      \textit{Theorem} \ref{thm:cauchyschwarzineq}, Page \pageref{thm:cauchyschwarzineq} \\
      \textit{Theorem} \ref{thm:triineq}, Page \pageref{thm:triineq} \\
      \textit{Definition} \ref{def:proj}, Page \pageref{def:proj} \\
      \textit{Theorem} \ref{thm:sumparperpproj}, Page \pageref{thm:sumparperpproj} \\
      \textit{Theorem} \ref{thm:projline}, Page \pageref{thm:projline} \\
      \textit{Definition} \ref{def:matmul}, Page \pageref{def:matmul} \\
      \textit{Theorem} \ref{thm:propmatmul}, Page \pageref{thm:propmatmul} \\
      \textit{Definition} \ref{def:matpow}, Page \pageref{def:matpow} \\
      \textbf{Chapter} \ref{chapter:syslineq}, Page \pageref{chapter:syslineq} \\
      \textit{Definition} \ref{def:lineq}, Page \pageref{def:lineq} \\
      \textit{Definition} \ref{def:syslineq}, Page \pageref{def:syslineq} \\
      \textit{Theorem} \ref{thm:charsollinsys}, Page \pageref{thm:charsollinsys} \\
      \textit{Theorem} \ref{thm:rowops}, Page \pageref{thm:rowops} \\
      \textit{Definition} \ref{def:rowechelon}, Page \pageref{def:rowechelon} \\
      \textit{Definition} \ref{def:redrowechelon}, Page \pageref{def:redrowechelon} \\
      \textit{Theorem} \ref{thm:numsollinsys}, Page \pageref{thm:numsollinsys} \\
      \textit{Definition} \ref{def:homosys}, Page \pageref{def:homosys} \\
      \textit{Theorem} \ref{thm:solstohomosys}, Page \pageref{thm:solstohomosys} \\
      \textit{Definition} \ref{def:sysequiv}, Page \pageref{def:sysequiv} \\
      \textit{Definition} \ref{def:rowequiv}, Page \pageref{def:rowequiv} \\
      \textit{Definition} \ref{def:equivrel}, Page \pageref{def:equivrel} \\
      \textit{Theorem} \ref{thm:equivrelrowsysequiv}, Page \pageref{thm:equivrelrowsysequiv} \\
      \textit{Theorem} \ref{thm:rowsyseq}, Page \pageref{thm:rowsyseq} \\
      \textit{Theorem} \ref{thm:uniquenessredrow}, Page \pageref{thm:uniquenessredrow} \\
      \textit{Definition} \ref{def:rank}, Page \pageref{def:rank} \\
      \textit{Theorem} \ref{thm:numsolshomosys}, Page \pageref{thm:numsolshomosys} \\
      \textit{Definition} \ref{def:lincomb}, Page \pageref{def:lincomb} \\
      \textit{Definition} \ref{def:rowspace}, Page \pageref{def:rowspace} \\
      \textit{Theorem} \ref{thm:translincomb}, Page \pageref{thm:translincomb} \\
      \textit{Theorem} \ref{thm:rowequivequalrowspc}, Page \pageref{thm:rowequivequalrowspc} \\
      \textit{Definition} \ref{def:linmaps}, Page \pageref{def:linmaps} \\
      \textit{Definition} \ref{def:inverse}, Page \pageref{def:inverse} \\
      \textit{Theorem} \ref{thm:invcommute}, Page \pageref{thm:invcommute} \\
      \textit{Definition} \ref{def:singularity}, Page \pageref{def:singularity} \\
      \textit{Theorem} \ref{thm:uniquenessinv}, Page \pageref{thm:uniquenessinv} \\
      \textit{Definition} \ref{def:nonsingmat}, Page \pageref{def:nonsingmat} \\
      \textit{Theorem} \ref{thm:propnonsingmat}, Page \pageref{thm:propnonsingmat} \\
      \textit{Theorem} \ref{thm:matexplaw}, Page \pageref{thm:matexplaw} \\
      \textit{Theorem} \ref{thm:2by2inv}, Page \pageref{thm:2by2inv} \\
      \textit{Theorem} \ref{thm:uniquenessofsol}, Page \pageref{thm:uniquenessofsol} \\
      \textbf{Chapter} \ref{chapter:deteigen}, Page \pageref{chapter:deteigen} \\
      \textit{Theorem} \ref{thm:areadet}, Page \pageref{thm:areadet} \\
      \textit{Theorem} \ref{thm:voldet}, Page \pageref{thm:voldet} \\
      \textit{Definition} \ref{def:submatrix}, Page \pageref{def:submatrix} \\
      \textit{Definition} \ref{def:minor}, Page \pageref{def:minor} \\
      \textit{Definition} \ref{def:cofactor}, Page \pageref{def:cofactor} \\
      \textit{Definition} \ref{def:det}, Page \pageref{def:det} \\
      \textit{Theorem} \ref{thm:uppertriangulardet}, Page \pageref{thm:uppertriangulardet} \\
      \textit{Theorem} \ref{thm:detrowops}, Page \pageref{thm:detrowops} \\
      \textit{Theorem} \ref{thm:invdet}, Page \pageref{thm:invdet} \\
      \textit{Theorem} \ref{thm:detprop}, Page \pageref{thm:detprop} \\
      \textit{Definition} \ref{def:similarity}, Page \pageref{def:similarity} \\
      \textit{Definition} \ref{def:diagonalizability}, Page \pageref{def:diagonalizability} \\
      \textit{Definition} \ref{def:eigenvaluesandvectors}, Page \pageref{def:eigenvaluesandvectors} \\
      \textit{Definition} \ref{def:eigenspace}, Page \pageref{def:eigenspace} \\
      \textit{Theorem} \ref{thm:findeigenvs}, Page \pageref{thm:findeigenvs} \\
      \textit{Theorem} \ref{thm:diagonalization}, Page \pageref{thm:diagonalization} \\
      \textit{Definition} \ref{def:linindep}, Page \pageref{def:linindep} \\
      \textit{Theorem} \ref{thm:diagonalizability}, Page \pageref{thm:diagonalizability} \\
      \textit{Theorem} \ref{thm:diagandrank}, Page \pageref{thm:diagandrank} \\
      \textit{Theorem} \ref{thm:raisematpow}, Page \pageref{thm:raisematpow} \\
      \textit{Definition} \ref{def:compnum}, Page \pageref{def:compnum} \\
      \textit{Definition} \ref{def:compvcspace}, Page \pageref{def:compvcspace} \\
      \textit{Definition} \ref{def:depcompops}, Page \pageref{def:depcompops} \\
      \textit{Theorem} \ref{thm:propcomp}, Page \pageref{thm:propcomp} \\
      \textit{Theorem} \ref{thm:fundthmalg}, Page \pageref{thm:fundthmalg} \\
      \textit{Definition} \ref{def:compvcadd}, Page \pageref{def:compvcadd} \\
      \textit{Definition} \ref{def:compscalmult}, Page \pageref{def:compscalmult} \\
      \textit{Definition} \ref{def:compdotprod}, Page \pageref{def:compdotprod} \\
      \textit{Definition} \ref{def:compmagn}, Page \pageref{def:compmagn} \\
      \textit{Definition} \ref{def:adjoint}, Page \pageref{def:adjoint} \\
      \textit{Theorem} \ref{thm:transposedot}, Page \pageref{thm:transposedot} \\
      \textit{Theorem} \ref{thm:adjointdot}, Page \pageref{thm:adjointdot} \\
      \textit{Definition} \ref{def:hermitskewhermit}, Page \pageref{def:hermitskewhermit} \\
      \textbf{Chapter} \ref{chapter:vcspcs}, Page \pageref{chapter:vcspcs} \\
      \textit{Definition} \ref{def:vcspc}, Page \pageref{def:vcspc} \\
      \textit{Theorem} \ref{thm:derpropvcspc1}, Page \pageref{thm:derpropvcspc1} \\
      \textit{Theorem} \ref{thm:derpropvcspc2}, Page \pageref{thm:derpropvcspc2} \\
      \textit{Theorem} \ref{thm:derpropvcspc3}, Page \pageref{thm:derpropvcspc3} \\
      \textit{Theorem} \ref{thm:derpropvcspc4}, Page \pageref{thm:derpropvcspc4} \\
      \textit{Definition} \ref{def:subspc}, Page \pageref{def:subspc} \\
      \textit{Theorem} \ref{thm:showsubspc}, Page \pageref{thm:showsubspc} \\
      \textit{Theorem} \ref{thm:eigsubspc}, Page \pageref{thm:eigsubspc} \\
      \textit{Definition} \ref{def:finitelincomb}, Page \pageref{def:finitelincomb} \\
      \textit{Theorem} \ref{thm:lincombsubspc}, Page \pageref{thm:lincombsubspc} \\
      \textit{Definition} \ref{def:span}, Page \pageref{def:span} \\
      \textit{Theorem} \ref{thm:spanchrs}, Page \pageref{thm:spanchrs} \\
      \textit{Theorem} \ref{thm:twosubvcspcspan}, Page \pageref{thm:twosubvcspcspan} \\
      \textit{Theorem} \ref{thm:spanintsubspc}, Page \pageref{thm:spanintsubspc} \\
      \textit{Theorem} \ref{thm:spanrowspc}, Page \pageref{thm:spanrowspc} \\
      \textit{Theorem} \ref{thm:vcspan}, Page \pageref{thm:vcspan} \\
      \textit{Definition} \ref{def:linindepdep}, Page \pageref{def:linindepdep} \\
      \textit{Definition} \ref{def:genlinindepdep}, Page \pageref{def:genlinindepdep} \\
      \textit{Theorem} \ref{thm:showlinindep1}, Page \pageref{thm:showlinindep1} \\
      \textit{Theorem} \ref{thm:showlinindep2}, Page \pageref{thm:showlinindep2} \\
      \textit{Theorem} \ref{thm:onetwosetsdep}, Page \pageref{thm:onetwosetsdep} \\
      \textit{Theorem} \ref{thm:finitesubsetcontain0}, Page \pageref{thm:finitesubsetcontain0} \\
      \textit{Theorem} \ref{thm:linindepnonemptsets}, Page \pageref{thm:linindepnonemptsets} \\
      \textit{Theorem} \ref{thm:lindeprn}, Page \pageref{thm:lindeprn} \\
      \textit{Theorem} \ref{thm:indlindep}, Page \pageref{thm:indlindep} \\
      \textit{Definition} \ref{def:basis}, Page \pageref{def:basis} \\
      \textit{Theorem} \ref{thm:basislemma}, Page \pageref{thm:basislemma} \\
      \textit{Theorem} \ref{thm:basisequivcard}, Page \pageref{thm:basisequivcard} \\
      \textit{Definition} \ref{def:dimension}, Page \pageref{def:dimension} \\
      \textit{Theorem} \ref{thm:spandim1}, Page \pageref{thm:spandim1} \\
      \textit{Theorem} \ref{thm:spandim2}, Page \pageref{thm:spandim2} \\
      \textit{Theorem} \ref{thm:lindim1}, Page \pageref{thm:lindim1} \\
      \textit{Theorem} \ref{thm:lindim2}, Page \pageref{thm:lindim2} \\
      \textit{Theorem} \ref{thm:dimsubspc}, Page \pageref{thm:dimsubspc} \\
      \textit{Theorem} \ref{thm:diagrev1}, Page \pageref{thm:diagrev1} \\
      \textit{Theorem} \ref{thm:basisforspan}, Page \pageref{thm:basisforspan} \\
      \textit{Theorem} \ref{thm:basisexpand}, Page \pageref{thm:basisexpand} \\
      \textit{Definition} \ref{def:orderedbases}, Page \pageref{def:orderedbases} \\
      \textit{Definition} \ref{def:coordswrtbasis}, Page \pageref{def:coordswrtbasis} \\
      \textit{Theorem} \ref{thm:coords}, Page \pageref{thm:coords} \\
      \textit{Theorem} \ref{thm:propcoords}, Page \pageref{thm:propcoords} \\
      \textit{Definition} \ref{def:transitionmatrix}, Page \pageref{def:transitionmatrix} \\
      \textit{Theorem} \ref{thm:transitionmatrix}, Page \pageref{thm:transitionmatrix} \\
      \textit{Theorem} \ref{thm:proptransmat}, Page \pageref{thm:proptransmat} \\
      \textit{Theorem} \ref{thm:diagrev2}, Page \pageref{thm:diagrev2} \\
      \textbf{Chapter} \ref{chapter:lintrans}, Page \pageref{chapter:lintrans} \\
      \textit{Definition} \ref{def:lineartransformation}, Page \pageref{def:lineartransformation} \\
      \textit{Theorem} \ref{thm:proplintrans}, Page \pageref{thm:proplintrans} \\
      \textit{Theorem} \ref{thm:compositionslintrans}, Page \pageref{thm:compositionslintrans} \\
      \textit{Definition} \ref{def:linearoperator}, Page \pageref{def:linearoperator} \\
      \textit{Definition} \ref{def:idlinop}, Page \pageref{def:idlinop} \\
      \textit{Definition} \ref{def:zerolinop}, Page \pageref{def:zerolinop} \\
      \textit{Theorem} \ref{thm:lintranssubspc}, Page \pageref{thm:lintranssubspc} \\
      \textit{Theorem} \ref{thm:lineartransformationsbases}, Page \pageref{thm:lineartransformationsbases} \\
      \textit{Theorem} \ref{thm:matlintrans}, Page \pageref{thm:matlintrans} \\
      \textit{Theorem} \ref{thm:matricesconsdiffbases}, Page \pageref{thm:matricesconsdiffbases} \\
      \textit{Theorem} \ref{thm:simmatlinops}, Page \pageref{thm:simmatlinops} \\
      \textit{Theorem} \ref{thm:matcomplintrans}, Page \pageref{thm:matcomplintrans} \\
      \textit{Definition} \ref{def:kernel}, Page \pageref{def:kernel} \\
      \textit{Definition} \ref{def:range}, Page \pageref{def:range} \\
      \textit{Theorem} \ref{thm:kerransubspc}, Page \pageref{thm:kerransubspc} \\
      \textit{Theorem} \ref{thm:findker}, Page \pageref{thm:findker} \\
      \textit{Theorem} \ref{thm:findrange}, Page \pageref{thm:findrange} \\
      \textit{Definition} \ref{def:nullity}, Page \pageref{def:nullity} \\
      \textit{Definition} \ref{def:ranklintrans}, Page \pageref{def:ranklintrans} \\
      \textit{Theorem} \ref{thm:dimthmrn}, Page \pageref{thm:dimthmrn} \\
      \textit{Theorem} \ref{thm:detinjsurj}, Page \pageref{thm:detinjsurj} \\
      \textit{Theorem} \ref{thm:detinjsurjequivdim}, Page \pageref{thm:detinjsurjequivdim} \\
      \textit{Theorem} \ref{thm:injlinindepsurjspan}, Page \pageref{thm:injlinindepsurjspan} \\
      \textit{Definition} \ref{def:isomorphisms}, Page \pageref{def:isomorphisms} \\
      \textit{Definition} \ref{def:invtrans}, Page \pageref{def:invtrans} \\
      \textit{Theorem} \ref{thm:isoinv}, Page \pageref{thm:isoinv} \\
      \textit{Theorem} \ref{thm:findinv}, Page \pageref{thm:findinv} \\
      \textit{Theorem} \ref{thm:isopreslinindepspan}, Page \pageref{thm:isopreslinindepspan} \\
      \textit{Definition} \ref{def:isovec}, Page \pageref{def:isovec} \\
      \textit{Theorem} \ref{thm:isoequivrel}, Page \pageref{thm:isoequivrel} \\
      \textit{Theorem} \ref{thm:dimensionthm}, Page \pageref{thm:dimensionthm} \\
      \textit{Theorem} \ref{thm:equivdim}, Page \pageref{thm:equivdim} \\
      \textit{Theorem} \ref{thm:allisofn}, Page \pageref{thm:allisofn} \\
      \textit{Definition} \ref{def:eigenvaluesandvectorslintrans}, Page \pageref{def:eigenvaluesandvectorslintrans} \\
      \textit{Definition} \ref{def:eigenspacelintrans}, Page \pageref{def:eigenspacelintrans} \\
      \textit{Theorem} \ref{thm:findeigenvslin}, Page \pageref{thm:findeigenvslin} \\
      \textit{Definition} \ref{def:diaglintrans}, Page \pageref{def:diaglintrans} \\
      \textit{Theorem} \ref{thm:diaglintrans}, Page \pageref{thm:diaglintrans} \\
      \textit{Theorem} \ref{thm:eigenvecsdisteigenvalslinindep}, Page \pageref{thm:eigenvecsdisteigenvalslinindep} \\
      \textit{Definition} \ref{def:algmultdiag}, Page \pageref{def:algmultdiag} \\
      \textit{Definition} \ref{def:geomultdiag}, Page \pageref{def:geomultdiag} \\
      \textit{Theorem} \ref{thm:alggeomultdiag}, Page \pageref{thm:alggeomultdiag} \\
      \textit{Theorem} \ref{thm:unioninterbaseseigenspc}, Page \pageref{thm:unioninterbaseseigenspc} \\
      \textit{Theorem} \ref{thm:procdiaglinops}, Page \pageref{thm:procdiaglinops} \\
      \textbf{Chapter} \ref{chapter:ortho}, Page \pageref{chapter:ortho} \\
      \textit{Definition} \ref{def:innerprod}, Page \pageref{def:innerprod} \\
      \textit{Theorem} \ref{thm:innerprodprops}, Page \pageref{thm:innerprodprops} \\
      \textit{Definition} \ref{def:norms}, Page \pageref{def:norms} \\
      \textit{Theorem} \ref{thm:propnorm}, Page \pageref{thm:propnorm} \\
      \textit{Theorem} \ref{thm:cauchyschwarzgen}, Page \pageref{thm:cauchyschwarzgen} \\
      \textit{Theorem} \ref{thm:triineqgen}, Page \pageref{thm:triineqgen} \\
      \textit{Definition} \ref{def:distance}, Page \pageref{def:distance} \\
      \textit{Definition} \ref{def:angle}, Page \pageref{def:angle} \\
      \textit{Definition} \ref{def:orthogonality}, Page \pageref{def:orthogonality} \\
      \textit{Theorem} \ref{thm:orthlinindep}, Page \pageref{thm:orthlinindep} \\
      \textit{Definition} \ref{def:orthogonalbasis}, Page \pageref{def:orthogonalbasis} \\
      \textit{Definition} \ref{def:orthonormalbasis}, Page \pageref{def:orthonormalbasis} \\
      \textit{Theorem} \ref{thm:gramschmidt}, Page \pageref{thm:gramschmidt} \\
      \textit{Theorem} \ref{thm:innerproductspaceorthobasis}, Page \pageref{thm:innerproductspaceorthobasis} \\
      \textit{Theorem} \ref{thm:coordsorthogonal}, Page \pageref{thm:coordsorthogonal} \\
      \textit{Theorem} \ref{thm:coordsorthonormal}, Page \pageref{thm:coordsorthonormal} \\
      \textit{Definition} \ref{def:orthocomp}, Page \pageref{def:orthocomp} \\
      \textit{Theorem} \ref{thm:lemmafindorthocomp}, Page \pageref{thm:lemmafindorthocomp} \\
      \textit{Theorem} \ref{thm:subsub}, Page \pageref{thm:subsub} \\
      \textit{Theorem} \ref{thm:subspcs}, Page \pageref{thm:subspcs} \\
      \textit{Theorem} \ref{thm:orthocompambientspc}, Page \pageref{thm:orthocompambientspc} \\
      \textit{Theorem} \ref{thm:orthocompzerovec}, Page \pageref{thm:orthocompzerovec} \\
      \textit{Definition} \ref{def:projectionssubspcog}, Page \pageref{def:projectionssubspcog} \\
      \textit{Theorem} \ref{thm:projectionssubspcon}, Page \pageref{thm:projectionssubspcon} \\
      \textit{Theorem} \ref{thm:projthm}, Page \pageref{thm:projthm} \\
      \textbf{Appendix} \ref{appendix:a}, Page \pageref{appendix:a} \\
      \textit{Definition} \ref{def:proofs}, Page \pageref{def:proofs} \\
      \textbf{Appendix} \ref{appendix:b}, Page \pageref{appendix:b} \\
      \textit{Definition} \ref{def:functions}, Page \pageref{def:functions} \\
      \textit{Definition} \ref{def:imagespreimage}, Page \pageref{def:imagespreimage} \\
      \textit{Definition} \ref{def:rangef}, Page \pageref{def:rangef} \\
      \textit{Definition} \ref{def:injectivefunctions}, Page \pageref{def:injectivefunctions} \\
      \textit{Definition} \ref{def:surjectivefunctions}, Page \pageref{def:surjectivefunctions} \\
      \textit{Definition} \ref{def:bijectivefunctions}, Page \pageref{def:bijectivefunctions} \\
      \textit{Definition} \ref{def:comp}, Page \pageref{def:comp} \\
      \textit{Theorem} \ref{thm:compinjsur}, Page \pageref{thm:compinjsur} \\
      \textit{Definition} \ref{def:invfunc}, Page \pageref{def:invfunc} \\
      \textit{Theorem} \ref{thm:existinv}, Page \pageref{thm:existinv} \\
      \textit{Theorem} \ref{thm:uniqueinv}, Page \pageref{thm:uniqueinv} \\
      \textit{Theorem} \ref{thm:pigeonhole}, Page \pageref{thm:pigeonhole} \\
      \textit{Theorem} \ref{thm:extpigeonhole}, Page \pageref{thm:extpigeonhole} \\
   \end{center}
\end{multicols}


 % ~~~~~~~~~~~~~~~~~~~~~~TENTATIVE: PERHAPS A FUTURE EDITION~~~~~~~~~~~~~~~~~~~~~~
 \begin{comment}
 \begin{savequote}
    \includegraphics[scale=0.4]{Graphics/optimization.png}
\end{savequote}
    \chapter{Optimization in \(\mathbb{R}^n\) and the Hessian Matrix} \label{appendix:c}
    
        % WILL NOT APPEAR IN FIRST EDITION
% WILL WAIT UNTIL ORTHOGONAL DIAGONALIZATION IS COVERED PROPERLY
\section{Taylor's Theorem in \(\mathbb{R}^n\)}

    Before stating Taylor's Theorem, consider the following definition.
    \begin{definition}{\Stop\,\,Hyperspheres}{hypersphere}

        An open hypersphere centered at \(\vec{x}_0\in\mathbb{R}^n\) is defined by
        \begin{equation*}
            \{\vec{x}\in\mathbb{R}^n:0<||\vec{x}-\vec{x}_0||<r;r\in\mathbb{R}\}.
        \end{equation*}
        Similarly, a closed hypersphere centered at \(\vec{x}_0\) is defined by
        \begin{equation*}
            \{\vec{x}\in\mathbb{R}^n:0<||\vec{x}-\vec{x}_0||\leq r;r\in\mathbb{R}\}.
        \end{equation*}

    \end{definition}
    \vphantom
    \\
    \\
    Now, consider Taylor's Theorem, stated without proof.
    \begin{theorem}{\Stop\,\,Taylor's Theorem in \(\mathbb{R}^n\)}{taylorsthm}

        Let \(A\) be an open hypersphere centered at \(\vec{x}_0\in\mathbb{R}^n\) and \(\vec{u}\in\mathbb{R}^n\) be a unit vector. Then, let \(t\in\mathbb{R}\) where \(\vec{x}_0+t\vec{u}\in A\). Suppose \(f:A\to\mathbb{R}\) has continuous second-order partial derivatives throughout \(A\). Then, there exists some \(c\) with \(0\leq c\leq t\) such that
        \begin{equation*}
            f(\vec{x}_0+t\vec{u})=f(\vec{x}_0)+\sum_{i=1}^n \frac{\partial f}{\partial x_i}\Big|_{\vec{x}_0}(tu_i)+\frac{1}{2}\sum_{i=1}^n \frac{\partial^2f}{\partial x_i^2}\Big|_{\vec{x}_0+c\vec{u}}(t^2u_i^2)+\sum_{i=1}^n\sum_{j=i+1}^n\frac{\partial^2f}{\partial x_i\partial x_j}\Big|_{\vec{x}_0+c\vec{u}}(t^2u_iu_j).
        \end{equation*}
        
    \end{theorem}
    \pagebreak
    \vphantom
    \\
    \\
    Theorem \ref{thm:taylorsthm} is derived from Taylor's Theorem in \(\mathbb{R}\) by applying it to
    \begin{equation*}
        g(t)=f(\vec{x}_0+t\vec{u}).
    \end{equation*}
    Consider the following definition.
    \begin{definition}{\Stop\,\,The Hessian Matrix}{hessian}

        The matrix
        \begin{equation*}
            H=\begin{bmatrix}
                \frac{\partial^2f}{\partial x_1^2} & \cdots & \frac{\partial^2f}{\partial x_1\partial x_n} \\
                \vdots & \ddots & \vdots \\
                \frac{\partial^2f}{\partial x_n\partial x_1} & \cdots & \frac{\partial^2f}{\partial x_n^2}
            \end{bmatrix}.
        \end{equation*}
        is the \(n\times n\) Hessian matrix.
        
    \end{definition}
    \vphantom
    \\
    \\
    Let \(\vec{v}=[tu_1,\ldots,tu_n]^T=[v_1,\ldots,v_n]^T\) and recall that \(\nabla f=\left[\frac{\partial f}{\partial x_1},\ldots,\frac{\partial f}{\partial x_n}\right]\). Then, we can rewrite the formula in Theorem \ref{thm:taylorsthm} as
    \begin{equation*}
        f(\vec{x}_0+\vec{v})=f(\vec{x}_0)+\left(\nabla f\Big|_{\vec{x}_0}\right)\cdot\vec{v}+\frac{1}{2}\vec{v}^T\left(H\Big|_{\vec{x}_0+k\vec{v}}\right)\vec{v}
    \end{equation*}
    where \(0\leq k=\frac{c}{t}\leq 1\). This result is difficult to see in \(\mathbb{R}^n\), but we will provide a brief explanation of why it is true in \(\mathbb{R}^2\). In \(\mathbb{R}^2\), Theorem \ref{thm:taylorsthm} states
    \begin{align*}
        f(\vec{x}_0+t\vec{u})=&f(\vec{x}_0)+\frac{\partial f}{\partial x}\Big|_{\vec{x}_0}(tu_1)+\frac{\partial f}{\partial y}\Big|_{\vec{x}_0}(tu_2)\\&+\frac{1}{2}\frac{\partial^2 f}{\partial x^2}\Big|_{\vec{x}_0+c\vec{u}}(t^2u_1^2)+\frac{1}{2}\frac{\partial^2 f}{\partial y^2}\Big|_{\vec{x}_0+c\vec{u}}(t^2u_2^2)+\frac{\partial^2f}{\partial x\partial y}\Big|_{\vec{x}_0+c\vec{u}}(t^2u_1u_2).
    \end{align*}
    and \(\vec{v}=[tu_1,tu_2]^T=[v_1,v_2]^T\). We see that \(\frac{\partial f}{\partial x}\Big|_{\vec{x}_0}(tu_1)+\frac{\partial f}{\partial y}\Big|_{\vec{x}_0}(tu_2)=\left(\nabla f\Big|_{\vec{x}_0}\right)\cdot\vec{v}\). Then, consider
    \begin{align*}
        \frac{1}{2}\frac{\partial^2 f}{\partial x^2}(t^2u_1^2)+\frac{1}{2}\frac{\partial^2 f}{\partial y^2}(t^2u_2^2)+\frac{\partial^2f}{\partial x\partial y}(t^2u_1u_2)&=\frac{1}{2}v_1\left(\frac{\partial^2f}{\partial x^2}v_1+\frac{\partial^2f}{\partial x\partial y}v_2\right)+\frac{1}{2}v_2\left(\frac{\partial^2f}{\partial y\partial x}v_1+\frac{\partial^2f}{\partial y^2}v_2\right) \\
        &=\frac{1}{2}\vec{v}^T\begin{bmatrix} \frac{\partial^2f}{\partial x^2} & \frac{\partial^2f}{\partial x\partial y} \\ \frac{\partial^2f}{\partial y\partial x} & \frac{\partial^2f}{\partial y^2} \end{bmatrix}\vec{v} \\
        &=\frac{1}{2}\vec{v}^TH\vec{v}.
    \end{align*}
    Making the necessary substitutions provides the desired result.

\pagebreak

\section{Critical Points}

    Consider the following theorems and definitions.
    \begin{definition}{\Stop\,\,Local Maximums and Local Minimums}{localmaxmin}

        Let \(A\subseteq\mathbb{R}^n\). Then, \(f:A\to\mathbb{R}\) has a local maximum at \(\vec{x}_0\in A\) if and only if there exists an open neighborhood \(\mathcal{U}\) of \(\vec{x}_0\) such that \(f(\vec{x}_0)\geq f(\vec{x})\) for all \(\vec{x}\in\mathcal{U}\). Similarly, \(f\) has a local minimum at \(\vec{x}_0\) if and only if there exists some \(\mathcal{U}\) such that \(f(\vec{x}_0)\leq f(\vec{x})\) for all \(\vec{x}\in\mathcal{U}\).
        
    \end{definition}

\end{comment}

\bibliographystyle{alpha}
\pagebreak
\bibliography{Auxillary/refs}

\end{document}