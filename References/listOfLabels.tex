\begin{multicols}{2}
      \setlength{\parindent}{0pt}
      \footnotesize{
         \textbf{Chapter} \ref{chapter:vecmat}, \textsc{Page} \pageref{chapter:vecmat} \\
         \textsc{Definition} \ref{def:zerovec}, \textsc{Page} \pageref{def:zerovec} \textit{The Zero Vector} \\
         \textsc{Definition} \ref{def:vecequal}, \textsc{Page} \pageref{def:vecequal} \textit{Vector Equality} \\
         \textsc{Definition} \ref{def:vecmagn}, \textsc{Page} \pageref{def:vecmagn} \textit{Vector Magnitude} \\
         \textsc{Definition} \ref{def:scalmult}, \textsc{Page} \pageref{def:scalmult} \textit{Scalar Multiplication} \\
         \textsc{Theorem} \ref{thm:scalmultandmagn}, \textsc{Page} \pageref{thm:scalmultandmagn} \textit{Scalar Multiplication and Magnitude} \\
         \textsc{Definition} \ref{def:vectdir}, \textsc{Page} \pageref{def:vectdir} \textit{Vector Direction} \\
         \textsc{Definition} \ref{def:unitvec}, \textsc{Page} \pageref{def:unitvec} \textit{Unit Vectors} \\
         \textsc{Theorem} \ref{thm:nonzerovecnonzeromagn}, \textsc{Page} \pageref{thm:nonzerovecnonzeromagn} \textit{Nonzero Vector Implies Nonzero Magnitude} \\
         \textsc{Theorem} \ref{thm:unitvecdir}, \textsc{Page} \pageref{thm:unitvecdir} \textit{Unit Vectors Represent Direction} \\
         \textsc{Definition} \ref{def:addsubvec}, \textsc{Page} \pageref{def:addsubvec} \textit{Vector Addition} \\
         \textsc{Theorem} \ref{thm:addscalmult}, \textsc{Page} \pageref{thm:addscalmult} \textit{Properties of Vector Addition and Scalar Multiplication} \\
         \textsc{Theorem} \ref{thm:scalmultzero}, \textsc{Page} \pageref{thm:scalmultzero} \textit{Scalar Multiplication Producing the Zero Vector} \\
         \textsc{Definition} \ref{def:matrices}, \textsc{Page} \pageref{def:matrices} \textit{Matrices} \\
         \textsc{Definition} \ref{def:sqmatrices}, \textsc{Page} \pageref{def:sqmatrices} \textit{Square Matrices} \\
         \textsc{Definition} \ref{def:diagmatrices}, \textsc{Page} \pageref{def:diagmatrices} \textit{Diagonal Matrices} \\
         \textsc{Definition} \ref{def:identitymatrices}, \textsc{Page} \pageref{def:identitymatrices} \textit{The Identity Matrix} \\
         \textsc{Definition} \ref{def:zeromatrices}, \textsc{Page} \pageref{def:zeromatrices} \textit{The Zero Matrix} \\
         \textsc{Definition} \ref{def:uppertriangularmatrices}, \textsc{Page} \pageref{def:uppertriangularmatrices} \textit{Upper Triangular Matrices} \\
         \textsc{Definition} \ref{def:lowertriangularmatrices}, \textsc{Page} \pageref{def:lowertriangularmatrices} \textit{Lower Triangular Matrices} \\
         \textsc{Definition} \ref{def:matset}, \textsc{Page} \pageref{def:matset} \textit{The Set of All \(m\times n\) Matrices} \\
         \textsc{Definition} \ref{def:matrixaddition}, \textsc{Page} \pageref{def:matrixaddition} \textit{Matrix Addition} \\
         \textsc{Definition} \ref{def:scalmultmat}, \textsc{Page} \pageref{def:scalmultmat} \textit{Scalar Multiplication With Matrices} \\
         \textsc{Definition} \ref{def:transpose}, \textsc{Page} \pageref{def:transpose} \textit{Matrix Transpose} \\
         \textsc{Definition} \ref{def:symmetricmatrices}, \textsc{Page} \pageref{def:symmetricmatrices} \textit{Symmetric and Skew-Symmetric Matrices} \\
         \textsc{Theorem} \ref{thm:transprop}, \textsc{Page} \pageref{thm:transprop} \textit{Properties of the Transpose} \\
         \textsc{Definition} \ref{def:trace}, \textsc{Page} \pageref{def:trace} \textit{Trace} \\
         \textsc{Theorem} \ref{thm:sumdiftrans}, \textsc{Page} \pageref{thm:sumdiftrans} \textit{Sum and Difference of Matrices and Their Transpose} \\
         \textsc{Theorem} \ref{thm:sqsymskew}, \textsc{Page} \pageref{thm:sqsymskew} \textit{The Relation Between Square, Symmetric, and Skew-Symmetric} \\
         \textsc{Definition} \ref{def:dotprod}, \textsc{Page} \pageref{def:dotprod} \textit{The Dot Product} \\
         \textsc{Theorem} \ref{thm:dotprodprop}, \textsc{Page} \pageref{thm:dotprodprop} \textit{Properties of the Dot Product} \\
         \textsc{Theorem} \ref{thm:angletwovec}, \textsc{Page} \pageref{thm:angletwovec} \textit{The Angle Between Two Vectors} \\
         \textsc{Theorem} \ref{thm:lemmacauchyschwarzineq}, \textsc{Page} \pageref{thm:lemmacauchyschwarzineq} \textit{A Useful Lemma for the Cauchy-Schwarz Inequality} \\
         \textsc{Theorem} \ref{thm:cauchyschwarzineq}, \textsc{Page} \pageref{thm:cauchyschwarzineq} \textit{The Cauchy-Schwarz Inequality} \\
         \textsc{Theorem} \ref{thm:triineq}, \textsc{Page} \pageref{thm:triineq} \textit{The Triangle Inequality} \\
         \textsc{Definition} \ref{def:proj}, \textsc{Page} \pageref{def:proj} \textit{The Projection Onto a Vector} \\
         \textsc{Theorem} \ref{thm:sumparperpproj}, \textsc{Page} \pageref{thm:sumparperpproj} \textit{Sum of Parallel and Perpendicular Projections} \\
         \textsc{Theorem} \ref{thm:projline}, \textsc{Page} \pageref{thm:projline} \textit{Projections Depend on Lines} \\
         \textsc{Definition} \ref{def:matmul}, \textsc{Page} \pageref{def:matmul} \textit{Matrix Multiplication} \\
         \textsc{Theorem} \ref{thm:propmatmul}, \textsc{Page} \pageref{thm:propmatmul} \textit{Properties of Matrix Multiplication} \\
         \textsc{Definition} \ref{def:matpow}, \textsc{Page} \pageref{def:matpow} \textit{Raising a Matrix to a Power} \\
         \textbf{Chapter} \ref{chapter:syslineq}, \textsc{Page} \pageref{chapter:syslineq} \\
         \textsc{Definition} \ref{def:lineq}, \textsc{Page} \pageref{def:lineq} \textit{Linear Equations} \\
         \textsc{Definition} \ref{def:syslineq}, \textsc{Page} \pageref{def:syslineq} \textit{Systems of Linear Equations} \\
         \textsc{Theorem} \ref{thm:charsollinsys}, \textsc{Page} \pageref{thm:charsollinsys} \textit{Characterizing Solutions of Linear Systems} \\
         \textsc{Theorem} \ref{thm:rowops}, \textsc{Page} \pageref{thm:rowops} \textit{Row Operations} \\
         \textsc{Definition} \ref{def:rowechelon}, \textsc{Page} \pageref{def:rowechelon} \textit{Row Echelon Form} \\
         \textsc{Definition} \ref{def:redrowechelon}, \textsc{Page} \pageref{def:redrowechelon} \textit{Reduced Row Echelon Form} \\
         \textsc{Theorem} \ref{thm:numsollinsys}, \textsc{Page} \pageref{thm:numsollinsys} \textit{Number of Solutions to a Linear System} \\
         \textsc{Definition} \ref{def:homosys}, \textsc{Page} \pageref{def:homosys} \textit{Homogeneous Systems} \\
         \textsc{Theorem} \ref{thm:solstohomosys}, \textsc{Page} \pageref{thm:solstohomosys} \textit{Solutions to Homogeneous Systems} \\
         \textsc{Definition} \ref{def:sysequiv}, \textsc{Page} \pageref{def:sysequiv} \textit{Equivalence of Linear Systems} \\
         \textsc{Definition} \ref{def:rowequiv}, \textsc{Page} \pageref{def:rowequiv} \textit{Row Equivalence} \\
         \textsc{Definition} \ref{def:equivrel}, \textsc{Page} \pageref{def:equivrel} \textit{Equivalence Relations} \\
         \textsc{Theorem} \ref{thm:equivrelrowsysequiv}, \textsc{Page} \pageref{thm:equivrelrowsysequiv} \textit{System Equivalence and Row Equivalence are Equivalence Relations} \\
         \textsc{Theorem} \ref{thm:rowsyseq}, \textsc{Page} \pageref{thm:rowsyseq} \textit{Row Equivalence Implies System Equivalence} \\
         \textsc{Theorem} \ref{thm:uniquenessredrow}, \textsc{Page} \pageref{thm:uniquenessredrow} \textit{Uniqueness of Reduced Row Echelon Form} \\
         \textsc{Definition} \ref{def:rank}, \textsc{Page} \pageref{def:rank} \textit{Rank} \\
         \textsc{Theorem} \ref{thm:numsolshomosys}, \textsc{Page} \pageref{thm:numsolshomosys} \textit{Number of Solutions to Homogeneous Systems} \\
         \textsc{Definition} \ref{def:lincomb}, \textsc{Page} \pageref{def:lincomb} \textit{Linear Combinations} \\
         \textsc{Definition} \ref{def:rowspace}, \textsc{Page} \pageref{def:rowspace} \textit{Row Space} \\
         \textsc{Theorem} \ref{thm:translincomb}, \textsc{Page} \pageref{thm:translincomb} \textit{Transitivity of Linear Combinations} \\
         \textsc{Theorem} \ref{thm:rowequivequalrowspc}, \textsc{Page} \pageref{thm:rowequivequalrowspc} \textit{Row Equivalence Implies Equal Row Space} \\
         \textsc{Definition} \ref{def:linmaps}, \textsc{Page} \pageref{def:linmaps} \textit{Linear Maps} \\
         \textsc{Definition} \ref{def:inverse}, \textsc{Page} \pageref{def:inverse} \textit{Multiplicative Inverse of a Matrix} \\
         \textsc{Theorem} \ref{thm:invcommute}, \textsc{Page} \pageref{thm:invcommute} \textit{Inverse Commutativity} \\
         \textsc{Definition} \ref{def:singularity}, \textsc{Page} \pageref{def:singularity} \textit{Singularity} \\
         \textsc{Theorem} \ref{thm:uniquenessinv}, \textsc{Page} \pageref{thm:uniquenessinv} \textit{Uniqueness of the Inverse} \\
         \textsc{Definition} \ref{def:nonsingmat}, \textsc{Page} \pageref{def:nonsingmat} \textit{Negative Integral Powers of a Nonsingular Matrices} \\
         \textsc{Theorem} \ref{thm:propnonsingmat}, \textsc{Page} \pageref{thm:propnonsingmat} \textit{Properties of Nonsingular Matrices} \\
         \textsc{Theorem} \ref{thm:matexplaw}, \textsc{Page} \pageref{thm:matexplaw} \textit{Matrix Exponent Laws} \\
         \textsc{Theorem} \ref{thm:2by2inv}, \textsc{Page} \pageref{thm:2by2inv} \textit{\(2\times 2\) Inverse} \\
         \textsc{Theorem} \ref{thm:uniquenessofsol}, \textsc{Page} \pageref{thm:uniquenessofsol} \textit{Uniqueness of Solutions to Linear Systems} \\
         \textbf{Chapter} \ref{chapter:deteigen}, \textsc{Page} \pageref{chapter:deteigen} \\
         \textsc{Theorem} \ref{thm:areadet}, \textsc{Page} \pageref{thm:areadet} \textit{The Determinant Determines the Area in \(\mathbb {R}^2\)} \\
         \textsc{Theorem} \ref{thm:voldet}, \textsc{Page} \pageref{thm:voldet} \textit{The Determinant Determines the Volume in \(\mathbb {R}^3\)} \\
         \textsc{Definition} \ref{def:submatrix}, \textsc{Page} \pageref{def:submatrix} \textit{The \((i,j)\) Submatrix} \\
         \textsc{Definition} \ref{def:minor}, \textsc{Page} \pageref{def:minor} \textit{The \((i,j)\) Minor} \\
         \textsc{Definition} \ref{def:cofactor}, \textsc{Page} \pageref{def:cofactor} \textit{The \((i,j)\) Cofactor} \\
         \textsc{Definition} \ref{def:det}, \textsc{Page} \pageref{def:det} \textit{The Determinant} \\
         \textsc{Theorem} \ref{thm:uppertriangulardet}, \textsc{Page} \pageref{thm:uppertriangulardet} \textit{The Determinant of an Upper Triangular Matrix} \\
         \textsc{Theorem} \ref{thm:detrowops}, \textsc{Page} \pageref{thm:detrowops} \textit{Determinants and Row Operations} \\
         \textsc{Theorem} \ref{thm:invdet}, \textsc{Page} \pageref{thm:invdet} \textit{Inverses and Determinants} \\
         \textsc{Theorem} \ref{thm:detprop}, \textsc{Page} \pageref{thm:detprop} \textit{Properties of Determinants} \\
         \textsc{Definition} \ref{def:similarity}, \textsc{Page} \pageref{def:similarity} \textit{Similarity} \\
         \textsc{Definition} \ref{def:diagonalizability}, \textsc{Page} \pageref{def:diagonalizability} \textit{Diagonalizability} \\
         \textsc{Definition} \ref{def:eigenvaluesandvectors}, \textsc{Page} \pageref{def:eigenvaluesandvectors} \textit{Eigenvalues and Eigenvectors} \\
         \textsc{Definition} \ref{def:eigenspace}, \textsc{Page} \pageref{def:eigenspace} \textit{Eigenspace} \\
         \textsc{Theorem} \ref{thm:findeigenvs}, \textsc{Page} \pageref{thm:findeigenvs} \textit{Finding Eigenvectors and Eigenvalues} \\
         \textsc{Theorem} \ref{thm:diagonalization}, \textsc{Page} \pageref{thm:diagonalization} \textit{The Process of Diagonalization} \\
         \textsc{Definition} \ref{def:linindep}, \textsc{Page} \pageref{def:linindep} \textit{Linear Independence} \\
         \textsc{Theorem} \ref{thm:diagonalizability}, \textsc{Page} \pageref{thm:diagonalizability} \textit{Diagonalizability} \\
         \textsc{Theorem} \ref{thm:diagandrank}, \textsc{Page} \pageref{thm:diagandrank} \textit{Diagonalizability and Rank} \\
         \textsc{Theorem} \ref{thm:raisematpow}, \textsc{Page} \pageref{thm:raisematpow} \textit{Raising Matrices to Powers} \\
         \textsc{Definition} \ref{def:compnum}, \textsc{Page} \pageref{def:compnum} \textit{The Complex Numbers} \\
         \textsc{Definition} \ref{def:compvcspace}, \textsc{Page} \pageref{def:compvcspace} \textit{The Set \(\mathbb {C}^n\)} \\
         \textsc{Definition} \ref{def:depcompops}, \textsc{Page} \pageref{def:depcompops} \textit{Definitions of Operations With Complex Numbers} \\
         \textsc{Theorem} \ref{thm:propcomp}, \textsc{Page} \pageref{thm:propcomp} \textit{Properties of Complex Numbers} \\
         \textsc{Theorem} \ref{thm:fundthmalg}, \textsc{Page} \pageref{thm:fundthmalg} \textit{The Fundamental Theorem of Algebra} \\
         \textsc{Definition} \ref{def:compvcadd}, \textsc{Page} \pageref{def:compvcadd} \textit{Vector Addition in \(\mathbb {C}^n\)} \\
         \textsc{Definition} \ref{def:compscalmult}, \textsc{Page} \pageref{def:compscalmult} \textit{Scalar Multiplication in \(\mathbb {C}^n\)} \\
         \textsc{Definition} \ref{def:compdotprod}, \textsc{Page} \pageref{def:compdotprod} \textit{The Dot Product in \(\mathbb {C}^n\)} \\
         \textsc{Definition} \ref{def:compmagn}, \textsc{Page} \pageref{def:compmagn} \textit{The Magnitude in \(\mathbb {C}^n\)} \\
         \textsc{Definition} \ref{def:adjoint}, \textsc{Page} \pageref{def:adjoint} \textit{The Adjoint} \\
         \textsc{Theorem} \ref{thm:transposedot}, \textsc{Page} \pageref{thm:transposedot} \textit{The Transpose and The Dot Product} \\
         \textsc{Theorem} \ref{thm:adjointdot}, \textsc{Page} \pageref{thm:adjointdot} \textit{The Adjoint and The Dot Product} \\
         \textsc{Definition} \ref{def:hermitskewhermit}, \textsc{Page} \pageref{def:hermitskewhermit} \textit{Hermitian and Skew-Hermitian Matrices} \\
         \textbf{Chapter} \ref{chapter:vcspcs}, \textsc{Page} \pageref{chapter:vcspcs} \\
         \textsc{Definition} \ref{def:vcspc}, \textsc{Page} \pageref{def:vcspc} \textit{Vector Spaces} \\
         \textsc{Theorem} \ref{thm:derpropvcspc1}, \textsc{Page} \pageref{thm:derpropvcspc1} \textit{Derived Property of Vector Space 1} \\
         \textsc{Theorem} \ref{thm:derpropvcspc2}, \textsc{Page} \pageref{thm:derpropvcspc2} \textit{Derived Property of Vector Space 2} \\
         \textsc{Theorem} \ref{thm:derpropvcspc3}, \textsc{Page} \pageref{thm:derpropvcspc3} \textit{Derived Property of Vector Space 3} \\
         \textsc{Theorem} \ref{thm:derpropvcspc4}, \textsc{Page} \pageref{thm:derpropvcspc4} \textit{Derived Property of Vector Space 4} \\
         \textsc{Definition} \ref{def:subspc}, \textsc{Page} \pageref{def:subspc} \textit{Subspaces} \\
         \textsc{Theorem} \ref{thm:showsubspc}, \textsc{Page} \pageref{thm:showsubspc} \textit{Showing a Vector Space is a Subspace} \\
         \textsc{Theorem} \ref{thm:eigsubspc}, \textsc{Page} \pageref{thm:eigsubspc} \textit{Eigenspaces are Subspaces} \\
         \textsc{Definition} \ref{def:finitelincomb}, \textsc{Page} \pageref{def:finitelincomb} \textit{Finite Linear Combinations} \\
         \textsc{Theorem} \ref{thm:lincombsubspc}, \textsc{Page} \pageref{thm:lincombsubspc} \textit{Subspaces are Closed Under Linear Combinations} \\
         \textsc{Definition} \ref{def:span}, \textsc{Page} \pageref{def:span} \textit{Span} \\
         \textsc{Theorem} \ref{thm:spanchrs}, \textsc{Page} \pageref{thm:spanchrs} \textit{A Complete Characterization of the Span} \\
         \textsc{Theorem} \ref{thm:twosubvcspcspan}, \textsc{Page} \pageref{thm:twosubvcspcspan} \textit{Two Subsets of a Vector Space and Their Spans} \\
         \textsc{Theorem} \ref{thm:spanintsubspc}, \textsc{Page} \pageref{thm:spanintsubspc} \textit{Span as an Intersection of Subspaces} \\
         \textsc{Theorem} \ref{thm:spanrowspc}, \textsc{Page} \pageref{thm:spanrowspc} \textit{Span and Row Space} \\
         \textsc{Theorem} \ref{thm:vcspan}, \textsc{Page} \pageref{thm:vcspan} \textit{Is a Vector in the Span?} \\
         \textsc{Definition} \ref{def:linindepdep}, \textsc{Page} \pageref{def:linindepdep} \textit{Linear Independence and Dependence} \\
         \textsc{Definition} \ref{def:genlinindepdep}, \textsc{Page} \pageref{def:genlinindepdep} \textit{Generalizing Linear Independence and Dependence to Infinite Sets} \\
         \textsc{Theorem} \ref{thm:showlinindep1}, \textsc{Page} \pageref{thm:showlinindep1} \textit{Showing Linear Independence 1} \\
         \textsc{Theorem} \ref{thm:showlinindep2}, \textsc{Page} \pageref{thm:showlinindep2} \textit{Showing Linear Independence 2} \\
         \textsc{Theorem} \ref{thm:onetwosetsdep}, \textsc{Page} \pageref{thm:onetwosetsdep} \textit{Linear Dependence of Sets With One or Two Elements} \\
         \textsc{Theorem} \ref{thm:finitesubsetcontain0}, \textsc{Page} \pageref{thm:finitesubsetcontain0} \textit{Linear Dependence of Finite Subsets of a Vector Space Containing \(\vec {0}\)} \\
         \textsc{Theorem} \ref{thm:linindepnonemptsets}, \textsc{Page} \pageref{thm:linindepnonemptsets} \textit{Linear Independence of Nonempty Sets} \\
         \textsc{Theorem} \ref{thm:lindeprn}, \textsc{Page} \pageref{thm:lindeprn} \textit{A Test for Linear Independence in \(\mathbb {R}^n\)} \\
         \textsc{Theorem} \ref{thm:indlindep}, \textsc{Page} \pageref{thm:indlindep} \textit{A Test for Linear Dependence in \(\mathbb {R}^n\)} \\
         \textsc{Definition} \ref{def:basis}, \textsc{Page} \pageref{def:basis} \textit{Basis} \\
         \textsc{Theorem} \ref{thm:basislemma}, \textsc{Page} \pageref{thm:basislemma} \textit{A Useful Lemma for Bases} \\
         \textsc{Theorem} \ref{thm:basisequivcard}, \textsc{Page} \pageref{thm:basisequivcard} \textit{Bases Have Equivalent Cardinality} \\
         \textsc{Definition} \ref{def:dimension}, \textsc{Page} \pageref{def:dimension} \textit{Dimension} \\
         \textsc{Theorem} \ref{thm:spandim1}, \textsc{Page} \pageref{thm:spandim1} \textit{General Statements About Span and Dimension 1} \\
         \textsc{Theorem} \ref{thm:spandim2}, \textsc{Page} \pageref{thm:spandim2} \textit{General Statements About Span and Dimension 2} \\
         \textsc{Theorem} \ref{thm:lindim1}, \textsc{Page} \pageref{thm:lindim1} \textit{General Statements About Linear Independence and Dimension 1} \\
         \textsc{Theorem} \ref{thm:lindim2}, \textsc{Page} \pageref{thm:lindim2} \textit{General Statements About Linear Independence and Dimension 2} \\
         \textsc{Theorem} \ref{thm:dimsubspc}, \textsc{Page} \pageref{thm:dimsubspc} \textit{Dimensions of Subspaces} \\
         \textsc{Theorem} \ref{thm:diagrev1}, \textsc{Page} \pageref{thm:diagrev1} \textit{Diagonalizability, Revisited: Part I} \\
         \textsc{Theorem} \ref{thm:basisforspan}, \textsc{Page} \pageref{thm:basisforspan} \textit{Finding a Basis for \(\Span (S)\) by Contraction} \\
         \textsc{Theorem} \ref{thm:basisexpand}, \textsc{Page} \pageref{thm:basisexpand} \textit{Finding a Basis by Expansion} \\
         \textsc{Definition} \ref{def:orderedbases}, \textsc{Page} \pageref{def:orderedbases} \textit{Ordered Bases} \\
         \textsc{Definition} \ref{def:coordswrtbasis}, \textsc{Page} \pageref{def:coordswrtbasis} \textit{Coordinates With Respect to a Basis} \\
         \textsc{Theorem} \ref{thm:coords}, \textsc{Page} \pageref{thm:coords} \textit{Coordinatization} \\
         \textsc{Theorem} \ref{thm:propcoords}, \textsc{Page} \pageref{thm:propcoords} \textit{Properties of Coordinatization} \\
         \textsc{Definition} \ref{def:transitionmatrix}, \textsc{Page} \pageref{def:transitionmatrix} \textit{Transition Matrices} \\
         \textsc{Theorem} \ref{thm:transitionmatrix}, \textsc{Page} \pageref{thm:transitionmatrix} \textit{Finding a Transition Matrix} \\
         \textsc{Theorem} \ref{thm:proptransmat}, \textsc{Page} \pageref{thm:proptransmat} \textit{Properties of Transition Matrices} \\
         \textsc{Theorem} \ref{thm:diagrev2}, \textsc{Page} \pageref{thm:diagrev2} \textit{Diagonalizability, Revisited: Part II} \\
         \textbf{Chapter} \ref{chapter:lintrans}, \textsc{Page} \pageref{chapter:lintrans} \\
         \textsc{Definition} \ref{def:lineartransformation}, \textsc{Page} \pageref{def:lineartransformation} \textit{Linear Transformations} \\
         \textsc{Theorem} \ref{thm:proplintrans}, \textsc{Page} \pageref{thm:proplintrans} \textit{Properties of Linear Transformations} \\
         \textsc{Theorem} \ref{thm:compositionslintrans}, \textsc{Page} \pageref{thm:compositionslintrans} \textit{Compositions of Linear Transformations} \\
         \textsc{Definition} \ref{def:linearoperator}, \textsc{Page} \pageref{def:linearoperator} \textit{Linear Operators} \\
         \textsc{Definition} \ref{def:idlinop}, \textsc{Page} \pageref{def:idlinop} \textit{The Identity Linear Operator} \\
         \textsc{Definition} \ref{def:zerolinop}, \textsc{Page} \pageref{def:zerolinop} \textit{The Zero Linear Operator} \\
         \textsc{Theorem} \ref{thm:lintranssubspc}, \textsc{Page} \pageref{thm:lintranssubspc} \textit{Linear Transformations and Subspaces} \\
         \textsc{Theorem} \ref{thm:lineartransformationsbases}, \textsc{Page} \pageref{thm:lineartransformationsbases} \textit{Linear Transformations and Bases} \\
         \textsc{Theorem} \ref{thm:matlintrans}, \textsc{Page} \pageref{thm:matlintrans} \textit{Matrices and Linear Transformations} \\
         \textsc{Theorem} \ref{thm:matricesconsdiffbases}, \textsc{Page} \pageref{thm:matricesconsdiffbases} \textit{Matrices for Linear Transformation, Considering Different Bases} \\
         \textsc{Theorem} \ref{thm:simmatlinops}, \textsc{Page} \pageref{thm:simmatlinops} \textit{Similar Matrices and Linear Operators} \\
         \textsc{Theorem} \ref{thm:matcomplintrans}, \textsc{Page} \pageref{thm:matcomplintrans} \textit{The Matrix of a Composition of Linear Transformations} \\
         \textsc{Definition} \ref{def:kernel}, \textsc{Page} \pageref{def:kernel} \textit{Kernel} \\
         \textsc{Definition} \ref{def:range}, \textsc{Page} \pageref{def:range} \textit{Range} \\
         \textsc{Theorem} \ref{thm:kerransubspc}, \textsc{Page} \pageref{thm:kerransubspc} \textit{Kernel and Range are Subspaces} \\
         \textsc{Theorem} \ref{thm:findker}, \textsc{Page} \pageref{thm:findker} \textit{Finding the Kernel of a Linear Transformation} \\
         \textsc{Theorem} \ref{thm:findrange}, \textsc{Page} \pageref{thm:findrange} \textit{Finding the Range of a Linear Transformation} \\
         \textsc{Definition} \ref{def:nullity}, \textsc{Page} \pageref{def:nullity} \textit{Nullity of a Linear Transformation} \\
         \textsc{Definition} \ref{def:ranklintrans}, \textsc{Page} \pageref{def:ranklintrans} \textit{Rank of a Linear Transformation} \\
         \textsc{Theorem} \ref{thm:dimthmrn}, \textsc{Page} \pageref{thm:dimthmrn} \textit{The Dimension Theorem (The Rank-Nullity Theorem), in \(\mathbb {R}^n\)} \\
         \textsc{Theorem} \ref{thm:detinjsurj}, \textsc{Page} \pageref{thm:detinjsurj} \textit{Determining Injectivity and Surjectivity} \\
         \textsc{Theorem} \ref{thm:detinjsurjequivdim}, \textsc{Page} \pageref{thm:detinjsurjequivdim} \textit{Determining Injectivity and Surjectivity With Equivalent Dimensions} \\
         \textsc{Theorem} \ref{thm:injlinindepsurjspan}, \textsc{Page} \pageref{thm:injlinindepsurjspan} \textit{Injectivity Implies Linear Independence, Surjectivity Implies Spanning} \\
         \textsc{Definition} \ref{def:isomorphisms}, \textsc{Page} \pageref{def:isomorphisms} \textit{Isomorphisms} \\
         \textsc{Definition} \ref{def:invtrans}, \textsc{Page} \pageref{def:invtrans} \textit{Invertible Linear Transformations} \\
         \textsc{Theorem} \ref{thm:isoinv}, \textsc{Page} \pageref{thm:isoinv} \textit{Isomorphism If And Only If Invertible} \\
         \textsc{Theorem} \ref{thm:findinv}, \textsc{Page} \pageref{thm:findinv} \textit{Finding an Inverse Matrix, if it Exists} \\
         \textsc{Theorem} \ref{thm:isopreslinindepspan}, \textsc{Page} \pageref{thm:isopreslinindepspan} \textit{Isomorphisms Preserve Linear Independence and Span} \\
         \textsc{Definition} \ref{def:isovec}, \textsc{Page} \pageref{def:isovec} \textit{Isomorphic Vector Spaces} \\
         \textsc{Theorem} \ref{thm:isoequivrel}, \textsc{Page} \pageref{thm:isoequivrel} \textit{\(\cong \) is an Equivalence Relation} \\
         \textsc{Theorem} \ref{thm:dimensionthm}, \textsc{Page} \pageref{thm:dimensionthm} \textit{The Dimension Theorem (The Rank-Nullity Theorem)} \\
         \textsc{Theorem} \ref{thm:equivdim}, \textsc{Page} \pageref{thm:equivdim} \textit{Isomorphism Implies Equivalent Dimension} \\
         \textsc{Theorem} \ref{thm:allisofn}, \textsc{Page} \pageref{thm:allisofn} \textit{All \(n\)-Dimensional Vector Spaces are Isomorphic to \(\mathbb {F}^n\)} \\
         \textsc{Definition} \ref{def:eigenvaluesandvectorslintrans}, \textsc{Page} \pageref{def:eigenvaluesandvectorslintrans} \textit{Eigenvalues and Eigenvectors} \\
         \textsc{Definition} \ref{def:eigenspacelintrans}, \textsc{Page} \pageref{def:eigenspacelintrans} \textit{Eigenspace} \\
         \textsc{Theorem} \ref{thm:findeigenvslin}, \textsc{Page} \pageref{thm:findeigenvslin} \textit{Finding Eigenvectors and Eigenvalues} \\
         \textsc{Definition} \ref{def:diaglintrans}, \textsc{Page} \pageref{def:diaglintrans} \textit{Diagonalizability of a Linear Operator} \\
         \textsc{Theorem} \ref{thm:diaglintrans}, \textsc{Page} \pageref{thm:diaglintrans} \textit{Diagonalizability of a Linear Operator} \\
         \textsc{Theorem} \ref{thm:eigenvecsdisteigenvalslinindep}, \textsc{Page} \pageref{thm:eigenvecsdisteigenvalslinindep} \textit{Eigenvectors With Distinct Eigenvalues are Linearly Independent} \\
         \textsc{Definition} \ref{def:algmultdiag}, \textsc{Page} \pageref{def:algmultdiag} \textit{Algebraic Multiplicity} \\
         \textsc{Definition} \ref{def:geomultdiag}, \textsc{Page} \pageref{def:geomultdiag} \textit{Geometric Multiplicity} \\
         \textsc{Theorem} \ref{thm:alggeomultdiag}, \textsc{Page} \pageref{thm:alggeomultdiag} \textit{Algebraic and Geometric Multiplicities and Diagonalizability} \\
         \textsc{Theorem} \ref{thm:unioninterbaseseigenspc}, \textsc{Page} \pageref{thm:unioninterbaseseigenspc} \textit{Union and Intersection of Bases for Eigenspaces} \\
         \textsc{Theorem} \ref{thm:procdiaglinops}, \textsc{Page} \pageref{thm:procdiaglinops} \textit{The Process of Diagonalization for Linear Operators} \\
         \textbf{Chapter} \ref{chapter:ortho}, \textsc{Page} \pageref{chapter:ortho} \\
         \textsc{Definition} \ref{def:innerprod}, \textsc{Page} \pageref{def:innerprod} \textit{Inner Products and Inner Product Spaces} \\
         \textsc{Theorem} \ref{thm:innerprodprops}, \textsc{Page} \pageref{thm:innerprodprops} \textit{Properties of Inner Products} \\
         \textsc{Definition} \ref{def:norms}, \textsc{Page} \pageref{def:norms} \textit{Norms} \\
         \textsc{Theorem} \ref{thm:propnorm}, \textsc{Page} \pageref{thm:propnorm} \textit{Properties of Norms} \\
         \textsc{Theorem} \ref{thm:cauchyschwarzgen}, \textsc{Page} \pageref{thm:cauchyschwarzgen} \textit{The Cauchy-Schwarz Inequality} \\
         \textsc{Theorem} \ref{thm:triineqgen}, \textsc{Page} \pageref{thm:triineqgen} \textit{The Triangle Inequality} \\
         \textsc{Definition} \ref{def:distance}, \textsc{Page} \pageref{def:distance} \textit{Distance} \\
         \textsc{Definition} \ref{def:angle}, \textsc{Page} \pageref{def:angle} \textit{Angle} \\
         \textsc{Definition} \ref{def:orthogonality}, \textsc{Page} \pageref{def:orthogonality} \textit{Orthogonality} \\
         \textsc{Theorem} \ref{thm:orthlinindep}, \textsc{Page} \pageref{thm:orthlinindep} \textit{Orthonormal Implies Linearly Independent} \\
         \textsc{Definition} \ref{def:orthogonalbasis}, \textsc{Page} \pageref{def:orthogonalbasis} \textit{Orthogonal Bases} \\
         \textsc{Definition} \ref{def:orthonormalbasis}, \textsc{Page} \pageref{def:orthonormalbasis} \textit{Orthonormal Bases} \\
         \textsc{Theorem} \ref{thm:gramschmidt}, \textsc{Page} \pageref{thm:gramschmidt} \textit{The Gram-Schmidt Process} \\
         \textsc{Theorem} \ref{thm:innerproductspaceorthobasis}, \textsc{Page} \pageref{thm:innerproductspaceorthobasis} \textit{Every Inner Product Space Has an Orthonormal Basis} \\
         \textsc{Theorem} \ref{thm:coordsorthogonal}, \textsc{Page} \pageref{thm:coordsorthogonal} \textit{Coordinatization With Respect to an Orthogonal Basis} \\
         \textsc{Theorem} \ref{thm:coordsorthonormal}, \textsc{Page} \pageref{thm:coordsorthonormal} \textit{Coordinatization With Respect to an Orthonormal Basis} \\
         \textsc{Definition} \ref{def:orthocomp}, \textsc{Page} \pageref{def:orthocomp} \textit{Orthogonal Complements} \\
         \textsc{Theorem} \ref{thm:lemmafindorthocomp}, \textsc{Page} \pageref{thm:lemmafindorthocomp} \textit{A Useful Lemma for Finding Orthogonal Complements} \\
         \textsc{Theorem} \ref{thm:subsub}, \textsc{Page} \pageref{thm:subsub} \textit{Subsets and Subspaces} \\
         \textsc{Theorem} \ref{thm:subspcs}, \textsc{Page} \pageref{thm:subspcs} \textit{Finite Dimensional Inner Product Spaces and Subspaces} \\
         \textsc{Theorem} \ref{thm:orthocompambientspc}, \textsc{Page} \pageref{thm:orthocompambientspc} \textit{The Orthogonal Complement of the Ambient Space} \\
         \textsc{Theorem} \ref{thm:orthocompzerovec}, \textsc{Page} \pageref{thm:orthocompzerovec} \textit{The Orthogonal Complement of the Zero Vector} \\
         \textsc{Definition} \ref{def:projectionssubspcog}, \textsc{Page} \pageref{def:projectionssubspcog} \textit{Projections Onto a Subspace With Orthogonal Bases} \\
         \textsc{Theorem} \ref{thm:projectionssubspcon}, \textsc{Page} \pageref{thm:projectionssubspcon} \textit{Projections Onto a Subspace With Orthonormal Bases} \\
         \textsc{Theorem} \ref{thm:projthm}, \textsc{Page} \pageref{thm:projthm} \textit{Projection Theorem} \\
         \textbf{Appendix} \ref{appendix:a}, \textsc{Page} \pageref{appendix:a} \\
         \textsc{Definition} \ref{def:proofs}, \textsc{Page} \pageref{def:proofs} \textit{Proofs} \\
         \textbf{Appendix} \ref{appendix:b}, \textsc{Page} \pageref{appendix:b} \\
         \textsc{Definition} \ref{def:functions}, \textsc{Page} \pageref{def:functions} \textit{Functions, Domains, and Codomains} \\
         \textsc{Definition} \ref{def:imagespreimage}, \textsc{Page} \pageref{def:imagespreimage} \textit{Images and Pre-Images} \\
         \textsc{Definition} \ref{def:rangef}, \textsc{Page} \pageref{def:rangef} \textit{Range} \\
         \textsc{Definition} \ref{def:injectivefunctions}, \textsc{Page} \pageref{def:injectivefunctions} \textit{Injective Functions} \\
         \textsc{Definition} \ref{def:surjectivefunctions}, \textsc{Page} \pageref{def:surjectivefunctions} \textit{Surjective Functions} \\
         \textsc{Definition} \ref{def:bijectivefunctions}, \textsc{Page} \pageref{def:bijectivefunctions} \textit{Bijective Functions} \\
         \textsc{Definition} \ref{def:comp}, \textsc{Page} \pageref{def:comp} \textit{Compositions} \\
         \textsc{Theorem} \ref{thm:compinjsur}, \textsc{Page} \pageref{thm:compinjsur} \textit{Compositions, Injectivity, and Surjectivity} \\
         \textsc{Definition} \ref{def:invfunc}, \textsc{Page} \pageref{def:invfunc} \textit{Inverse Functions} \\
         \textsc{Theorem} \ref{thm:existinv}, \textsc{Page} \pageref{thm:existinv} \textit{Existence of Inverse Functions} \\
         \textsc{Theorem} \ref{thm:uniqueinv}, \textsc{Page} \pageref{thm:uniqueinv} \textit{Uniqueness of Inverse Functions} \\
         \textsc{Theorem} \ref{thm:pigeonhole}, \textsc{Page} \pageref{thm:pigeonhole} \textit{The Pigeonhole Principle} \\
         \textsc{Theorem} \ref{thm:extpigeonhole}, \textsc{Page} \pageref{thm:extpigeonhole} \textit{The Extended Pigeonhole Principle} \\         
      }
   %\end{center}
\end{multicols}