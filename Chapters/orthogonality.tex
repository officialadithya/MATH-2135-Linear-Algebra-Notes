\section{Lecture 35: November 28, 2022}

    \subsection{Inner Product Spaces}

        Consider the following definition.
        \begin{definition}{\Stop\,\,Inner Products and Inner Product Spaces}{innerprod}

            An \(\mathbb{F}\) valued inner product on a vector space \(V\) is a function 
            \begin{equation*}
                \langle\cdot,\cdot\rangle:V\times V\to \mathbb{F}
            \end{equation*}
            such that
            \begin{enumerate}
                \item \(\forall\vec{v}\in V,\iprod{\vec{v}}{\vec{v}}\geq 0\).
                \item \(\vec{v}=\vec{0}_V\iff\iprod{\vec{v}}{\vec{v}}\).
                \item \(\forall\vec{u},\vec{v}\in V,\iprod{\vec{u}}{\vec{v}}=\bar{\iprod{\vec{v}}{\vec{u}}}\).
                \item \(\forall\vec{u},\vec{v},\vec{w}\in V,\iprod{\vec{u}+\vec{v}}{\vec{w}}=\iprod{\vec{u}}{\vec{w}}+\iprod{\vec{v}}{\vec{w}}\).
                \item \(\forall c\in\mathbb{F},\forall\vec{u},\vec{v}\in V,\iprod{c\vec{u}}{\vec{v}}=c\iprod{\vec{u}}{\vec{v}}\).
            \end{enumerate}
            \vphantom
            \\
            \\
            The pair \((V,\langle\cdot,\cdot\rangle)\) is called an inner product space.
            
        \end{definition}
        \pagebreak
        \vphantom
        \\
        \\
        Consider the following inner product spaces.
        \begin{enumerate}
            \item The pair \((\mathbb{R}^n, \iprod{\vec{u}}{\vec{v}}=\vec{u}\cdot\vec{v}=u_1v_1+\cdots+u_nv_n)\) is a real inner product space.
            \item The pair \((\mathbb{C}^n, \iprod{\vec{u}}{\vec{v}}=\vec{u}\cdot\vec{v}=u_1\bar{v_1}+\cdots+u_n\bar{v_n})\) is a complex inner product space.
            \item The pair 
                \begin{equation*}
                    \left(V,\iprod{\vec{f}}{\vec{g}}=\int_0^1 f(x)g(x)\dd x\right)
                \end{equation*}
                is an real inner product space, for 
                \begin{equation*}
                    V=\left\{f:[0,1]\to\mathbb{R}:\forall c, \lim_{x\to c}=f(c)\right\}.
                \end{equation*}
            \item The pair
            \begin{equation*}
                \left(\mathcal{P},\iprod{\vec{p}}{\vec{q}}=\int_0^\infty p(x)q(x)e^{-x}\dd x\right)
            \end{equation*}
            is a real inner product space.
        \end{enumerate}
        \pagebreak
        \vphantom
        \\
        \\
        Consider the following theorems and definitions.
        \begin{theorem}{\Stop\,\,Properties of Inner Products}{innerprodprops}

            Let \(V\) be an inner product space. Suppose \(\vec{u},\vec{v},\vec{w}\in V\) and \(c\in\mathbb{F}\). Then,
            \begin{enumerate}
                \item \(F:V\to \mathbb{F},\vec{v}\mapsto\iprod{\vec{v}}{\vec{u}}\) is a linear transformation.
                \begin{proof}
                    Consider \(\vec{v}_1,\vec{v}_2\in V\). Then,
                    \begin{align*}
                        F(\vec{v}_1+\vec{v}_2)&=\iprod{\vec{v}_1+\vec{v}_2}{\vec{u}} \\
                        &=\iprod{\vec{v}_1}{u}+\iprod{\vec{v}_2}{\vec{u}} \\
                        &=F(\vec{v}_1)+F(\vec{v}_2).
                    \end{align*}
                    For \(c\in\mathbb{F}\), we have
                    \begin{align*}
                        F(c\vec{v}_1)&=\iprod{c\vec{v}_1}{\vec{u}} \\
                        &=c\iprod{\vec{v}_1}{\vec{u}},
                    \end{align*}
                    as desired.
                \end{proof}
                \item \(\iprod{\vec{0}_V}{\vec{v}}=\vec{0}_V=\iprod{\vec{v}}{\vec{0}_V}\).
                \begin{proof}
                    The first equality is derived by the first part since, for all linear transformations \(L\), \(\vec{0}_V\in\ker(L)\). The second equality is derived from conjugate symmetry.
                \end{proof}
                \item \(\iprod{\vec{u}}{\vec{v}+\vec{w}}=\iprod{\vec{u}}{\vec{v}}+\iprod{\vec{u}}{\vec{w}}\).
                \begin{proof}
                    Consider \(\iprod{\vec{u}}{\vec{v}+\vec{w}}=\bar{\iprod{\vec{v}+\vec{w}}{\vec{u}}}=\bar{\iprod{\vec{v}}{\vec{u}}}+\bar{\iprod{\vec{w}}{\vec{u}}}=\iprod{\vec{u}}{\vec{v}}+\iprod{\vec{u}}{\vec{w}}\).
                \end{proof}
                \item \(\iprod{\vec{u}}{c\vec{v}}=\bar{c}\iprod{\vec{u}}{\vec{v}}\).
                \begin{proof}
                    Consider \(\iprod{\vec{u}}{c\vec{v}}=\bar{\iprod{c\vec{v}}{\vec{u}}}=\bar{c}\bar{\iprod{\vec{v}}{\vec{w}}}=\bar{c}\iprod{\vec{u}}{\vec{v}}\).
                \end{proof}
            \end{enumerate}
            
        \end{theorem}
        \begin{definition}{\Stop\,\,Norms}{norms}

            The norm associated to the inner product \(\langle\cdot,\cdot\rangle\) is
            \begin{equation*}
                ||\cdot||:V\to[0,\infty),\vec{v}\mapsto\sqrt{\langle\vec{v},\vec{v}\rangle}.
            \end{equation*}
            
        \end{definition}
        \pagebreak
        \begin{theorem}{\Stop\,\,}{...2}
            
            Let \(V\) be an inner product space. Suppose \(\vec{v},\vec{w}\in V\) and \(c\in\mathbb{F}\). Then,
            \begin{enumerate}
                \item \(||c\vec{v}||=|c|||\vec{v}||\).
                \item \(\langle \vec{v},\vec{w}\rangle\leq ||\vec{v}||||\vec{w}||\).
                \item \(||\vec{v}+\vec{w}||\leq||\vec{v}||+||\vec{w}||\).
            \end{enumerate}

        \end{theorem}
        \begin{definition}{\Stop\,\,Distance}{distance}

            Let \(V\) be an inner product space. The distance betwen \(\vec{v},\vec{w}\in V\) is \(||\vec{v}-\vec{w}||\).
            
        \end{definition}
        \begin{definition}{\Stop\,\,Angle}{angle}

            Let \(V\) be an inner product space. Let \(\mathbb{F}=\mathbb{R}\). The angle betwen \(\vec{v},\vec{w}\neq\vec{0}_V\) is given by
            \begin{equation*}
                \theta =\arccos\left(\frac{\langle\vec{v},\vec{w}\rangle}{||\vec{v}||||\vec{w}||}\right)
            \end{equation*}
            
        \end{definition}
        \begin{definition}{\Stop\,\,Orthogonality}{orthogonality}

            Let \(V\) be an inner product space. Then, 
            \begin{enumerate}
                \item \(\vec{v},\vec{w}\in V\) are orthogonal if and only if \(\langle\vec{v},\vec{w}\rangle=0\).
                \item A set \(S\subseteq V\) is orthogonal if and only if for each \(\vec{v},\vec{w}\in S\), \(\langle\vec{v},\vec{w}\rangle=0\).
                \item A set \(S\subseteq V\) is orthonormal if and only if it is orthogonal and each \(\vec{v}\in S\) has \(||\vec{v}||=1\).
            \end{enumerate}

        \end{definition}
        \vphantom
        \\
        \\
        Consider the following example.
        \begin{example}{\Difficulty\,\,The Standard Basis of \(\mathbb{R}^n\)}{stdbasisrn}

            The set \(\{\vec{e}_1,\ldots,\vec{e}_n\}\) is orthonormal when considered as a subset of \(\mathbb{R}^n\) or \(\mathbb{C}^n\).
            
        \end{example}
        \begin{theorem}{\Stop\,\,Orthonormal Implies Linearly Independent}{orthlinindep}
            
            If \(\{\vec{v}_1,\ldots,\vec{v}_n\}\) is orthonormal, \(\{\vec{v}_1,\ldots,\vec{v}_n\}\) is linearly independent.
            \begin{proof}
                Suppose
                \begin{equation*}
                    c_1\vec{v}_1+\cdots+c_n\vec{v}_n=\vec{0}_V
                \end{equation*}
                for some \(c\in\mathbb{F}\). We must show \(c_1=\cdots=c_n=0\). For each \(i\), \(1\leq i\leq n\), consider 
                \begin{align*}
                    0&=\langle c_1\vec{v}_1+\cdots+c_n\vec{v}_n,\vec{v}_i \rangle \\
                    &=\langle c_1\vec{v}_1,\vec{v}_i\rangle+\cdots+\langle c_n\vec{v}_n,\vec{v}_i\rangle \\
                    &=c_1\langle \vec{v}_1,\vec{v}_i\rangle+\cdots+c_n\langle \vec{v}_n,\vec{v}_i\rangle \\
                    &=c_i\langle \vec{v}_i,\vec{v}_i\rangle
                \end{align*}
                We see that \(\langle\vec{v}_i,\vec{v}_i\rangle>0\), since \(\vec{v}_i\neq\vec{0}_V\), we have \(0=c_i||\vec{v}_i||^2\), implying that \(c_i=0\) for each \(i\).
            \end{proof}
            The converse is false.

        \end{theorem}
        \pagebreak
        \vphantom
        \\
        \\
        Given a linearly indepedent set \(S_1=\{\vec{w}_1,\ldots,\vec{w}_n\}\), we seek to construct a set \(S_2=\{\vec{v}_1,\ldots,\vec{v}_k\}\) such that \(S\) is orthogonal and \(\Span(S_1)=\Span(S_2)\). We present the Gram-Schmidt process.
        \begin{theorem}{\Stop\,\,Gram-Schmidt Process}{gramschmidt}
            
            Let \(S_1=\{\vec{w}_1,\ldots,\vec{w}_n\}\) such that \(S_1\) is linearly independent.
            \begin{enumerate}
                \item Let \(\vec{v}_1=\vec{w}_1\)
                \item Let \(\vec{v}_2=\vec{w}_2-\left(\frac{\langle\vec{w}_2,\vec{v}_1\rangle}{\langle(\vec{v}_1,\vec{v}_1\rangle)}\right)\vec{v}_1\)
                \item \(\vdots\)
                \item Let \(\vec{v}_k=\)
            \end{enumerate}
            \DOTHISLATER

        \end{theorem}
        \vphantom
        \\
        \\
        We can also normalize an orthogonal set into an orthonormal one by just dividing by the magnitude of each \(\vec{v}_i\).
        \DOTHISLATER
        \begin{definition}{\Stop\,\,Orthonormal Bases}{orthonormalbasis}

            A set \(B\subseteq V\) is an orthonormal basis if and only if \(B\) is a basis of \(V\) and \(B\) is orthonormal.
            
        \end{definition}
        \begin{theorem}{\Stop\,\,Every Inner Product Space Has an Orthonormal Basis}{innerproductspaceorthobasis}

            Every inner product space has an orthonormal basis.
            \begin{proof}
                Suppose \(V\) is a finite dimensional inner product space. Let \(B=\{\vec{w}_1,\ldots,\vec{w}_n\}\) be a basis for \(V\). Apply the Gram-Schmidt process to find orthogonal \(\{\vec{v}_1,ldots,\vec{v}_n\}\) and normalize it. This set is orthonormal and, therefore, linearly indepedent, and the spans are the same.
                \DOTHISLATER
            \end{proof}
            
        \end{theorem}

        \pagebreak

\section{Lecture 36: November 30, 2022}

    \subsection{Orthogonal Complements and Projections}

        Consider the following definition.
        \begin{definition}{\Stop\,\,Orthogonal Complements}{orthocomp}

            Let \(V\) be an inner product space. Let \(S\subseteq V\). Then,
            \begin{equation*}
                S^{\perp}=\{\vec{v}\in V:\forall \vec{w}\in S, \iprod{\vec{v}}{\vec{w}}=0\}.
            \end{equation*}

        \end{definition}
        \vphantom
        \\
        \\
        Consider the following example.
        \begin{example}{\Difficulty\,\Difficulty\,\,Find Orthogonal Complement 1}{findorthocomp1}

            Let \(W=\Span([1,0,0],[0,1,0])\subseteq\mathbb{R}^3\). Find \(W^\perp\).
            \\
            \\
            We see that 
            \begin{equation*}
                W^\perp=\Span([0,0,1]).
            \end{equation*}
            Geometrically, \(W\) is the \(xy\) plane and \(W^\perp\) is the \(z\) axis.
            
        \end{example}
        \begin{example}{\Difficulty\,\Difficulty\,\,Find Orthogonal Complement 2}{findorthocomp2}

           Let \(V\) be an inner product space. Find \(\{\vec{0_V}\}^\perp\).
           \\
           \\
           We see \(\{\vec{0_V}\}^\perp=V\). Also, \(V^\perp=\{\vec{0}_V\}\).
        
        \end{example}
        \pagebreak
        \vphantom
        \\
        \\
        Consider the following theorems.
        \begin{theorem}{\Stop\,\,Subsets and Subspaces}{subsub}

            Let \(V\) be an inner product space. Let \(S\subseteq V\). Then,
            \begin{enumerate}
                \item \(S^\perp\) is a subspace of \(V\).
                \begin{proof}
                    Outline: Show nonempty, show closure properties.
                \end{proof}
                \item If \(W\) is a subspace of \(V\), \(W\cap W^\perp=\{\vec{0}_V\}\).
                \begin{proof}
                    Since \(W\) is a subspace of \(V\), \(\vec{0}_V\in W\). Since \(W^\perp\) is a subspace of \(V\) because \(W\subseteq V\), \(\vec{0}_V\in W^\perp\). Then, suppose \(\vec{w}\in W\cap W^\perp\). By Definition \ref{def:orthocomp}, \(\iprod{\vec{w}}{\vec{w}}=0\), so \(\vec{w}=\vec{0}_V\).
                \end{proof}
                \item \(S\subseteq (S^\perp)^\perp\).
            \end{enumerate}
            
        \end{theorem}
        \begin{theorem}{\Stop\,\,Finite Dimensional Inner Product Spaces and Subspaces}{subspcs}
            
            Suppose \(V\) is a finite dimensional inner product space and \(W\) is a subspace of \(V\). Then,
            \begin{enumerate}
                \item If \(\{\vec{v}_1,\ldots,\vec{v}_k\}\) is an orthonormal basis of \(W\) and \(\{\vec{v}_1,\ldots,\vec{v}_k,\vec{w}_1,\ldots,\vec{w}_{\ell}\}\) is an orthonormal basis for \(V\), \(\{\vec{w}_1,\ldots,\vec{w}_{\ell}\}\) is an orthonormal basis for \(W^\perp\).
                \begin{proof}
                    Since \(\{\vec{w}_1,\ldots,\vec{w}_{\ell}\}\) is orthonormal, it is linearly independent. We wish to show that \(\vec{w}_i\in W^\perp\) for \(1\leq i \leq \ell\) and \(\Span(\{\vec{w}_1,\ldots,\vec{w}_\ell\})=W^\perp\). For the spanning property, let \(\vec{w}\in W^\perp\). Then, \(\vec{w}\in V\), so
                    \begin{equation*}
                        \vec{w}=c_1\vec{v}_1+\cdots+c_k\vec{v}_k+d_1\vec{w}_1+\cdots+d_\ell\vec{v}_\ell
                    \end{equation*}
                    for \(c_1,\ldots,c_k\in\mathbb{F}\) and \(d_1,\ldots,d_\ell\in\mathbb{F}\). We wish to show \(c_1=\cdots=c_k=0\). Since \(\vec{w}\in W^\perp\) and \(\vec{v}_i\in W\), we have that \(\iprod{\vec{w}}{\vec{v}_i}=0\) for \(1\leq i\leq k\). But, we have
                    \begin{equation*}
                        \iprod{\vec{w}}{\vec{v}}=\iprod{\vec{w}_1=c_1\vec{v}_1+\cdots+c_k\vec{v}_k+d_1\vec{w}_1+\cdots+d_\ell\vec{v}_\ell}{\vec{v}_i}=c_i\iprod{\vec{v}_i}{\vec{v}_i}.
                    \end{equation*}
                    This implies \(c_i\iprod{\vec{v}_i}{\vec{v}_i}=0\). Since \(\vec{v}_i\neq\vec{0}_V\), \(c_i=0\), so
                    \begin{equation*}
                        \vec{w}=d_1\vec{w}_1+\cdots+d_\ell\vec{v}_\ell,
                    \end{equation*}
                    so \(W=\Span(\{\vec{w}_1,\ldots,\vec{w}_\ell\})\).
                \end{proof}
                \item \(\dim V=\dim W+\dim W^\perp\).
                \item \(W=(W^\perp)^\perp\).
            \end{enumerate}

        \end{theorem}
        \pagebreak
        \vphantom
        \\
        \\
        Consider the following definitions.
        \begin{definition}{\Stop\,\,Projections Onto a Subspace}{projectionssubspc}

            Let \(V\) be an inner product space. Let \(W\) be a subspace of \(V\).
            
        \end{definition}
        \DOTHISLATER
        \DOTHISLATER