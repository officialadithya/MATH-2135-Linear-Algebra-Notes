\section{Graphs and Digraphs}

    Consider the following definitions.
    \begin{definition}{\Stop\,\,Graphs}{graphs}
        
        A graph is a finite collection of vertices together with a finite collection of edges, each of which has two, not necessarily distinct, vertices, as endpoints.

    \end{definition}
    \begin{definition}{\Stop\,\,Digraphs}{digraphs}
        
        A digraph, or directed graph, is a special type of graph in which each edge is assigned a ``direction.'' Edges of a digraph are referred to as directed edges.

    \end{definition}
    \vphantom
    \\
    \\
    Graphs can be represented using matrices. Consider the following definition.
    \begin{definition}{\Stop\,\,The Adjacency Matrix}{adjmat}

        The adjacency matrix of a graph having vertices \(P_1,\ldots,P_n\) is the \(n\times n\) matrix whose \((i,j)\) entry is the number of edges connecting \(P_i\) and \(P_j\). Similarly, the adjacency matrix of a digraph having vertices \(P_1,\ldots,P_n\) is the \(n\times n\) matrix whose \((i,j)\) entry is the number of directed edges from \(P_i\) to \(P_j\).
        
    \end{definition}
    \pagebreak
    \vphantom
    \\
    \\
    Consider the following examples.
    \begin{example}{\Difficulty\,\Difficulty\,\,Finding an Adjacency Matrix 1}{findadj1}
        
    \end{example}