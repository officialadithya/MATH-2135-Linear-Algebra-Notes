To the interested reader,
\\
\\
This document is a compilation of lecture notes taken during the Fall 2022 semester for MATH2135: Linear Algebra for Mathematics Majors at the University of Colorado Boulder. The course used \textit{Elementary Linear Algebra}\footnote{Andrilli, S., \& Hecker, D. (2016). \textit{Elementary Linear Algebra} (5th ed.). Academic Press.} by Stephen Andrilli and David Hecker as its primary text. Supplemental texts included Sergei Treil's \textit{Linear Algebra Done Wrong}\footnote{Treil, S. (2017). \textit{Linear Algebra Done Wrong}. Sergei Treil.} and Sheldon Axler's \textit{Linear Algebra Done Right}\footnote{Axler, S. (2015). \textit{Linear Algebra Done Right} (3rd ed.). Springer. }. As such, many theorems, definitions, and content may be quoted or derived from the aforementioned books. This course was taught by Robin Deeley, Ph. D.
\\
\\
Appendix \ref{appendix:a} is provided as a nonexhaustive refresher on proof techniques commonly covered in a discrete mathematics course. Appendix \ref{appendix:a} is a result of lecture notes compiled during the Summer 2022 semester for MATH2520: Discrete Mathematics at Front Range Community College. This course was taught by Kenneth M. Monks, Ph. D. The primary text was Oscar Levin's \textit{Discrete Mathematics: An Open Introduction}\footnote{Levin, O. (2019). \textit{Discrete Mathematics: An Open Introduction} (3rd ed.). Oscar Levin.}, but Richard Hammack's \textit{Book of Proof}\footnote{Hammack, R. (2018). \textit{Book of Proof} (3rd ed.). Richard Hammack.} was also used as supplement.
\\
\\
Appendix \ref{appendix:b} is provided as a nonexhaustive refresher on functions, as preparation for Chapter \ref{chapter:lintrans}, the chapter on linear transformations. Appendix \ref{appendix:b} is a result of both content covered in \textit{Elementary Linear Algebra} and MATH2520 lecture notes.
\\
\\
Each chapter in this text has an accompanying quote, or accompanying image, from \href{https://xkcd.com/license.html}{XKCD}. These additions are tied to the content of the chapter and are meant to add lighthearted humor to the material. 
\\
\\
The author would like to provide the following resources for students currently taking a linear algebra course:
\begin{enumerate}
    \item \href{https://www.math.brown.edu/streil/papers/LADW/LADW_2017-09-04.pdf}{Sergei Treil's \textit{Linear Algebra Done Wrong}.}
    \item \href{https://link.springer.com/book/10.1007/978-3-319-11080-6?utm_medium=affiliate&utm_source=commission_junction_authors&utm_campaign=CONR_BOOKS_ECOM_GL_PHSS_ALWYS_DEEPLINK&utm_content=deeplink&utm_term=PID100197440&CJEVENT=f9f74b076a4c11ed80fe023d0a1c0e0d}{Sheldon Axler's \textit{Linear Algebra Done Right}.}
    \item \href{https://youtube.com/playlist?list=PL221E2BBF13BECF6C}{Gilbert Strang's Linear Algebra Lectures From Fall 2011.}
    \item \href{https://www.youtube.com/playlist?app=desktop&list=PLZHQObOWTQDPD3MizzM2xVFitgF8hE_ab}{3Blue1Brown's \textit{Essence of Linear Algebra}.}
\end{enumerate}
\vphantom
\\
\\
While much effort has been put in to remove typos and mathematical errors, it is very likely that some errors, both small and large, are present. Please keep in mind that the author wrote this resource during the first semester of his undergraduate studies. If an error needs to be resolved, please contact Adithya Bhaskara at \href{mailto:adithya.bhaskara@colorado.edu}{adithya.bhaskara@colorado.edu}.
\pagebreak
\\
\\
The current edition of this text consists purely of concepts covered in MATH2135; however, the author would like to add additional topics, not covered in the course, to provide for a more complete text. Possible future topics include, but are not limited to, orthogonal and unitary diagonalization procedures, matrix factorizations, optimization in \(\mathbb{R}^n\) with the Hessian matrix, and applications of linear algebra to differential equations. As MATH2135 is a proof-based course, the author will prioritize rigor in the presentation of any additional topics and will refrain from implementing sections where the necessary level of rigor has not been reached.
\\
\\
Because of this, future editions are more likely to resolve errata or reframe the discussion of existing material instead of implementing new sections entirely. By the end of MATH2135, a student will have a degree of mathematical maturity enough to make inroads into additional topics in linear algebra without direct instruction.
\\
\\
Finally, the author would like to dedicate this set of lecture notes to \textit{Aidan Janney}, \textit{Erika Sj\"{o}blom}, and \textit{Tate McDonald}, three of the author's closest friends who plan to take Linear Algebra in the upcoming semester; Spring 2023, at the time of writing.
\\
\\
\rightline{Best Regards,}
\rightline{Adithya Bhaskara}
\vfill
\rightline{\textbf{REVISED: \today}}