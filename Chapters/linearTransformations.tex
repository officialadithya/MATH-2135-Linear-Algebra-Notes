\section{Lecture 30: November 4, 2022}

    \subsection{An Introduction to Linear Transformations}

        Before proceeding into linear transformations, for a review of functions and associated terminology, consult Appendix \ref{appendix:b}. Consider the following definition.
        \begin{definition}{\Stop\,\,Linear Transformations}{lineartransformation}

            Let \(V\) and \(W\) be vector spaces. Let \(F:V\to W\) be a function. Then, \(F\) is a linear transformation if and only if both the following conditions hold:
            \begin{enumerate}
                \item \(\forall \vec{v}_1,\vec{v}_2\in V, F(\vec{v}_1)+F(\vec{v}_2)=F(\vec{v}_1)+F(\vec{v}_2)\).
                \item \(\forall c\in\mathbb{F},\forall\vec{v}\in V, F(c\vec{v})=cF(\vec{v})\).
            \end{enumerate}
            
        \end{definition}
        \vphantom
        \\
        \\
        We remark that a linear transformation ``preserves'' the operations that give structure to the vector spaces involved: vector addition and scalar multiplication.
        \pagebreak
        \\
        \\
        Consider the following examples.
        \begin{example}{\Difficulty\,\Difficulty\,\,Is it a Linear Transformation? 1}{lintrans1}

            Let \(F:\mathcal{M}_{mn}\to \mathcal{M}_{nm}\) where \(F(A)=A^T\). Is \(F\) a linear transformation?
            \\
            \\
            For matrices \(A_1,A_2\in\mathcal{M}_{mn}\) and scalar \(c\in\mathbb{R}\), we have
            \begin{align*}
                F(A_1+A_2)&=(A_1+A_2)^T \\
                &=A_1^T+A_2^T \\
                &=F(A_1)+F(A_2)
            \end{align*}
            and
            \begin{align*}
                F(cA_1)&=(cA_1)^T \\
                &=cA_1^T \\
                &=cF(A_1).
            \end{align*}
            Thus, \(F\) is a linear transformation.
            
        \end{example}
        \begin{example}{\Difficulty\,\Difficulty\,\,Is it a Linear Transformation? 2}{lintrans2}

            Let \(F:\mathcal{P}_{n}\to\mathcal{P}_{n-1}\) where \(F(\vec{p})=\vec{p}'\), the derivative of \(\vec{p}\). Is \(F\) a linear transformation?
            \\
            \\
            For \(\vec{p}_1,\vec{p}_2\in P\), we know, from Calculus, that the derivative of a sum is the sum of the derivatives, so
            \begin{align*}
                F(\vec{p}_1+\vec{p_2})&=(\vec{p}_1+\vec{p}_2)' \\
                &=\vec{p}_1'+\vec{p}_2' \\
                &=F(\vec{p}_1)+F(\vec{p}_2).
            \end{align*}
            For \(c\in\mathbb{R}\), the constant multiple rule, from Calculus, tells us that
            \begin{align*}
                F(c\vec{p}_1)&=(c\vec{p}_1)' \\
                &=c\vec{p_1}' \\
                &=cF(\vec{p_1}).
            \end{align*}
            Thus, \(F\) is a linear transformation.
        \end{example}
        \pagebreak
        \begin{example}{\Difficulty\,\Difficulty\,\,Is it a Linear Transformation? 3}{lintrans3}

            Let \(V\) be a vector space with \(\dim V=n\). Let \(B\) be an ordered basis for \(B\). Then, every \(\vec{v}\in V\) has coordinatization \([\vec{v}]_B\) with respect to \(B\). Consider the function \(F:V\to\mathbb{R}^n\) given by 
            \begin{equation*}
                F(\vec{v})=[\vec{v}]_B.
            \end{equation*}
            Is \(F\) a linear transformation?
            \\
            \\
            By Theorem \ref{thm:propcoords}, \(F\) is a linear transformation.

        \end{example}