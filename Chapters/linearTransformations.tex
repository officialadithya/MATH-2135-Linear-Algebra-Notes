\section{Lecture 30: November 4, 2022}

    \subsection{An Introduction to Linear Transformations}

        Before proceeding into linear transformations, for a review of functions and associated terminology, consult Appendix \ref{appendix:b}. Consider the following definition.
        \begin{definition}{\Stop\,\,Linear Transformations}{lineartransformation}

            Let \(V\) and \(W\) be vector spaces. Let \(F:V\to W\) be a function. Then, \(F\) is a linear transformation if and only if both the following conditions hold:
            \begin{enumerate}
                \item \(\forall \vec{v}_1,\vec{v}_2\in V, F(\vec{v}_1+\vec{v}_2)=F(\vec{v}_1)+F(\vec{v}_2)\).
                \item \(\forall c\in\mathbb{F},\forall\vec{v}\in V, F(c\vec{v})=cF(\vec{v})\).
            \end{enumerate}
            
        \end{definition}
        \vphantom
        \\
        \\
        We remark that a linear transformation ``preserves'' the operations that give structure to the vector spaces involved: vector addition and scalar multiplication.
        \pagebreak
        \\
        \\
        Consider the following examples.
        \begin{example}{\Difficulty\,\Difficulty\,\,Is it a Linear Transformation? 1}{lintrans1}

            Let \(F:\mathcal{M}_{mn}\to \mathcal{M}_{nm}\) where \(F(A)=A^T\). Is \(F\) a linear transformation?
            \\
            \\
            For matrices \(A_1,A_2\in\mathcal{M}_{mn}\) and scalar \(c\in\mathbb{R}\), we have
            \begin{align*}
                F(A_1+A_2)&=(A_1+A_2)^T \\
                &=A_1^T+A_2^T \\
                &=F(A_1)+F(A_2)
            \end{align*}
            and
            \begin{align*}
                F(cA_1)&=(cA_1)^T \\
                &=cA_1^T \\
                &=cF(A_1).
            \end{align*}
            Thus, \(F\) is a linear transformation.
            
        \end{example}
        \begin{example}{\Difficulty\,\Difficulty\,\,Is it a Linear Transformation? 2}{lintrans2}

            Let \(F:\mathcal{P}_n\to\mathcal{P}_{n-1}\) where \(F(\vec{p})=\vec{p}'\), the derivative of \(\vec{p}\). Is \(F\) a linear transformation?
            \\
            \\
            For \(\vec{p}_1,\vec{p}_2\in \mathcal{P}_n\), we know, from Calculus, that the derivative of a sum is the sum of the derivatives, so
            \begin{align*}
                F(\vec{p}_1+\vec{p_2})&=(\vec{p}_1+\vec{p}_2)' \\
                &=\vec{p}_1'+\vec{p}_2' \\
                &=F(\vec{p}_1)+F(\vec{p}_2).
            \end{align*}
            For \(c\in\mathbb{R}\), the constant multiple rule, from Calculus, tells us that
            \begin{align*}
                F(c\vec{p}_1)&=(c\vec{p}_1)' \\
                &=c\vec{p_1}' \\
                &=cF(\vec{p_1}).
            \end{align*}
            Thus, \(F\) is a linear transformation.
        \end{example}
        \pagebreak
        \begin{example}{\Difficulty\,\Difficulty\,\,Is it a Linear Transformation? 3}{lintrans3}

            Let \(F:\mathcal{P}_{n}\to W\) where \(W=\Span\left\{\frac{1}{s},\ldots,\frac{1}{s^{n+1}}\right\}\) and
            \begin{align*}
                F(\vec{p})&=\laplace{\vec{p}(t)}(s) \\
                &=\int_0^\infty e^{-st}\vec{p}(t)\dd t.
            \end{align*}
            Is \(F\) a linear transformation?
            \\
            \\
            For \(\vec{p}_1(t),\vec{p}_2(t)\in\mathcal{P}_n\), we have
            \begin{align*}
                F(\vec{p}_1(t)+\vec{p}_2(t))&=\int_0^\infty e^{-st}(\vec{p}_1(t)+\vec{p}_2(t))\dd t \\
                &=\int_0^\infty e^{-st}\vec{p}_1(t)+e^{-st}\vec{p}_2(t)\dd t \\
                &=\int_0^\infty e^{-st}\vec{p}_1(t)\dd t+\int_0^\infty e^{-st}\vec{p}_2(t)\dd t \\
                &=F(\vec{p}_1(t))+F(\vec{p}_2(t)).
            \end{align*}
            For \(c\in\mathbb{R}\), we have 
            \begin{align*}
                F(c\vec{p}_1(t))&=\int_0^\infty ce^{-st}\vec{p}_1(t)\dd t \\
                &=c\int_0^\infty e^{-st}\vec{p}_1(t)\dd t \\
                &=cF(\vec{p}_1(t)).
            \end{align*}
            Thus, \(F\) is a linear transformation.
        \end{example}
        \begin{example}{\Difficulty\,\Difficulty\,\,Is it a Linear Transformation? 4}{lintrans4}

            Let \(V\) be a vector space with \(\dim V=n\). Let \(B\) be an ordered basis for \(B\). Then, every \(\vec{v}\in V\) has coordinatization \([\vec{v}]_B\) with respect to \(B\). Consider the function \(F:V\to\mathbb{R}^n\) given by 
            \begin{equation*}
                F(\vec{v})=[\vec{v}]_B.
            \end{equation*}
            Is \(F\) a linear transformation?
            \\
            \\
            By Theorem \ref{thm:propcoords}, \(F\) is a linear transformation.

        \end{example}
        \pagebreak
        \vphantom
        \\
        \\
        We now state some properties of linear transformations.
        \begin{theorem}{\Stop\,\,Properties of Linear Transformations}{proplintrans}

            Let \(V\) and \(W\) be vector spaces, and let \(L:V\to W\) be a linear transformation. Let \(\vec{0}_V\) be the zero vector in \(V\) and \(\vec{0}_W\) be the zero vector in \(W\). Then,
            \begin{enumerate}
                \item \(L(\vec{0}_V)=L(\vec{0}_W)\).
                \begin{proof}
                    Consider \(L(\vec{0}_V)=L(0\vec{0}_V)=0L(\vec{0}_V)=\vec{0}_W\), as desired.
                \end{proof}
                \item \(L(-\vec{v})=-L(\vec{v})\).
                \begin{proof}
                    Consider \(L(-\vec{v})=L(-1\vec{v})=-L(\vec{v})\), as desired.
                \end{proof}
                \item \(L(c_1\vec{v}_1+\cdots+c_n\vec{v}_n)=c_1L(\vec{v}_1)+\cdots+c_nL(\vec{v}_n)\) for \(c_1,\ldots,c_n\in\mathbb{F}\) and \(\vec{v}_1,\ldots,\vec{v}_n\in V\) with \(n\geq2\).
                \begin{proof}
                    We proceed by induction. For the base case when \(n=2\), we have
                    \begin{align*}
                        L(c_1\vec{v}_1+c_2\vec{v}_2)&=L(c_1\vec{v}_1)+L(c_2\vec{v}_2) \\
                        &=c_1L(\vec{v}_1)+c_2L(\vec{v}_2).
                    \end{align*}
                    Then, suppose that for all \(n=k\), 
                    \begin{equation*}
                        L(c_1\vec{v}_1+\cdots+c_k\vec{v}_k)=c_1L(\vec{v}_1)+\cdots+c_kL(\vec{v}_k).
                    \end{equation*}
                    Then, we have
                    \begin{align*}
                        L(c_1\vec{v}_1+\cdots+c_k\vec{v}_k+c_{k+1}v_{k+1})&=c_1L(\vec{v}_1)+\cdots+c_kL(\vec{v}_k)+L(c_{k+1}\vec{v}_{k+1}) \\
                        &=c_1L(\vec{v}_1)+\cdots+c_kL(\vec{v}_k)+c_{k+1}L(\vec{v}_{k+1}),
                    \end{align*}
                    as desired.
                \end{proof}
            \end{enumerate}
            
        \end{theorem}
        \vphantom
        \\
        \\
        We remark that not every function between vector spaces is a linear transformation. To show that some function between vector spaces is not a linear transformation, we must show a counterexample of the conditions in Definition \ref{def:lineartransformation}.
        \pagebreak
        \\
        \\
        We now turn to compositions of linear transformations.
        \begin{theorem}{\Stop\,\,Compositions of Linear Transformations}{compositionslintrans}

            Let \(V_1\), \(V_2\), and \(V_3\) be vector spaces and \(L_1:V_1\to V_2\) and \(L_2:V_2\to V_3\) be linear transformations. Then, \((L_2\circ L_1):V_1\to V_3\) with \((L_2\circ L_1)(\vec{v})=L_2(L_1(\vec{v}))\) is a linear transformation.
            \begin{proof}
                For \(\vec{v}_1,\vec{v}_2\in V_1\), we have
                \begin{align*}
                    (L_2\circ L_1)(\vec{v}_1+\vec{v}_2)&=L_2(L_1(\vec{v}_1+\vec{v}_2)) \\
                    &=L_2(L_1(\vec{v}_1)+L_1(\vec{v}_2)) \\
                    &=L_2(L_1(\vec{v}_1))+L_2(L_1(\vec{v}_2)) \\
                    &=(L_2\circ L_1)(\vec{v}_1)+(L_2\circ L_1)(\vec{v}_2).
                \end{align*}
                Then, for \(c\in\mathbb{F}\), we have
                \begin{align*}
                    (L_2\circ L_1)(c\vec{v}_1)&=L_2(L_1(c\vec{v}_1)) \\
                    &=L_2(cL_1(\vec{v}_1)) \\
                    &=cL_2(L_1(\vec{v}_1)) \\
                    &=c(L_2\circ L_1)(\vec{v}_1),
                \end{align*}
                as desired.
            \end{proof}            
        \end{theorem}
        \vphantom
        \\
        \\
        We now define a special case of linear transformations: linear operators.
        \begin{definition}{\Stop\,\,Linear Operators}{linearoperator}

            Let \(V\) be a vector space. A linear operator on \(V\) is a linear transformation whose domain and codomain are both \(V\).
            
        \end{definition}