\section{Lecture 20: October 7, 2022}

    \subsection{The Process of Abstraction}

        Consider the following definition.
        \begin{definition}{\Stop\,\,Vector Spaces}{vcspc}

            Let \(\mathbb{F}\) be a field of scalars. For now, \(\mathbb{F}=\mathbb{R}\vee\mathbb{C}\). A vector space \(V\) over \(\mathbb{F}\) is a set with two operations: 
            \begin{enumerate}
                \item Vector Addition: \(V\times V\to V, (\vec{v},\vec{w})\mapsto \vec{v}+\vec{w}\).
                \item Scalar Multiplication: \(\mathbb{F}\times V\to V, (c,\vec{v})\mapsto c\vec{v}\).
            \end{enumerate}
            \vphantom
            \\
            \\
            The following axioms must hold for each \(\vec{u},\vec{v},\vec{w}\in V\) and \(c_1,c_2\in\mathbb{F}\).
            \begin{enumerate}
                \item \(\vec{u}+\vec{v}=\vec{v}+\vec{u}\)
                \item \(\vec{u}+(\vec{v}+\vec{w})=(\vec{u}+\vec{v})+\vec{w}\).
                \item \(\exists \vec{0}\in V, \forall \vec{v}\in V, \vec{0}+\vec{v}=\vec{v}\)
                \item \(\forall \vec{v}\in V, \exists! (-\vec{v})\in V, \vec{v}+(-\vec{v})=\vec{0}\).
                \item \(c_1(\vec{u}+\vec{v})=c_1\vec{u}+c_1\vec{v}\).
                \item \((c_1+c_2)\vec{u}=c_1\vec{u}+c_2\vec{u}\).
                \item \((c_1c_2)\vec{u}=c_1(c_2\vec{u})\).
                \item \(1\vec{u}=\vec{u}\).
            \end{enumerate}
            
        \end{definition}
        \vphantom
        \\
        \\
        We remark that \(0\vec{v}=\vec{0}\) is \textit{not} an axiom of a vector space; we must \textit{prove} that it holds. Now, we will justify our use of abstraction.
        \begin{enumerate}
            \item The set \(\mathbb{R}^n\) is a vector space.
            \item The set \(\mathbb{C}^n\) is a vector space.
            \item The set \(\{\vec{0}\}\), with \(\vec{0}+\vec{0}+\vec{0}\) and \(c\vec{0}=\vec{0}\), is a vector space.
            \item The set \(\mathcal{M}_{mn}\) is a vector space.
            \item Let \(S\) be a non-empty set and \(F(S)=\{f:S\to\mathbb{R}\}\) with \((f+g)(s)=f(s)+g(s)\) and \((cf)(s)=cf(s)\). The set \(F(S)\) is a vector space.
            \item The set \(\{a_nx^n+\cdots+a_0x^0:a_0,\ldots a_n\in\mathbb{R}\}\) is a vector space.
            \item The set \(\{a_nx^n+\cdots+a_0x^0:a_0,\ldots a_n\in\mathbb{R}, \text{degree is less than or equal to \(n\)}\}\) is a vector space.
        \end{enumerate}
        \vphantom
        \\
        \\
        We note that all the above examples need \textit{proof}. We remark that, for now, we will primarily consider vector spaces over \(\mathbb{R}\). We will emphasize when we consider vector spaces over \(\mathbb{C}\).  Consider the following examples of sets that are not vector spaces.
        \begin{enumerate}
            \item The set \(\mathbb{R}^+=\{x\in\mathbb{R}:x\geq0\}\), with usual addition and usual multiplication in \(\mathbb{R}\), is not a vector space. The set does not satisfy the closure axiom for scalar multiplication; \((-1)\cdot1=-1\nin\mathbb{R}^+\).
            \item The set \(S=\{f:\mathbb{R}\to\mathbb{R}:f(0)=1\}\), with \((f+g)(x)=f(x)+g(x)\) and \((cf)(x)=cf(x)\), is not a vector space. The set does not satisfy the closure axiom for vector addition; \((f+g)(0)=f(0)+g(0)=2\), so \((f+g)(x)\nin S\).
        \end{enumerate}

\pagebreak

\section{Lecture 21: October 14, 2022}

    \subsection{Derived Properties of Vector Spaces}

        Consider the following theorem.
        \begin{theorem}{\Stop\,\,Derived Properties of Vector Spaces}{derpropvcspc}
            
            Suppose \(V\) is a vector space, \(\vec{v}\in V\), and \(c\in\mathbb{F}\). Then,
            \begin{enumerate}
                \item \(c\vec{0}=\vec{0}\).
                \begin{proof}
                    We note that by axiom \(3\) of Definition \ref{def:vcspc},
                    \begin{equation*}
                        c\vec{0}=c\vec{0}+\vec{0}.
                    \end{equation*}
                    By axiom \(4\),
                    \begin{equation*}
                        c\vec{0}=c\vec{0}+c\vec{0}+(-c\vec{0}).
                    \end{equation*}
                    Then, by axiom \(5\),
                    \begin{equation*}
                        c\vec{0}=(\vec{0}+\vec{0})+(-c\vec{0}).
                    \end{equation*}
                    By axiom \(3\), we have
                    \begin{equation*}
                        c\vec{0}+(-c\vec{0}).
                    \end{equation*}
                    Then, by axiom \(4\), we have
                    \begin{equation*}
                        c\vec{0}=\vec{0},
                    \end{equation*}
                    and we have proved the proposition.
                \end{proof}
                \item \(0\vec{v}=\vec{0}\).
                \begin{proof}
                    Consider, by axiom \(3\),
                    \begin{equation*}
                        0\vec{v}=0\vec{v}+\vec{0}.
                    \end{equation*}
                    Then, by axiom \(4\), we have
                    \begin{equation*}
                        0\vec{v}=0\vec{v}+0\vec{v}+(-0\vec{v}).
                    \end{equation*}
                    By axiom \(6\),
                    \begin{equation*}
                        0\vec{v}=(0+0)\vec{v}+(-0\vec{v})=0\vec{v}+(-0\vec{v}).
                    \end{equation*}
                    Finally, by axiom \(4\), 
                    \begin{equation*}
                        0\vec{v}=\vec{0}
                    \end{equation*}
                    and we have proved the proposition.
                \end{proof}
                \pagebreak
                \item \((-1)\vec{v}=-\vec{v}\).
                \begin{proof}
                    Consider, by axiom \(8\),
                    \begin{equation*}
                        \vec{v}+(-1)\vec{v}=1\vec{v}+(-1)\vec{v}.
                    \end{equation*}
                    Then, by axiom \(6\),
                    \begin{equation*}
                        \vec{v}+(-1)\vec{v}=(1+(-1))\vec{v}=0\vec{v}.
                    \end{equation*}
                    By the above proof, 
                    \begin{equation*}
                        \vec{v}+(-1)\vec{v}=\vec{0},
                    \end{equation*}
                    which implies \((-1)\vec{v}=-\vec{v}\), the additive inverse of \(\vec{v}\).
                \end{proof}
                \item \(c\vec{v}=\vec{0}\iff c=0\vee \vec{v}=\vec{0}\).
                \begin{proof}
                    The statement \(c=0\vee\vec{v}=\vec{0}\implies c\vec{v}=\vec{0}\) is governed by the first two derived results. Then, to show that \(c\vec{v}=\vec{0}\implies c=0\vee\vec{v}=\vec{0}\), we suppose that \(c\neq0\) and wish to show that \(\vec{v}=\vec{0}\). We have
                    \begin{equation*}
                        c\vec{v}=\vec{0}
                    \end{equation*}
                    with \(c\neq0\). Then,
                    \begin{equation*}
                        \left(\frac{1}{c}\right)(c\vec{v})=\frac{1}{c}\vec{0}=\vec{0},
                    \end{equation*}
                    by the second result. But, by axiom \(7\), 
                    \begin{equation*}
                        \frac{1}{c}(c\vec{v})=\left(\frac{1}{c}c\vec{v}\right)=\vec{v},
                    \end{equation*}
                    which implies \(\vec{v}=\vec{0}\).
                \end{proof}

            \end{enumerate}

        \end{theorem}

\pagebreak

\section{Lecture 22: October 17, 2022}

    \subsection{Subspaces}

        Consider the following definition.
        \begin{definition}{\Stop\,\,Subspaces}{subspc}

            Suppose \(V\) is a vector space and the set \(W\subseteq V\). Then, \(W\) is a subspace of \(V\) if and only if \(W\) is a vector space with the same operations as \(V\).
            
        \end{definition}